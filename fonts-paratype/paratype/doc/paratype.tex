%% Copyright 2011 Pavel Farar
%
% This work may be distributed and/or modified under the
% conditions of the LaTeX Project Public License, either version 1.3
% of this license or (at your option) any later version.
% The latest version of this license is in
%   http://www.latex-project.org/lppl.txt
% and version 1.3 or later is part of all distributions of LaTeX
% version 2005/12/01 or later.
%
% This work has the LPPL maintenance status `maintained'.
% 
% The Current Maintainer of this work is Pavel Farar


\documentclass[oneside]{scrartcl}

\usepackage[utf8]{inputenc}
\usepackage[czech, english]{babel}
\usepackage[IL2, T1]{fontenc}
\usepackage{paratype}

\usepackage{booktabs}

\newcommand{\code}{\texttt}
\newcommand{\program}{\texttt}

\usepackage{color}
\newcommand{\faked}{\textcolor{red}}


\usepackage{textcomp}
\usepackage[colorlinks]{hyperref}
\hypersetup{pdfauthor={Pavel Farar}}


\title{Support package for free fonts by ParaType}
\author{Pavel Farář\\
\href{mailto:pavel.farar@centrum.cz}{pavel.farar@centrum.cz}}

\begin{document}

\selectlanguage{english}%

\maketitle

\tableofcontents


\section{Introduction}

This package contains the LaTeX support for the fonts PT Sans,
PT Serif and PT Mono released by ParaType.
PT Sans has four basic styles, two narrow styles and two caption styles.
PT Serif has four basic styles and two caption styles. PT Mono has just
the regular and bold styles.

\begin{figure}[!htb]
\centering{%
  \begin{tabular}{ll}
    {\usefont{T1}{PTSans-TLF}{m}{n}PT Sans Regular} &
    {\usefont{T1}{PTSans-TLF}{b}{n}PT Sans Bold} \\
    {\usefont{T1}{PTSans-TLF}{m}{it}PT Sans Italic} &
    {\usefont{T1}{PTSans-TLF}{b}{it}PT Sans Bold Italic} \\
    {\usefont{T1}{PTSansNarrow-TLF}{m}{n}PT Sans Narrow} &
    {\usefont{T1}{PTSansNarrow-TLF}{b}{n}PT Sans Narrow Bold} \\
    {\usefont{T1}{PTSansCaption-TLF}{m}{n}PT Sans Caption} &
    {\usefont{T1}{PTSansCaption-TLF}{b}{n}PT Sans Caption Bold}
  \end{tabular}
\caption{The styles of PT Sans}}
\end{figure}


\begin{figure}[!htb]
\centering{%
  \begin{tabular}{ll}
    {\usefont{T1}{PTSerif-TLF}{m}{n}PT Serif Regular} &
    {\usefont{T1}{PTSerif-TLF}{b}{n}PT Serif Bold} \\
    {\usefont{T1}{PTSerif-TLF}{m}{it}PT Serif Italic} &
    {\usefont{T1}{PTSerif-TLF}{b}{it}PT Serif Bold Italic} \\
    {\usefont{T1}{PTSerifCaption-TLF}{m}{n}PT Serif Caption} &
    {\usefont{T1}{PTSerifCaption-TLF}{m}{it}PT Serif Caption Italic}
  \end{tabular}
\caption{The styles of PT Serif}}
\end{figure}

\begin{figure}[!htb]
\centering{%
  \begin{tabular}{ll}
    {\usefont{T1}{PTMono-TLF}{m}{n}PT Mono Regular} &
    {\usefont{T1}{PTMono-TLF}{b}{n}PT Mono Bold}
  \end{tabular}
\caption{The styles of PT Mono}}
\end{figure}

The fonts cover standard Western, Central European and Cyrillic code
pages and contain also characters of all title languages of the
Russian Federation.This package supports encodings T1, OT1, OT2, IL2,
TS1, T2A, T2B, T2C and X2. The fonts are included in the original
TrueType format and in the Type 1 format (without PT Mono Bold).

The fonts were developed by ParaType for the project \emph{Public Types
of Russian Federation} and released under an open user license.
They were designed by Alexandra Korolkova, Olga Umpeleva, Isabella Chaeva
and Vladimir Yefimov. For more information see the
web-site of \href{http://www.paratype.com}{ParaType}.


\section{License}

The fonts (both TrueType and Type 1) are licensed under Paratype Free
Font License (the license is included).

The fonts in TrueType format are original files designed by ParaType. The
fonts in Type 1 format are also original files kindly provided by ParaType.

All the support files are licensed under \LaTeX\ Project Public License,
either version 1.3 of this license or (at  your option) any later version.


\section{Using the package}

The package was created using \code{fontools}, but with some changes. From
\code{fontools} also come the mixed case names and the suffix \code{-TLF}
(tabular lining figures) of the font families and also the mixed case
names of the packages for the individual font families.


\subsection{Choosing the right map file}

There are two map files: \code{paratype-truetype.map} and
\code{paratype-type1.map}.
You should use just one of these map files. 

The map file \code{paratype-type1.map} is good for \program{dvips}. It sets
Type~1 fonts for PT Sans, PT Serif and PT Mono.
This map file is a safe choice if you do not know which map file to use.

The map file \code{paratype-truetype.map} is good for \program{pdf\TeX}. It sets
TrueType fonts for PT Sans, PT Serif and PT Mono. You can use it if you prefer
TrueType fonts.

You install the required map file with a command like:
\begin{verbatim}
updmap-sys --enable Map=paratype-type1.map
\end{verbatim}

If you want to change the used map file, remember to uninstall the old map file
before you install the new one:
\begin{verbatim}
updmap-sys --disable paratype-type1.map
updmap-sys --enable Map=paratype-truetype.map
\end{verbatim}


\subsection{Setting the fonts in a document}

There are several packages that you can use to set the fonts. You
must explicitly use the package \code{fontenc} or \code{textcomp} (if
needed). The easiest way to use the fonts is the package \code{paratype}:
\begin{verbatim}
\usepackage[T1]{fontenc}
\usepackage{paratype}
\end{verbatim}

This sets the font \code{PTSerif-TLF} as the default serif family,
the font \code{PTSans-TLF} as the default sans serif family and the
font \code{PTMono-TLF} as the default monospaced family.

There are also several packages that set the individual font families.
They have the same name as the font family, but without the suffix:
\code{PTSerif}, \code{PTSerifCaption},
\code{PTSans}, \code{PTSansNarrow}, \code{PTSansCaption}
and \code{PTMono}.

For example the package \code{PTSerif} sets the font \code{PTSerif-TLF}
and you use it this way:
\begin{verbatim}
\usepackage[T1]{fontenc}
\usepackage{PTSerif}
\end{verbatim}


If you want to use this font together with another font that has different
x-height, you can use the option \code{scaled}:
\begin{verbatim}
\usepackage[scaled=0.9]{PTSerif}
\end{verbatim}

You can also typeset some text in a desired font like this:
\begin{verbatim}
{\usefont{T1}{PTSerifCaption-TLF}{m}{it}Text in PT Serif
Caption Italic}
\end{verbatim}


\subsection{Faked styles}
\label{faked-styles}

PT Serif and PT Serif Caption have faked slanted and upright italic styles. PT Serif has them also in bold weight.
PT Sans Caption, PT Sans Narrow and PT Mono have faked slanted and bold
slanted styles.

As all faked styles, they can not reach the quality of the true designed
styles and there can be some problems. In PT Serif, there are problems with
e.g. mathematical operators---look at the font samples, it is especially obvious
for the upright italic style.

However, these styles can be useful when used reasonably and not too often.
You have the choice now.

Different engines treat these styles differently, \program{dvips} embeds just
the original fonts in the document, while \program{pdf\TeX} embeds faked fonts.
Moreover, \program{pdf\TeX} behaves differently for Type 1 and TrueType fonts.
In the former case it slants the Type 1 fonts and embeds them, in the latter
case it does not slant the TrueType fonts and embeds bitmap Type 3 fonts instead.


\subsection{The available series/shape combinations}

You can see the available series/shape combinations in the
table~\ref{tableShape}. The table shows the combinations that really
exist in the fonts in black and faked styles in red.

There is also automatic substitution. If you request the bold extended series,
you will get the bold series instead.

\begin{table}[!htb]
\centering{%
  \begin{tabular}{ll}
    \toprule
    font family & series/shape combinations \\
    \midrule
    PTSans-TLF & m/n, m/it, b/n, b/it, c/n, bc/n \\
    PTSansNarrow-TLF & m/n, b/n, \faked{m/sl}, \faked{b/sl} \\
    PTSansCaption-TLF & m/n, b/n, \faked{m/sl}, \faked{b/sl} \\
    PTSerif-TLF & m/n, m/it, b/n, b/it,
      \faked{m/sl}, \faked{m/ui}, \faked{b/sl}, \faked{b/ui} \\
    PTSerifCaption-TLF & m/n, m/it, \faked{m/sl}, \faked{m/ui} \\
    PTMono-TLF & m/n, b/n, \faked{m/sl}, \faked{b/sl} \\
    \bottomrule
  \end{tabular}
\caption{The series/shape combinations for the fonts}\label{tableShape}}
\end{table}

The family PTSans-TLF has also both narrow styles. The caption styles
for both PT Sans and PT Serif are available only as separate families.
Especially the caption styles in PT Serif are more different from the
four basic styles than it is usual for other fonts, therefore everyone
should decide if and where he will use them.


\subsection{Notes for the Czech and Slovak users}

I changed the width of some letters in the encoding IL2: Lcaron
in PT Serif Bold and PT Serif Bold Italic; tcaron in PT Serif Bold;
dcaron in PT Sans Regular and PT Sans Bold and tcaron in PT Sans Bold.
In all cased I made the width equal to that of the unaccented letter.
Now all letters dcaron, tcaron, lcaron and Lcaron have the same width
as the unaccented letter in the encoding IL2. The encoding T1 was left
untouched. You can see the differences in the following samples (first
line is in T1 and second in IL2).

\selectlanguage{czech}
V kódování IL2 jsem změnil šířku znaku Ľ ve fontech PT Serif Bold a
PT Serif Bold Italic; dále šířku znaku ť ve fontu PT Serif Bold; pak
šířku znaku ď ve fontech PT Sans Regular a PT Sans Bold a nakonec
šířku znaku ť ve fontu PT Sans Bold. Ve všech případech jsem změnil
šířku tak, že je stejná jako šířka stejného písmene bez diakritiky.
Nyní mají v kódování IL2 všechny znaky ď, ť, ľ a Ľ stejnou šířku
jako stejné znaky bez háčku. Kódování T1 bylo ponecháno beze změny.
Rozdíly je možno vidět v následujících ukázkách (první řádka je pro
kódování T1, druhá pro IL2).

\begin{figure}[!htb]
\centering{%
  {\Huge%
  \begin{tabular}{llllll}%
    {\usefont{T1}{PTSerif-TLF}{b}{n}Ľubomír} &
    {\usefont{T1}{PTSerif-TLF}{b}{it}Ľubomír} &
    {\usefont{T1}{PTSerif-TLF}{b}{n}šťovík} &
    {\usefont{T1}{PTSerif-TLF}{b}{n}laťka}\\
    {\usefont{IL2}{PTSerif-TLF}{b}{n}Ľubomír} &
    {\usefont{IL2}{PTSerif-TLF}{b}{it}Ľubomír} &
    {\usefont{IL2}{PTSerif-TLF}{b}{n}šťovík} &
    {\usefont{IL2}{PTSerif-TLF}{b}{n}laťka}
  \end{tabular}}
  \caption{Changes in PT Serif}}
\end{figure}


\begin{figure}[!htb]
\centering{%
  {\Huge%
  \begin{tabular}{llllll}%
    {\usefont{T1}{PTSans-TLF}{m}{n}ďas} &
    {\usefont{T1}{PTSans-TLF}{m}{n}loďka} &
    {\usefont{T1}{PTSans-TLF}{b}{n}ďas} &
    {\usefont{T1}{PTSans-TLF}{b}{n}loďka} &
    {\usefont{T1}{PTSans-TLF}{b}{n}šťovík} &
    {\usefont{T1}{PTSans-TLF}{b}{n}laťka}\\
    {\usefont{IL2}{PTSans-TLF}{m}{n}ďas} &
    {\usefont{IL2}{PTSans-TLF}{m}{n}loďka} &
    {\usefont{IL2}{PTSans-TLF}{b}{n}ďas} &
    {\usefont{IL2}{PTSans-TLF}{b}{n}loďka} &
    {\usefont{IL2}{PTSans-TLF}{b}{n}šťovík} &
    {\usefont{IL2}{PTSans-TLF}{b}{n}laťka}
  \end{tabular}}
  \caption{Changes in PT Sans}}
\end{figure}

\selectlanguage{english}


\section{Known bugs}

I put the letters \emph{Cyrillic letter short I with tail} and \emph{Cyrillic
letter EM with tail} to the slots where the same letters with descender
should be. They are somewhat similar and some languages should use exactly
these letters. I think that it is better than leaving the slots empty.

Please \href{mailto:pavel.farar@centrum.cz}{send me} bug reports and
suggestions about this package.

\end{document}
