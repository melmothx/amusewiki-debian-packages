\documentclass[12pt]{article}
\usepackage[T1]{fontenc}
\usepackage[latin1]{inputenc}
%\usepackage{geometry}
\usepackage{pst-all}
\usepackage{showexpl}
\usepackage{tabularx}
\SpecialCoor
%\usepackage[ps2pdf,colorlinks,linktocpage]{hyperref}
\usepackage[colorlinks,linktocpage]{hyperref}
\def\UrlFont{\small\ttfamily}
\makeatletter
\def\verbatim@font{\small\normalfont\ttfamily}
\makeatother
%\usepackage{color}
\definecolor{hellgelb}{rgb}{1,1,0.8}
\definecolor{colKeys}{rgb}{0,0,1}
\definecolor{colIdentifier}{rgb}{0,0,0}
\definecolor{colComments}{rgb}{1,0,0}
\definecolor{colString}{rgb}{0,0.5,0}
%
\usepackage{listings}
\lstset{%
    language=PSTricks,%
    float=hbp,%
    basicstyle=\ttfamily\small, %
    identifierstyle=\color{colIdentifier}, %
    keywordstyle=\color{colKeys}, %
    stringstyle=\color{colString}, %
    commentstyle=\color{colComments}, %
    columns=flexible, %
    tabsize=4, %
    frame=single, %
    extendedchars=true, %
    showspaces=false, %
    showstringspaces=false, %
    numbers=left, %
    numberstyle=\tiny, %
    breaklines=true, %
%    backgroundcolor=\color{hellgelb}, %
    breakautoindent=true, %
    captionpos=b,%
	xleftmargin=0pt%
}

%\parindent=0pt
\newcommand\verbI[1]{{\small\texttt{#1}}}
\newcommand\CMD[1]{{\texttt{\textbackslash#1}}}
%
%\psset{subgriddiv=0,griddots=5,gridlabels=7pt}
%
\DeclareRobustCommand\cs[1]{\texttt{\char`\\#1}}
\def\PS{PostScript}
%
\begin{document}
\title{\texttt{PSTricks -- version 1.11}\\new macros and bugfixes for \texttt{pstricks}}
\author{Herbert Vo�\thanks{%
\url{Herbert.Voss@perce.de}}}
\date{\today}

\maketitle

\begin{abstract}
This new version of \texttt{pstricks.tex} depends on the also new prologue file
\texttt{pstricks.pro} (v 1.00), which should go into the local \TeX-directoory \url{$TEXMFLOCAL/dvips/}.
\end{abstract}


\tableofcontents

%--------------------------------------------------------------------------------------
\section{New macro names}
%--------------------------------------------------------------------------------------
In general \texttt{PSTricks} uses macronames with a preceeding \verb+ps+ to prevent 
clashes with other packages. However, some macros have names without the \verb+ps+ and
these ones have now new names:

\begin{verbatim}
  \scalebox    -> \psscalebox
  \scaleboxto  -> \psscaleboxto
  \rotateleft  -> \psrotateleft
  \rotateright -> \psrotateright
  \rotatedown  -> \psrotatedown
\end{verbatim}

The first change is important, because there were a lot of problems in the past;
\verb+graphicx+ also defines a \verb+scalebox+ but with diffent syntax.

%--------------------------------------------------------------------------------------
\section{New fill options}
%--------------------------------------------------------------------------------------
For the fillstyles \verb+hlines+, \verb+vlines+ and \verb+crosshatch+ there are two new
options to get increasing line widths and/or increasing whitespace. Both options are 
lengths and can be set as usual for PSTricks, with or without a unit.

\bigskip\noindent
\begin{tabularx}{\linewidth}{lXc}
\emph{name} & \emph{meaning} & \emph{default}\\\hline
\verb|hatchsepinc| & additional increasing space between two hatch lines & 0\tabularnewline
\verb|hatchwidthinc| & value for the increasing line width of two hatch lines & 0
\end{tabularx}



\bigskip
\begin{LTXexample}[pos=t]
\begin{pspicture}(\linewidth,3)
  \psframe[fillstyle=vlines,hatchangle=0,hatchsep=.5pt,%
     hatchwidth=1pt,hatchwidthinc=0.25pt](\linewidth,3)
\end{pspicture}
\end{LTXexample}

\begin{LTXexample}[pos=t]
\begin{pspicture}(\linewidth,3)
  \psframe[fillstyle=hlines,hatchangle=0,%
     hatchwidth=1pt,hatchsep=0.5pt,hatchsepinc=0.1pt](\linewidth,3)
\end{pspicture}
\end{LTXexample}

\begin{LTXexample}[pos=t]
\begin{pspicture}(\linewidth,3)
  \psframe[fillstyle=vlines,hatchangle=0,hatchsep=0.6pt,%
     hatchwidth=1pt,hatchwidthinc=0.3pt,hatchangle=60,
     hatchcolor=red](\linewidth,3)
\end{pspicture}
\end{LTXexample}

\begin{LTXexample}[pos=t]
\begin{pspicture}(\linewidth,3)
  \psframe[fillstyle=hlines,hatchangle=0,hatchangle=-60,%
     hatchwidth=1pt,hatchsep=0.5pt,hatchsepinc=0.1pt,
     hatchcolor=blue](\linewidth,3)
\end{pspicture}
\end{LTXexample}

\begin{LTXexample}[pos=t]
\begin{pspicture}(\linewidth,4)
  \pscircle[fillstyle=vlines,hatchangle=0,hatchsep=0.6pt,%
     hatchwidth=1pt,hatchwidthinc=0.3pt,hatchangle=90,
     hatchcolor=red](2,2){2}
  \pscircle[fillstyle=vlines,hatchangle=0,hatchsep=0.6pt,%
     hatchwidth=1pt,hatchwidthinc=0.3pt,hatchangle=-45,
     hatchcolor=green](7,2){2}
  \pscircle[fillstyle=hlines,hatchangle=0,hatchsep=0.6pt,%
     hatchwidth=1pt,hatchwidthinc=0.3pt,hatchangle=45,
     hatchcolor=blue](12,2){2}
\end{pspicture}
\end{LTXexample}

\begin{LTXexample}[pos=t]
\begin{pspicture}(\linewidth,3)
  \psframe[fillstyle=crosshatch,hatchangle=0,hatchangle=-90,%
     hatchwidth=1pt,hatchsep=0.5pt,hatchsepinc=0.1pt,
     hatchcolor=blue](\linewidth,3)
\end{pspicture}
\end{LTXexample}

%--------------------------------------------------------------------------------------
\section{Other changes}
%--------------------------------------------------------------------------------------
\texttt{pstricks.tex} defined the PostScript subroutines for arcs of an ellipse.
This code now moved into the appropriate \texttt{pstricks.pro}, which holds the
pure PostScript code of \texttt{PSTricks}. This in not important for user until
the newest \texttt{pstricks.pro} \textbf{and} \texttt{pstricks.tex} are installed. 


\end{document}
