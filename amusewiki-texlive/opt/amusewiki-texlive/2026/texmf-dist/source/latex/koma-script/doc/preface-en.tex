% ======================================================================
% preface-en.tex
% Copyright (c) Markus Kohm, 2008-2022
%
% This file is part of the LaTeX2e KOMA-Script bundle.
%
% This work may be distributed and/or modified under the conditions of
% the LaTeX Project Public License, version 1.3c of the license.
% The latest version of this license is in
%   http://www.latex-project.org/lppl.txt
% and version 1.3c or later is part of all distributions of LaTeX
% version 2005/12/01 or later and of this work.
%
% This work has the LPPL maintenance status "author-maintained".
%
% The Current Maintainer and author of this work is Markus Kohm.
%
% This work consists of all files listed in MANIFEST.md.
% ======================================================================

\KOMAProvidesFile{preface-en.tex}
                 [$Date: 2025-11-25 12:50:16 +0100 (Di, 25. Nov 2025) $
                  preface to dedicated version]
\translator{Markus Kohm\and DeepL}

\addchap{Preface to \KOMAScript~3.49}

Recent \KOMAScript{} releases have been marked by changes due to external
developments. These included changes to the \LaTeX{} kernel as well as to
other packages. In addition, there were many internal changes, but also the
implementation of frequently requested features, even though I had already
announced in December~2019 with version~3.28 that there would be no more
completely new features.

For \KOMAScript~3.49, work on changes as far-reaching as those triggered by
Axel Sommerfeldt's discontinuation of maintenance of the caption package in
version 3.46 was already underway when I was once again distracted by other
external circumstances. This time, the decisive factors were support for
\Macro{DocumentMetadata} and PDF tagging. Suffice it to say that using
\Macro{DocumentMetadata} replaces many internal \LaTeX{} functions. This
includes a considerable number of macros from the interface for class and
package authors, which I was forced to replace with extended versions in
\KOMAScript{} long ago. When tagging is enabled, further changes relevant to
\KOMAScript{} occur in the \LaTeX{} kernel. In order to support all of this, a
lot of code would also have to be added to \KOMAScript{} depending on the use
of \Macro{DocumentMetadata} and any features enabled with it. To make matters
more difficult, parts of the new code in \LaTeX{} itself are located in the
\texttt{latex-lab} area and are even prefixed with
\texttt{latex-lab-testphase}. Changes are to be expected here at any time, and
it will be necessary to react quickly to them.

Although Marei Peischl has thankfully been working on providing the necessary
support for some time now, part of her work would have been undone by my
aforementioned work on \KOMAScript~3.49. Even in a larger team, coordinating
such different code branches is time-consuming. In a small team, it means
constant extra work, which repeatedly brings any planning to a standstill.

I have therefore decided on the following:
\begin{itemize}
\item Most of my modifications for KOMA-Script 3.49 were removed from the
  code.
\item Until further notice, there will be no new features.
\item Internal improvements to the code will only be made if they facilitate
  the work of supporting \Macro{DocumentMetadata} or PDF tagging.
\item Requested features will not be implemented for the time being.
\item Until further notice, KOMA-Script will no longer include workarounds for
  problems with third-party packages. Problems with third-party packages
  should therefore be reported to the maintainers of the third-party packages.
\item Even bugs will only be fixed in exceptional cases. Anything that hasn't
  been noticed so far shouldn't be particularly serious.
\item Adjustments to new \LaTeX{} releases will also only be made if they are
  absolutely necessary.
\item The integration of the code developed independently by Marei to date
  must be planned and tackled at the appropriate time.
\item The consequences for the management, maintainance and further
  development of \KOMAScript{} must be considered and planned.
\end{itemize}
So I have essentially closed some tracks so that trains can continue to run on
the remaining ones without constantly changing the points.

To be clear: I cannot say today whether there will ever be a version of
\KOMAScript{} that will work both with and without the use of
\Macro{DocumentMetadata}. In the medium term, it is to be expected that
\KOMAScript{} will require the current \LaTeX{} release in the future and, for
the sake of code simplification, both the option
\DescRef{maincls.option.version} will disappear and the use of
\Macro{DocumentMetadata} will become mandatory.

\bigskip\noindent
Markus Kohm, Neckarhausen in November 2025.

\endinput

%%% Local Variables: 
%%% mode: latex
%%% TeX-master: "scrguide-en.tex"
%%% coding: utf-8
%%% ispell-local-dictionary: "en_GB"
%%% eval: (flyspell-mode 1)
%%% End: 
