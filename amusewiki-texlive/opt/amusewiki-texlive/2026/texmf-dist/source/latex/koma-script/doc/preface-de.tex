% ======================================================================
% preface-de.tex
% Copyright (c) Markus Kohm, 2008-2025
%
% This file is part of the LaTeX2e KOMA-Script bundle.
%
% This work may be distributed and/or modified under the conditions of
% the LaTeX Project Public License, version 1.3c of the license.
% The latest version of this license is in
%   http://www.latex-project.org/lppl.txt
% and version 1.3c or later is part of all distributions of LaTeX
% version 2005/12/01 or later and of this work.
%
% This work has the LPPL maintenance status "author-maintained".
%
% The Current Maintainer and author of this work is Markus Kohm.
%
% This work consists of all files listed in MANIFEST.md.
% ======================================================================

\KOMAProvidesFile{preface-de.tex}
                 [$Date: 2025-11-25 12:50:16 +0100 (Di, 25. Nov 2025) $
                  preface to a dedicated version]

\addchap{Vorwort zu \KOMAScript~3.49}

Die \KOMAScript-Releases der letzten Zeit waren von Umbauten aufgrund äußerer
Veränderungen gekennzeichnet. Dazu gehörten neben Änderungen am \LaTeX-Kern
auch solche bei anderen Paketen. Daneben gab es viele interne Umbauten aber
auch immer wieder die Implementierung nachgefragter Features, obwohl ich schon
bei Version~3.28 im Dezember~2019 verkündet hatte, dass es keine komplett
neuen Features mehr geben werde.

Für \KOMAScript~3.49 waren die Arbeiten für ähnlich tiefgreifende Änderungen
wie die durch Axel Sommerfeldts Einstellung der Pflege des
\Package{caption}-Pakets in Version~3.46 ausgelösten Umbauten bereits im
Gange, als ich wieder einmal durch andere äußere Umstände davon abgelenkt
wurde. Dieses Mal ging es maßgeblich um die Unterstützung von
\Macro{DocumentMetadata} und PDF-Tagging. Dazu sei nur soviel erklärt, dass
bei Verwendung von \Macro{DocumentMetadata} viele Interna von \LaTeX{} ersetzt
werden. Darunter ist auch eine beträchtliche Anzahl an Makros aus der
Schnittstelle für Klassen- und Paketautoren, die ich gezwungenermaßen in
\KOMAScript{} längst durch erweiterte Versionen ersetzt habe. Bei aktiviertem
Tagging ergeben sich im \LaTeX-Kern weitere für \KOMAScript{} relevante
Änderungen. Um all das zu unterstützen, müsste in \KOMAScript{} ebenfalls
reichlich Code in Abhängigkeit von der Verwendung von \Macro{DocumentMetadata}
und gegebenenfalls damit aktivierten Features hinzugefügt werden. Erschwerend
kommt hinzu dass bei \LaTeX{} selbst Teile des neuen Codes im Bereich
\texttt{latex-lab} untergebracht und dort sogar mit dem Präfix
\texttt{latex-lab-testphase} versehen sind. Hier ist jederzeit mit Änderungen
zu rechnen, auf die dann auch schnell reagiert werden muss.

Zwar arbeitet Marei Peischl dankenswerter Weise schon seit einiger Zeit an
entsprechender Unterstützung, allerdings wäre ein Teil ihrer Arbeiten durch
meine bereits erwähnen Arbeiten an \KOMAScript~3.49 wieder zunichte gemacht
worden. Selbst in einem größeren Team ist die Koordination solcher
unterschiedlicher Code-Zweige aufwändig. In einem winzigen Team bedeutet es
ständige Mehrarbeit, durch die jegliche Planung immer wieder ins Stocken
gerät.

Ich habe mich deshalb zu folgendem entschlossen:
\begin{itemize}
\item Die meisten meiner Umbauten für \KOMAScript~3.49 wurden wieder aus dem
  Code geworfen.
\item Bis auf Weiteres wird es wirklich keine neuen Features mehr geben.
\item Interne Verbesserungen des Codes finden nur dann statt, wenn sie die
  Arbeit für die Unterstützung von \Macro{DocumentMetadata} oder PDF-Tagging
  erleichtern.
\item Nachgefragte Features werden vorerst nicht implementiert.
\item In \KOMAScript{} werden bis auf Weiteres keine Workarounds für Probleme
  mit Drittpaketen mehr eingebaut. Probleme mit Drittpaketen sollten daher den
  Entwicklern der Drittpakete gemeldet werden.
\item Selbst Bugs werden nur in Ausnahmefällen beseitigt werden. Was bisher
  nicht aufgefallen ist, sollte auch nicht sonderlich schwerwiegend sein.
\item Anpassungen an neue \LaTeX-Versionen werden ebenfalls nur dann
  stattfinden, wenn diese zwingend erforderlich sind.
\item Die Integration des von Marei bisher unabhängig entwickelten Codes ist
  zu planen und zu gegebener Zeit in Angriff zu nehmen.
\item Folgen für die Verwaltung und Weiterentwicklung von \KOMAScript{}
  sind zu bedenken und zu planen.
\end{itemize}
Ich habe also quasi einige Gleise gesperrt, damit auf den verbliebenen ohne
ständige Änderung der Weichenstellung weiter gefahren werden kann.

Um es deutlich zu sagen: Ob es je eine \KOMAScript-Version geben wird, die
sowohl mit als auch ohne Verwendung von \Macro{DocumentMetadata} funktionieren
wird, kann ich heute nicht sagen. Mittelfristig ist zu erwarten, dass
\KOMAScript{} künftig die jeweils aktuelle \LaTeX-Version voraussetzen und
zwecks Code-Vereinfachung sowohl Option \DescRef{maincls.option.version}
verschwinden als auch die Verwendung von \Macro{DocumentMetadata}
obligatorisch werden wird.

\bigskip\noindent
Markus Kohm, Neckarhausen im November 2025
\endinput

%%% Local Variables: 
%%% mode: latex
%%% TeX-master: "scrguide-de.tex"
%%% coding: utf-8
%%% ispell-local-dictionary: "de_DE"
%%% eval: (flyspell-mode 1)
%%% End: 

% LocalWords:  Dokumentationsklasse Teststruktur Benutzeranleitung Releases
% LocalWords:  Pseudolängen Buchausgaben Sommerfeldts Paketautoren Marei
% LocalWords:  Peischl Drittpaketen Drittpakete Workarounds Tagging
