% \CheckSum{1877}
% \iffalse meta-comment
% ======================================================================
% scrhack.dtx
% Copyright (c) Markus Kohm, 2008-2020
%
% This file is part of the LaTeX2e KOMA-Script bundle.
%
% This work may be distributed and/or modified under the conditions of
% the LaTeX Project Public License, version 1.3c of the license.
% The latest version of this license is in
%   http://www.latex-project.org/lppl.txt
% and version 1.<3c or later is part of all distributions of LaTeX
% version 2005/12/01 or later and of this work.
%
% This work has the LPPL maintenance status "author-maintained".
%
% The Current Maintainer and author of this work is Markus Kohm.
%
% This work consists of all files listed in manifest.txt.
%
% To create `scrhack.sty' run `tex scrhack.dtx'.  Using LaTeX instead
% of TeX would generate the implementation documentation.
% ----------------------------------------------------------------------
% scrhack.dtx
% Copyright (c) Markus Kohm, 2008-2020
%
% Dieses Werk darf nach den Bedingungen der LaTeX Project Public Lizenz,
% Version 1.3c, verteilt und/oder veraendert werden.
% Die neuste Version dieser Lizenz ist
%   http://www.latex-project.org/lppl.txt
% und Version 1.3c ist Teil aller Verteilungen von LaTeX
% Version 2005/12/01 oder spaeter und dieses Werks.
%
% Dieses Werk hat den LPPL-Verwaltungs-Status "author-maintained"
% (allein durch den Autor verwaltet).
%
% Der Aktuelle Verwalter und Autor dieses Werkes ist Markus Kohm.
%
% Dieses Werk besteht aus den in manifest.txt aufgefuehrten Dateien.
%
% `scrhack.sty' kann durch den Aufruf `tex scrhack.dtx' erzeugt 
% werden. Bei Verwendung von LaTeX statt TeX wird hingegen die
% Implementierungsdokumentation erzeugt.
% ======================================================================
% \fi
%
% \CharacterTable
%  {Upper-case    \A\B\C\D\E\F\G\H\I\J\K\L\M\N\O\P\Q\R\S\T\U\V\W\X\Y\Z
%   Lower-case    \a\b\c\d\e\f\g\h\i\j\k\l\m\n\o\p\q\r\s\t\u\v\w\x\y\z
%   Digits        \0\1\2\3\4\5\6\7\8\9
%   Exclamation   \!     Double quote  \"     Hash (number) \#
%   Dollar        \$     Percent       \%     Ampersand     \&
%   Acute accent  \'     Left paren    \(     Right paren   \)
%   Asterisk      \*     Plus          \+     Comma         \,
%   Minus         \-     Point         \.     Solidus       \/
%   Colon         \:     Semicolon     \;     Less than     \<
%   Equals        \=     Greater than  \>     Question mark \?
%   Commercial at \@     Left bracket  \[     Backslash     \\
%   Right bracket \]     Circumflex    \^     Underscore    \_
%   Grave accent  \`     Left brace    \{     Vertical bar  \|
%   Right brace   \}     Tilde         \~}
%
% \iffalse
%%% From File: $Id: scrhack.dtx 3474 2020-04-08 09:34:56Z kohm $
%<package&identify>%%% using: package,identify
%<package&option>%%% using: package,option
%<package&body>%%% using: package,body
%<package&identity>\NeedsTeXFormat{LaTeX2e}[1995/06/01]
%<*dtx>
\ifx\ProvidesFile\undefined\def\ProvidesFile#1[#2]{}\fi
\begingroup
  \def\filedate$#1: #2-#3-#4 #5${\gdef\filedate{#2/#3/#4}}
  \filedate$Date: 2020-04-08 11:34:56 +0200 (Wed, 08 Apr 2020) $
  \def\filerevision$#1: #2 ${\gdef\filerevision{r#2}}
  \filerevision$Revision: 1962 $
\endgroup
\ProvidesFile{scrhack.dtx}[\filedate\space\filerevision\space
%</dtx>
%<package&identify>\ProvidesPackage{scrhack}[%
%<hyperref&identify>\ProvidesFile{hyperref.hak}[%
%<float&identify>\ProvidesFile{float.hak}[%
%<floatrow&identify>\ProvidesFile{floatrow.hak}[%
%<listings&identify>\ProvidesFile{listings.hak}[%
%<setspace&identify>\ProvidesFile{setspace.hak}[%
%<lscape&identify>\ProvidesFile{lscape.hak}[%
%<nomencl&identify>\ProvidesFile{nomencl.hak}[%
%<standardsections&identify>\ProvidesFile{standardsections.hak}[%
%<*dtx|identify>
%!KOMAScriptVersion
%<*dtx>
  source of
%</dtx>
%<*dtx|package>
  package (hacking other packages)%
%</dtx|package>
%<*hack>
%<hack>  scrhack part
%<!standardsections>  (hacking package
%<hyperref&identify>    hyperref)%
%<float&identify>    float)%
%<floatrow&identify>    floatrow)%
%<listings&identify>    listings)%
%<setspace&identify>    setspace)%
%<lscape&identify>    lscape)%
%<nomencl&identify>    nomencl)%
%<standardsections&identify>  (defining standard LaTeX class section headings)%
%</hack>
]
%</dtx|identify>
%<*dtx>
\ifx\documentclass\undefined
  \input scrdocstrip.tex
  \@@input scrkernel-version.dtx
  \@@input scrstrip.inc
  \KOMAdefVariable{COPYRIGHTFROM}{2008}
  \generate{\usepreamble\defaultpreamble
    \file{scrhack.sty}{%
      \from{scrkernel-version.dtx}{package,scrhack}%
      \from{scrhack.dtx}{package,identify}%
      \from{scrkernel-basics.dtx}{load}%
      \from{scrhack.dtx}{package,option}%
      \from{scrhack.dtx}{package,body}%
      \from{scrlogo.dtx}{logo}%
    }%
    \file{hyperref.hak}{%
      \from{scrkernel-version.dtx}{file,hyperref.hak}%
      \from{scrhack.dtx}{hack,hyperref,identify}%
      \from{scrhack.dtx}{hack,hyperref,body}%
    }%
    \file{float.hak}{%
      \from{scrkernel-version.dtx}{file,float.hak}%
      \from{scrhack.dtx}{hack,float,identify}%
      \from{scrhack.dtx}{hack,float,body}%
    }%
    \file{floatrow.hak}{%
      \from{scrkernel-version.dtx}{file,floatrow.hak}%
      \from{scrhack.dtx}{hack,floatrow,identify}%
      \from{scrhack.dtx}{hack,floatrow,body}%
    }%
    \file{listings.hak}{%
      \from{scrkernel-version.dtx}{file,listings.hak}%
      \from{scrhack.dtx}{hack,listings,identify}%
      \from{scrhack.dtx}{hack,listings,body}%
    }%
    \file{setspace.hak}{%
      \from{scrkernel-version.dtx}{file,setspace.hak}%
      \from{scrhack.dtx}{hack,setspace,identify}%
      \from{scrhack.dtx}{hack,setspace,body}%
    }%
    \file{lscape.hak}{%
      \from{scrkernel-version.dtx}{file,lscape.hak}%
      \from{scrhack.dtx}{hack,lscape,identify}%
      \from{scrhack.dtx}{hack,lscape,body}%
    }%
    \file{nomencl.hak}{%
      \from{scrkernel-version.dtx}{file,nomencl.hak}%
      \from{scrhack.dtx}{hack,nomencl,identify}%
      \from{scrhack.dtx}{hack,nomencl,body}%
    }%
    \file{standardsections.hak}{%
      \from{scrkernel-version.dtx}{file,standardsections.hak}%
      \from{scrhack.dtx}{hack,standardsections,identify}%
      \from{scrhack.dtx}{hack,standardsections,body}%
    }%
  }
  \@@input scrstrop.inc
\else
  \let\endbatchfile\relax
\fi
\endbatchfile
\documentclass[parskip=half-]{scrdoc}
\usepackage[english,ngerman]{babel}
\usepackage[T1]{fontenc}
\usepackage{lmodern}
\newcommand*{\Environment}[1]{\texttt{\mbox{#1}}}
\CodelineIndex
\RecordChanges
\begin{document}
\GetFileInfo{scrhack.dtx}
\DocInput{scrhack.dtx}
\end{document}
%</dtx>
%\fi
%
% \selectlanguage{english}
%
% \let\PName\meta
% \let\Parameter\marg
% \providecommand*{\PParameter}[1]{\texttt{\{#1\}}}
% \providecommand*{\Counter}[1]{\texttt{\mbox{#1}}}
% \providecommand*{\Option}[1]{\texttt{#1}}
% \providecommand*{\OptionValue}[2]{\texttt{\mbox{#1=}\linebreak[3]\mbox{#2}}}
% \let\PValue\texttt
%
% \title{\KOMAScript{} \partname\ \texttt{\filename}%
%   \thanks{This file is version \fileversion\ of \texttt{\filename}.}}
% \date{\filedate}
% \author{Markus Kohm\thanks{mailto:komascript(at)gmx.info}}
% \maketitle
% \begin{abstract}
%   Some packages from other authors may have problems with \KOMAScript{}.  In
%   my opinion some packages could be improved. With some packages this makes
%   only sense, if \KOMAScript{} was used. With some other packages the
%   package author has another opinion. Sometimes proposals was never
%   answered. Package \Package{scrhack} contains all those improvement
%   proposals for other packages. This means, \Package{scrhack} redefines
%   macros of packages from other authors! The redefinitions are only
%   activated, if those packages were loaded. Users may prevent
%   \Package{scrhack} from redefining macros of individual packages.
% \end{abstract}
% \tableofcontents
%
% \selectlanguage{ngerman}
% \StopEventually{\PrintIndex\PrintChanges}
% \changes{v3.03}{2009/03/12}{erste Version des Pakets}%^^A
%
% \section{Implementation of \Package{scrhack}}
%
% \changes{v3.04b}{2009/08/05}{Die Reihenfolge von Anweisungen und Optionen
%   grundlegend geändert, um das Paket \textsf{scrhack} unabhängiger von der
%   Reihenfolge beim Laden von Paketen zu machen.}%^^A
%
% \subsection{Optionen}
%
% \iffalse
%<*package&option>
% \fi
%
% Das Paket bedient sich \cs{KOMAoptions} etc. aus \textsf{scrkbase} (dieses
% wird übrigens direkt per \texttt{scrkbase.dtx} geladen).
%
% Per Option kann gewählt werden, welche Manipulationen geladen werden
% sollen. Alle diese Optionen können jedoch nur bis zum Laden des
% entsprechenden Pakets oder dem Laden von \textsf{scrhack} gesetzt
% werden (es zählt, was später kommt). Anschließend sind sie wirkungslos.
%
% \subsection{Verwendete Anweisungen}
%
% \iffalse
%</package&option>
%<*package&body>
% \fi
%
% \begin{macro}{\scr@ifexpected}
% Wenn die im ersten Argument angegebene Anweisung nach Ausführung der im
% zweiten Argument angegebenen Anweisungen unverändert ist, dann soll das
% dritte Argument ausgeführt werden, sonst das vierte.
%    \begin{macrocode}
\newcommand{\scr@ifexpected}[2]{%
  \begingroup
    \let\@tempa#1
    #2
    \ifx\@tempa#1
      \aftergroup\@firstoftwo
    \else
      \aftergroup\@secondoftwo
    \fi
  \endgroup
}
%    \end{macrocode}
% \end{macro}
%
% \begin{macro}{\scr@hack@load}
% Wenn die Datei mit dem Namen des zweiten Arguments und der Endung des ersten
% Arguments so geladen wurde, dass \LaTeX{} eine Versionsinfo dazu gespeichert
% hat, dann soll zusätzlich der entsprechende Hack geladen werden.
%    \begin{macrocode}
\newcommand*{\scr@hack@load}[2]{%
  \expandafter\ifx\csname ver@#2.#1\endcsname\relax
    \expandafter\@secondoftwo
  \else
%    \end{macrocode}
% Allerdings wird jeder Hack nur genau einmal geladen:
%    \begin{macrocode}
    \expandafter\ifx\csname ver@#2.hak\endcsname\relax
      \expandafter\expandafter\expandafter\@firstoftwo
    \else
      \expandafter\expandafter\expandafter\@secondoftwo
    \fi
  \fi
  {%
    \PackageInfo{scrhack}{loading #2 hack}%
    \edef\reserved@a{%
      \noexpand\makeatletter\noexpand\input{#2.hak}%
      \noexpand\catcode`\noexpand\@\the\catcode`\@\relax
    }\reserved@a
  }{%
    \PackageInfo{scrhack}{ignorring #2 hack}%
  }%
}
%    \end{macrocode}
% \end{macro}
%
% \iffalse
%</package&body>
% \fi
%
% \subsection{Der \textsf{hyperref}-Hack}
%
% \textsf{hyperref} setzt den Anker zu der Stern-Variante einer Überschrift
% hinter die Überschrift, während es bei der nicht Stern-Variante den Anker
% auch dann vor die Überschrift setzt, wenn die Überschrift aufgrund von
% \texttt{secnumdepth} nicht nummeriert wird. Der Hack setzt den Anker
% einheitlich vor die Überschrift.
%
% \begin{option}{hyperref}
% \changes{v3.12}{2013/03/05}{Signalisierung mit
%     \cs{FamilyKeyStateProcessed}}%^^A
% \changes{v3.17}{2015/03/09}{Defaulteinstellung mit
%     \cs{KOMAExecuteOptions}}%^^A
%    \begin{macrocode}
%<*package&option>
\KOMA@ifkey{hyperref}{@scrhack@hyperref}%
\KOMAExecuteOptions{hyperref=true}%
%</package&option>
%<*package&body>
%    \end{macrocode}
% \changes{v3.04b}{2009/11/09}{\textsf{hyperref}-Hack wird früher geladen}%^^A
% \changes{v3.17}{2015/03/09}{Neuere \textsf{hyperref}-Version deaktiviert
%   ggf. Option \texttt{hyperref}}%^^A
% \changes{v3.18}{2015/05/22}{Neuere \KOMAScript-Versionen deaktivieren
%   ggf. Option \texttt{hyperref}}%^^A
% Hier muss ein wenig trickreicher gearbeitet werden, weil \textsf{hyperref}
% die Treiberdatei per \cs{AtEndOfPackage} lädt und der Hack erst danach
% installiert werden darf. Mit \cs{AfterPackage*} alleine, würde der Hack aber
% vor dem Laden der Treiberdatei installiert. Dafür können wir aber sicher
% sein, dass ein innerhalb von \cs{AfterPackage*} aufgerufenes
% \cs{AtEndOfPackage} garantiert nach dem Laden der Treiberdatei ausgeführt
% wird. Das funktioniert auch noch, wenn \textsf{hyperref} bereits geladen
% wurde. In dem Fall wird der Code einfach nach dem Ende von \textsf{scrhack}
% statt nach dem Ende von \textsf{hyperref} ausgeführt.
%    \begin{macrocode}
\BeforePackage{hyperref}{%
  \scr@ifundefinedorrelax{hy@insteadofrefstepcounter}{}{%
    \PackageInfo{scrhack}{hyperref hack deactivated because of\MessageBreak
      detection of KOMA-Script class, that doesn't\MessageBreak
      need that hack,}%
    \KOMAExecuteOptions[.scrhack.sty]{hyperref=false}%
  }%
}
\AfterPackage*{hyperref}{%
  \if@scrhack@hyperref
    \@ifpackagelater{hyperref}{2009/11/24}{%
      \PackageInfo{scrhack}{hyperref hack deactivated because of\MessageBreak
        detection of hyperref version, that doesn't\MessageBreak
        need that hack,}%
      \KOMAExecuteOptions[.scrhack.sty]{hyperref=false}%
    }{%
      \AtEndOfPackage{%
        \KOMA@key[.scrhack.sty]{hyperref}{%
          \PackageWarning{scrhack}{option `hyperref=#1' ignored}%
          \FamilyKeyStateProcessed
        }%
        \if@scrhack@hyperref\scr@hack@load\@pkgextension{hyperref}\fi
      }%
    }%
  \fi
}
%</package&body>
%    \end{macrocode}
% \end{option}
%
%
% \begin{macro}{\@schapter}
% \begin{macro}{\@spart}
% \begin{macro}{\@ssect}
% Eigentlich wird hier gar nicht \texttt{hyperref.sty} verändert, sondern
% diverse Treiberdateien. Sobald das Paket \textsf{hyperref} geladen ist, ist
% auch die passende Treiberdatei geladen und außerdem sind alle
% Treiberdateien, die entsprechende Definitionen vornehmen, gleichermaßen
% betroffen. Also kann der entsprechende Patch einfach erfolgen, wenn hyperref
% geladen ist (was bereits von \cs{scr@hack@load} getestet wurde). Es muss
% also nur noch sichergestellt werden, dass die umzudefinierenden Macros
% derzeit den erwarteten Inhalt haben.
%    \begin{macrocode}
%<*hyperref&body>
\scr@ifexpected\@schapter{%
  \def\@schapter#1{%
    \H@old@schapter{#1}%
    \begingroup
      \let\@mkboth\@gobbletwo
      \Hy@GlobalStepCount\Hy@linkcounter
      \xdef\@currentHref{\Hy@chapapp*.\the\Hy@linkcounter}%
      \Hy@raisedlink{%
        \hyper@anchorstart{\@currentHref}\hyper@anchorend
      }%
    \endgroup
  }%
}{%
  \PackageInfo{scrhack}{redefining \string\@schapter}%
  \def\@schapter#1{%
    \begingroup
      \let\@mkboth\@gobbletwo
      \Hy@GlobalStepCount\Hy@linkcounter
      \xdef\@currentHref{\Hy@chapapp*.\the\Hy@linkcounter}%
      \Hy@raisedlink{%
        \hyper@anchorstart{\@currentHref}\hyper@anchorend
      }%
    \endgroup
    \H@old@schapter{#1}%
  }%
}{%
  \scr@ifexpected\@schapter{%
    \def\@schapter#1{%
      \begingroup
        \let\@mkboth\@gobbletwo
        \Hy@GlobalStepCount\Hy@linkcounter
        \xdef\@currentHref{\Hy@chapapp*.\the\Hy@linkcounter}%
        \Hy@raisedlink{%
          \hyper@anchorstart{\@currentHref}\hyper@anchorend
        }%
      \endgroup
      \H@old@schapter{#1}%
    }%
  }{}{%
    \PackageWarningNoLine{scrhack}{unknown \string\@schapter\space
      definition found!\MessageBreak
      Maybe you are using a unsupported hyperref version}%
  }%
}
\scr@ifexpected\@spart{%
  \def\@spart#1{%
    \H@old@spart{#1}%
    \Hy@GlobalStepCount\Hy@linkcounter
    \xdef\@currentHref{part*.\the\Hy@linkcounter}%
    \Hy@raisedlink{%
      \hyper@anchorstart{\@currentHref}\hyper@anchorend
    }%
  }%
}{%
  \PackageInfo{scrhack}{redefining \string\@spart}%
  \def\@spart#1{%
    \Hy@GlobalStepCount\Hy@linkcounter
    \xdef\@currentHref{part*.\the\Hy@linkcounter}%
    \Hy@raisedlink{%
      \hyper@anchorstart{\@currentHref}\hyper@anchorend
    }%
    \H@old@spart{#1}%
  }%
}{%
  \scr@ifexpected\@spart{%
    \def\@spart#1{%
      \Hy@GlobalStepCount\Hy@linkcounter
      \xdef\@currentHref{part*.\the\Hy@linkcounter}%
      \Hy@raisedlink{%
        \hyper@anchorstart{\@currentHref}\hyper@anchorend
      }%
      \H@old@spart{#1}%
    }%
  }{}{%
    \PackageWarningNoLine{scrhack}{unknown \string\@spart\space
      definition found!\MessageBreak
      Maybe you are using a unsupported hyperref version}%
  }%
}
\scr@ifexpected\@ssect{%
  \def\@ssect#1#2#3#4#5{%
    \H@old@ssect{#1}{#2}{#3}{#4}{#5}%
    \phantomsection
  }%
}{%
  \PackageInfo{scrhack}{redefining \string\@ssect}%
  \def\@ssect#1#2#3#4#5{%
    \H@old@ssect{#1}{#2}{#3}{#4}{\phantomsection\ignorespaces#5}%
  }%
}{%
  \scr@ifexpected\@ssect{%
    \def\@ssect#1#2#3#4#5{%
      \H@old@ssect{#1}{#2}{#3}{#4}{\phantomsection\ignorespaces#5}%
    }%
  }{}{%
    \PackageWarningNoLine{scrhack}{unknown \string\@ssect\space
      definition found!\MessageBreak
      Maybe you are using a unsupported hyperref version}%
  }%
}
%</hyperref&body>
%    \end{macrocode}
% \end{macro}
% \end{macro}
% \end{macro}
%
%
% \subsection{Der \textsf{float}-Hack}
%
% Das \textsf{float}-Paket verwendet das Makro \cs{float@listhead} zum
% Setzen der Überschriften. Dies wird seit \KOMAScript~3 nicht mehr empfohlen
% und fliegt demnächst komplett aus der Unterstützung. Stattdessen wird
% empfohlen, dass Pakete \textsf{tocbasic} unterstützen. Der Aufwand dafür ist
% sehr gering und wird mit vielen neuen Möglichkeiten belohnt.
%
% Dieser Hack rüstet die \textsf{tocbasic}-Unterstützung für \textsf{float}
% nach.
%
% \begin{option}{float}
% \changes{v3.12}{2013/03/05}{Signalisierung mit
%     \cs{FamilyKeyStateProcessed}}%^^A
% \changes{v3.17}{2015/03/09}{Defaulteinstellung mit
%     \cs{KOMAExecuteOptions}}%^^A
%    \begin{macrocode}
%<*package&option>
\KOMA@ifkey{float}{@scrhack@float}%
\KOMAExecuteOptions{float=true}%
%</package&option>
%<*package&body>
\AfterPackage*{float}{%
  \KOMA@key[.scrhack.sty]{float}{%
    \PackageWarning{scrhack}{option `float' ignored}%
    \FamilyKeyStateProcessed
  }%
  \if@scrhack@float\scr@hack@load\@pkgextension{float}\fi
}
%</package&body>
%    \end{macrocode}
% \end{option}
%
%
% \begin{macro}{\newfloat}
% \changes{v3.28}{2019/11/19}{\cs{ifattoclist} replaced by
%   \cs{Ifattoclist}}%^^A
% Über die Anweisung \cs{newfloat} wird eine neue Gleitumgebung
% definiert. Hier muss die neue Erweiterung aus dem dritten Argument
% \textsf{tocbasic} bekannt gemacht werden.
% \begin{macro}{\listof}
% Über die Anweisung \cs{listof} wird ein Verzeichnis für Gleitumgebungen
% ausgegeben. Hier muss schlicht die entsprechende Anweisung von
% \textsf{tocbasic} verwendet werden.
% \begin{macro}{\float@addtolists}
% Diese Anweisung wird nicht länger benötigt und daher auf die ursprüngliche
% Definition zurückgesetzt.
%    \begin{macrocode}
%<*float&body>
\scr@ifexpected{\newfloat}{%
  \long\def\newfloat#1#2#3{\@namedef{ext@#1}{#3}
    \let\float@do=\relax
    \xdef\@tempa{\noexpand\float@exts{\the\float@exts \float@do{#3}}}%
    \@tempa
    \floatplacement{#1}{#2}%
    \@ifundefined{fname@#1}{\floatname{#1}{#1}}{}
    \expandafter\edef\csname ftype@#1\endcsname{\value{float@type}}%
    \addtocounter{float@type}{\value{float@type}}
    \restylefloat{#1}%
    \expandafter\edef\csname fnum@#1\endcsname%
    {\expandafter\noexpand\csname fname@#1\endcsname{}
      \expandafter\noexpand\csname the#1\endcsname}
    \@ifnextchar[%]
    {\float@newx{#1}}%
    {\@ifundefined{c@#1}{\newcounter{#1}\@namedef{the#1}{\arabic{#1}}}%
      {}}}%
}{%
  \scr@ifexpected{\listof}{%
    \def\listof#1#2{%  
      \@ifundefined{ext@#1}{\float@error{#1}}{%
        \@namedef{l@#1}{\@dottedtocline{1}{1.5em}{2.3em}}%
        \float@listhead{#2}%
        \begingroup\setlength{\parskip}{\z@}%
        \@starttoc{\@nameuse{ext@#1}}%
        \endgroup}}%
  }{%
    \RequirePackage{tocbasic}%
    \PackageInfo{scrhack}{redefining \string\newfloat}%
    \renewcommand\newfloat[3]{%
      \Ifattoclist{#3}{%
        \PackageError{scrhack}{extension `#3' already in use}{%
          Each extension may be used only once.\MessageBreak
          You, the class, or another package already uses extension
          `#3'.\MessageBreak
          \string\newfloat\space command will be ignored!}%
      }{%
        \addtotoclist[float]{#3}%
        \setuptoc{#3}{chapteratlist}%
        \@namedef{ext@#1}{#3}%
        \let\float@do=\relax
        \xdef\@tempa{\noexpand\float@exts{\the\float@exts \float@do{#3}}}%
        \@tempa
        \floatplacement{#1}{#2}%
        \@ifundefined{fname@#1}{\floatname{#1}{#1}}{}%
        \expandafter\edef\csname ftype@#1\endcsname{\value{float@type}}%
        \addtocounter{float@type}{\value{float@type}}
        \restylefloat{#1}%
        \expandafter\edef\csname fnum@#1\endcsname%
        {\expandafter\noexpand\csname fname@#1\endcsname{}
          \expandafter\noexpand\csname the#1\endcsname}%
        \@ifnextchar[%]
        {\float@newx{#1}}%
        {\@ifundefined{c@#1}{\newcounter{#1}\@namedef{the#1}{\arabic{#1}}}%
          {}}}%
    }%
    \PackageInfo{scrhack}{redefining \string\listof}%
    \renewcommand*\listof[2]{%
      \@ifundefined{ext@#1}{\float@error{#1}}{%
        \@ifundefined{l@#1}{\expandafter\let\csname l@#1\endcsname\l@figure
          \@ifundefined{l@#1}{%
%    \end{macrocode}
% \changes{v3.25}{2018/03/16}{Verwendung von \cs{DeclareTOCStyleEntry}}%^^A
% Bis Version 3.24 wurde hier nun wie üblich |\l@#1| unter Verwendung von
% \cs{@dottedtocline} definiert:
% \begin{verbatim}
%   \@namedef{l@#1}{\@dottedtocline{1}{1.5em}{2.3em}}%
% \end{verbatim}
% Da aber \Package{tocbasic} inzwischen diese schöne Schnittstelle für die
% Verzeichniseinträge bietet, verwenden wir die hier nun auch:
%    \begin{macrocode}
            \DeclareTOCStyleEntry[level=1,numwidth=2.3em,indent=1.5em]
            {default}{#1}%
          }{}%
        }{}%
        \listoftoc[{#2}]{\csname ext@#1\endcsname}%
      }%
    }%
    \scr@ifexpected{\float@addtolists}{%
      \long\def\float@addtolists#1{%
        \def\float@do##1{\addtocontents{##1}{#1}} \the\float@exts}%
    }{%
      \PackageInfo{scrhack}{undefining \string\float@addtolists}%
      \let\float@addtolists\relax
    }{%
      \PackageWarningNoLine{scrhack}{unkown \string\float@addtolists\space
        definition found!\MessageBreak
        Maybe you are using a unsupported float version}%
    }%
  }{%
    \PackageWarningNoLine{scrhack}{unknown \string\listof\space
      definition found!\MessageBreak
      Maybe you are using a unsupported float version}%
  }%
}{%
  \PackageWarningNoLine{scrhack}{unknown \string\newfloat\space
    definition found!\MessageBreak
    Maybe you are using a unsupported float version}%
}
%</float&body>
%    \end{macrocode}
% \end{macro}
% \end{macro}
% \end{macro}
% 
%
% \subsection{Der \textsf{floatrow}-Hack}
%
% Das \textsf{floatrow}-Paket verwendet das Makro \cs{float@listhead} zum
% Setzen der Überschriften. Dies wird seit \KOMAScript~3 nicht mehr empfohlen
% und fliegt demnächst komplett aus der Unterstützung. Stattdessen wird
% empfohlen, dass Pakete \textsf{tocbasic} unterstützen. Der Aufwand dafür ist
% sehr gering und wird mit vielen neuen Möglichkeiten belohnt.
%
% Dieser Hack rüstet die \textsf{tocbasic}-Unterstützung für \textsf{floatrow}
% nach.
%
% \begin{option}{floatrow}
% \changes{v3.12}{2013/03/05}{Signalisierung mit
%     \cs{FamilyKeyStateProcessed}}%^^A
% \changes{v3.17}{2015/03/09}{Defaulteinstellung mit
%     \cs{KOMAExecuteOptions}}%^^A
%    \begin{macrocode}
%<*package&option>
\KOMA@ifkey{floatrow}{@scrhack@floatrow}
\KOMAExecuteOptions{floatrow=true}
%</package&option>
%<*package&body>
\AfterPackage*{floatrow}{%
  \KOMA@key[.scrhack.sty]{floatrow}{%
    \PackageWarning{scrhack}{option `floatrow' ignored}%
    \FamilyKeyStateProcessed
  }%
  \if@scrhack@floatrow\scr@hack@load\@pkgextension{floatrow}\fi
}
%</package&body>
%    \end{macrocode}
% \end{option}
%
%
% \begin{macro}{\DeclareNewFloatType}
% Über die Anweisung \cs{DeclareNewFloatType} wird eine neue Gleitumgebung
% definiert. Hier muss die neue Erweiterung aus dem dritten Argument
% \textsf{tocbasic} bekannt gemacht werden.
% \begin{macro}{\listof}
% Über die Anweisung \cs{listof} wird ein Verzeichnis für Gleitumgebungen
% ausgegeben. Hier muss schlicht die entsprechende Anweisung von
% \textsf{tocbasic} verwendet werden.
% \begin{macro}{\float@addtolists}
% Diese Anweisung wird nicht länger benötigt und daher auf die ursprüngliche
% Definition zurückgesetzt.
%    \begin{macrocode}
%<*floatrow&body>
\scr@ifexpected{\DeclareNewFloatType}{%
  \long\def\DeclareNewFloatType#1#2{\def\FB@captype{#1}%
    \expandafter\edef\csname ftype@#1\endcsname{\the\c@float@type}%
    \addtocounter{float@type}{\value{float@type}}%
    \@namedef{#1name}{#1}\newcounter{#1}%
    \expandafter\edef\csname fnum@#1\endcsname
    {\expandafter\noexpand\csname #1name\endcsname\nobreakspace
      \expandafter\noexpand\csname the#1\endcsname}%
    \@namedef{the#1}{\arabic{#1}}\flnew@ext{lo#1}\@namedef{fps@#1}{tbp}%
    \@namedef{l@#1}{\@dottedtocline{1}{1.5em}{2.3em}}%
    \caption@setkeys[floatrow]{newfloat}{#2}\let\FR@tmp=\relax
    \xdef\@tempa{\noexpand\flrow@types{\the\flrow@types \FR@tmp{#1}}}%
    \@tempa}%
}{%
  \scr@ifexpected{\listof}{%
    \def\listof#1#2{%  
      \@ifundefined{ext@#1}{\flrow@error{Unknown float style `#1'}}{%
        \expandafter\providecommand\csname l@#1\endcsname
        {\@dottedtocline{1}{1.5em}{2.3em}}%
        \float@listhead{#2}%
        \begingroup\setlength{\parskip}{\z@}%
        \@starttoc{\@nameuse{ext@#1}}%
        \endgroup}}%
  }{%
    \RequirePackage{tocbasic}%
    \PackageInfo{scrhack}{redefining \string\DeclareNewFloatType}%
%    \end{macrocode}
% Eigentlich wäre es besser, wie im \textsf{float}-Hack einen Test
% vorzuschalten, ob die Dateiendung bereits in Gebrauch ist. Aber das würde
% voraussetzen, dass die Reihenfolge der Anweisungen geändert wird. Dazu
% stecke ich aber im Code von \textsf{floatrow} zu wenig
% drin. (\emph{\foreignlanguage{english}{Note: It would be better to first
%     test, if the new extension is already in use like done at the
%     \textsf{float} hack. But I don't know the \textsf{floatrow} code good
%     enough to make such a change!}})
%    \begin{macrocode}
    \renewcommand\DeclareNewFloatType[2]{\def\FB@captype{#1}%
      \expandafter\edef\csname ftype@#1\endcsname{\the\c@float@type}%
      \addtocounter{float@type}{\value{float@type}}%
      \@namedef{#1name}{#1}\newcounter{#1}%
      \expandafter\edef\csname fnum@#1\endcsname
      {\expandafter\noexpand\csname #1name\endcsname\nobreakspace
        \expandafter\noexpand\csname the#1\endcsname}%
      \@namedef{the#1}{\arabic{#1}}\flnew@ext{lo#1}\@namedef{fps@#1}{tbp}%
%    \end{macrocode}
% \changes{v3.25}{2018/03/16}{Verwendung von \cs{DeclareTOCStyleEntry}}%^^A
% Bis Version 3.24 wurde hier nun wie üblich |\l@#1| unter Verwendung von
% \cs{@dottedtocline} definiert:
% \begin{verbatim}
%   \@namedef{l@#1}{\@dottedtocline{1}{1.5em}{2.3em}}%
% \end{verbatim}
% Da aber \Package{tocbasic} inzwischen diese schöne Schnittstelle für die
% Verzeichniseinträge bietet, verwenden wir die hier nun auch:
%    \begin{macrocode}
      \DeclareTOCStyleEntry[level=1,numwidth=2.3em,indent=1.5em]{default}{#1}%
      \caption@setkeys[floatrow]{newfloat}{#2}\let\FR@tmp=\relax
      \xdef\@tempa{\noexpand\flrow@types{\the\flrow@types \FR@tmp{#1}}}%
      \@tempa
      \xdef\@tempa{\noexpand\addtotoclist[float]{\@nameuse{ext@\FB@captype}}%
        \noexpand\setuptoc{\@nameuse{ext@\FB@captype}}{chapteratlist}%
      }%
      \@tempa
    }%
    \PackageInfo{scrhack}{redefining \string\listof}%
    \renewcommand*\listof[2]{%
      \@ifundefined{ext@#1}{\flrow@error{Unknown float style `#1'}}{%
        \@ifundefined{l@#1}{\expandafter\let\csname l@#1\endcsname\l@figure
          \@ifundefined{l@#1}{%
%    \end{macrocode}
% Siehe die Anmerkung zur Verwendung von \cs{DeclareTOCStyleEntry} oben.
%    \begin{macrocode}
            \DeclareTOCStyleEntry[level=1,numwidth=2.3em,indent=1.5em]
            {default}{#1}%
          }{}%
        }{}%
        \listoftoc[{#2}]{\csname ext@#1\endcsname}%
      }%
    }%
    \scr@ifexpected{\float@addtolists}{%
      \long\def\float@addtolists#1{%
        \def\float@do##1{\addtocontents{##1}{#1}} \the\float@exts}%
    }{%
      \PackageInfo{scrhack}{undefining \string\float@addtolists}%
      \let\float@addtolists\relax
    }{%
      \PackageWarningNoLine{scrhack}{unkown \string\float@addtolists\space
        definition found!\MessageBreak
        Maybe you are using a unsupported floatrow version}%
    }%
  }{%
    \PackageWarningNoLine{scrhack}{unknown \string\listof\space
      definition found!\MessageBreak
      Maybe you are using a unsupported floatrow version}%
  }%
}{%
  \PackageWarningNoLine{scrhack}{unknown \string\DeclareNewFloatType\space
    definition found!\MessageBreak
    Maybe you are using a unsupported floatrow version}%
}
%</floatrow&body>
%    \end{macrocode}
% \end{macro}
% \end{macro}
% \end{macro}
% 
%
% \subsection{Der \textsf{listings}-Hack}
%
% Das \textsf{listings}-Paket verwendet das Makro \cs{float@listhead} zum
% Setzen der Überschriften. Dies wird seit \KOMAScript~3 nicht mehr empfohlen
% und fliegt demnächst komplett aus der Unterstützung. Stattdessen wird
% empfohlen, dass Pakete \textsf{tocbasic} unterstützen. Der Aufwand dafür ist
% sehr gering und wird mit vielen neuen Möglichkeiten belohnt.
%
% Dieser Hack rüstet die \textsf{tocbasic}-Unterstützung für \textsf{listings}
% nach.
%
% \begin{option}{listings}
% \changes{v3.12}{2013/03/05}{Signalisierung mit
%     \cs{FamilyKeyStateProcessed}}%^^A
% \changes{v3.17}{2015/03/09}{Defaulteinstellung mit
%     \cs{KOMAExecuteOptions}}%^^A
%    \begin{macrocode}
%<*package&option>
\KOMA@ifkey{listings}{@scrhack@listings}
\KOMAExecuteOptions{listings=true}
%</package&option>
%<*package&body>
\AfterPackage*{listings}{%
  \KOMA@key[.scrhack.sty]{listings}{%
    \PackageWarning{scrhack}{option `listings' ignored}%
    \FamilyKeyStateProcessed
  }%
  \if@scrhack@listings\scr@hack@load\@pkgextension{listings}\fi
}
%</package&body>
%    \end{macrocode}
% \end{option}
%
%
% \begin{macro}{\scr@do@hack@listings}
% \begin{macro}{\lstlistoflistings}
% Über dieses Macro wird das Verzeichnis der Listings gesetzt. Die gesamte
% Funktionalität dafür kann \Package{tocbasic} überlassen werden.
% \begin{macro}{\float@addtolists}
% Diese Anweisung wird nicht länger benötigt und daher auf die ursprüngliche
% Definition zurückgesetzt. Da \Package{listings} ihre Definition mit
% \Macro{AtBeginDocument} verzögert, muss dies hier ebenfalls geschehen.
%    \begin{macrocode}
%<*listings&body>
\newcommand*{\scr@do@hack@listings}{%
  \RequirePackage{tocbasic}%
  \addtotoclist[float]{lol}%
  \setuptoc{lol}{chapteratlist}%
  \PackageInfo{scrhack}{redefining \string\lstlistoflistings}%
  \renewcommand*{\lstlistoflistings}{\listoftoc[{\lstlistlistingname}]{lol}}%
  \AtBeginDocument{%
    \scr@ifexpected{\float@addtolists}{%
      \def\float@addtolists##1{\addtocontents{lol}{##1}}%
    }{%
      \PackageInfo{scrhack}{undefining \string\float@addtolists}%
      \let\float@addtolists\relax
    }{%
      \scr@ifexpected{\float@addtolists}{%
        \def\float@addtolists##1{\addtocontents{lol}{##1}%
          \orig@float@addtolists{##1}}%
      }{%
        \PackageInfo{scrhack}{setting \string\float@addtolists\MessageBreak
          to \string\orig@float@addtolists}%
        \let\float@addtolists\orig@float@addtolists
      }{%
        \PackageWarningNoLine{scrhack}{unkown \string\float@addtolists\space
          definition found!\MessageBreak
          Maybe you are using a unsupported listings version}%
      }%
    }%
  }%
%    \end{macrocode}
% \changes{v3.30}{2020/04/01}{setzen von \cs{@captype} ergänzt}%^^A
% Ab Version~3.30 wird zusätzlich \cs{lst@makecaption} um das Setzen von
% \cs{@captype} auf \texttt{lstlisting} ergänzt. Dadurch wird auch die
% Verwendung von \cs{raggedlstlistingcaption} ermöglicht, falls dieses
% definiert wird. Die Umdefinierung erfolgt in Form einer Erweiterung der
% Definition aus \textsf{listings}. Das sollte so lange funktionieren, solange
% die Anweisung in \textsf{listings} keine Argumente hat.
%    \begin{macrocode}
  \expandafter\def\expandafter\lst@makecaption\expandafter{%
    \expandafter\def\expandafter\@captype\expandafter{%
      \expandafter l\expandafter s\expandafter t\expandafter l%
      \expandafter i\expandafter s\expandafter t\expandafter i%
      \expandafter n\expandafter g\expandafter}\lst@makecaption
  }%
  \let\scr@do@hack@listings\relax
}
\scr@ifexpected{\lstlistoflistings}{%
  \def\lstlistoflistings{\bgroup
    \let\contentsname\lstlistlistingname
    \let\lst@temp\@starttoc \def\@starttoc##1{\lst@temp{lol}}%
    \tableofcontents \egroup}%
}{%
  \scr@do@hack@listings
}{%
  \scr@ifexpected{\lstlistoflistings}{%
    \def\lstlistoflistings{%
      \begingroup
        \@ifundefined{@restonecoltrue}{}{%
          \if@twocolumn
            \@restonecoltrue\onecolumn
          \else
            \@restonecolfalse
          \fi
        }%
        \float@listhead{\lstlistlistingname}%
        \parskip\z@\parindent\z@\parfillskip \z@ \@plus 1fil%
        \@starttoc{lol}%
        \@ifundefined{@restonecoltrue}{}{%
          \if@restonecol\twocolumn\fi
        }%
      \endgroup
    }%
  }{%
    \scr@do@hack@listings
  }{%
    \PackageWarningNoLine{scrhack}{unknown \string\lstlistoflistings\space
      definition found!\MessageBreak
      Maybe you are using a unsupported listings version}%
  }%
}
%</listings&body>
%    \end{macrocode}
% \end{macro}
% \end{macro}
% \end{macro}
%
%
%
% \subsection{Der \textsf{nomencl}-Hack}
%
% Das \textsf{nomencl}-Paket harmoniert eigentlich nur in einer Kleinigkeit
% bezüglich des Eintrags in das Inhaltsverzeichnis nicht sehr gut mit
% \KOMAScript-Klassen bzw. Paket \textsf{tocbasic}. Allerdings kann es von den
% Vorteilen von \textsf{tocbasic} auch in weiterer Hinsicht profitieren.
%
% Dieser Hack rüstet weitgehende \textsf{tocbasic}-Unterstützung für
% \textsf{nomencl} nach. Dafür wird Paket \textsf{xpatch} verwendet.
%
% \begin{option}{nomencl}
% \changes{v3.23}{2017/03/31}{neue Option für neuen Hack}%^^A
%    \begin{macrocode}
%<*package&option>
\RequirePackage{xpatch}%
\KOMA@ifkey{nomencl}{@scrhack@nomencl}
\KOMAExecuteOptions{nomencl=true}
%</package&option>
%<*package&body>
\AfterPackage*{nomencl}{%
  \KOMA@key[.scrhack.sty]{nomencl}{%
    \PackageWarning{scrhack}{option `nomencl' ignored}%
    \FamilyKeyStateProcessed
  }%
  \if@scrhack@nomencl\scr@hack@load\@pkgextension{nomencl}\fi
}
%</package&body>
%    \end{macrocode}
% \end{option}%^^A nomencl
%
% \begin{macro}{\thenomenclature}
% \changes{v3.23}{2017/03/31}{Umgebung \texttt{thenomenclature} wird
%     gepatcht}%^^A
% Statt die Fallunterscheidung aus der Originaldefinition direkt hier
% nachzubilden, patchen wir diese komplett heraus und ersetzen sie durch eine
% neue Anweisung.
%    \begin{macrocode}
%<*nomencl&body>
\RequirePackage{tocbasic}
%    \end{macrocode}
% With \Package{nomencl} > 5.0 this should be all.
%    \begin{macrocode}
\@ifpackagelater{nomencl}{2019/01/01}{%
  \PackageInfo{scrhack}{only `tocbasic' loaded for `nomencl'}%
  \expandafter\endinput
}{}
%    \end{macrocode}
% Otherwise we have to patch it.
%    \begin{macrocode}
\addtotoclist[nomencl]{nlo}
\addtotoclist[nomencl]{nls}
\xpatchcmd{\thenomenclature}{%
  \@ifundefined{chapter}%
  {
    \section*{\nomname}
    \if@intoc\addcontentsline{toc}{section}{\nomname}\fi%
  }%
  {
    \chapter*{\nomname}
    \if@intoc\addcontentsline{toc}{chapter}{\nomname}\fi%
  }%
}{%
  \scr@hack@nomencl@head
}{%
  \PackageInfo{scrhack}{\string\thenomenclature\space successfully patched}%
}{%
%    \end{macrocode}
% \changes{v3.26b}{2019/01/23}{new \textsf{nomencl} 5.0}%^^A
% From \textsf{nomencl} 5.0 another code is used if the nomenclature is a
% chapter. So we need another patch:
%    \begin{macrocode}
  \xpatchcmd{\thenomenclature}{%
    \@ifundefined{chapter}%
    {
      \section*{\nomname}
      \if@intoc\addcontentsline{toc}{section}{\nomname}\fi%
    }%
    {
      \chapter*{\nomname}
      \markboth{\nomname}{\nomname}%
      \if@intoc\addcontentsline{toc}{chapter}{\nomname}\fi%
    }%
  }{%
    \scr@hack@nomencl@head
  }{%
    \PackageInfo{scrhack}{\string\thenomenclature\space successfully patched}%
  }{%
%    \end{macrocode}
% \changes{v3.26b}{2019/01/23}{prepared for hoply comming \textsf{nomencl}}%^^A
% But this code is also wrong, because it breaks manual running heads, e.g.,
% of page style \texttt{myheadings}. So I hope, there will be another code,
% that needs a somehow different patch:
%    \begin{macrocode}
    \xpatchcmd{\thenomenclature}{%
      \@ifundefined{chapter}%
      {
        \section*{\nomname}
        \if@intoc\addcontentsline{toc}{section}{\nomname}\fi%
      }%
      {
        \chapter*{\nomname}
        \@mkboth{\nomname}{\nomname}%
        \if@intoc\addcontentsline{toc}{chapter}{\nomname}\fi%
      }%
    }{%
      \scr@hack@nomencl@head
    }{%
      \PackageInfo{scrhack}{\string\thenomenclature\space successfully patched}%
    }{%
%    \end{macrocode}
% \changes{v3.25}{2019/01/23}{emergency patch added}%^^A
% \changes{v3.30}{2020/02/25}{spurious space in warning message removed}%^^A
% If this also does not work, we try an emergency patch, that could break
% things.
%    \begin{macrocode}
      \xpatchcmd{\thenomenclature}{%
        \@ifundefined{chapter}%
      }{%
        \scr@hack@nomencl@head
        \@gobbletwo
      }{%
        \PackageWarning{scrhack}{%
          using emergency patch of \string\thenomenclature.\MessageBreak
          This could happen if you are using an unknown\MessageBreak
          release of package `nomencl'.\MessageBreak
          Mostly this patch does work too, but it also could\MessageBreak
          break `nomencl`. If this happens, please deactivate\MessageBreak
          the `nomencl` hack of `scrhack`%
        }%
      }{%
        \PackageWarning{scrhack}{%
          cannot patch \string\thenomenclature.\MessageBreak 
          This could happen if you are using an unknown\MessageBreak
          release of package `nomencl'%
        }%
      }%
    }%
  }%
}
%</nomencl&body>
%    \end{macrocode}
% \end{macro}%^^A \thenomenclature
%
% \begin{macro}{\scr@hack@nomencl@head}
% \changes{v3.23}{2017/03/31}{Überschrift für \texttt{thenomenclature} unter
%     Verwendung von \textsf{tocbasic}}%^^A
% Im Gegensatz zu Gleitumgebungen benötigen wir hier keine Gruppe, weil
% \texttt{thenomenclature} bereits die Gruppe bereitstellt.
%    \begin{macrocode}
%<*nomencl&body>
\newcommand*{\scr@hack@nomencl@head}{%
  \providecommand*{\listofnlsname}{\nomname}%
  \let\list@fname\listofnlsname
  \def\@currext{nls}%
  \tocbasic@listhead{\list@fname}%
}
%</nomencl&body>
%    \end{macrocode}
% Es ist ggf. noch zu beachten, dass von den ganzen Haken, die
% \textsf{tocbasic} bereitstellt, für dieses Verzeichnis natürlich nur die
% Überschriftenhaken zum Einsatz kommen. Verzeichnishaken können zwar
% definiert werden, kommen aber nie zur Anwendung.
% \end{macro}%^^A \scr@hack@nomencl@head
%
% \begin{macro}{\if@intoc}
% \changes{3.23}{2017/03/31}{Reaktion auf \cs{if@intoc}}
%   Natürlich muss jetzt noch auf \textsf{nomencl}'s Option \texttt{intoc}
%   reagiert werden, also entsprechend der Option das \textsf{tocbasic}
%   Feature \texttt{totoc} passend gesetzt werden. Etwaige Änderungen des
%   Features sind also erst nach dem Laden von \textsf{nomencl} und
%   \textsf{scrhack} möglich. Spätere Änderungen von \cs{if@intoc} werden
%   außerdem nicht berücksichtigt.
%    \begin{macrocode}
%<*nomencl&body>
\if@intoc
  \setuptoc{nls}{totoc}
\fi
%</nomencl&body>
%    \end{macrocode}
% \end{macro}
%
%
% \subsection{Der \textsf{setspace}-Hack}
%
% Das \textsf{setspace}-Paket verwendet \cs{@ptsize} auf ungünstige Art, indem
% es davon ausgeht, dass es immer eine ganze Zahl enthält. Das ist aber bei
% \KOMAScript{} keineswegs zwingend. Außerdem ist der Wert für \texttt{11pt}
% falsch, weil \LaTeX{} in diesem Fall tatsächlich eine 10,95\,pt-Schrift mit
% einem Zeilenabstand von 13,6\,pt einstellt. Damit wäre der korrekte Wert für
% \cs{onehalfspacing} beispielsweise:
% \[ 10{,}95\,\mathrm{pt} \dot 1{,5} / 13{,}6\,\mathrm{pt} \equiv 1{,}208 \]
% Tatsächlich stellt \textsf{setspace} aber einen Wert von 1,213 ein, was
% einer effektiven Schriftgröße von 11\,pt entsprechen würde. Ebenso stellt es
% den aktuellen Abstand bei \cs{onehalfspacing} nicht relativ zur aktuellen
% Schriftgröße ein, sondern zur Grundschriftgröße. Damit erhält man bei
% \begin{verbatim}
% \documentclass[10pt]{article}
% \usepackage{setspace}
% \begin{document}
% \large\onehalfspacing\raggedright
% Fontsize: \csname f@size\endcsname pt\\
% Normal baselineskip: \csname f@baselineskip\endcsname\\
% baselineskip: \the\baselineskip
% \end{document}
% \end{verbatim}
% einen anderen Abstand als bei
% \begin{verbatim}
% \documentclass[11pt]{article}
% \usepackage{setspace}
% \begin{document}
% \large\onehalfspacing\raggedright
% Fontsize: \csname f@size\endcsname pt\\
% Normal baselineskip: \csname f@baselineskip\endcsname\\
% baselineskip: \the\baselineskip
% \end{document}
% \end{verbatim}
% obwohl beide Male dieselbe Schriftgröße verwendet wird. Streng genommen
% müsste also bei jeder Änderung der Schriftgröße der Wert Abstand angepasst
% werden. So weit geht dieser Hack nicht. Stattdessen wird der Wert abhängig
% von der tatsächlichen Schriftgröße und dem tatsächlichen Basisabstand beim
% Aufruf der Anweisungen eingestellt. Das ergibt immerhin in den obigen
% Beispielen gleiche Ergebnisse.
%
% \begin{option}{setspace}
% \changes{v3.17}{2015/04/16}{Neu}%^^A
%    \begin{macrocode}
%<*package&option>
\KOMA@ifkey{setspace}{@scrhack@setspace}
\KOMAExecuteOptions{setspace=true}
%</package&option>
%<*package&body>
\AfterPackage*{setspace}{%
  \KOMA@key[.scrhack.sty]{setspace}{%
    \PackageWarning{scrhack}{option `setspace' ignored}%
    \FamilyKeyStateProcessed
  }%
  \if@scrhack@setspace\scr@hack@load\@pkgextension{setspace}\fi
}
%</package&body>
%    \end{macrocode}
% \end{option}
%
%
% \begin{macro}{\onehalfspacing}
% \changes{v3.17}{2015/04/16}{Neu}%^^A
% Über diese Anweisung wird der eineinhalbzeilige Satz eingestellt. Ein auf
% drei Nachkommastellen genauer Wert erscheint mir ausreichend genau.
%    \begin{macrocode}
%<*setspace&body>
\scr@ifexpected{\onehalfspacing}{%
  \long\def\onehalfspacing{%
    \setstretch{1.25}%  default
    \ifcase \@ptsize \relax % 10pt
      \setstretch {1.25}%
    \or % 11pt
      \setstretch {1.213}%
    \or % 12pt
      \setstretch {1.241}%
    \fi
  }%
}{%
  \renewcommand*{\onehalfspacing}{%
    \@tempdima=\dimexpr (\f@size pt)*1500/
                        (\dimexpr \f@baselineskip\relax)*\p@/1000\relax
    \expandafter\setstretch\expandafter{\strip@pt\@tempdima}%
  }%
}{%
  \PackageWarning{scrhack}{unknown \string\onehalfspacing\space
    definition found!\MessageBreak
    Maybe you are using a unsupported setpace version}%
}
%    \end{macrocode}
% \end{macro}
% \begin{macro}{\doublespacing}
% \changes{v3.17}{2015/04/16}{Neu}%^^A
% Über diese Anweisung wird der zweizeilige Satz eingestellt. Ein auf
% drei Nachkommastellen genauer Wert erscheint mir ausreichend genau.
%    \begin{macrocode}
\scr@ifexpected{\doublespacing}{%
  \long\def\doublespacing{%
    \setstretch{1.667}%  default
    \ifcase \@ptsize \relax % 10pt
      \setstretch {1.667}%
    \or % 11pt
      \setstretch {1.618}%
    \or % 12pt
      \setstretch {1.655}%
    \fi
  }%
}{%
  \renewcommand*{\doublespacing}{%
    \@tempdima=\dimexpr (\f@size pt)*2000/
                        (\dimexpr \f@baselineskip\relax)*\p@/1000\relax
    \expandafter\setstretch\expandafter{\strip@pt\@tempdima}%
  }%
}{%
  \PackageWarning{scrhack}{unknown \string\doublespacing\space
    definition found!\MessageBreak
    Maybe you are using a unsupported setpace version}%
}
%</setspace&body>
%    \end{macrocode}
% \end{macro}
%
%
% \subsection{Der \textsf{lscape}-Hack}
% \changes{v3.18a}{2015/07/07}{Neuer \textsf{lscape}-Hack}%^^A
%
% Das \textsf{lscape}-Paket setzt innerhalb der
% \texttt{landscape}-Umgebung die Länge \cs{textheight} auf den Wert
% von \cs{textwidth} obwohl es auf der anderen Seite \cs{textwidth}
% nicht auf den Wert von \cs{textheight} setzt. Das ist inkonsequent. Da
% David~Carlisle außerdem angibt, dass \cs{textwidth} nicht verändert wird,
% weil das zu Problemen führen konnte, ist es unverständlich, dass
% \cs{textheight} verändert wird, obwohl das ebenfalls zu Problemen führen
% kann, beispielsweise für \textsf{showframe} oder \textsf{scrpage}. Daher
% verändere ich die Definition so, dass auch \cs{textheight} unverändert
% bleibt. Dabei muss allerdings auch \textsf{pdflscape} berücksichtigt
% werden. Das ist am einfachsten mit \textsf{xpatch}.
%
% \begin{option}{lscape}
% \changes{v3.18a}{2015/07/07}{Neue Option für neuen Hack}%^^A
% \changes{v3.20}{2016/05/07}{\textsf{xpatch} wird bereits im Optionenteil
%     geladen}%^^A
%    \begin{macrocode}
%<*package&option>
\RequirePackage{xpatch}%
\KOMA@ifkey{lscape}{@scrhack@lscape}%
\KOMAExecuteOptions{lscape=true}%
%</package&option>
%<*package&body>
\AfterPackage*{lscape}{%
  \if@scrhack@lscape\scr@hack@load\@pkgextension{lscape}\else
    \KOMA@key[.scrhack.sty]{lscape}{%
      \PackageWarning{scrhack}{option `lscape' ignored}%
      \FamilyKeyStateProcessed
    }%
  \fi
}
%</package&body>
%    \end{macrocode}
% \end{option}
%
% \begin{macro}{\landscape}
% \changes{v3.18a}{2015/07/07}{Neu}%^^A
% \changes{v3.20}{2016/05/07}{\textsf{xpatch} wird bereits im Optionenteil
%     geladen}%^^A
% Über diese Anweisung wird die \texttt{landscape}-Umgebung von
% \textsf{lscape} gestartet. Genau diese muss gepatcht werden. Dafür wird das
% Paket \textsf{xpatch} benötigt. Da der Patch nur geladen wird, wenn die
% Option dafür gesetzt ist, kann die Option daher nur ein- und ausgeschaltet
% werden, wenn sie bis zum Laden des Pakets aktiviert wurde.
%    \begin{macrocode}
%<*lscape&body>
\xpatchcmd{\landscape}{\textheight=\vsize}{%
  \if@scrhack@lscape
%    \end{macrocode}
% \changes{v3.18a}{2015/08/11}{Patchen von \cs{@outputpage}}%^^A
% Es gibt allerdings in der Tat eine Stelle, an der ein verändertet Wert von
% \cs{textheight} benötigt wird. Das ist wenn innerhalb von \cs{@outputpage}
% der Wert von \cs{@colht} reinitialisiert wird. Also wird das entsprechend
% auch noch hinein gepatcht.
%    \begin{macrocode}
    \scrh@LT@textheight=\vsize
    \let\scrh@LT@outputpage\@outputpage
    \def\@outputpage{\scrh@LT@outputpage\global\@colht\scrh@LT@textheight}% 
  \else
    \textheight=\vsize
  \fi
}{%
  \PackageInfo{scrhack}{\string\landscape\space patched to make
    \string\textheight\space change optional}%
}{%
  \PackageWarning{scrhack}{Cannot patch \string\landscape!\MessageBreak
    Maybe you are using a unsupported lscape version}%
  \@scrhack@lscapefalse
}
%    \end{macrocode}
% \begin{Length}{\scrh@LT@textheight}
% \changes{v3.18a}{2015/08/11}{neue Länge (intern)}%^^A
%    \begin{macrocode}
\newlength{\scrh@LT@textheight}
%</lscape&body>
%    \end{macrocode}
% \end{Length}%^^A \scrh@LT@textheight
% \end{macro}%^^A \landscape
%
%
% \selectlanguage{english}
% \subsection{The standard classes sections hack}
% \label{sec:standardsections}
% The \emph{standard classes sections hack} differs from each other
% hack. First of all it does not patch a specific package but the class
% definitions of section commands from \cs{part} till
% \cs{subparagraph}. Additionally it resets several internal \LaTeX{} macros
% which are changed by \KOMAScript{} classes to their original
% definition. Last but not least it is not activated by default, but the user
% has to activate it, if he needs it. The hack has more potential to break
% things than to make things better. Because of this, there is not any support
% for using this hack and the hack shows several warning messages.
%
% \begin{option}{standardsections}
%    \begin{macrocode}
%<*package&option>
\KOMA@ifkey{standardsections}{@scrhack@standardsections}
\KOMAExecuteOptions{standardsections=false}
%</package&option>
%<*package&body>
\AtEndOfPackage{%
  \if@scrhack@standardsections
    \PackageInfo{scrhack}{loading standard classes sections hack}%
    \input{standardsections.hak}%
  \fi
}
%</package&body>
%    \end{macrocode}
% \end{option}%^^A standardsections
%
% \begin{macro}{\part}
% \begin{macro}{\chapter}
% \begin{macro}{\section}
% \begin{macro}{\subsection}
% \begin{macro}{\subsubsection}
% \begin{macro}{\paragraph}
% \begin{macro}{\subparagraph}
% The hack redefines all these commands to a definition copied from the
% standard \LaTeX{} classes version v1.4i. Copyright of this code is by the
% \LaTeX3{} Project. The complete, unmodified code can be found on
% \url{https://ctan.org/tex-archive/macros/latex/base/classes.dtx}.
%    \begin{macrocode}
%<*standardsections&body>
\PackageWarningNoLine{scrhack}{%
  You are loosing every kind of support because of\MessageBreak
  using the dangerous standard classes sections hack!\MessageBreak
  This hack can break several classes and packages.\MessageBreak
  However, sometimes it is needed to make a\MessageBreak
  package work%
}%
\PackageInfo{scrhack}{%
  Using code from the standard LaTeX classes.\MessageBreak
  For more information about this code and the\MessageBreak
  license of this code, please see:\MessageBreak
  https://ctan.org/tex-archive/macros/latex/base/classes.dtx
}%
\@ifpackageloaded{hyperref}{%
  \PackageWarning{scrhack}{this hack should not be loaded after `hyperref'}%
}{}
\@ifpackageloaded{titlesec}{%
  \PackageWarning{scrhack}{this hack should not be loaded after `titlesec'}%
}{}
\@ifpackageloaded{sectsty}{%
  \PackageWarning{scrhack}{this hack should not be loaded after `sectsty'}%
}{}
\@ifpackageloaded{sectsty}{%
  \PackageWarning{scrhack}{this hack should not be loaded after `sectsty'}%
}{}
\scr@ifundefinedorrelax{chapter}{%
%    \end{macrocode}
% It seems to be an article class, so we use the code from \textsf{article}.
%    \begin{macrocode}
  \scr@ifundefinedorrelax{c@part}{\newcounter{part}}{}%
  \scr@ifundefinedorrelax{c@section}{\newcounter{section}}{}%
  \scr@ifundefinedorrelax{c@subsection}{\newcounter{subsection}[section]}{}%
  \scr@ifundefinedorrelax{c@subsubsection}{%
    \newcounter{subsubsection}[subsection]}{}%
  \scr@ifundefinedorrelax{c@paragraph}{%
    \newcounter{paragraph}[subsubsection]}{}%
  \scr@ifundefinedorrelax{c@subparagraph}{%
    \newcounter{subparagraph}[paragraph]}{}%
  \renewcommand\thepart {\@Roman\c@part}%
  \renewcommand\thesection {\@arabic\c@section}%
  \renewcommand\thesubsection   {\thesection.\@arabic\c@subsection}%
  \renewcommand\thesubsubsection{\thesubsection.\@arabic\c@subsubsection}%
  \renewcommand\theparagraph    {\thesubsubsection.\@arabic\c@paragraph}%
  \renewcommand\thesubparagraph {\theparagraph.\@arabic\c@subparagraph}%
  \def\part{%
     \if@noskipsec \leavevmode \fi
     \par
     \addvspace{4ex}%
     \@afterindentfalse
     \secdef\@part\@spart}%
  \def\@part[#1]#2{%
      \ifnum \c@secnumdepth >\m@ne
        \refstepcounter{part}%
        \addcontentsline{toc}{part}{\thepart\hspace{1em}#1}%
      \else
        \addcontentsline{toc}{part}{#1}%
      \fi
      {\parindent \z@ \raggedright
       \interlinepenalty \@M
       \normalfont
       \ifnum \c@secnumdepth >\m@ne
         \Large\bfseries \partname\nobreakspace\thepart
         \par\nobreak
       \fi
       \huge \bfseries #2%
       \markboth{}{}\par}%
      \nobreak
      \vskip 3ex
      \@afterheading}%
  \def\@spart#1{%
      {\parindent \z@ \raggedright
       \interlinepenalty \@M
       \normalfont
       \huge \bfseries #1\par}%
       \nobreak
       \vskip 3ex
       \@afterheading}%
  \def\section{\@startsection {section}{1}{\z@}%
                                     {-3.5ex \@plus -1ex \@minus -.2ex}%
                                     {2.3ex \@plus.2ex}%
                                     {\normalfont\Large\bfseries}}%
  \def\subsection{\@startsection{subsection}{2}{\z@}%
                                       {-3.25ex\@plus -1ex \@minus -.2ex}%
                                       {1.5ex \@plus .2ex}%
                                       {\normalfont\large\bfseries}}%
  \def\subsubsection{\@startsection{subsubsection}{3}{\z@}%
                                       {-3.25ex\@plus -1ex \@minus -.2ex}%
                                       {1.5ex \@plus .2ex}%
                                       {\normalfont\normalsize\bfseries}}%
  \def\paragraph{\@startsection{paragraph}{4}{\z@}%
                                      {3.25ex \@plus1ex \@minus.2ex}%
                                      {-1em}%
                                      {\normalfont\normalsize\bfseries}}%
  \def\subparagraph{\@startsection{subparagraph}{5}{\parindent}%
                                         {3.25ex \@plus1ex \@minus .2ex}%
                                         {-1em}%
                                        {\normalfont\normalsize\bfseries}}%
}{%
%    \end{macrocode}
% It seems to be either a book or report class.
%    \begin{macrocode}
  \scr@ifundefinedorrelax{c@part}{\newcounter{part}}{}%
  \scr@ifundefinedorrelax{c@chapter}{\newcounter{chapter}}{}%
  \scr@ifundefinedorrelax{c@section}{\newcounter{section}[chapter]}{}%
  \scr@ifundefinedorrelax{c@subsection}{\newcounter{subsection}[section]}{}%
  \scr@ifundefinedorrelax{c@subsubsection}{%
    \newcounter{subsubsection}[subsection]}{}%
  \scr@ifundefinedorrelax{c@paragraph}{%
    \newcounter{paragraph}[subsubsection]}{}%
  \scr@ifundefinedorrelax{c@subparagraph}{%
    \newcounter{subparagraph}[paragraph]}{}%
  \def\@chapapp{\chaptername}%
  \scr@ifundefinedorrelax{frontmatter}{%
%    \end{macrocode}
% It seems not to be a book class.
%    \begin{macrocode}
    \def\part{%
      \if@openright
        \cleardoublepage
      \else
        \clearpage
      \fi
      \thispagestyle{plain}%
      \if@twocolumn
        \onecolumn
        \@tempswatrue
      \else
        \@tempswafalse
      \fi
      \null\vfil
      \secdef\@part\@spart}

    \def\@part[#1]#2{%
        \ifnum \c@secnumdepth >-2\relax
          \refstepcounter{part}%
          \addcontentsline{toc}{part}{\thepart\hspace{1em}#1}%
        \else
          \addcontentsline{toc}{part}{#1}%
        \fi
        \markboth{}{}%
        {\centering
         \interlinepenalty \@M
         \normalfont
         \ifnum \c@secnumdepth >-2\relax
           \huge\bfseries \partname\nobreakspace\thepart
           \par
           \vskip 20\p@
         \fi
         \Huge \bfseries #2\par}%
        \@endpart}
    \def\@spart#1{%
        {\centering
         \interlinepenalty \@M
         \normalfont
         \Huge \bfseries #1\par}%
        \@endpart}
    \def\@endpart{\vfil\newpage
                  \if@twoside
                   \if@openright
                    \null
                    \thispagestyle{empty}%
                    \newpage
                   \fi
                  \fi
                  \if@tempswa
                    \twocolumn
                  \fi}
    \def\chapter{\if@openright\cleardoublepage\else\clearpage\fi
                        \thispagestyle{plain}%
                        \global\@topnum\z@
                        \@afterindentfalse
                        \secdef\@chapter\@schapter}
    \def\@chapter[#1]#2{\ifnum \c@secnumdepth >\m@ne
                             \refstepcounter{chapter}%
                             \typeout{\@chapapp\space\thechapter.}%
                             \addcontentsline{toc}{chapter}%
                                       {\protect\numberline{\thechapter}#1}%
                        \else
                          \addcontentsline{toc}{chapter}{#1}%
                        \fi
                        \chaptermark{#1}%
                        \addtocontents{lof}{\protect\addvspace{10\p@}}%
                        \addtocontents{lot}{\protect\addvspace{10\p@}}%
                        \if@twocolumn
                          \@topnewpage[\@makechapterhead{#2}]%
                        \else
                          \@makechapterhead{#2}%
                          \@afterheading
                        \fi}
    \def\@makechapterhead#1{%
      \vspace*{50\p@}%
      {\parindent \z@ \raggedright \normalfont
        \ifnum \c@secnumdepth >\m@ne
            \huge\bfseries \@chapapp\space \thechapter
            \par\nobreak
            \vskip 20\p@
        \fi
        \interlinepenalty\@M
        \Huge \bfseries #1\par\nobreak
        \vskip 40\p@
      }}
    \def\@schapter#1{\if@twocolumn
                       \@topnewpage[\@makeschapterhead{#1}]%
                     \else
                       \@makeschapterhead{#1}%
                       \@afterheading
                     \fi}
    \def\@makeschapterhead#1{%
      \vspace*{50\p@}%
      {\parindent \z@ \raggedright
        \normalfont
        \interlinepenalty\@M
        \Huge \bfseries  #1\par\nobreak
        \vskip 40\p@
      }}
    \def\section{\@startsection {section}{1}{\z@}%
                                       {-3.5ex \@plus -1ex \@minus -.2ex}%
                                       {2.3ex \@plus.2ex}%
                                       {\normalfont\Large\bfseries}}%
    \def\subsection{\@startsection{subsection}{2}{\z@}%
                                         {-3.25ex\@plus -1ex \@minus -.2ex}%
                                         {1.5ex \@plus .2ex}%
                                         {\normalfont\large\bfseries}}%
    \def\subsubsection{\@startsection{subsubsection}{3}{\z@}%
                                         {-3.25ex\@plus -1ex \@minus -.2ex}%
                                         {1.5ex \@plus .2ex}%
                                         {\normalfont\normalsize\bfseries}}%
    \def\paragraph{\@startsection{paragraph}{4}{\z@}%
                                        {3.25ex \@plus1ex \@minus.2ex}%
                                        {-1em}%
                                        {\normalfont\normalsize\bfseries}}%
    \def\subparagraph{\@startsection{subparagraph}{5}{\parindent}%
                                           {3.25ex \@plus1ex \@minus .2ex}%
                                           {-1em}%
                                          {\normalfont\normalsize\bfseries}}%
  }{%
%    \end{macrocode}
% It seems to be a book class.
%    \begin{macrocode}
    \scr@ifundefinedorrelax{if@mainmatter}{%
      \expandafter\newif\csname if@mainmatter\endcsname
    }{}%
    \def\frontmatter{%
        \cleardoublepage
      \@mainmatterfalse
      \pagenumbering{roman}}%
    \def\mainmatter{%
        \cleardoublepage
      \@mainmattertrue
      \pagenumbering{arabic}}%
    \def\backmatter{%
      \if@openright
        \cleardoublepage
      \else
        \clearpage
      \fi
      \@mainmatterfalse}%
    \def\part{%
      \if@openright
        \cleardoublepage
      \else
        \clearpage
      \fi
      \thispagestyle{plain}%
      \if@twocolumn
        \onecolumn
        \@tempswatrue
      \else
        \@tempswafalse
      \fi
      \null\vfil
      \secdef\@part\@spart}%

    \def\@part[#1]#2{%
        \ifnum \c@secnumdepth >-2\relax
          \refstepcounter{part}%
          \addcontentsline{toc}{part}{\thepart\hspace{1em}#1}%
        \else
          \addcontentsline{toc}{part}{#1}%
        \fi
        \markboth{}{}%
        {\centering
         \interlinepenalty \@M
         \normalfont
         \ifnum \c@secnumdepth >-2\relax
           \huge\bfseries \partname\nobreakspace\thepart
           \par
           \vskip 20\p@
         \fi
         \Huge \bfseries #2\par}%
        \@endpart}%
    \def\@spart#1{%
        {\centering
         \interlinepenalty \@M
         \normalfont
         \Huge \bfseries #1\par}%
        \@endpart}%
    \def\@endpart{\vfil\newpage
                  \if@twoside
                   \if@openright
                    \null
                    \thispagestyle{empty}%
                    \newpage
                   \fi
                  \fi
                  \if@tempswa
                    \twocolumn
                  \fi}%
    \def\chapter{\if@openright\cleardoublepage\else\clearpage\fi
                        \thispagestyle{plain}%
                        \global\@topnum\z@
                        \@afterindentfalse
                        \secdef\@chapter\@schapter}%
    \def\@chapter[#1]#2{\ifnum \c@secnumdepth >\m@ne
                           \if@mainmatter
                             \refstepcounter{chapter}%
                             \typeout{\@chapapp\space\thechapter.}%
                             \addcontentsline{toc}{chapter}%
                                       {\protect\numberline{\thechapter}#1}%
                           \else
                             \addcontentsline{toc}{chapter}{#1}%
                           \fi
                        \else
                          \addcontentsline{toc}{chapter}{#1}%
                        \fi
                        \chaptermark{#1}%
                        \addtocontents{lof}{\protect\addvspace{10\p@}}%
                        \addtocontents{lot}{\protect\addvspace{10\p@}}%
                        \if@twocolumn
                          \@topnewpage[\@makechapterhead{#2}]%
                        \else
                          \@makechapterhead{#2}%
                          \@afterheading
                        \fi}%
    \def\@makechapterhead#1{%
      \vspace*{50\p@}%
      {\parindent \z@ \raggedright \normalfont
        \ifnum \c@secnumdepth >\m@ne
          \if@mainmatter
            \huge\bfseries \@chapapp\space \thechapter
            \par\nobreak
            \vskip 20\p@
          \fi
        \fi
        \interlinepenalty\@M
        \Huge \bfseries #1\par\nobreak
        \vskip 40\p@
      }}%
    \def\@schapter#1{\if@twocolumn
                       \@topnewpage[\@makeschapterhead{#1}]%
                     \else
                       \@makeschapterhead{#1}%
                       \@afterheading
                     \fi}%
    \def\@makeschapterhead#1{%
      \vspace*{50\p@}%
      {\parindent \z@ \raggedright
        \normalfont
        \interlinepenalty\@M
        \Huge \bfseries  #1\par\nobreak
        \vskip 40\p@
      }}%
    \def\section{\@startsection {section}{1}{\z@}%
                                       {-3.5ex \@plus -1ex \@minus -.2ex}%
                                       {2.3ex \@plus.2ex}%
                                       {\normalfont\Large\bfseries}}%
    \def\subsection{\@startsection{subsection}{2}{\z@}%
                                         {-3.25ex\@plus -1ex \@minus -.2ex}%
                                         {1.5ex \@plus .2ex}%
                                         {\normalfont\large\bfseries}}%
    \def\subsubsection{\@startsection{subsubsection}{3}{\z@}%
                                         {-3.25ex\@plus -1ex \@minus -.2ex}%
                                         {1.5ex \@plus .2ex}%
                                         {\normalfont\normalsize\bfseries}}%
    \def\paragraph{\@startsection{paragraph}{4}{\z@}%
                                        {3.25ex \@plus1ex \@minus.2ex}%
                                        {-1em}%
                                        {\normalfont\normalsize\bfseries}}%
    \def\subparagraph{\@startsection{subparagraph}{5}{\parindent}%
                                           {3.25ex \@plus1ex \@minus .2ex}%
                                           {-1em}%
                                          {\normalfont\normalsize\bfseries}}%
  }%
}% End of the derived code.
%    \end{macrocode}
% \end{macro}
% \end{macro}
% \end{macro}
% \end{macro}
% \end{macro}
% \end{macro}
% \end{macro}
%
% \begin{macro}{\@startsection}
% \KOMAScript{} classes redefine this internal \LaTeX{} kernel macro. In this
% case \cs{scr@saved@startsection} is defined, differs from
% \cs{@startsection} and has the definition found before redefining the
% original macro. We do nothing else but resetting \cs{@startsection} to this
% saved definition.
%    \begin{macrocode}
\scr@ifundefinedorrelax{scr@saved@startsection}{}{%
  \ifx\@startsection\scr@saved@startsection\else
    \PackageWarningNoLine{scrhack}{%
      Resetting \string\@startsection,\MessageBreak
      because of loading standard classes sections hack%
    }%
    \let\@startsection\scr@saved@startsection
  \fi
}
%    \end{macrocode}
% \end{macro}%^^A \@startsection
%
% \begin{macro}{\@sect}
% \begin{macro}{\@ssect}
% \begin{macro}{\@xsect}
% All these should be resetted to their original definitions stored in
% \cs{scr@latex@\dots}.
%    \begin{macrocode}
\scr@ifundefinedorrelax{scr@latex@sect}{}{%
  \ifx\@sect\scr@latex@sect\else
    \PackageWarningNoLine{scrhack}{%
      Resetting \string\@sect,\MessageBreak
      because of loading standard classes sections hack%
    }%
    \let\@sect\scr@latex@sect
  \fi
}
\scr@ifundefinedorrelax{scr@latex@ssect}{}{%
  \ifx\@ssect\scr@latex@ssect\else
    \PackageWarningNoLine{scrhack}{%
      Resetting \string\@ssect,\MessageBreak
      because of loading standard classes sections hack%
    }%
    \let\@sect\scr@latex@ssect
  \fi
}
\scr@ifundefinedorrelax{scr@latex@xsect}{}{%
  \ifx\@xsect\scr@latex@xsect\else
    \PackageWarningNoLine{scrhack}{%
      Resetting \string\@xsect,\MessageBreak
      because of loading standard classes sections hack%
    }%
    \let\@sect\scr@latex@sect
  \fi
}
%    \end{macrocode}
% \end{macro}%^^A \@xsect
% \end{macro}%^^A \@ssect
% \end{macro}%^^A \@sect
%
% \begin{macro}{\DeclareSectionCommand}
% \begin{macro}{\RedeclareSectionCommand}
% \begin{macro}{\DeclareNewSectionCommand}
% \begin{macro}{\ProvideSectionCommand}
% \begin{macro}{\DeclareSectionCommands}
% \begin{macro}{\RedeclareSectionCommands}
% \begin{macro}{\DeclareNewSectionCommands}
% \begin{macro}{\ProvideSectionCommands}
% These \KOMAScript{} commands do not make sense any longer after forced
% redefinition of \cs{part} \dots{} \cs{subparagraph}. They even would not
% work correctly after resetting, e.g., \cs{@startsection}. So best reaction
% is to \emph{undefine} them.
%    \begin{macrocode}
\scr@ifundefinedorrelax{DeclareSectionCommand}{}{%
  \PackageWarningNoLine{scrhack}{%
    \string\DeclareSectionCommand\space
    and all derivates deactivated,\MessageBreak
    because of loading standard classes sections hack%
  }%
  \let\DeclareSectionCommand\undefined
  \let\DeclareSectionCommands\undefined
  \let\RedeclareSectionCommand\undefined
  \let\RedeclareSectionCommands\undefined
  \let\DeclareNewSectionCommand\undefined
  \let\DeclareNewSectionCommands\undefined
  \let\ProvideSectionCommand\undefined
  \let\ProvideSectionCommands\undefined
}
%    \end{macrocode}
% \begin{macro}{\scr@chapter@before@hyperref@patch}
% \begin{macro}{\scr@chapter@after@hyperref@patch}
% Now, definitions without respecting \textsf{hyperref} are used, so
% \textsf{hyperref} should again be able to patch them. So the hack should
% deactivate the patch deactivation of the \KOMAScript{} classes.
%    \begin{macrocode}
\let\scr@chapter@before@hyperref@patch\relax
\let\scr@chapter@after@hyperref@patch\relax
%    \end{macrocode}
% \end{macro}
% \end{macro}
% \end{macro}
% \end{macro}
% \end{macro}
% \end{macro}
% \end{macro}
% \end{macro}
% \end{macro}
% \end{macro}%^^A \DeclareSectionCommand
%
% \begin{macro}{\scr@osectarg}
% The \KOMAScript{} classes use this macro to store how to use the optional
% argument of section commands. But with the standard class definition the
% only possibility is to use it as running head and write it into the table of
% contents without any extended handling. So the only value, that makes sense
% is 0.
%    \begin{macrocode}
\scr@ifundefinedorrelax{scr@osectarg}{}{%
  \def\reserved@a{0}%
  \ifx\scr@osectarg\reserved@a\else
    \let\scr@osectarg\reserved@a
    \PackageWarningNoLine{scrhack}{%
      Extended option handling for section commands deactivated,\MessageBreak
      because of loading standard classes sections hack%
    }%
  \fi
}%
%    \end{macrocode}
% \begin{macro}{\scr@activate@xsection}
% Further changing of the value has to be deactivated.
%    \begin{macrocode}
\scr@ifundefinedorrelax{scr@activate@xsection}{}{%
  \renewcommand*{\scr@activate@xsection}[1]{%
    \edef\scr@osectarg{#1}\def\reserved@a{0}%
    \ifx\scr@osectarg\reserved@a
    \else
      \let\scr@osectarg\reserved@a
      \PackageWarning{scrhack}{%
        Because of loading the standard classes\MessageBreak
        sections hack, the extended option handling\MessageBreak
        of section commands failed%
      }%
    \fi
  }%
}
%    \end{macrocode}
% \end{macro}%^^A \scr@activate@xsection
% \end{macro}%^^A \scr@osectarg
%
% \begin{macro}{\partformat}
% \begin{macro}{\chapterformat}
% Most \cs{\dots format} macros of \KOMAScript{} are still used, because of
% the redefinition of \cs{@seccntformat}. But \cs{part} and \cs{chapter} are
% defined without using them.
%    \begin{macrocode}
\scr@ifundefinedorrelax{partformat}{}{%
  \PackageWarningNoLine{scrhack}{%
    Usage of \string\partformat\space
    deactivated,\MessageBreak
    because of loading the standard classes sections hack%
  }%
}
\scr@ifundefinedorrelax{chapterformat}{}{%
  \PackageWarningNoLine{scrhack}{%
    Usage of \string\chapterformat\space
    deactivated,\MessageBreak
    because of loading the standard classes sections hack%
  }%
}
%    \end{macrocode}
% \end{macro}%^^A \chapterformat
% \end{macro}%^^A \partformat
%
% \begin{macro}{\sectionformat}
% \begin{macro}{\subsectionformat}
% \begin{macro}{\subsubsectionformat}
% \begin{macro}{\paragraphformat}
% \begin{macro}{\subparagraphformat}
% All these commands have to be resetted to not use the automatic dot and to
% add a \cs{quad} instead of an \cs{enskip}.
%    \begin{macrocode}
\scr@ifundefinedorrelax{sectionformat}{}{%
  \PackageWarningNoLine{scrhack}{%
    \string\sectionformat\space set to
    `\string\thesection\string\quad',\MessageBreak
    because of loading the standard classes sections hack%
  }%
  \renewcommand*{\sectionformat}{\thesection\quad}%
}
\scr@ifundefinedorrelax{subsectionformat}{}{%
  \PackageWarningNoLine{scrhack}{%
    \string\subsectionformat\space set to
    `\string\thesubsection\string\quad'\MessageBreak
    because of loading the standard classes sections hack%
  }%
  \renewcommand*{\subsectionformat}{\thesubsection\quad}%
}
\scr@ifundefinedorrelax{subsubsectionformat}{}{%
  \PackageWarningNoLine{scrhack}{%
    \string\subsubsectionformat\space set to
    `\string\thesubsubsection\string\quad'\MessageBreak
    because of loading the standard classes sections hack%
  }%
  \renewcommand*{\subsubsectionformat}{\thesubsubsection\quad}%
}
\scr@ifundefinedorrelax{paragraphformat}{}{%
  \PackageWarningNoLine{scrhack}{%
    \string\paragraphformat\space set to
    `\string\theparagraph\string\quad'\MessageBreak
    because of loading the standard classes sections hack%
  }%
  \renewcommand*{\paragraphformat}{\theparagraph\quad}%
}
\scr@ifundefinedorrelax{subparagraphformat}{}{%
  \PackageWarningNoLine{scrhack}{%
    \string\subparagraphformat\space set to
    `\string\thesubparagraph\string\quad'\MessageBreak
    because of loading the standard classes sections hack%
  }%
  \renewcommand*{\subparagraphformat}{\thesubparagraph\quad}%
}
%    \end{macrocode}
% \end{macro}
% \end{macro}
% \end{macro}
% \end{macro}
% \end{macro}
%
% \begin{macro}{\partlineswithprefixformat}
% \begin{macro}{\chapterlineswithprefixformat}
% \begin{macro}{\chapterlinesformat}
% \begin{macro}{\sectionlinesformat}
% \begin{macro}{\sectioncatchphraseformat}
% These \KOMAScript{} commands are not used any longer.
%    \begin{macrocode}
\scr@ifundefinedorrelax{partlineswithprefixformat}{}{%
  \PackageWarningNoLine{scrhack}{%
    Usage of \string\partlineswithprefixformat\space deactivated,\MessageBreak
    because of loading the standard classes sections hack%
  }%
}
\scr@ifundefinedorrelax{chapterlineswithprefixformat}{}{%
  \PackageWarningNoLine{scrhack}{%
    Usage of \string\chapterlineswithprefixformat\space and\MessageBreak
    \string\chapterlinesformat\space deactivated,\MessageBreak
    because of using standard classes sections hack%
  }%
}
\scr@ifundefinedorrelax{sectionlinesformat}{}{%
  \PackageWarningNoLine{scrhack}{%
    Usage of \string\sectionlinesformat\space and\MessageBreak
    \string\sectioncatchphraseformat\space deactivated,\MessageBreak
    because of using standard classes sections hack%
  }%
}
%    \end{macrocode}
% \end{macro}%^^A \sectioncatchphraseformat
% \end{macro}%^^A \sectionlinesformat
% \end{macro}%^^A \chapterlinesformat
% \end{macro}%^^A \chapterlineswithprefixformat
% \end{macro}%^^A \partlineswithprefixformat
%
% \begin{option}{headings}
% The \KOMAScript{} option does not make sense any longer. So it is
% deactivated and usage will throw an error.
% \begin{macrocode}
\scr@ifundefinedorrelax{KOMAClassName}{}{%
  \PackageWarningNoLine{scrhack}{%
    KOMA-Script option `headings' removed,\MessageBreak
    because of loading standard classes sections hack%
  }%
  \RelaxFamilyKey[.\KOMAClassFileName]{KOMA}{headings}%
  \KOMA@kav@removekey{.\KOMAClassFileName}{headings}%
}
%    \end{macrocode}
% \end{option}%^^A headings
%
% \begin{macro}{\scr@class@titlesec@warning}
% The warning about loading \textsf{titlesec} has to be changes, because we do
% not expect errors any longer.
%    \begin{macrocode}
\def\scr@class@titlesec@warning{%
  \PackageWarning{scrhack}{%
    Usage of `titlesec' should work, because of\MessageBreak
    loading the standard classes sections hack.\MessageBreak
    However, you should know, there is not any\MessageBreak
    support for this combination by the\MessageBreak
    KOMA-Script author%
  }%
}
%</standardsections&body>
%    \end{macrocode}
% \end{macro}%^^A \scr@class@titlesec@warning
% \selectlanguage{ngerman}%^^A
%
% \subsection{Optionen ausführen}
%
% Zum Schluss noch die Optionen ausführen. Im Paket wird diese Anweisung
% allerdings vor den Anweisungen der Hacks und den Anweisungen aus dem
% Abschnitt »Verwendete Anweisungen« stehen.
%    \begin{macrocode}
%<*package&option>
\KOMAProcessOptions\relax
%</package&option>
%    \end{macrocode}
%
%
% \Finale
%
\endinput
%
% end of file `scrhack.dtx'
%%% Local Variables:
%%% mode: doctex
%%% mode: flyspell
%%% coding: utf-8
%%% ispell-local-dictionary: "en_GB"
%%% TeX-master: t
%%% End:

%  LocalWords:  eineinhalbzeilige
