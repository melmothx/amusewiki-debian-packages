% ======================================================================
% preface.tex
% Copyright (c) Markus Kohm, 2008-2019
%
% This file is part of the LaTeX2e KOMA-Script bundle.
%
% This work may be distributed and/or modified under the conditions of
% the LaTeX Project Public License, version 1.3c of the license.
% The latest version of this license is in
%   http://www.latex-project.org/lppl.txt
% and version 1.3c or later is part of all distributions of LaTeX
% version 2005/12/01 or later and of this work.
%
% This work has the LPPL maintenance status "author-maintained".
%
% The Current Maintainer and author of this work is Markus Kohm.
%
% This work consists of all files listed in manifest.txt.
% ----------------------------------------------------------------------
% preface.tex
% Copyright (c) Markus Kohm, 2008-2019
%
% Dieses Werk darf nach den Bedingungen der LaTeX Project Public Lizenz,
% Version 1.3c, verteilt und/oder veraendert werden.
% Die neuste Version dieser Lizenz ist
%   http://www.latex-project.org/lppl.txt
% und Version 1.3c ist Teil aller Verteilungen von LaTeX
% Version 2005/12/01 oder spaeter und dieses Werks.
%
% Dieses Werk hat den LPPL-Verwaltungs-Status "author-maintained"
% (allein durch den Autor verwaltet).
%
% Der Aktuelle Verwalter und Autor dieses Werkes ist Markus Kohm.
%
% Dieses Werk besteht aus den in manifest.txt aufgefuehrten Dateien.
% ======================================================================

\KOMAProvidesFile{preface.tex}
                 [$Date: 2019-12-19 10:20:31 +0100 (Thu, 19 Dec 2019) $
                  Preface to version 3.25]
\translator{Markus Kohm\and Karl Hagen\and DeepL}

% Date of the translated German file: 2019-12-19

\addchap{Preface to \KOMAScript~3.28}

The \KOMAScript~3.28 manual,\,---\,not only the German version\,---\,once
again benefits from the fact that a new edition of the print version
\cite{book:komascript} and the eBook version \cite{ebook:komascript} will be
published at almost the same time as this version. This has led to many
improvements which also affect the free manual, in both the German and the
English version.

In \KOMAScript~3.28 there are also some significant changes. In some cases,
compatibility with earlier versions has been waived. Thus a recommendation
from the ranks of \emph{The LaTeX Project Team} regarding \Macro{if\dots}
statements is complied with. If you use such statements, you should refer to
the manual again.

It is not just about the manual that I now receive little criticism. I
conclude from this fact that \KOMAScript{} has reached the level that it
fulfils all desires. At the same time, the project has\,---\,not only starting
with the current release\,---\,reached a scale that makes it almost impossible
for a single person to accomplish
\begin{itemize}
\item the search for and elimination of errors,
\item the development and implementation of new functions,
\item the observation of changes in other packages and the \LaTeX{} kernel
  with regard to effects on \KOMAScript,
\item the rapid response to such changes,
\item the maintenance of the guides in two languages,
\item help for beginners far beyond the functions of \KOMAScript{} down to the
  basic operation of a computer,
\item assistance in the implementation of tricky solutions for advanced users
  and experts,
\item moderation and participation in the maintenance of a forum for all kind
  of help around \KOMAScript.
\end{itemize}
While I am personally have most fun with the development of new functions, I
consider troubleshooting in existing features, compatibility with new \LaTeX{}
kernel versions, and above all instructing users for the most important
tasks. Therefore I will focus in the future on and new functions will be
available only in exceptional cases. Therefore already in \KOMAScript~3.28
some experimental functions and packages have been removed. In future releases
this should be continued.

This, of course, also reduces the effort for the documentation of new
functions.  Readers of this free, screen version, however, still have to live
with some restrictions. So some information\,---\,mainly intended for advanced
users or capable of turning an ordinary user into an advanced one\,---\,is
reserved for the printed book, which currently exists only in German. As a
result, some links in this manual lead to a page that simply mentions this
fact. In addition, the free version is scarcely suitable for making a
hard-copy. The focus, instead, is on using it on screen, in parallel with the
document you are working on. It still has no optimized wrapping but is almost
a first draft, in which both the paragraph and page breaks are in some cases
quite poor. Corresponding optimizations are reserved for the German book
editions.

Another important improvement to the English guide has been accomplished by
Karl Hagen, who has continued the translation of the entire manual. Many, many
thanks to him! Everything that is fine in this English manual is because of
him. Everything that is not good in this manual\,---\,like the translation of
this preface\,---\,is because of me. Additional editors or translators,
however, would still be welcome!

But the biggest thanks go to my family and above all to my wife. They absorb
all my unpleasant experiences on the Internet. They have also tolerated it for
more than 25~years, when I am again not approachable, because I am completely
lost in \KOMAScript{} or some \LaTeX{} problems. The fact that I can afford to
invest an incredible amount of time in such a project is entirely thanks to my
wife.

\bigskip\noindent
Markus Kohm, Neckarhausen in the foggy December of 2019.

\endinput

%%% Local Variables: 
%%% mode: latex
%%% mode: flyspell
%%% ispell-local-dictionary: "english"
%%% coding: us-ascii
%%% TeX-master: "../guide"
%%% End: 

