% Copyright 2012 by Till Tantau
%
% This file may be distributed and/or modified
%
% 1. under the LaTeX Project Public License and/or
% 2. under the GNU Free Documentation License.
%
% See the file doc/generic/pgf/licenses/LICENSE for more details.


\section{Writing Graph Drawing Algorithms in C}
\label{section-algorithms-in-c}

\noindent{\emph{by Till Tantau}}
\bigskip
\ifluatex
\else
    This section of the manual can only be typeset using Lua\TeX.
    \expandafter\endinput
\fi

In the present section we have a look at how graph drawing
algorithms written in the C programming language (or in C++) can be
used in the graph drawing framework.

\begin{quote}
    \emph{Warning:} Graph drawing algorithms written in C can be incredibly
    fast if you use the facilities of C correctly. \emph{However,} C code is
    much less portable than Lua code in the sense that it has to be compiled
    for the specific platform used by the user and that it has to be linked
    dynamically during a run of the \TeX\ program. All of this in possible (and
    works, as demonstrated by the linking of the \textsc{ogdf} framework), but
    it is \emph{much} harder to get right than writing Lua code.

    Bottom line, \emph{you really should be using this method only if it is
    really necessary (namely, when Lua code is simply not fast enough).}
\end{quote}

In the following, I first explain how the link between \TeX\ and C code works,
in general. Then, in the subsequent sections, we go over the different kinds of
programming languages and frameworks for which there is direct support for such
a link.


\subsection{How C and \TeX\ Communicate}

In order to use C code for graph drawing algorithms during a run of the \TeX\
program, there is no need to build a new version of \TeX. Rather, it is
possible that C code is linked into the \TeX\ executable at runtime. This is
made possible by the fact that Lua (which part of Lua\TeX$\dots$) is able to
link C libraries at runtime -- provided a strict regime of rules is adhered to:
%
\begin{enumerate}
    \item When you say |require| in Lua, it will normally look for a |.lua|
        file; but it will also try to find a |.so| file (a shared C library) as
        a fallback.
    \item If it finds such a shared library, Lua(\TeX) will try to link this
        library dynamically at runtime.
    \item Inside the library, there must be a function (called an entry point)
        with a special name (it must start with |luaopen_| and it must
        otherwise be the path and name of the library with slashes replaced by
        underscores).
    \item This function gets called by Lua, once. Its job is to setup the
        library so that it can be used by Lua. Mainly, this means that certain
        C functions get registered in such a way that Lua can call them.
    \item At this point, control returns to Lua and, now, certain functions
        have become available on the Lua layer that, when called, actually
        invoke the C code of our linked library.
\end{enumerate}

For each of the above points, there are some bells and whistles:
%
\begin{enumerate}
    \item Lua\TeX\ looks at slightly inconvenient places for shared libraries:
        By default, (currently, 2013) it looks in a |lib| subdirectory of the
        directory containing the Lua\TeX\ executable. The logic behind is that
        the shared libraries depend on the specific architecture of the
        executable. Thus, unlike normal Lua files, the library needs to be
        installed ``far away'' from the actual package of which it is part.
    \item Certain versions of Lua\TeX\ have a broken handling of filenames of
        libraries written in C. The TL2013 version of Lua\TeX, for instance,
        crashes when the filename of a shared library does not contain the
        complete path (while this works for normal file). Hopefully, this, too,
        will be fixed in future versions.
    \item On certain platforms, the support for dynamic linking against
        Lua\TeX\ is broken since the symbol table of the Lua library has been
        stripped away. Hopefully, this will be fixed soon; in the meantime, a
        highly fragile workaround is to link in another copy of the Lua
        library.
    \item The entry point that is called by Lua requires a certain signature
        (it takes a Lua state as its only parameter) and must return the number
        of objects it returns on the Lua stack.
    \item The registration process of C functions is somewhat tricky and
        changes from Lua version to Lua version.
    \item C functions that get called by Lua must follow all sorts of tricky
        rules in order to communicate with Lua correctly.
\end{enumerate}

Despite the above obstacles, one can use graph drawing algorithms written in C
inside Lua, in principle, as follows: One loads an appropriately prepared and
located C library using |require| and this library uses commands like |declare|
to register its own functions into the graph drawing system so that when the
|run| method is called, a C functions gets called instead.

Unfortunately, the above approach is extremely tedious and error-prone and it
is ``no fun'' to access Lua data structures (such as the syntactic digraph)
from C. For this reason, I have written some libraries that encapsulate (as
much as possible) of this communication between C and Lua. Indeed, when you use
these libraries, you can focus entirely on the graph drawing issues and you
will not even notice that your code ``is talking to Lua''. (Except for the name
of the entry point, which is fixed to start with |luaopen_| and it is
impossible to change this without disrupting a lot inside Lua's module system).

There are libraries available for simplifying the communication between the
graph drawing system and graph drawing algorithms written in
%
\begin{itemize}
    \item C, see Section~\ref{section-gd-c},
    \item C++, see Section~\ref{section-gd-c++},
    \item Open Graph Drawing Framework, see
        Section~\ref{section-gd-ogdf-interface}.
\end{itemize}


\subsection{Writing Graph Drawing Algorithms in C}
\label{section-gd-c}

\subsubsection{The Hello World of Graph Drawing in C}

As our first example, as always, the ``hello world'' of graph drawing simply
places nodes on a circle. For this, we implement a function |fast_hello_world|
in a file |SimpleDemoC.c|. It starts as follows:
%
\begin{codeexample}[code only, tikz syntax=false]
#include <pgf/gd/interface/c/InterfaceFromC.h>
#include <math.h>

static void fast_hello_world (pgfgd_SyntacticDigraph* graph) {
  ...
}
\end{codeexample}

As we can see, we first include a special header file of a rather small library
that does all the hard work of translating between Lua and C for us
(|InterfaceFromC|). These header files reside in the |c| subdirectory of the
|pgf| package. Note that we do \emph{not} have to include the headers of the
Lua library; indeed, you do not need access to the source of Lua to use the
interface headers. As a side effect, we will, however, have to write
|struct lua_State| instead of the more common |lua_State| once in our code,
namely in the declaration of the entry point; but that is the only downside.

The library |InterfaceFromC| declares the type |pgfgd_SyntacticDigraph|. In a
moment, we will see that we can setup a key |fast simple demo layout| such that
when this key is used on the display layer, the function |fast_hello_world|
gets called. When it is called, the |graph| parameter will be a full
representation of the to-be-laid-out graph. We can access the fields of the
graph and even directly modify some of its fields (in particular, we can modify
the |pos| fields of the vertices). Here is the complete code of the algorithm:
%
\begin{codeexample}[code only, tikz syntax=false]
static void fast_hello_world (pgfgd_SyntacticDigraph* graph) {
  double angle  = 6.28318530718 / graph->vertices.length;
  double radius = pgfgd_tonumber(graph->options, "fast simple demo radius");

  int i;
  for (i = 0; i < graph->vertices.length; i++) {
    pgfgd_Vertex* v = graph->vertices.array[i];
    v->pos.x = cos(angle*i) * radius;
    v->pos.y = sin(angle*i) * radius;
  }
}
\end{codeexample}

That is all that is needed; the C library will take care of both creating the
|graph| object as all well as of deleting it and of copying back the computed
values of the |pos| fields of the vertices.

Our next task is to setup the key |fast simple demo layout|. We can (and must)
also do this from C, using the following code:
%
\begin{codeexample}[code only, tikz syntax=false]
int luaopen_pgf_gd_examples_c_SimpleDemoC (struct lua_State *state) {

  pgfgd_Declaration* d = pgfgd_new_key ("fast simple demo layout");
  pgfgd_key_summary          (d, "The C version of the hello world of graph drawing");
  pgfgd_key_algorithm        (d, fast_hello_world);
  pgfgd_key_add_precondition (d, "connected");
  pgfgd_key_add_precondition (d, "tree");
  pgfgd_declare              (state, d)
  pgfgd_free_key             (d);
\end{codeexample}

The function |luaopen_pgf_gd_examples_c_SimpleDemoC| is the function that will
be called by Lua (we will come to that). More important for us, at the moment,
is the declaration of the key: We use |pgfgd_new_key| to create a declaration
record and then fill the different fields using appropriate function calls. In
particular, the call |pgfgd_key_algorithm| allows us to link the key with a
particular C function. The |pgfgd_declare| will then pass the whole declaration
back to Lua, so the effect of the above is essentially the same as if you had
written in Lua:
%
\begin{codeexample}[code only, tikz syntax=false]
declare {
  key = "fast simple demo layout",
  summary = "The C version of the hello world of graph drawing",
  preconditions = {
    connected = true,
    tree = true,
  },
  algorithm = {
    run =  -- something magic we cannot express in Lua
  }
}
\end{codeexample}

In our algorithm, in addition to the above key, we also use the
|fast simple demo radius| key, which is a simple length key. This key, too, can
be declared on the C layer:
%
\begin{codeexample}[code only, tikz syntax=false]
  d = pgfgd_new_key ("fast simple demo radius");
  pgfgd_key_summary (d, "A radius value for the hello world of graph drawing");
  pgfgd_key_type    (d, "length");
  pgfgd_key_initial (d, "1cm");
  pgfgd_declare     (state, d);
  pgfgd_free_key    (d);

  return 0;
}
\end{codeexample}

We simply add this code to the startup function above.

Now it is time to compile and link the code. For this, you must, well, compile
it, link it against the library |InterfaceFromC|, and build a shared library
out of it. Also, you must place it somewhere where Lua\TeX\ will find it. You
will find a Makefile that should be able to achieve all of this in the
directory |pgf/c/graphdrawing/pgf/gd/examples/c|, where you  will also find the
code of the above example.

Now, all you need to do to use it is to write in Lua (after you have loaded the
|pgf.gd| library, of course), would normally be the call
%
\begin{codeexample}[code only, tikz syntax=false]
require 'pgf.gd.examples.c.SimpleDemoC'
\end{codeexample}
%
or in \tikzname
%
\begin{codeexample}[code only]
\usegdlibrary {examples.c.SimpleDemoC}
\end{codeexample}

This should cause Lua\TeX\ to find the shared library, load it, and then call
the function in that library with the lengthy name (the name is always
|luaopen_| followed by the path and filename with slashes replaced by
underscores).

\emph{Remark:} Unfortunately, the above does not work with the \TeX Live 2013
versions of Lua\TeX\ due to a bugs that causes the ``replace dots by slashes''
to fail. For this reason, we currently need to rename our shared library file
to
%
\begin{codeexample}[code only, tikz syntax=false]
pgf_gd_examples_c_SimpleDemoC.so
\end{codeexample}
%
and then say
%
\begin{codeexample}[code only, tikz syntax=false]
require 'pgf_gd_examples_c_SimpleDemoC'
\end{codeexample}
%
or in \tikzname
%
\begin{codeexample}[code only]
\usegdlibrary {pgf_gd_examples_c_SimpleDemoC}
\end{codeexample}

In future versions of Lua\TeX, things should be ``back to normal'' in this
regard. Also, the bug only concerns shared libraries; you can still create a
normal Lua file with a nice name and place at a nice location and the only
contents of this file is then the above |require| command.

Anyway, once we have loaded the shared library we can say:
%
\begin{codeexample}[code only]
\tikz \graph [fast simple demo layout, fast simple demo radius=1.25cm]
{ a -> b -> c -> d -> e -> a };
\end{codeexample}


\subsubsection{Documenting Algorithms Written in C}
\label{section-gd-documenting-c-algos}

In our above example, we included a summary with the keys in the C code. It
would be even better if we added a longer documentation and some examples that
show how the key works; but this is a bit impracticable in C since multi-line
strings are hard to write down in~C. The trick is to use the |documentation_in|
field of a key: It allows us to specify the name of a Lua file that should be
loaded (using |require|) to install the missing documentation fields. As
explained in Section~\ref{section-gd-documentation-in}, this Lua file may make
good use the |pgf.gd.doc| package. Note, also, that for keys documented in this
way the documentation can easily be included in this manual through the use of
the |\includedocumentationof| command.

In our example, we would first add the following line twice in the C code (once
for each key), assuming that the documentation resides in the file
|pgf/gd/doc/examples/SimpleDemoC.lua|:
%
\begin{codeexample}[code only, tikz syntax=false]
  pgfgd_key_documentation_in (d, "pgf.gd.doc.examples.SimpleDemoC");
\end{codeexample}

Note that since the documentation is a normal Lua file, it will be searched in
the usual places Lua files are located (in the texmf trees) and not, like the C
shared library, in the special |lib| subdirectory of the Lua\TeX\ binary.

Here are typical contents of the documentation file:
%
\begin{codeexample}[code only, tikz syntax=false]
-- File pgf/gd/doc/examples/SimpleDemoC.lua
local key           = require 'pgf.gd.doc'.key
local documentation = require 'pgf.gd.doc'.documentation
local summary       = require 'pgf.gd.doc'.summary
local example       = require 'pgf.gd.doc'.example

key           "fast simple demo layout"
documentation
[[
This layout is used...
]]
example
[[
\tikz \graph [fast simple example layout]
{ a -- b -- c -- d -- e; };
]]

key           "fast simple demo radius"
documentation
[[
The radius parameter is used to ...
]]
example
[[
\tikz \graph [fast simple demo layout, fast simple demo radius=1.25cm]
{ a -> b -> c -> d -> e -> a };
]]
\end{codeexample}


\subsubsection{The Interface From C}

In the above example, we already saw some of the functions from the library
|InterfaceFromC| that translated from Lua to C for us. For a complete list of
all functions available, currently please see
|graphdrawing/c/pgf/gd/interface/c/InterfaceFromC.h| directly.

Currently, the library provides C functions to directly access all aspects of
the syntactic digraph and also of the graphs computed by the preprocessing of
the layout pipeline. What is missing, however, is access to the tree of
(sub)layouts and to collections. Hopefully, these will be added in the future.


\subsection{Writing Graph Drawing Algorithms in C++}
\label{section-gd-c++}

Built on top of the C interface presented in the previous section, there is
also a C++ interface available. It encapsulates as much of the C functions as
possible in C++ classes. Thus, this interface is mostly there for convenience,
it does not offer fundamentally new functionality.


\subsubsection{The Hello World of Graph Drawing in C++}

Let us have a look at how our beloved hello world of graph drawing looks in
C++. Although it is still possible to put graph drawing algorithms inside
functions, it is more natural in C++ to turn them into methods of a class.
Thus, we start the code of |SimpleDemoCPlusPlus.c++| as follows:
%
\begin{codeexample}[code only, tikz syntax=false]
#include <pgf/gd/interface/c/InterfaceFromC++.h>
#include <pgf/gd/interface/c/InterfaceFromC.h>

#include <math.h>

struct FastLayout : scripting::declarations, scripting::runner {
  ...
}
\end{codeexample}

As can be seen, we do not only include the interface from C++, but also that
from C (since, currently, not all functionality of the C library is
encapsulated in C++).

The interesting part is the |struct FastLayout|, which will contain our
algorithm (you could just as well have used a |class| instead of a |struct|).
It is derived from two classes: First, from a |declarations| class and,
secondly, from a |runner| class. Both of them, just like everything else from
the interface, reside in the namespace |scripting|. This name was chosen since
the main purpose of the interface is to provide ``scripting facilities'' to C
code through the use of Lua.

We are currently interested in the class |runner|. This class has a virtual
function |run| that gets called when, on the Lua side, someone has selected the
algorithm represented by the class. Thus, we place our algorithm in this
method:
%
\begin{codeexample}[code only, tikz syntax=false]
void run () {
  pgfgd_SyntacticDigraph* graph = parameters->syntactic_digraph;

  double angle  = 6.28318530718 / graph->vertices.length;
  double radius = parameters->option<double>("fast simple demo radius c++");

  for (int i = 0; i < graph->vertices.length; i++) {
    pgfgd_Vertex* v = graph->vertices.array[i];
    v->pos.x = cos(angle*i) * radius;
    v->pos.y = sin(angle*i) * radius;
  }
}
\end{codeexample}

The |run| method has access to the member variable |parameters|, which contains
all sorts of information concerning the to-be-drawn graph. In particular, the
|syntactic_digraph| field gives us access to the syntactic digraph structure
that was already available in the interface from plain~C. However, we can also
see that a template function like |option| allows us to access the graph's
option table in a simple way.

As for C code, our next task is to setup a key that, when used on the
\tikzname\ layer, will run our algorithm. For this, we can use an object
derived from a |declarations|. In our example, the |FastLayout| is both derived
from a |runner| (since it contains an algorithm) and also from |declarations|
(since it also contains the code necessary for declaring this algorithm). If
you prefer, you can split this into two classes. A |declarations| object must
override the |declare| method. This method gets a |script| object as input,
which is the ``representation'' of Lua inside the C++ code:
%
\begin{codeexample}[code only, tikz syntax=false]
void declare(scripting::script s) {
  using namespace scripting;

  s.declare(key ("fast simple demo layout c++")
            .summary ("The C++ version of the hello world of graph drawing")
            .precondition ("connected")
            .precondition ("tree")
            .algorithm (this));

  s.declare(key ("fast simple demo radius c++")
            .summary ("A radius value for the hello world of graph drawing")
            .type ("length")
            .initial ("1cm"));
}
\end{codeexample}

For each key that we wish to declare, we call the script's |declare| method
once. This method takes a |key| object as input, which can be configured
through a sequence of calls to different member functions (like |summary| or
|algorithm|). Most of these member functions are rather self-explaining; only
|algorithm| is a bit trickier: It does not take a function as input, but rather
an object of type |runner| and it will call the |run| method of this object
whenever the algorithm is run.

Lastly, we also need to write the entry point:
%
\begin{codeexample}[code only, tikz syntax=false]
extern "C" int luaopen_pgf_gd_examples_c_SimpleDemoCPlusPlus (struct lua_State *state) {
  scripting::script s (state);
  s.declare (new FastLayout);
  return 0;
}
\end{codeexample}

Note that it is the job of the interface classes to free the passed
|declarations| object. For this reason, you really need to call |new| and
cannot pass the address of a temporary object.

As before, because of the bug in some Lua\TeX\ versions, to actually load the
library at runtime, we need to rename it to
%
\begin{codeexample}[code only, tikz syntax=false]
pgf_gd_examples_c_SimpleDemoCPlusPlus.so
\end{codeexample}
%
and then say
%
\begin{codeexample}[code only, tikz syntax=false]
require 'pgf_gd_examples_c_SimpleDemoCPlusPlus'
\end{codeexample}
%
or in \tikzname
%
\begin{codeexample}[code only]
\usegdlibrary {pgf_gd_examples_c_SimpleDemoCPlusPlus}
\end{codeexample}

We can now use it:
%
\begin{codeexample}[code only]
\tikz \graph [fast simple demo layout c++, fast simple demo radius c++=1.25cm]
{ a -> b -> c -> d -> e -> a };
\end{codeexample}


\subsubsection{The Interface From C++}

The header |graphdrawing/c/pgf/gd/interface/c/InterfaceFromC++.h| contains, as
the name suggest, the interface from C++. A complete documentation is still
missing, but let us go over the main ideas:


\medskip
\noindent\textbf{Runners.}
Algorithms are represented by objects of type |runner|. An algorithm will
overwrite the |run| method, as we saw in the example, and it should modify the
|parameters| of the runner object.

In addition to the |run| method, there are also two more virtual methods,
called |bridge| and |unbrigde|. The first is called before the |run| method is
called and the second afterwards. The idea is that another framework, such as
\textsc{ogdf}, can implement a new class |ogdf_runner| that overrides these two
methods in order to transform the Lua/C representation of the input graph into
an \textsc{ogdf} representation prior to the |run| method being called. The
|run| method can then access additional member variables that store the graph
representations in \textsc{ogdf} form (or another form, depending on the
framework). The |unbridge| method allows the framework to translate back.

Although a |runner| object must be created for every algorithm, an algorithm
can also reside in a function. The class |function_runner| is a simple wrapper
that turns a function into such an object.


\medskip
\noindent\textbf{Keys.}
A key object is a temporary object that is passed to the |declare| method of a
script. It represents the table that is passed to the Lua function |declare|.
In order to make setting its field easy, for each field name there is a
corresponding function (like |summary|) that takes the string that should be
set to this field and returns the key object once more, so that we can chain
calls.

The |algorithm| method gets a runner object as parameter and will store a
pointer to this object inside Lua. Each time the algorithm is used, this object
will be used to ``run'' the algorithm, that is, the methods |prepare|,
|bridge|, |run|, and |unbridge| will be called in that order. Since the object
is reused each time, only one object is needed; but this object may not be
freed prematurely. Indeed, you will normally create the object using |new| once
and will then never delete it.

A typical idiom you may find in the code is
%
\begin{codeexample}[code only, tikz syntax=false]
s.declare (key (...)
           .algorithm(this)
           ...);
\end{codeexample}
%
This code is seen inside the |declare| method of objects that are both
declarations and runners. They register ``themselves'' via the above code.
Note, however, that this requires that the |this| pointer is not a temporary
object. (The typing rules of C++ make it hard for this situation to happen, but
it can be achieved.)


\medskip
\noindent\textbf{Reading options.} Once options have been declared, your C++
algorithms will wish to read them back. For this, the |parameters| field of a
runner object provides a number of templated methods:
%
\begin{itemize}
    \item The |option_is_set| method returns |true| if the passed option has
        been set \emph{and} can be cast to the type of the template. So,
        |option_is_set<double>("node distance")| will return true if the
        |node distance| key has been set for the graph as a whole (currently,
        there is no way to read the options of a vertex or an edge from C++,
        use the C methods instead).
    \item The |option| function comes in two flavours: First, it takes a single
        option name and just returns the option's value. If, however, the
        option has not been set or has another type, some sort of null value is
        returned. So, |option<double>("node distance")| will return the
        configured node distance as a double. When an option has an initial
        value, this call will always return a sensible value.

        The second flavour of |option| allows you to pass a reference to an
        object in which the option's value should be stored and the function
        will return true if the option is set (and, thus, something was written
        into the reference). This is the ``safest'' way to access an option:
        %
\begin{codeexample}[code only, tikz syntax=false]
double dist;
if (parameters->option ("node distance", dist))
  ...
\end{codeexample}

        Caution must be taken for |char*| options: The returned string must be
        explicitly freed; it will be a copy of the string stored in the Lua
        table.
    \item The |configure_option| method is used to set a member of an object
        based on the value of a certain option. For this, you must pass a
        pointer to a member function that sets the member. Here is an example:
        %
\begin{codeexample}[code only, tikz syntax=false]
class MyClass {
public:
  void setMyDistance (double distance);
...
};
...

MyClass m;
parameters->configure_option("node distance", &MyClass::setMyDistance, m);
\end{codeexample}
        %
        If the option has not been set or does not have the correct type, the
        member function is not called.
\end{itemize}


\medskip
\noindent\textbf{Factories and modules.}
A Lua key is normally either a Boolean, a double, or a string. However, in C++,
we may also sometimes wish Lua users to configure which C function is used to
achieve something. One could do this using strings or numbers and then use
search algorithms or a long |switch|, but this would neither be flexible nor
elegant.

Instead, it is possible to store \emph{factories} in Lua keys. A factory is a
class derived from |factory| that implements the virtual function |make|. This
function will return a new object of a template type. You can store such a
factory in a key.

The |make| method of a parameters object allows you to invoke the factory
stored in a key. (If no factory is stored in it, |null| is returned).

The |configure_module| method is akin to |configure_option|, only the result of
applying the factory is passed to the member function of the class.


\medskip
\noindent\textbf{Scripts.}
A ``script'' is the abstraction of the communication between Lua and C++. From
C++'s point of view, the script object offers different |declare| methods that
allow us to ``make objects and function scriptable'' in the sense that they can
then be called and configured from Lua. The script must be initialized with a
Lua state and will be bound to that state (basically, the script only stores
this single pointer).

When you call |declare|, you either pass a single key object (which is then
declared on the Lua layer) or you pass a |declarations| object, whose virtual
|declare| method is then called. The |declarations| objects are used to bundle
several declarations into a single one.


\subsection{Writing Graph Drawing Algorithms Using OGDF}
\label{section-gd-ogdf-interface}

Built on top of the C++ interface, a small interface allows you to easily link
algorithms written for the \textsc{ogdf} (Open Graph Drawing Framework) with
graph drawing in Lua.


\subsubsection{The Hello World of Graph Drawing in OGDF -- From Scratch}

We start with some startup code:
%
\begin{codeexample}[code only, tikz syntax=false]
#include <pgf/gd/ogdf/c/InterfaceFromOGDF.h>
#include <math.h>

using namespace ogdf;
using namespace scripting;
\end{codeexample}

Note that the interface from \textsc{ogdf} resides in the |ogdf| folder, not in
the |interface| folder.

Like in the plain C++ interface, we must now subclass the |runner| class and
the |declarations| class. Also like the plain C++ interface, we can use
multiple inheritance. The difference lies in the fact that we do not directly
subclass form |runner|, but rather from |ogdf_runner|. This class implements
the complicated ``bridging'' or ``translation'' process between the world of
|InterfaceFromC++| and \textsc{ogdf}:
%
\begin{codeexample}[code only, tikz syntax=false]
struct FastLayoutOGDF : declarations, ogdf_runner {

  void run () {
    double angle  = 6.28318530718 / graph.numberOfNodes();
    double radius = parameters->option<double>("my radius ogdf");

    int i = 0;
    for (node v = graph.firstNode(); v; v=v->succ(), i++) {
      graph_attributes.x(v) = cos(angle*i) * radius;
      graph_attributes.y(v) = sin(angle*i) * radius;
    }
  }
\end{codeexample}

As can be seen, in a subclass of |ogdf_runner|, the |run| method will have
access to a member called |graph| and to another member called
|graph_attributes|. These will have been setup with the graph from the Lua
layer and, after the algorithm has run, the information stored in the |x| and
|y| fields of the graph attributes and also the bend information of the edges
will be written back automatically.

Next, we need to declare the algorithm. This is done as in the plain
C++ interface:
%
\begin{codeexample}[code only, tikz syntax=false]
  void declare(script s) {
    using namespace scripting;

    s.declare(key ("fast simple demo layout ogdf")
          .summary ("The OGDF version of the hello world of graph drawing")
          .precondition ("connected")
          .algorithm (this));

    s.declare(key ("my radius ogdf")
          .summary ("A radius value for the hello world of graph drawing")
          .type ("length")
          .initial ("1cm"));
  }
};
\end{codeexample}

Finally, we need the entry point, which is also ``as usual'':
%
\begin{codeexample}[code only, tikz syntax=false]
extern "C" int luaopen_pgf_gd_examples_c_SimpleDemoOGDF (struct lua_State *state) {
  script (state).declare (new FastLayoutOGDF);
  return 0;
}
\end{codeexample}

Yet again, we need to rename the resulting shared library and then say
|require| on it. We can now use it:
%
\begin{codeexample}[code only]
\tikz \graph [fast simple demo layout ogdf, my radius ogdf=1cm]
{ a -> b -> c -> d -> e -> a };
\end{codeexample}


\subsubsection{The Hello World of Graph Drawing in OGDF -- Adapting Existing Classes}

In the previous example we implemented a graph drawing algorithm using
\textsc{ogdf} for use with Lua ``from scratch''. In particular, the whole
algorithm was contained in the |run| method of our main class. In practice,
however, graph drawing algorithms are typically placed in classes that ``know
nothing about scripting''. For instance, our hello world of graph drawing might
actually be implemented like this:
%
\begin{codeexample}[code only, tikz syntax=false]
// File HelloWorldLayout.h
#include <ogdf/module/LayoutModule.h>

class HelloWorldLayout : puplic ogdf::LayoutModule {
public:

  virtual void call(ogdf::GraphAttributes &GA)
  {
    using namespace ogdf;

    const Graph &graph = GA.constGraph();
    double angle  = 6.28318530718 / graph.numberOfNodes();
    int i = 0;
    for (node v = graph.firstNode(); v; v=v->succ(), i++) {
      GA.x(v) = cos(angle*i) * radius;
      GA.y(v) = sin(angle*i) * radius;
    }
  }

  void setRadius (double r) { radius = r; }

private:

  double radius;
};
\end{codeexample}

Now, what we actually want to do is to ``make this class scriptable''. For
this, we setup a new class whose |run| method will produce a new
|HelloWorldLayout|, configure it, and then run it. Here is this run method:
%
\begin{codeexample}[code only, tikz syntax=false]
void run ()
{
  HelloWorldLayout layout;
  parameters->configure_option("HelloWorldLayout.radius", &HelloWorldLayout::setRadius, layout);
  layout.call(graph_attributes);
}
\end{codeexample}

Next, we need to write the declarations code. This is very similar to the
``from scratch'' version:
%
\begin{codeexample}[code only, tikz syntax=false]
void declare(script s) {
  using namespace scripting;

  s.declare(key ("HelloWorldLayout")
            .summary ("The OGDF version of the hello world of graph drawing")
            .precondition ("connected")
            .algorithm (this));

  s.declare(key ("HelloWorldLayout.radius")
            .summary ("A radius value for the hello world of graph drawing")
            .type ("length")
            .alias ("radius"));
}
\end{codeexample}

Two remarks are in order: First, it is customary to name the keys for the
display system the same way as the classes. Second, the different configuration
options of the algorithm are named with the class name followed by the option
name. This makes it clear who, exactly, is being configured. However, these
keys should then also get an |alias| field set, which will cause an automatic
forwarding of the key to something more ``user friendly'' like just |radius|.

It remains to put the above methods in a ``script'' file. It is this file that,
when compiled, must be linked at runtime against Lua\TeX.
%
\begin{codeexample}[code only, tikz syntax=false]
// File HelloWorldLayout_script.c++

#include <pgf/gd/ogdf/c/InterfaceFromOGDF.h>
#include <HelloWorldLayout.h>

using namespace ogdf;
using namespace scripting;

struct HelloWorldLayout_script : declarations, ogdf_runner {
  void run ()             { ... see above ... }
  void declare (script s) { ... see above ... }
};

extern "C" int luaopen_my_path_HelloWorldLayout_script (struct lua_State *state) {
  script (state).declare (new HelloWorldLayout_script);
  return 0;
}
\end{codeexample}


\subsubsection{Documenting OGDF Algorithms}

As explained in Section~\ref{section-gd-documenting-c-algos}, we can add
external documentation to algorithms written in C and, using the
|documentation_in| method of the |key| class, we can use the exact same method
to document \textsc{ogdf} algorithms.

I strongly recommend making use of this feature since, currently, the
documentation of many \textsc{ogdf} classes is sketchy at best and using
\tikzname\ examples seems to be a good way of explaining the effect of the
different parameters algorithms offer.


\subsubsection{The Interface From OGDF}

The support for \textsc{ogdf} offered inside |InterfaceFromOGDF.h| is just the
class |ogdf_runner| we saw already in the example. In addition, there is also a
wrapper class |ogdf_function_runner| that allows you to wrap an algorithm
implemented in a function that uses \textsc{ogdf}, but I expect this to be the
case only rarely.
