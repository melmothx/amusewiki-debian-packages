%% $Id: pst-news.tex 444 2017-04-19 10:55:35Z herbert $
\documentclass[11pt,english,BCOR10mm,DIV12,bibliography=totoc,parskip=false,smallheadings
    headexclude,footexclude,oneside]{pst-doc}
\listfiles
\let\Lfile\LFile
\usepackage[utf8]{inputenc}
\usepackage{pstricks,pst-plot,xkvview}
\let\pstplotFV\fileversion
\let\pstplotFD\filedate
\usepackage{pst-eucl,pst-func}
\renewcommand\bgImage{\psscalebox{15}{\color{blue!20}2011}}
\def\textat{\char064}
\lstset{explpreset={pos=l,width=-99pt,overhang=0pt,hsep=\columnsep,vsep=\bigskipamount,rframe={}},
    escapechar=?}
\begin{document}

%\psset{PstDebug=1}
\title{\texttt{News}\\ \Large new macros and bugfixes for the
basic package \nxLFile{pstricks}}
\subtitle{Summary of the yearly posted news}
\author{Herbert Voß}
\date{\today}

\maketitle

\clearpage
\tableofcontents

\clearpage
\part{\texttt{pstricks} -- package}

\section{General}
There exists a new document class \LClass{pst-doc} for writing PSTricks documentations,
like this news document. It depends on the KOMA-Script document class \LClass{scrartcl}.
\LClass{pst-doc} defines a lot of special macros to create a good index. Take one of
the already existing package documentation and look into the source file. Then it will be
easy to understand, how all these macros have to be used.

When running \Lprog{pdflatex} the title page is created with boxes and inserted 
with the macro \Lcs{AddToShipoutPicture} from the package \LPack{eso-pic}. It
inserts the background title page image \Lfile{pst-doc-pdf} to use directly
\Lprog{pdflatex}.
When running \Lprog{latex} the title page
 is created with \PST\ macros.This allows to use the Perl script \Lprog{pst2pdf} or
the package \LPack{pst-pdf} or \LPack{auto-pst-pdf} or any other program/package which
supports \PS\ code in the document.


%--------------------------------------------------------------------------------------
\section{\texttt{pstricks.sty}}
%--------------------------------------------------------------------------------------
\subsection{New optional argument}

\begin{description}
\item[\texttt{noxcolor}] load package \LPack{color} instead of  \LPack{xcolor};
\item[\texttt{plain}] do nothing else as a \Lcs{input}\Largb{\nxLPack{pstricks}}; 
\item[\texttt{DIA}] a bug fix for the \verb+PSTricks+-export of the grafic program DIA.
\end{description}


%--------------------------------------------------------------------------------------
\section{\texttt{pstricks.tex} (\pstricksFV -- \pstricksFD)}
%--------------------------------------------------------------------------------------

\subsection{New and modified option for {pspicture}}\label{sec:option}

Table~\ref{tab:pspicture} shows the two new options for the  \verb+pspicture+ environment.

\begin{table}[htb]
\caption{Optionen der \texttt{pspicture}-Umgebung}\label{tab:pspicture}
\centering
\begin{tabular}{@{}lll@{}}
\textrm{\emph{name}}  & \emph{meaning}   & \emph{default}\\\hline
\Lkeyword{shift}  & vertical shift & 0 \\
\Lkeyword{showgrid} & show grid & \verb+false+\\
\end{tabular}
\end{table}


% ---------------------------------------------------------------------------------------
\subsubsection{\nxLkeyword{shift}}\label{subsubsec:shift}
% ---------------------------------------------------------------------------------------
This option is the known one from older \texttt{PSTricks} versions, but now with the
common syntax for options. The shift is relative to the height of the defined \Lenv{pspicture}
environment, its lower left corner is by deafult on the base line. For older versions
the shift depends with its value to the baseline, a negative value raised up the \Lenv{pspicture}
box. Now the \verb+shift+ option works similiar to the known \Lcs{raisebox} makro, except that
\Lkeyword{shift} is relative to the box height. A positive \Lkeyword{shift} value raises up the box
and vice versa for a negative value.

\begin{figure}[htb]
\centering
\textcolor{red}{\rule{5mm}{1pt}}%
\begin{pspicture}[shift=0.5](-0.5,-0.5)(0.5,0.5)
	\psframe[linecolor=blue](-0.5,-0.5)(0.5,0.5)\rput(0,0){-0.5}
\end{pspicture}%
\textcolor{red}{\rule{5mm}{1pt}}
\hspace{1cm}%
\textcolor{red}{\rule{5mm}{1pt}}%
\begin{pspicture}(-0.5,-0.5)(0.5,0.5)
	\psframe[linecolor=blue](-0.5,-0.5)(0.5,0.5)\rput(0,0){0}
\end{pspicture}\textcolor{red}{\rule{5mm}{1pt}}
\hspace{1cm}%
\textcolor{red}{\rule{5mm}{1pt}}%
\begin{pspicture}[shift=-0.5](-0.5,-0.5)(0.5,0.5)
	\psframe[linecolor=blue](-0.5,-0.5)(0.5,0.5)\rput(0,0){0.5}
\end{pspicture}%
\textcolor{red}{\rule{5mm}{1pt}}
\caption{Meaning of the \texttt{shift} option}\label{fig:baseline}
\end{figure}

%\begin{lstlisting}[caption={Vertikale Verschiebung der Baseline}]
\begin{lstlisting}
\textcolor{red}{\rule{5mm}{1pt}}%
\begin{pspicture}[shift=0.5](-0.5,-0.5)(0.5,0.5)
	\psframe[linecolor=blue](-0.5,-0.5)(0.5,0.5)\rput(0,0){-0.5}
\end{pspicture}%
\textcolor{red}{\rule{5mm}{1pt}}
\hspace{1cm}%
\textcolor{red}{\rule{5mm}{1pt}}%
\begin{pspicture}(-0.5,-0.5)(0.5,0.5)
	\psframe[linecolor=blue](-0.5,-0.5)(0.5,0.5)\rput(0,0){0}
\end{pspicture}\textcolor{red}{\rule{5mm}{1pt}}
\hspace{1cm}%
\textcolor{red}{\rule{5mm}{1pt}}%
\begin{pspicture}[shift=-0.5](-0.5,-0.5)(0.5,0.5)
	\psframe[linecolor=blue](-0.5,-0.5)(0.5,0.5)\rput(0,0){0.5}
\end{pspicture}%
\textcolor{red}{\rule{5mm}{1pt}}
\end{lstlisting}

With \Lkeyword{shift}=\Lkeyval{*}, instead of a value or a length
it is possible to center the \Lenv{pspicture} box vertically to the baseline
of the current line.

\begin{LTXexample}[width=4cm]
\usepackage{pstricks}
\rule{5mm}{0.5pt}%
\psframebox{%
\begin{pspicture}[showgrid=true,
    shift=*](-0.3,-0.4)(3.2,3.3)
  \psarc[showpoints=true](1,1){2}{-45}{120} 
\end{pspicture}}\rule{5mm}{0.5pt}
\end{LTXexample}


% ---------------------------------------------------------------------------------------
\subsubsection{\texttt{showgrid}}\label{subsubsec:showgrid}
% ---------------------------------------------------------------------------------------
This version of \texttt{PSTricks} defines internally a special grid style

\begin{lstlisting}
\newpsstyle{gridstyle}{subgriddiv=0,gridcolor=lightgray,griddots=10,gridlabels=8pt}
\end{lstlisting}

which can be overwritten by the user. This style is only used for the \Lkeyword{showgrid}
option of the \Lenvpspicture} environment. The macro \Lcs{psgrid} doesn't use this predefined
style and works in the usual way. However, the user can use it like all
other self defined styles: \Lcs{psgrid}\Largs{\Lkeyset{style=gridstyle}}.


\begin{LTXexample}[width=3.5cm]
\begin{pspicture}[showgrid=true](-1,0)(2,1)
\end{pspicture}
\end{LTXexample}

\begin{LTXexample}[width=3.5cm]
\newpsstyle{gridstyle}{%
  subgriddiv=2,subgridcolor=lightgray}
\begin{pspicture}[showgrid=true](-1,0)(2,1)
\end{pspicture}
\end{LTXexample}

\begin{LTXexample}[width=3.5cm]
\newpsstyle{gridstyle}{}
\begin{pspicture}[showgrid=true](-1,0)(2,1)
\end{pspicture}
\end{LTXexample}

\begin{LTXexample}[width=3.5cm]
\begin{pspicture}(-1,0)(2,1)
\end{pspicture}
\end{LTXexample}

\begin{LTXexample}[width=3.5cm]
\begin{pspicture}(-1,0)(2,1)
  \psgrid
\end{pspicture}
\end{LTXexample}

Depending to the internal structure of the \Lenv{pspicture} environment it is not possible
to set the \Lkeyword{shift} option global by \Lcs{psset}, it must always be locally defined
with optional part of the parameter, as seen in the above examples.

\subsection{Option \nxLkeyword{gridfont}}
By default the \Index{grid label}s were printed always in \Index{Helvetica}. With the new keyword \Lkeyword{gridfont}
one can define another \Index{PostScript Font}. Available are at least

\medskip
{\ttfamily\noindent
\Lkeyval{Helvetica} (default) -- \Lkeyval{Helvetica-Narrow} -- \Lkeyval{Times-Roman} -- \Lkeyval{Courier} -- \Lkeyval{AvantGard} --\Lkeyval{NewCenturySchlbk} -- 
\Lkeyval{Palatino-Roman} -- \Lkeyval{Bookman-Demi} -- \linebreak \Lkeyval{ZapfDingbats} -- \Lkeyval{Symbol}}

\begin{LTXexample}[width=4cm]
\usepackage{pstricks}
\begin{pspicture}[showgrid=true](3,2)
\end{pspicture}\\[20pt]
\begin{pspicture}(3,2)
  \psgrid[style=gridstyle,gridfont=AvantGard-Demi]
\end{pspicture}\\[20pt]
\begin{pspicture}(3,2)
  \psgrid[style=gridstyle,gridfont=ZapfDingbats]
\end{pspicture}
\end{LTXexample}




\subsection{Macro \nxLcs{psLoop}}
\PST\marginpar[2.17]{2.17} already knows \Lcs{psforeach} and \Lcs{psForeach} for loops. The new
macro \Lcs{psLoop} allows a loop without defining a variable:

\begin{BDef}
\Lcs{psLoop}\Largb{n}\Largb{argument}
\end{BDef}

However, the internal \TeX\ counter \Lctr{psLoopIndex} can be used for own purposes.

\begin{LTXexample}[width=7cm]
  \psLoop{4}{PSTricks }
\end{LTXexample}

\begin{LTXexample}[width=7cm]
\tabular{|c|c|c|c|}
  \psLoop{3}{PSTricks &}\\\hline
  A & B & C & D\\\hline
\endtabular
\end{LTXexample}

\begin{LTXexample}[width=6cm]
\begin{pspicture}[showgrid](3,3)
\psLoop{4}{%
  \psdots(\the\psLoopIndex,\the\psLoopIndex)}
\end{pspicture}
\end{LTXexample}


%--------------------------------------------------------------------------------------
\section{The PostScript header files}
\subsection{\nxLFile{pstricks.pro}}
%--------------------------------------------------------------------------------------


%--------------------------------------------------------------------------------------
\subsection{\nxLFile{pst-algparser.pro}}
%--------------------------------------------------------------------------------------

\section{\nxLcs{psforeach} and \nxLcs{psForeach}}
%--------------------------------------------------------------------------------------

\section{List of all optional arguments for \texttt{pstricks}}

\xkvview{family=pstricks,columns={key,type,default}}



\nocite{*}
\bibliographystyle{plain}
\bibliography{PSTricks}

\printindex


\end{document}
