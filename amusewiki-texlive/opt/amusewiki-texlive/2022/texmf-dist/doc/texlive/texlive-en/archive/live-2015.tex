% $Id: texlive-en.tex 37205 2015-05-05 21:36:33Z karl $
% TeX Live documentation.  Originally written by Sebastian Rahtz and
% Michel Goossens, now maintained by Karl Berry and others.
% Public domain.
%
\documentclass{article}
\let\tldocenglish=1  % for live4ht.cfg
\usepackage{tex-live}
\usepackage[latin1]{inputenc} % translators: use your preferred encodings.
\usepackage[T1]{fontenc}

\begin{document}

\title{%
  {\huge \textit{The \TeX\ Live Guide---2015}}
}

\author{Karl Berry, editor \\[3mm]
        \url{http://tug.org/texlive/}
       }

\date{May 2015}

\maketitle

\begin{multicols}{2}
\tableofcontents
%\listoftables
\end{multicols}

\section{Introduction}
\label{sec:intro}

\subsection{\TeX\ Live and the \TeX\ Collection}

This document describes the main features of the \TL{} software
distribution\Dash \TeX{} and related programs for \GNU/Linux
and other Unix flavors, \MacOSX, and Windows systems.

You may have acquired \TL{} by downloading, or on the \TK{} \DVD, which
\TeX{} user groups distribute among their members, or in other ways.
Section \ref{sec:tl-coll-dists} briefly describes the contents of the
\DVD.  Both \TL{} and the \TK{} are cooperative efforts by the \TeX{}
user groups. This document mainly describes \TL{} itself.

\TL{} includes executables for \TeX{}, \LaTeXe{}, \ConTeXt,
\MF, \MP, \BibTeX{} and many other programs; an extensive collection
of macros, fonts and documentation; and support for typesetting in
many different scripts from around the world.

For a brief summary of the major changes in this edition of \TL{},
see the end of the document, section~\ref{sec:history}
(\p.\pageref{sec:history}).


\htmlanchor{platforms}
\subsection{Operating system support}
\label{sec:os-support}

\TL{} contains binaries for many Unix-based platforms, including
\GNU/Linux, \MacOSX, and Cygwin.  The included sources can be compiled
on platforms for which we do not provide binaries.

As to Windows: Windows \acro{XP} and later are supported.  Windows~2000
will probably still mostly work.  There are no special 64-bit
executables for Windows, but the 32-bit executables should run on 64-bit
systems.

See section~\ref{sec:tl-coll-dists} for alternate solutions
for Windows and \MacOSX.

\subsection{Basic installation of \protect\TL{}}
\label{sec:basic}

You can install \TL{} either from \DVD{} or over the Internet
(\url{http://tug.org/texlive/acquire.html}). The net installer itself is
small, and downloads everything requested from the Internet.

The \DVD{} installer lets you install to a local disk.  You cannot run
\TL{} directly from the \TK{} \DVD{} (or its \code{.iso} image), but you
can prepare a runnable installation on, e.g., a \USB{} stick (see
section~\ref{sec:portable-tl}).  Installation is described in later
sections (\p.\pageref{sec:install}), but here is a quick start:

\begin{itemize*}

\item The installation script is named \filename{install-tl}.  It can
  operate in a ``wizard mode'' given the option \code{-gui=wizard}
  (default for Windows), a text mode given \code{-gui=text} (default for
  everything else), and an expert \GUI{} mode given \code{-gui=perltk}.

\item One of the installed items is the `\TL\ Manager' program,
  named \prog{tlmgr}.  Like the installer, it can be used in both \GUI{}
  mode and in text mode. You can use it to install and uninstall
  packages and do various configuration tasks.

\end{itemize*}


\htmlanchor{security}
\subsection{Security considerations}
\label{sec:security}

To the best of our knowledge, the core \TeX\ programs themselves are
(and always have been) extremely robust.  However, the contributed
programs in \TeX\ Live may not reach the same level, despite everyone's
best efforts.  As always, you should be careful when running programs on
untrusted input; for maximum safety, use a new subdirectory.

This need for care is especially urgent on Windows, since in general
Windows finds programs in the current directory before anything else,
regardless of the search path.  This opens up a wide variety of possible
attacks.  We have closed many holes, but undoubtedly some remain,
especially with third-party programs.  Thus, we recommend checking for
suspicious files in the current directory, especially executables
(binaries or scripts).  Ordinarily they should not be present, and
definitely should not normally be created by merely processing a document.

Finally, \TeX\ (and its companion programs) are able to write files when
processing documents, a feature that can also be abused in a wide
variety of ways.  Again, processing unknown documents in a new
subdirectory is the safest bet.


\subsection{Getting help}
\label{sec:help}

The \TeX{} community is active and friendly, and most serious questions
end up getting answered.  However, the support is informal, done by
volunteers and casual readers, so it's especially important that you do
your homework before asking.  (If you prefer guaranteed commercial
support, you can forgo \TL{} completely and purchase a vendor's system;
\url{http://tug.org/interest.html#vendors} has a list.)

Here is a list of resources, approximately in the order we recommend
using them:

\begin{description}
\item [Getting Started] If you are new to \TeX, the web page
\url{http://tug.org/begin.html} gives a brief introduction to the system.

\item [\TeX{} \acro{FAQ}] The \TeX{} \acro{FAQ} is a huge compendium
  of answers to all sorts of questions, from the most basic to the
  most arcane.  It is included on \TL{} in
  \OnCD{texmf-dist/doc/generic/FAQ-en/}, and is available
  on the Internet through \url{http://www.tex.ac.uk/faq}.  Please
  check here first.

\item [\TeX{} Catalogue] If you are looking for a specific package,
font, program, etc., the \TeX{} Catalogue is the place to look.  It is a
huge collection of all \TeX{}-related items.  See
\url{http://mirror.ctan.org/help/Catalogue/}.

\item [\TeX{} Web Resources] The web page
\url{http://tug.org/interest.html} has many \TeX{}-related links, in
particular for numerous books, manuals, and articles on all aspects of
the system.

\item [support archives] Principal support forums include the
Usenet newsgroup \url{news:comp.text.tex}, the mailing list
\email{texhax@tug.org}, and \url{http://tex.stackexchange.com}, a
collaboratively edited question and answer site.  Their archives have
years of past questions and answers for your searching pleasure, via
\url{http://groups.google.com/group/comp.text.tex/topics} and
\url{http://tug.org/mail-archives/texhax}, respectively.  And a general
web search, for example on \url{http://google.com}, never hurts.

\item [asking questions] If you cannot find an answer, you can post to
\dirname{comp.text.tex} through Google or your newsreader, or to
\email{texhax@tug.org} through email.  But before you post,
\emph{please} read this \acro{FAQ} entry, to maximize
your chances of getting a useful answer:
\url{http://www.tex.ac.uk/cgi-bin/texfaq2html?label=askquestion}.
Also worth mentioning are the \LaTeX{} Community site at
\url{http://www.latex-community.org/} and its forum
\url{http://www.latex-community.org/forum/}, and 
\TeX\ StackExchange at \url{http://tex.stackexchange.com/}.

\item [\TL{} support] If you want to report a bug or have
suggestions or comments on the \TL{} distribution, installation, or
documentation, the mailing list is \email{tex-live@tug.org}.  However,
if your question is about how to use a particular program included in
\TL{}, please write to that program's maintainer or
mailing list.  Often running a program with the \code{-{}-help} option
will provide a bug reporting address.

\end{description}

The other side of the coin is helping others who have questions.  Both
\dirname{comp.text.tex} and \code{texhax} are open to anyone, so feel
free to join, start reading, and help out where you can.


% don't use \TL so the \uppercase in the headline works.  Also so
% tex4ht ends up with the right TeX.  Likewise the \protect's.
\section{Overview of \protect\TeX\protect\ Live}
\label{sec:overview-tl}

This section describes the contents of \TL{} and the \TK{} of which it
is a part.

\subsection{The \protect\TeX\protect\ Collection: \TL, pro\TeX{}t, Mac\TeX}
\label{sec:tl-coll-dists}

The \TK{} \DVD{} comprises the following:

\begin{description}

\item [\TL] A complete \TeX{} system to be installed to disk.  Home
page: \url{http://tug.org/texlive/}.

\item [Mac\TeX] for \MacOSX, this adds a native \MacOSX\ installer and other
Mac applications to \TL{}.  Home page:
\url{http://tug.org/mactex/}.

\item [pro\TeX{}t] An enhancement of the \MIKTEX\ distribution for Windows,
\ProTeXt\ adds a few extra tools to \MIKTEX, and simplifies
installation.  It is entirely independent of \TL{}, and has its own
installation instructions.  Home page:
\url{http://tug.org/protext/}.

\item [CTAN] A snapshot of the \CTAN{} repository (\url{http://www.ctan.org/}).

\end{description}

\CTAN{} and \pkgname{protext} do not follow the same copying conditions
as \TL{}, so be careful when redistributing or modifying.


\subsection{Top level \TL{} directories}
\label{sec:tld}

Here is a brief listing and description of the top level directories in a
\TL{} installation.  

\begin{ttdescription}
\item[bin] The \TeX{} system programs, arranged by platform.
%
\item[readme-*.dir] Quick overview and useful links for \TL{},
in various languages, in both \HTML{} and plain text.
%
\item[source] The source to all included programs, including the main
  \Webc{}-based \TeX{} distributions.
%
\item[texmf-dist] The principal tree; see \dirname{TEXMFDIST} below.
%
\item[tlpkg] Scripts, programs and data for managing the
  installation, and special support for Windows.
\end{ttdescription}

In addition to the directories above, the installation scripts and
\filename{README} files (in various languages) are at the top level of
the distribution.

For documentation, the comprehensive links in the top-level file
\OnCD{doc.html} may be helpful.  The documentation for nearly everything
(packages, formats, fonts, program manuals, man pages, Info files) is in
\dirname{texmf-dist/doc}.  You can use the \cmdname{texdoc} program to
find documentation wherever it is located.

This \TL\ documentation itself is in \dirname{texmf-dist/doc/texlive},
available in several languages:

\begin{itemize*}
\item{Czech/Slovak:} \OnCD{texmf-dist/doc/texlive/texlive-cz}
\item{German:} \OnCD{texmf-dist/doc/texlive/texlive-de}
\item{English:} \OnCD{texmf-dist/doc/texlive/texlive-en}
\item{French:} \OnCD{texmf-dist/doc/texlive/texlive-fr}
\item{Italian:} \OnCD{texmf-dist/doc/texlive/texlive-it}
\item{Polish:} \OnCD{texmf-dist/doc/texlive/texlive-pl}
\item{Russian:} \OnCD{texmf-dist/doc/texlive/texlive-ru}
\item{Serbian:} \OnCD{texmf-dist/doc/texlive/texlive-sr}
\item{Simplified Chinese:} \OnCD{texmf-dist/doc/texlive/texlive-zh-cn}
\end{itemize*}

\subsection{Overview of the predefined texmf trees}
\label{sec:texmftrees}

This section lists the predefined variables specifying the texmf trees
used by the system, and their intended purpose, and the default layout
of \TL{}. The command \texttt{tlmgr~conf} shows the values of these
variables, so that you can easily find out how they map to particular
directories in your installation.

All of the trees, including the personal ones, should follow the \TeX\
Directory Structure (\TDS, \url{http://tug.org/tds}), with all its
myriad subdirectories, or files may not be found.  Section
\ref{sec:local-personal-macros} (\p.\pageref{sec:local-personal-macros})
describes this in more detail.

\begin{ttdescription}
\item [TEXMFDIST] The tree which holds nearly all of the files in the original
  distribution---configuration files, scripts, packages, fonts, etc.
  (The main exception are the per-platform executables, which are stored
  in a sibling directory \code{bin/}.)
\item [TEXMFLOCAL] The tree which administrators can use for system-wide
  installation of additional or updated macros, fonts, etc.
\item [TEXMFHOME] The tree which users can use for their own individual
  installations of additional or updated macros, fonts, etc.
  The expansion of this variable dynamically adjusts for each user to
  their own individual directory.
\item [TEXMFCONFIG] The (personal) tree used by the utilities
  \verb+texconfig+, \verb+updmap+, and \verb+fmtutil+ to store modified
  configuration data.
\item [TEXMFSYSCONFIG] The (site-wide) tree used by the utilities
  \verb+texconfig-sys+, \verb+updmap-sys+, and \verb+fmtutil-sys+ to
  store modified configuration data.
\item [TEXMFVAR] The (personal) tree used by \verb+texconfig+,
  \verb+updmap+ and \verb+fmtutil+ to store (cached) runtime data such
  as format files and generated map files.
\item [TEXMFSYSVAR] The (site-wide) tree used by \verb+texconfig-sys+,
  \verb+updmap-sys+ and \verb+fmtutil-sys+, and also by \verb+tlmgr+, to
  store (cached) runtime data such as format files and generated map files.
\item [TEXMFCACHE] The tree(s) used by \ConTeXt\ Mk\acro{IV} and Lua\LaTeX\
  to store (cached) runtime data; defaults to \code{TEXMFSYSVAR},
  or (if that's not writable), \code{TEXMFVAR}.
\end{ttdescription}

\noindent
The default layout is:
\begin{description}
  \item[system-wide root] can span multiple \TL{} releases:
  \begin{ttdescription}
    \item[2014] A previous release.
    \item[2015] The current release.
    \begin{ttdescription}
      \item [bin] ~
      \begin{ttdescription}
        \item [i386-linux] \GNU/Linux binaries
        \item [...]
        \item [universal-darwin] \MacOSX\ binaries
        \item [win32] Windows binaries
      \end{ttdescription}
      \item [texmf-dist\ \ ]      \envname{TEXMFDIST} and \envname{TEXMFMAIN}
      \item [texmf-var \ \ ]      \envname{TEXMFSYSVAR}, \envname{TEXMFCACHE}
      \item [texmf-config]        \envname{TEXMFSYSCONFIG}
    \end{ttdescription}
    \item [texmf-local] \envname{TEXMFLOCAL}, intended to be
      retained from release to release.
  \end{ttdescription}
  \item[user's home directory] (\texttt{\$HOME} or
      \texttt{\%USERPROFILE\%})
    \begin{ttdescription}
      \item[.texlive2014] Privately generated and configuration data
        for a previous release.
      \item[.texlive2015] Privately generated and configuration data
        for the current release.
      \begin{ttdescription}
        \item [texmf-var\ \ \ ] \envname{TEXMFVAR}, \envname{TEXMFCACHE}
        \item [texmf-config]    \envname{TEXMFCONFIG}
      \end{ttdescription}
    \item[texmf] \envname{TEXMFHOME} Personal macros, etc.
  \end{ttdescription}
\end{description}


\subsection{Extensions to \protect\TeX}
\label{sec:tex-extensions}

Knuth's original \TeX{} itself is frozen, apart from rare bug fixes. It
is present in \TL\ as the program \prog{tex}, and will remain so
for the foreseeable future.  \TL{} also contains several extended versions of
\TeX (also known as \TeX\ engines):

\begin{description}

\item [\eTeX] adds a set of new primitives
\label{text:etex} (related to macro expansion, character scanning,
classes of marks, additional debugging features, and more) and the
\TeXXeT{} extensions for bidirectional typesetting.  In default mode,
\eTeX{} is 100\% compatible with ordinary \TeX. See
\OnCD{texmf-dist/doc/etex/base/etex_man.pdf}.

\item [pdf\TeX] builds on the \eTeX\ extensions, adding support for
writing \acro{PDF} output as well as \dvi{}, and many non-output-related
extensions.  This is the program invoked for most formats, e.g.,
\prog{etex}, \prog{latex}, \prog{pdflatex}.  Its web site is
\url{http://www.pdftex.org/}.  See
\OnCD{texmf-dist/doc/pdftex/manual/pdftex-a.pdf} for the manual, and
\OnCD{texmf-dist/doc/pdftex/manual/samplepdf/samplepdf.tex} for example
usage of some of its features.

\item [Lua\TeX] is the designated successor of pdf\TeX,
and is mostly (but not entirely) backward-compatible.  It is also
intended to be a functional superset of Aleph (see below), though
technical compatibility is not intended. The incorporated Lua
interpreter (\url{http://www.lua.org/}) enables elegant solutions for
many thorny \TeX{} problems. When called as \filename{texlua}, it
functions as a standalone Lua interpreter, and is already used as such
within \TL.  Its web site is \url{http://www.luatex.org/}, and the
reference manual is \OnCD{texmf-dist/doc/luatex/base/luatexref-t.pdf}.

\item [\XeTeX] adds support for Unicode input and OpenType- and system
fonts, implemented using standard third-party libraries.  See
\url{http://tug.org/xetex}.

\item [\OMEGA\ (Omega)] is based on Unicode (16-bit characters), thus
supports working with almost all the world's scripts simultaneously. It
also supports so-called `\OMEGA{} Translation Processes' (\acro{OTP}s),
for performing complex transformations on arbitrary input.  Omega is no
longer included in \TL{} as a separate program; only Aleph is provided:

\item [Aleph] combines the \OMEGA\ and \eTeX\ extensions.
See \OnCD{texmf-dist/doc/aleph/base}.

\end{description}


\subsection{Other notable programs in \protect\TL}

Here are a few other commonly-used programs included in \TL{}:

\begin{cmddescription}

\item [bibtex] bibliography support.

\item [makeindex, xindy] index support.

\item [dvips] convert \dvi{} to \PS{}.

\item [xdvi] \dvi{} previewer for the X Window System.

\item [dviconcat, dviselect] cut and paste pages
from \dvi{} files.

\item [dvipdfmx] convert \dvi{} to \acro{PDF}, an alternative approach
to pdf\TeX\ (mentioned above).

\item [psselect, psnup, \ldots] \PS{} utilities.

\item [pdfjam, pdfjoin, \ldots] \acro{PDF} utilities.

\item [context, mtxrun] Con\TeX{}t and \acro{PDF} processor.

\item [htlatex, \ldots] \cmdname{tex4ht}: \AllTeX{} to \acro{HTML} (and
\acro{XML} and more) converter.

\end{cmddescription}


\htmlanchor{installation}
\section{Installation}
\label{sec:install}

\subsection{Starting the installer}
\label{sec:inst-start}

To begin, get the \TK{} \DVD{} or download the \TL{} net installer.  See
\url{http://tug.org/texlive/acquire.html} for more information and other
methods of getting the software.

\begin{description}
\item [Net installer, .zip or .tar.gz:] Download from \CTAN, under
\dirname{systems/texlive/tlnet}; the url
\url{http://mirror.ctan.org/systems/texlive/tlnet} should redirect to a
nearby, up-to-date, mirror.  You can retrieve either
\filename{install-tl.zip} which can be used under Unix and Windows, or
the considerably smaller \filename{install-unx.tar.gz} for Unix
only. After unpacking, \filename{install-tl} and
\filename{install-tl-windows.bat} will be in the \dirname{install-tl}
subdirectory.

\item[Net installer, Windows .exe:] Download from \CTAN{} as above,
and double-click. This starts up a first-stage installer and unpacker;
see Figure~\ref{fig:nsis}. It gives three choices: ``Simple install''
starts up the wizard installer, ``Custom install'' the expert \GUI{}
installer, as described in section~\ref{sec:wininst}. A third option is
just unpacking.

\item [\TeX{} Collection \DVD:] go to the \DVD's \dirname{texlive}
subdirectory. Under Windows, the installer normally starts automatically
when you insert the \DVD.  You can get the \DVD\ by becoming a member of
a \TeX\ user group (highly recommended,
\url{http://tug.org/usergroups.html}), or purchasing it separately
(\url{http://tug.org/store}), or burning your own from the \ISO\ image.
You can also mount the \ISO\ directly on most systems.  After installing
from \DVD\ or \ISO, if you want to get continuing updates from the
Internet, please see \ref{sec:dvd-install-net-updates}.

\end{description}

\begin{figure}[tb]
\tlpng{nsis_installer}{.6\linewidth}
\caption{First stage of Windows .exe installer}\label{fig:nsis}
\end{figure}

The same installer program is run, whatever the source.  The most
visible difference between the two is that with the net installer, what
you end up with is the packages that are currently available.  This is
in contrast to the \DVD\ and \ISO\ images, which are not updated between
the major public releases.

If you need to download through proxies, use a \filename{~/.wgetrc} file
or environment variables with the proxy settings for Wget
(\url{http://www.gnu.org/software/wget/manual/html_node/Proxies.html}).
\TL{} always uses \GNU\ Wget to download.  Of course, this should not
matter if you are installing from the \DVD\ or \ISO\ image.

\noindent
The following sections explain installer start-up in more detail.

\subsubsection{Unix}

Below, \texttt{>} denotes the shell prompt; user input is
\Ucom{\texttt{bold}}.
The script \filename{install-tl} is a Perl script.  The simplest way
to start it on a Unix-compatible system is as follows:
\begin{alltt}
> \Ucom{perl /path/to/installer/install-tl}
\end{alltt}
(Or you can invoke \Ucom{/path/to/installer/install-tl} if it stayed
executable, or \texttt{cd} to the directory first, etc.; we won't repeat
all the variations.)  You may have to enlarge your terminal window so
that it shows the full text installer screen
(Figure~\ref{fig:text-main}).

To install in expert \GUI\ mode (figure~\ref{fig:gui-main}), you'll
need the \dirname{Perl::TK} module compiled with \acro{XFT} support,
which is generally the case with \GNU/Linux, but often not so with other
systems.  Given that, you can run:
\begin{alltt}
> \Ucom{perl install-tl -gui}
\end{alltt}

For a complete listing of the various options:
\begin{alltt}
> \Ucom{perl install-tl -help}
\end{alltt}

\textbf{Warning about Unix permissions:} Your \code{umask} at the time
of installation will be respected by the \TL{} installer.  Therefore, if
you want your installation to be usable by users other than you, make
sure your setting is sufficiently permissive, for instance, \code{umask
002}.  For more information about \code{umask}, consult your system
documentation.

\textbf{Special considerations for Cygwin:} Unlike other
Unix-compatible systems, Cygwin does not by default include all of the
prerequisite programs needed by the \TL{} installer.  See
Section~\ref{sec:cygwin}.


\subsubsection{\MacOSX}
\label{sec:macosx}

As mentioned in section~\ref{sec:tl-coll-dists}, a separate distribution
is prepared for \MacOSX, named Mac\TeX\ (\url{http://tug.org/mactex}).
We recommend using the native Mac\TeX\ installer instead of the \TL\
installer on \MacOSX, because the native installer makes a few
Mac-specific adjustments, in particular to allow easily switching
between the various \TeX\ distributions for \MacOSX\ (Mac\TeX, Fink,
MacPorts, \ldots) using the so-called \TeX{}Dist data structure.

Mac\TeX\ is firmly based on \TL, and the main \TeX\ trees and binaries
are precisely the same.  It adds a few extra folders with Mac-specific
documentation and applications.


\subsubsection{Windows}\label{sec:wininst}

If you are using the unpacked downloaded zip file, or the \DVD\
installer failed to start automatically, double-click
\filename{install-tl-windows.bat}.  If you want more customization
options, e.g., selection of specific package collections, run
\filename{install-tl-advanced.bat} instead.

You can also start the installer from the command-prompt.  Below,
\texttt{>} denotes the prompt; user input is \Ucom{\texttt{bold}}.  If
you are in the installer directory, run just:
\begin{alltt}
> \Ucom{install-tl-windows}
\end{alltt}

Or you can invoke it with an absolute location, such as:
\begin{alltt}
> \Ucom{D:\bs{}texlive\bs{}install-tl-windows}
\end{alltt}
for the \TK\ \DVD, supposing that \dirname{D:} is the optical
drive. Figure~\ref{fig:wizard-w32} displays the wizard installer, which
is the default for Windows.

To install in text mode, use:
\begin{alltt}
> \Ucom{install-tl-windows -no-gui}
\end{alltt}

For a complete listing of the various options:
\begin{alltt}
> \Ucom{install-tl-windows -help}
\end{alltt}

\begin{figure}[tb]
\begin{boxedverbatim}
Installing TeX Live 2015 from: ...
Platform: i386-linux => 'GNU/Linux on Intel x86'
Distribution: inst (compressed)
Directory for temporary files: /tmp
...
 Detected platform: GNU/Linux on Intel x86
 
 <B> binary platforms: 1 out of 19

 <S> set installation scheme (scheme-full)

 <C> customizing installation collections
     47 collections out of 48, disk space required: 4022 MB

 <D> directories:
   TEXDIR (the main TeX directory):
     /usr/local/texlive/2015
   ...

 <O> options:
   [ ] use letter size instead of A4 by default
   ...
 
 <V> set up for portable installation

Actions:
 <I> start installation to hard disk
 <H> help
 <Q> quit
\end{boxedverbatim}
\caption{Main text installer screen (\GNU/Linux)}\label{fig:text-main}
\end{figure}

\begin{figure}[tb]
\tlpng{install-lnx-main}{\linewidth}
\caption{Expert \GUI{} installer screen (\GNU/Linux)}\label{fig:gui-main}
\end{figure}

\begin{figure}[tb]
\tlpng{wizard-w32}{\linewidth}
\caption{Wizard installer screen (Windows)}\label{fig:wizard-w32}
\end{figure}


\htmlanchor{cygwin}
\subsubsection{Cygwin}
\label{sec:cygwin}

Before beginning the installation, use Cygwin's \filename{setup.exe} program to
install the \filename{perl} and \filename{wget} packages if you have
not already done so.  The following additional packages are
recommended:
\begin{itemize*}
\item \filename{fontconfig} [needed by \XeTeX\ and Lua\TeX]
\item \filename{ghostscript} [needed by various utilities]
\item \filename{libXaw7} [needed by \code{xdvi}]
\item \filename{ncurses} [provides the \code{clear} command used by the installer]
\end{itemize*}


\subsubsection{The text installer}

Figure~\ref{fig:text-main} displays the main text mode screen under
Unix.  The text installer is the default on Unix.

This is only a command-line installer; there is no cursor support at
all.  For instance, you cannot tab around checkboxes or input fields.
You just type something (case-sensitive) at the prompt and press the
Enter key, and then the entire terminal screen will be rewritten, with
adjusted content.

The text installer interface is this primitive for a reason: it
is designed to run on as many platforms as possible, even with a
very barebones Perl.

\subsubsection{The expert graphical installer}

Figure~\ref{fig:gui-main} displays the expert graphical installer under
\GNU/Linux.  Other than using buttons and menus, this does
not differ much from the text installer.

This mode can be invoked explicitly with
\begin{alltt}
> \Ucom{install-tl -gui=perltk}
\end{alltt}


\subsubsection{The simple wizard installer}

Under Windows, the default is to run the simplest installation method we
could devise, called the ``wizard'' installer
(figure~\ref{fig:wizard-w32}).  It installs everything and asks almost
no questions.  If you want to customize your setup, you should run one
of the other installers.

On other platforms, this mode can be invoked explicitly with
\begin{alltt}
> \Ucom{install-tl -gui=wizard}
\end{alltt}


\subsection{Running the installer}
\label{sec:runinstall}

The installer is intended to be mostly self-explanatory, but following are a
few notes about the various options and submenus.

\subsubsection{Binary systems menu (Unix only)}
\label{sec:binary}

\begin{figure}[tb]
\begin{boxedverbatim}
Available platforms:
===============================================================================
   a [ ] Cygwin on Intel x86 (i386-cygwin)
   b [ ] Cygwin on x86_64 (x86_64-cygwin)
   c [ ] MacOSX/Darwin universal binaries (universal-darwin)
   d [ ] MacOSX/Darwin on x86_64 (x86_64-darwin)
   e [ ] FreeBSD on x86_64 (amd64-freebsd)
   f [ ] FreeBSD on Intel x86 (i386-freebsd)
   g [ ] GNU/Linux on DEC Alpha (alpha-linux)
   h [ ] GNU/Linux on ARM (armel-linux)
   i [ ] GNU/Linux on ARMhf (armhf-linux)
   j [X] GNU/Linux on Intel x86 (i386-linux)
   k [ ] GNU/Linux on MIPSel (mipsel-linux)
   l [ ] GNU/Linux on PowerPC (powerpc-linux)
   m [ ] GNU/Linux on x86_64 (x86_64-linux)
   o [ ] NetBSD on x86_64 (amd64-netbsd)
   p [ ] NetBSD on Intel x86 (i386-netbsd)
   s [ ] Solaris on Intel x86 (i386-solaris)
   t [ ] Solaris on Sparc (sparc-solaris)
   u [ ] Solaris on x86_64 (x86_64-solaris)
   v [ ] Windows (win32)
\end{boxedverbatim}
\caption{Binaries menu}\label{fig:bin-text}
\end{figure}

Figure~\ref{fig:bin-text} displays the text mode binaries menu.  By
default, only the binaries for your current platform will be installed.
From this menu, you can select installation of binaries for other
platforms as well.  This can be useful if you are sharing a \TeX\
tree across a network of heterogeneous machines, or for a dual-boot
system.

\subsubsection{Selecting what is to be installed}
\label{sec:components}

\begin{figure}[tbh]
\begin{boxedverbatim}
Select scheme:
===============================================================================
 a [X] full scheme (everything)
 b [ ] medium scheme (small + more packages and languages)
 c [ ] small scheme (basic + xetex, metapost, a few languages)
 d [ ] basic scheme (plain and latex)
 e [ ] minimal scheme (plain only)
 f [ ] ConTeXt scheme
 g [ ] GUST TeX Live scheme
 h [ ] teTeX scheme (more than medium, but nowhere near full)
 i [ ] XML scheme
 j [ ] custom selection of collections
\end{boxedverbatim}
\caption{Scheme menu}\label{fig:scheme-text}
\end{figure}

Figure~\ref{fig:scheme-text} displays the \TL\ scheme menu; from here,
you choose a ``scheme'', which is an overall set of package collections.
The default \optname{full} scheme installs everything available.  This
is recommended, but you can also choose the \optname{basic} scheme for a
small system, \optname{minimal} for testing purposes, and
\optname{medium} or \optname{teTeX} to get something in between.  There
are also various specialized and country-specific schemes.

\begin{figure}[tb]
\centering \tlpng{stdcoll}{.7\linewidth}
\caption{Collections menu}\label{fig:collections-gui}
\end{figure}

You can refine your scheme selection with the `collections' menu
(figure~\ref{fig:collections-gui}, shown in \GUI\ mode for a change).

Collections are one level more detailed than schemes\Dash in essence, a
scheme consists of several collections, a collection consists of one or
more packages, and a package (the lowest level grouping in \TL) contains
the actual \TeX\ macro files, font files, and so on.

If you want more control than the collection menus provide, you can use
the \TeX\ Live Manager (\prog{tlmgr}) program after installation (see
section~\ref{sec:tlmgr}); using that, you can control the installation
at the package level.

\subsubsection{Directories}
\label{sec:directories}

The default layout is described in section~\ref{sec:texmftrees},
\p.\pageref{sec:texmftrees}.  The default installation directory is
\dirname{/usr/local/texlive/2015} on Unix and
|%SystemDrive%\texlive\2015| on Windows.  This arrangement enables
having many parallel \TL\ installations, such as one for each release
(typically by year, as here), and you can switch between them merely by
altering your search path.

That installation directory can be overridden by setting the so-called
\dirname{TEXDIR} in the installer.  The \GUI\ screen for this and other
options is shown in figure~\ref{fig:gui-main}.  The most common reasons
to change it are either lacking enough disk space in that partition (the
full \TL\ needs several gigabytes), or lacking write permission for the
default location (you don't have to be root or administrator to install
\TL, but you do need write access to the target directory).

The installation directories can also be changed by setting a variety of
environment variables before running the installer (most likely,
\envname{TEXLIVE\_INSTALL\_PREFIX} or
\envname{TEXLIVE\_INSTALL\_TEXDIR}); see the documentation from
|install-tl --help| (available online at
\url{http://tug.org/texlive/doc/install-tl.html}) for the full list and
more details.

A reasonable alternative destination is a directory under your home,
especially if you will be the sole user. Use `|~|' to indicate this, as
in `|~/texlive/2015|'.

We recommend including the year in the name, to enable keeping different
releases of \TL{} side by side.  (You may wish to also maintain a
version-independent name, such as \dirname{/usr/local/texlive-cur}, via a
symbolic link, which you can then repoint after testing the new release.)

Changing \dirname{TEXDIR} in the installer will also change
\dirname{TEXMFLOCAL}, \dirname{TEXMFSYSVAR} and
\dirname{TEXMFSYSCONFIG}.

\dirname{TEXMFHOME} is the recommended location for personal macro files
or packages.  The default value is |~/texmf|.  In contrast to
\dirname{TEXDIR}, here a |~| is preserved in the newly-written
configuration files, since it usefully refers to the home directory of
the user running \TeX.  It expands to \dirname{$HOME} on Unix and
\verb|%USERPROFILE%| on Windows.  Special redundant note:
\envname{TEXMFHOME}, like all trees, must be organized according to the
\TDS, or files may not be found.

\dirname{TEXMFVAR} is the location for storing most cached runtime data
specific to each user.  \dirname{TEXMFCACHE} is the variable name used
for that purpose by Lua\LaTeX and \ConTeXt\ Mk\acro{IV} (see
section~\ref{sec:context-mkiv}, \p.\pageref{sec:context-mkiv}); its
default value is \dirname{TEXMFSYSVAR}, or (if that's not writable),
\dirname{TEXMFVAR}.


\subsubsection{Options}
\label{sec:options}

\begin{figure}[tbh]
\begin{boxedverbatim}
Options setup:
===============================================================================
 <P> use letter size instead of A4 by default: [ ]
 <E> execution of restricted list of programs: [X]
 <F> create format files:                      [X]
 <D> install font/macro doc tree:              [X]
 <S> install font/macro source tree:           [X]
 <L> create symlinks in standard directories:  [ ]
            binaries to:
            manpages to:
                info to:
 <Y> after installation, get package updates from CTAN: [X]
\end{boxedverbatim}
\caption{Options menu (Unix)}\label{fig:options-text}
\end{figure}

Figure~\ref{fig:options-text} shows the text mode options menu.
More info on each:

\begin{description}
\item[use letter size instead of A4 by default:] The default paper
  size selection.  Of course, individual documents can and should
  specify a specific paper size, if desired.

\item[execution of restricted list of programs:] As of \TL\ 2010,
  execution of a few external programs is allowed by default.  The (very
  short) list of allowed programs is given in the \filename{texmf.cnf}.
  See the 2010 news (section~\ref{sec:2010news}) for more details.

\item[create format files:] Although unnecessary format files
  take time to generate and disk space to store, it is still recommended
  to leave this option checked: if you don't, then format files will be
  generated in people's private \dirname{TEXMFVAR} tree as they are
  needed.  In that location, they will not be updated automatically if
  (say) binaries or hyphenation patterns are updated in the
  installation, and thus you could end up with incompatible format files.

\item[install font/macro \ldots\ tree:] Omit downloading/installing the
  documentation and source files included in most packages.  Not recommended.

\item[create symlinks in standard directories:]
  This option (Unix only) bypasses the need to change environment
  variables. Without this option, \TL{} directories usually have to be
  added to \envname{PATH}, \envname{MANPATH} and \envname{INFOPATH}. You
  will need write permissions to the target directories.  It is strongly
  advised \emph{not} to overwrite a \TeX\ system that came with your
  system with this option.  It is intended primarily for accessing the
  \TeX\ system through directories that are already known to users, such
  as \dirname{/usr/local/bin}, which don't already contain any \TeX\
  files.

\item[after installation \ldots\ \CTAN:] When installing from \DVD, this
  option is enabled by default, since usually one wants to take any
  subsequent package updates from the \CTAN\ area that is updated
  throughout the year.  The only likely reason to disable it is if you
  install only a subset from the \DVD\ and plan to augment the
  installation later.  In any case, the package repository for the
  installer, and for updates after installation, can be set
  independently as needed; see section~\ref{sec:location} and
  section~\ref{sec:dvd-install-net-updates}.
\end{description}

When all the settings are to your liking, you can type `I' to start the
installation process. When it is done, skip to
section~\ref{sec:postinstall} to read what else needs to be done, if
anything.

\subsection{Command-line install-tl options}
\label{sec:cmdline}

Type
\begin{alltt}
> \Ucom{install-tl -help}
\end{alltt}
for a listing of command-line options.  Either |-| or |--| can be used
to introduce option names.  These are the most common ones:

\begin{ttdescription}
\item[-gui] If possible, use the \GUI{} installer. This requires the
  Perl/Tk module (\url{http://tug.org/texlive/distro.html#perltk})
  compiled with \acro{XFT} support; if Perl/Tk is not available,
  installation continues in text mode.

\item[-no-gui] Force using the text mode installer, even under
  Windows.

\item[-lang {\sl LL}] Specify the installer interface
  language as a standard (usually two-letter) code.  The installer tries
  to automatically determine the right language but if it fails, or if
  the right language is not available, then it uses English as a
  fallback.  Run \verb|install-tl --help| to get the list of available
  languages.

\item[-portable] Install for portable use on, e.g., a \USB{} stick.
  Also selectable from within the text installer with the \code{V}
  command, and from the \GUI{} installer.  See
  section~\ref{sec:portable-tl}.

\item[-profile {\sl file}] Load the installation profile \var{file} and
  do the installation with no user interaction.  The installer always
  writes a file \filename{texlive.profile} to the \dirname{tlpkg}
  subdirectory of your installation.  That file can be given as the
  argument to redo the exact same installation on a different system,
  for example.  Alternatively, you can use a custom profile, most easily
  created by starting from a generated one and changing values, or an
  empty file, which will take all the defaults.

\item [-repository {\sl url-or-directory}] Specify package
  repository from which to install; see following.

\htmlanchor{opt-in-place}
\item[-in-place] (Documented only for completeness: Do not use this
  unless you know what you are doing.) If you already have an rsync, svn,
  or other copy of \TL{} (see
  \url{http://tug.org/texlive/acquire-mirror.html}) then this option
  will use what you've got, as-is, and do only the necessary
  post-install.  Be warned that the file \filename{tlpkg/texlive.tlpdb}
  may be overwritten; saving it is your responsibility.  Also, package
  removal has to be done manually. This option cannot be toggled via the
  installer interface.
\end{ttdescription}


\subsubsection{The \optname{-repository} option}
\label{sec:location}

The default network package repository is a \CTAN{} mirror chosen
automatically via \url{http://mirror.ctan.org}.

If you want to override that, the location value can be a url starting
with \texttt{ftp:}, \texttt{http:}, or \texttt{file:/}, or a plain
directory path.  (When giving an \texttt{http:}\ or \texttt{ftp:}\
location, trailing `\texttt{/}' characters and/or a trailing
`\texttt{/tlpkg}' component are ignored.)

For example, you could choose a particular \CTAN\ mirror with something
like: \url{http://ctan.example.org/tex-archive/systems/texlive/tlnet/},
substituting a real hostname and its particular top-level \CTAN\ path
for |ctan.example.org/tex-archive|.  The list of \CTAN\ mirrors is
maintained at \url{http://ctan.org/mirrors}.

If the given argument is local (either a path or a \texttt{file:/} url),
compressed files in an \dirname{archive} subdirectory of the repository
path are used (even if uncompressed files are available as well).


\subsection{Post-install actions}
\label{sec:postinstall}

Some post-installation may be required.

\subsubsection{Environment variables for Unix}
\label{sec:env}

If you elected to create symlinks in standard directories (described in
section~\ref{sec:options}), then there is no need to edit environment
variables.  Otherwise, on Unix systems, the directory of the binaries
for your platform must be added to the search path.  (On Windows, the
installer takes care of this.)

Each supported platform has its own subdirectory under
\dirname{TEXDIR/bin}. See figure~\ref{fig:bin-text} for the list of
subdirectories and corresponding platforms.

Optionally, you can also add the documentation man and Info directories
to their respective search paths, if you want the system tools to find
them.  The man pages might be found automatically after the addition to
\envname{PATH}.

For Bourne-compatible shells such as \prog{bash}, and using Intel x86
GNU/Linux and a default directory setup as an example, the file to edit
might be \filename{$HOME/.profile} (or another file sourced by
\filename{.profile}, and the lines to add would look like this:

\begin{sverbatim}
PATH=/usr/local/texlive/2015/bin/i386-linux:$PATH; export PATH
MANPATH=/usr/local/texlive/2015/texmf-dist/doc/man:$MANPATH; export MANPATH
INFOPATH=/usr/local/texlive/2015/texmf-dist/doc/info:$INFOPATH; export INFOPATH
\end{sverbatim}

For csh or tcsh, the file to edit is typically \filename{$HOME/.cshrc}, and
the lines to add might look like:

\begin{sverbatim}
setenv PATH /usr/local/texlive/2015/bin/i386-linux:$PATH
setenv MANPATH /usr/local/texlive/2015/texmf-dist/doc/man:$MANPATH
setenv INFOPATH /usr/local/texlive/2015/texmf-dist/doc/info:$INFOPATH
\end{sverbatim}

If you already have settings somewhere in your ``dot'' files, naturally
the \TL\ directories should be merged in as appropriate.


\subsubsection{Environment variables: Global configuration}
\label{sec:envglobal}

If you want to make these changes globally, or for a user newly added to
the system, then you are on your own; there is just too much variation
between systems in how and where these things are configured.

Our two hints are: 1)~you may want to check for a file
\filename{/etc/manpath.config} and, if present, add lines such as

\begin{sverbatim}
MANPATH_MAP /usr/local/texlive/2015/bin/i386-linux \
            /usr/local/texlive/2015/texmf-dist/doc/man
\end{sverbatim}

And 2)~check for a file \filename{/etc/environment} which may define the
search path and other default environment variables.

In each (Unix) binary directory, we also create a symbolic link named
\code{man} to the directory \dirname{texmf-dist/doc/man}.  Some \code{man}
programs, such as the standard \MacOSX\ \code{man}, will automatically
find that, obviating the need for any man page setup.


\subsubsection{Internet updates after \DVD\ installation}
\label{sec:dvd-install-net-updates}

If you installed \TL\ from \DVD\ and then wish to get updates from the
Internet, you need to run this command---\emph{after} you've updated
your search path (as described in the previous section):

\begin{alltt}
> \Ucom{tlmgr option repository http://mirror.ctan.org/systems/texlive/tlnet}
\end{alltt}

This tells \cmdname{tlmgr} to use a nearby \CTAN\ mirror for future updates.
This is done by default when installing from \DVD, via the option
described in section~\ref{sec:options}.

If there are problems with the automatic mirror selection, you can
specify a particular \CTAN\ mirror from the list at
\url{http://ctan.org/mirrors}.  Use the exact path to the
\dirname{tlnet} subdir on that mirror, as shown above.


\htmlanchor{xetexfontconfig}  % keep historical anchor working
\htmlanchor{sysfontconfig}
\subsubsection{System font configuration for \XeTeX\ and Lua\TeX}
\label{sec:font-conf-sys}

\XeTeX\ and Lua\TeX\ can use any font installed on the system, not just
those in the \TeX\ trees.  They do these via related but not identical
methods.

On Windows, fonts shipped with \TL\ are made available to \XeTeX\
automatically.  But if you have installed the \filename{xetex} package
on a Unix-compatible system, you need to configure your system to be
able to find the fonts shipped with \TL\ via system name lookup, and not
just filename lookup.

To facilitate this, when the \pkgname{xetex} package is installed
(either at initial installation or later), the necessary configuration
file is created in
\filename{TEXMFSYSVAR/fonts/conf/texlive-fontconfig.conf}.

To set up the \TL{} fonts for system-wide use (assuming you have
suitable privileges), proceed as follows:
\begin{enumerate*}
\item Copy the \filename{texlive-fontconfig.conf} file to
\dirname{/etc/fonts/conf.d/09-texlive.conf}.
\item Run \Ucom{fc-cache -fsv}.
\end{enumerate*}

If you do not have sufficient privileges to carry out the steps above,
or if you want to make the \TL{} fonts available to only one user,
you can do the following:
\begin{enumerate*}
\item Copy the \filename{texlive-fontconfig.conf} file to
      \filename{~/.fonts.conf}, where \filename{~} is your home directory.
\item Run \Ucom{fc-cache -fv}.
\end{enumerate*}

You can run \code{fc-list} to see the names of the system fonts.  The
incantation \code{fc-list : family style file spacing} (all arguments
are literal strings) shows some generally interesting information.


\subsubsection{\ConTeXt{} Mark \acro{IV}}
\label{sec:context-mkiv}

Both the `old' \ConTeXt{} (Mark \acro{II}) and the `new' \ConTeXt{}
(Mark \acro{IV}) should run out of the box after \TL{} installation,
and should need no special attention as long as you stick to
using \verb+tlmgr+ for updates.

However, because \ConTeXt{} Mk\acro{IV} does not use the kpathsea
library, some setup will be required whenever you install new 
files manually (without using \verb+tlmgr+). After each such 
installation, each Mk\acro{IV} user must run:
\begin{sverbatim}
context --generate
\end{sverbatim}
to refresh the \ConTeXt{} disk cache data.
The resulting files are stored under \code{TEXMFCACHE}.
whose default value in \TL\ is \verb+TEXMSYSVAR;TEXMFVAR+. 

\ConTeXt\ Mk\acro{IV} will read from
all paths mentioned in \verb+TEXMFCACHE+, and write to the first
path that is writable. While reading, the last found match will 
take precedence in the case of duplicated cache data.

For more information, see
\url{http://wiki.contextgarden.net/Running_Mark_IV}.


\subsubsection{Integrating local and personal macros}
\label{sec:local-personal-macros}

This is already mentioned implicitly in section~\ref{sec:texmftrees}:
\dirname{TEXMFLOCAL} (by default,
\dirname{/usr/local/texlive/texmf-local} or
\verb|%SystemDrive%\texlive\texmf-local| on Windows)
is intended for system-wide local fonts and macros; and
\dirname{TEXMFHOME} (by default, \dirname{$HOME/texmf} or
\verb|%USERPROFILE%\texmf|), is for personal fonts and macros.  These
directories are intended to stick around from release to release, and
have their content seen automatically by a new \TL{} release.
Therefore, it is best to refrain from changing the definition of
\dirname{TEXMFLOCAL} to be too far away from the main \TL{} directory,
or you will need to manually change future releases.

For both trees, files should be placed in their proper \TeX\ Directory
Structure (\TDS) subdirectories; see \url{http://tug.org/tds} or consult
\filename{texmf-dist/web2c/texmf.cnf}. For instance, a \LaTeX{} class file or
package should be placed in \dirname{TEXMFLOCAL/tex/latex} or
\dirname{TEXMFHOME/tex/latex}, or a subdirectory thereof.

\dirname{TEXMFLOCAL} requires an up-to-date filename database, or files
will not be found.  You can update it with the command
\cmdname{mktexlsr} or use the `Reinit file database' button on the
configuration tab of the \TeX\ Live Manager \GUI.

By default, each of these variables is defined to be a single directory,
as shown.  This is not a hard-and-fast requirement.  If you need to
easily switch back and forth between different versions of large
packages, for example, you can maintain multiple trees for your own
purposes.  This is done by setting \dirname{TEXMFHOME} to the
list of directories, within braces, separated by commas:

\begin{verbatim}
  TEXMFHOME = {/my/dir1,/mydir2,/a/third/dir}
\end{verbatim}

Section~\ref{sec:brace-expansion} describes brace expansion further.


\subsubsection{Integrating third-party fonts}

This is unfortunately a messy topic.  Forget about it unless you want to
delve into many details of the \TeX{} installation.  Many fonts are
included in \TL\ already, so take a look if you don't already know 
that what you want isn't there.

A possible alternative is to use \XeTeX\ or Lua\TeX\ (see
section~\ref{sec:tex-extensions}), which let you use operating system
fonts without any installation in \TeX.

If you do need to do this, see
\url{http://tug.org/fonts/fontinstall.html} for our best effort at
describing the procedure.


\subsection{Testing the installation}
\label{sec:test-install}

After installing \TL{}, you naturally want to test it out, so you can
start creating beautiful documents and\slash or fonts.

One thing you may immediately be looking for is a front-end with which
to edit files.  \TL{} installs \TeX{}works
(\url{http://tug.org/texworks}) on Windows (only), and Mac\TeX\ installs
TeXShop (\url{http://pages.uoregon.edu/koch/texshop}.  On other Unix
systems, it's left up to you to choose an editor.  There are many
choices available, some of which are listed in the next section; see
also \url{http://tug.org/interest.html#editors}.  Any plain text editor
will work; something \TeX-specific is not required.

The rest of this section gives some basic procedures for testing that
the new system is functional.  We give Unix commands here; under
\MacOSX{} and Windows, you're more likely to run the tests through a
graphical interface, but the principles are the same.

\begin{enumerate}

\item Make sure that you can run the \cmdname{tex} program in the first
place:
\begin{alltt}
> \Ucom{tex -{}-version}
TeX 3.14159265 (TeX Live ...)
Copyright ... D.E. Knuth.
...
\end{alltt}
If this comes back with `command not found' instead of version and
copyright information, or with an older version, most likely you don't
have the correct \dirname{bin} subdirectory in your \envname{PATH}.  See
the environment-setting information on \p.\pageref{sec:env}.

\item Process a basic \LaTeX{} file:
\begin{alltt}
> \Ucom{latex sample2e.tex}
This is pdfTeX 3.14...
...
Output written on sample2e.dvi (3 pages, 7484 bytes).
Transcript written on sample2e.log.
\end{alltt}
If this fails to find \filename{sample2e.tex} or other files, most
likely you have interference from old environment variables or
configuration files; we recommend unsetting all \TeX-related environment
variables for a start.  (For a deep analysis, you can ask \TeX{} to
report on exactly what it is searching for, and finding; see ``Debugging
actions'' on page~\pageref{sec:debugging}.)

\item Preview the result online:
\begin{alltt}
> \Ucom{xdvi sample2e.dvi}    # Unix
> \Ucom{dviout sample2e.dvi}  # Windows
\end{alltt}
You should see a new window with a nice document explaining some of the
basics of \LaTeX{}.  (Well worth reading, by the way, if you're new to
\TeX.)  You do have to be running under X for \cmdname{xdvi} to work; if
you're not, or your \envname{DISPLAY} environment variable is set
incorrectly, you'll get an error \samp{Can't open display}.

\item Create a \PS{} file for printing or display:
\begin{alltt}
> \Ucom{dvips sample2e.dvi -o sample2e.ps}
\end{alltt}

\item Create a \acro{PDF} file instead of \dvi{}; this processes the
\filename{.tex} file and writes \acro{PDF} directly:
\begin{alltt}
> \Ucom{pdflatex sample2e.tex}
\end{alltt}

\item Preview the \acro{PDF} file:
\begin{alltt}
> \Ucom{gv sample2e.pdf}
\textrm{or:}
> \Ucom{xpdf sample2e.pdf}
\end{alltt}
Neither \cmdname{gv} nor \cmdname{xpdf} are included in \TL{}, so you
must install them separately.  See \url{http://www.gnu.org/software/gv}
and \url{http://www.foolabs.com/xpdf}, respectively.  There are plenty
of other \acro{PDF} viewers, too.  For Windows, we recommend trying
Sumatra PDF (\url{http://blog.kowalczyk.info/software/sumatrapdf}).

\item Standard test files you may find useful in addition to
\filename{sample2e.tex}:

\begin{ttdescription}
\item [small2e.tex] A simpler document than \filename{sample2e}, to
reduce the input size if you're having troubles.
\item [testpage.tex] Test if your printer introduces any offsets.
\item [nfssfont.tex] For printing font tables and tests.
\item [testfont.tex] Also for font tables, but using plain \TeX{}.
\item [story.tex] The most canonical (plain) \TeX{} test file of all.
You must type \samp{\bs bye} to the \code{*} prompt after \samp{tex
story.tex}.
\end{ttdescription}

\item If you have installed the \filename{xetex} package, you can test
its access to system fonts as follows:
\begin{alltt}
> \Ucom{xetex opentype-info.tex}
This is XeTeX, Version 3.14\dots
...
Output written on opentype-info.pdf (1 page).
Transcript written on opentype-info.log.
\end{alltt}

If you get an error message saying ``Invalid fontname `Latin Modern
Roman/ICU'\dots'', then you need to configure your system so that the
fonts shipped with \TL\ can be found.  See
Section~\ref{sec:font-conf-sys}.

\end{enumerate}

\subsection{Links for additional downloadable software}

If you are new to \TeX{}, or otherwise need help with actually writing
\TeX{} or \LaTeX{} documents, please visit
\url{http://tug.org/begin.html} for some introductory resources.

Links for some other tools you may consider installing:
\begin{description}
\item[Ghostscript] \url{http://www.cs.wisc.edu/~ghost/}
\item[Perl] \url{http://www.perl.org/} with
      supplementary packages from \acro{CPAN}, \url{http://www.cpan.org/}
\item[ImageMagick] \url{http://www.imagemagick.com}, for graphics
      processing and conversion
\item[NetPBM] \url{http://netpbm.sourceforge.net/}, also for graphics.

\item[\TeX-oriented editors] There is a wide choice, and it is a matter of the
      user's taste. Here is a selection in alphabetical order (a few
      here are for Windows only).
  \begin{itemize*}
  \item \cmdname{GNU Emacs} is available natively under Windows, see
        \url{http://www.gnu.org/software/emacs/windows/ntemacs.html}.
  \item \cmdname{Emacs with Auc\TeX} for Windows is available from \CTAN.
        The AuC\TeX\ home page is \url{http://www.gnu.org/software/auctex}.
  \item \cmdname{LEd} is available from
        \url{http://mirror.ctan.org/support/LEd}.
  \item \cmdname{SciTE} is available from
        \url{http://www.scintilla.org/SciTE.html}.
  \item \cmdname{Texmaker} is free software, available from
        \url{http://www.xm1math.net/texmaker}.
  \item \cmdname{TeXstudio} started out as a fork of \cmdname{Texmaker} with
        additional features; \url{http://texstudio.sourceforge.net}.
  \item \cmdname{TeXnicCenter} is free software, available from
        \url{http://www.texniccenter.org} and in the pro\TeX{}t distribution.
  \item \cmdname{TeXworks} is free software, available from
        \url{http://tug.org/texworks} and installed as part of \TL\ for
        Windows (only).
  \item \cmdname{Vim} is free software, available from
        \url{http://www.vim.org}.
  \item \cmdname{WinEdt} is shareware available though
        \url{http://tug.org/winedt} or \url{http://www.winedt.com}.
  \item \cmdname{WinShell} is available from \url{http://www.winshell.de}.
  \end{itemize*}
\end{description}
For a much longer list of packages and programs, see
\url{http://tug.org/interest.html}.


\section{Specialized installations}

The previous sections described the basic installation process.  Here we
turn to some specialized cases.

\htmlanchor{tlsharedinstall}
\subsection{Shared-user (or cross-machine) installations}
\label{sec:sharedinstall}

\TL{} has been designed to be sharable between different users on one
system, and/or between different systems on a network. With a standard
directory layout, no hard paths are configured: the locations for files
needed by \TL{} programs are found relative to the programs.  You can
see this in the principal configuration file
\filename{$TEXMFDIST/web2c/texmf.cnf}, which contains lines such as
\begin{sverbatim}
TEXMFROOT = $SELFAUTOPARENT
...
TEXMFDIST = $TEXMFROOT/texmf-dist
...
TEXMFLOCAL = $SELFAUTOPARENT/../texmf-local
\end{sverbatim}
This means that adding the directory for \TL{} executables for their
platform to their search path is sufficient to get a working setup.

By the same token, you can also install \TL{} locally and then move
the entire hierarchy afterwards to a network location.

For Windows, a sample network installation script named
\filename{w32client} can be downloaded through
\url{http://tug.org/texlive/w32client.html}.  It creates settings and
menu shortcuts for using an existing \TL{} installation on a \acro{LAN}.
It also registers an uninstaller \filename{w32unclient}, available
in the same zip file.  See the web page for more information.


\htmlanchor{tlportable}
\subsection{Portable (\USB{}) installations}
\label{sec:portable-tl}

The \code{-portable} installer option (or \code{V} command in the text
installer or corresponding \GUI{} option) creates a completely
self-contained \TL{} installation under a common root and forgoes system
integration. You can create such an installation directly on a \USB{}
stick, or copy it to a \USB{} stick afterwards.

To run \TeX\ using this portable installation, you need to add the
appropriate binary directory to the search path during your terminal
session, as usual.  On Windows, you can double-click
\filename{tl-tray-menu} at the root of the installation to choose
between a few common tasks, as shown in this screenshot:

\medskip
\tlpng{tray-menu}{4cm}
\smallskip

\noindent The `More\ldots' entry explains how you can customize this menu.


\htmlanchor{tlisoinstall}
\subsection{\ISO\ (or \DVD) installations}
\label{sec:isoinstall}

If you don't need to update or otherwise modify your installation often,
and\slash or have several systems on which to use \TL{}, it may be
convenient to create an \ISO\ of your \TL{} installation, because:

\begin{itemize}
\item Copying an \ISO\ between different computers is much
  faster than copying a normal installation.
\item If you are dual-booting between different operating systems
  and want them to share a \TL{} installation, an \ISO
  installation is not tied to the idiosyncrasies and limitations of
  other mutually supported filesystems (\acro{FAT32}, \acro{NTFS},
  \acro{HFS+}).
\item Virtual machines can simply mount such an \ISO.
\end{itemize}

Of course you can also burn an \ISO\ to \DVD, if that is useful for you.

Desktop \GNU/Linux/Unix systems, including \MacOSX, are able to
mount an \ISO. Windows 8 is the first(!) Windows version which can
do this. Apart from that, nothing changes compared to a normal hard
disk installation, see section \ref{sec:env}.

When preparing such an \ISO\ installation, it is best to omit the
subdirectory for the release year, and have
\filename{texmf-local} at the same level as the other trees
(\filename{texmf-dist}, \filename{texmf-var}, etc.).  You can do this with
the normal directory options in the installer.

For a physical (rather than virtual) Windows system, you can burn
the \ISO\ to \acro{DVD}. However, it may be worth your while to
investigate free \ISO-mounting options such as WinCDEmu at
\url{http://wincdemu.sysprogs.org/}.

For Windows system integration, you can include the \filename{w32client}
scripts described in section~\ref{sec:sharedinstall} and at
\url{http://tug.org/texlive/w32client.html}, which work just as well for
an \ISO\ as for a network installation.

On \MacOSX, TeXShop will be able to use the \acro{DVD}
installation if a symlink \filename{/usr/texbin} points to the
appropriate binary directory, e.g.,
\begin{verbatim}
sudo ln -s /Volumes/MyTeXLive/bin/universal-darwin /usr/texbin
\end{verbatim}

Historical note: \TL{} 2010 was the first \TL{} edition which was no
longer distributed `live'.  However, it always required some acrobatics
to run from \DVD\ or \ISO; in particular, there was no way around
setting at least one extra environment variable.  If you create your
\ISO\ from an existing installation then there is no need for this.


\htmlanchor{tlmgr}
\section{\cmdname{tlmgr}: Managing your installation}
\label{sec:tlmgr}

\begin{figure}[tb]
\tlpng{tlmgr-gui}{\linewidth}
\caption{\prog{tlmgr} in \GUI\ mode: main window, after `Load'.}
\label{fig:tlmgr-gui}
\end{figure}

\begin{figure}[tb]
\hbox to \hsize{%
  \vtop{\hsize=.59\linewidth
    \null  % make figures align at the top
    \tlpng{tlmgr-general-options}{\hsize}
    \caption{\texttt{tlmgr} in \GUI\ mode: General options}
    \label{fig:tlmgr-general-options}
  }
  \hfil
  \vtop{\hsize=.39\linewidth
    \null
    \tlpng{tlmgr-paper-options}{\hsize}
    \caption{\texttt{tlmgr} in \GUI\ mode: Paper size options}
    \label{fig:tlmgr-paper-options}
  }
}
\end{figure}

\TL{} includes a program named \prog{tlmgr} for managing \TL{} after the
initial installation.  Its capabilities include:

\begin{itemize*}
\item installing, updating, backing up, restoring, and uninstalling
  individual packages, optionally taking dependencies into account;
\item searching for and listing packages and their descriptions;
\item listing, adding, and removing platforms;
\item changing installation options such as paper size and source
  location (see section~\ref{sec:location}).
\end{itemize*}

\prog{tlmgr}'s functionality completely subsumes the \prog{texconfig}
program.  We still distribute and maintain \prog{texconfig} for the sake
of anyone used to its interface, but we recommend using \prog{tlmgr}
nowadays.

\subsection{\cmdname{tlmgr} \GUI{} mode}
\prog{tlmgr} can be started in \GUI{} mode (figure~\ref{fig:tlmgr-gui}) with
\begin{alltt}
> \Ucom{tlmgr -gui}
\end{alltt}
or in Windows via the Start menu: \texttt{Start}, \texttt{Programs},
\texttt{TeX Live ...}, \texttt{TeX Live Manager}. After clicking `Load'
it displays a list of available and installed packages.  This assumes of
course that the installation source is valid and reachable.

Figures~\ref{fig:tlmgr-general-options} and~\ref{fig:tlmgr-paper-options}
show the general and paper size option screens.

\subsection{Sample \cmdname{tlmgr} command-line invocations}

After the initial installation, you can update your system to the latest
versions available with:
\begin{alltt}
> \Ucom{tlmgr update -all}
\end{alltt}
If this makes you nervous, first try
\begin{alltt}
> \Ucom{tlmgr update -all -dry-run}
\end{alltt}
or (less verbose):
\begin{alltt}
> \Ucom{tlmgr update -list}
\end{alltt}

This more complex example adds a collection, for the engine \XeTeX, from
a local directory:

\begin{alltt}
> \Ucom{tlmgr -repository /local/mirror/tlnet install collection-xetex}
\end{alltt}
It generates the following output (abridged):
\begin{fverbatim}
install: collection-xetex
install: arabxetex
...
install: xetex
install: xetexconfig
install: xetex.i386-linux
running post install action for xetex
install: xetex-def
...
running mktexlsr
mktexlsr: Updating /usr/local/texlive/2015/texmf-dist/ls-R...
...
running fmtutil-sys --missing
...
Transcript written on xelatex.log.
fmtutil: /usr/local/texlive/2015/texmf-var/web2c/xetex/xelatex.fmt installed.
\end{fverbatim}

As you can see, \prog{tlmgr} installs dependencies, and takes care of any
necessary post-install actions, including updating the filename database
and (re)generating formats.  In the above, we generated new formats for
\XeTeX.

To describe a package (or collection or scheme):
\begin{alltt}
> \Ucom{tlmgr show collection-latexextra}
\end{alltt}
which produces output like this:
\begin{fverbatim}
package:    collection-latexextra
category:   Collection
shortdesc:  LaTeX supplementary packages
longdesc:   A very large collection of add-on packages for LaTeX.
installed:  Yes
revision:   32768
\end{fverbatim}

Last and most important, for full documentation see
\url{http://tug.org/texlive/tlmgr.html}, or:
\begin{alltt}
> \Ucom{tlmgr -help}
\end{alltt}


\section{Notes on Windows}
\label{sec:windows}

\subsection{Windows-specific features}
\label{sec:winfeatures}

Under Windows, the installer does some extra things:
\begin{description}
\item[Menus and shortcuts.] A new `\TL{}' submenu of the
  Start menu is installed, which contains entries for some \GUI{}
  programs (\prog{tlmgr}, \prog{texdoctk}, the PS\_View (\prog{psv})
  PostScript previewer) and some documentation.
\item[File associations.] If enabled, \prog{TeXworks}, \prog{Dviout}
  and \prog{PS\_view} become either the default program for
  their respective filetypes, or get an entry in the `Open with'
  right-click menus of those filetypes.
\item[Bitmap to eps converter.] Various bitmapped formats get an
  entry \cmdname{bitmap2eps} in their `Open with' right-click
  menu. Bitmap2eps is a simple script which lets \cmdname{sam2p} or
  \cmdname{bmeps} do the real work.
\item[Automatic path adjustment.] No manual configuration steps are required.
\item[Uninstaller.] The installer creates an entry under `Add/Remove
  Programs' for \TL. The uninstall tab of the \TeX\ Live Manager \GUI\
  refers to this.
\end{description}

\subsection{Additional software included on Windows}

To be complete, a \TL{} installation needs support packages that are not
commonly found on a Windows machine. \TL{} provides the missing pieces.
These programs are all installed as part of \TL{} only on Windows.

\begin{description}
\item[Perl and Ghostscript.] Because of the importance of Perl and
  Ghostscript, \TL{} includes `hidden' copies of these
  programs. \TL{} programs that need them know where to find them,
  but they don't betray their presence through environment variables
  or registry settings. They aren't full-scale installations, and
  shouldn't interfere with any system installations of Perl or
  Ghostscript.

\item[PS\_View.] Also installed is PS\_View, a \PS{} and \acro{PDF}
  viewer; see figure~\ref{fig:psview}.

\begin{figure}[tb]
\tlpng{psview}{.6\linewidth}
\caption{PS\_View: very high magnifications available!}\label{fig:psview}
\end{figure}

\item[dviout.] Also installed is \prog{dviout}, a \acro{DVI} viewer.
  At first, when you preview files with \cmdname{dviout}, it will create
  fonts, because screen fonts were not installed. After a while, you
  will have created most of the fonts you use, and you will rarely see
  the font-creation window.  More information can be found in the
  (highly recommended) on-line help.

\item[\TeX{}works.]  \TeX{}works is a \TeX-oriented editor with
  an integrated \acro{PDF} viewer.

\item[Command-line tools.] A number of Windows ports of common Unix
  command-line programs are installed along with the usual \TL{}
  binaries. These include \cmdname{gzip}, \cmdname{unzip} and the
  utilities from the \cmdname{xpdf} suite (\cmdname{pdfinfo},
  \cmdname{pdffonts}, \ldots).  The \cmdname{xpdf} viewer itself is not
  available for Windows. Instead, you can download the Sumatra
  \acro{PDF} viewer, which is based on \cmdname{xpdf}, from
  \url{http://blog.kowalczyk.info/software/sumatrapdf}, among other options.

\item[fc-list, fc-cache, \ldots] The tools from the fontconfig library allow
  \XeTeX{} to handle system fonts on Windows.  You can use
  \prog{fc-list} to determine the font names to pass to \XeTeX's
  extended \cs{font} command. If necessary, run \prog{fc-cache}
  first to update font information.

\end{description}


\subsection{User Profile is Home}
\label{sec:winhome}

The Windows counterpart of a Unix home directory is the
\verb|%USERPROFILE%| directory.  Under Windows \acro{XP}, this
is usually \verb|C:\Documents and Settings\<username>|, and under
Windows Vista and later it is \verb|C:\Users\<username>|.  In the
\filename{texmf.cnf} file, and \KPS{} in general, \verb|~| will expand
appropriately on both Windows and Unix.


\subsection{The Windows registry}
\label{sec:registry}

Windows stores nearly all configuration data in its registry.  The
registry contains a set of hierarchically organized keys, with several
root keys. The most important ones for installation programs are
\path{HKEY_CURRENT_USER} and \path{HKEY_LOCAL_MACHINE}, \path{HKCU} and
\path{HKLM} in short. The \path{HKCU} part of the registry is in the
user's home directory (see section~\ref{sec:winhome}).  \path{HKLM} is
normally in a subdirectory of the Windows directory.

In some cases, system information could be obtained from environment
variables but for other information, for example the location of
shortcuts, it is necessary to consult the registry.  Setting environment
variables permanently also requires registry access.


\subsection{Windows permissions}
\label{sec:winpermissions}

In later versions of Windows, a distinction is made between regular
users and administrators, where only the latter have free access to the
entire operating system. We have made an effort to make \TL{}
installable without administrative privileges.

If the installer is started with administrative permissions, there is an
option to install for all users.  If this option is chosen, shortcuts
are created for all users, and the system environment is
modified. Otherwise, shortcuts and menu entries are created for the
current user, and the user environment is modified.

Regardless of administrator status, the default root of \TL{} proposed
by the installer is always under \verb|%SystemDrive%|. The installer
always tests whether the root is writable for the current user.

A problem may arise if the user is not an administrator and \TeX{}
already exists in the search path.  Since the effective path
consists of the system path followed by the user path, the new \TL{}
would never get precedence.  As a safeguard, the installer creates a
shortcut to the command-prompt in which the new \TL{} binary
directory is prepended to the local search path.  The new \TL{} will
be always usable from within such a command-prompt. The shortcut for
\TeX{}works, if installed, also prepends \TL{} to the search path, so it
should also be immune to this path problem.

For Windows Vista and later, you should be aware that even if you are
logged in as administrator, you need to explicitly ask for administrator
privileges. In fact, there is not much point in logging in as
administrator. Instead, right-clicking on the program or shortcut that
you want to run usually gives you a choice `Run as administrator'.


\subsubsection{Protecting the \TL{} directory}

An admin install does \emph{not} protect the \TL{} directory from
tampering by regular users. This will have to be done in a separate
step by setting a suitable \acro{ACL} (Access Control List) for the
directory. To this end, check out the \filename{icacls} Windows
command-line utility.


\subsection{Increasing maximum memory on Windows and Cygwin}
\label{sec:cygwin-maxmem}

Windows and Cygwin (see section~\ref{sec:cygwin} for Cygwin installation
specifics) users may find that they run out of memory when running some
of the programs shipped with \TL.  For example, \prog{asy} might run out
of memory if you try to allocate an array of 25,000,000 reals, and
Lua\TeX\ might run out of memory if you try to process a document with a
lot of big fonts.

For Cygwin, you can increase the amount of available memory by following
the instructions in the Cygwin User's Guide
(\url{http://www.cygwin.com/cygwin-ug-net/setup-maxmem.html}).

For Windows, you have to create a file, say \code{moremem.reg}, with
these four lines:

\begin{sverbatim}
Windows Registry Editor Version 5.00

[HKEY_LOCAL_MACHINE\Software\Cygwin]
"heap_chunk_in_mb"=dword:ffffff00
\end{sverbatim}

\noindent and then execute the command \code{regedit /s moremem.reg} as
administrator.  (If you want to change memory only for the current user
instead of system-wide, use \code{HKEY\_CURRENT\_USER}.)


\section{A user's guide to Web2C}

\Webc{} is an integrated collection of \TeX-related programs: \TeX{}
itself, \MF{}, \MP, \BibTeX{}, etc.  It is the heart of \TL{}.  The home
page for \Webc{}, with the current manual and more, is
\url{http://tug.org/web2c}.

A bit of history: The original implementation was by Tomas Rokicki who,
in 1987, developed a first \TeX{}-to-C system based on change files
under Unix, which were primarily the original work of Howard Trickey and
Pavel Curtis.  Tim Morgan became the maintainer of the system, and
during this period the name changed to Web-to-C\@.  In 1990, Karl Berry
took over the work, assisted by dozens of additional contributors, and
in 1997 he handed the baton to Olaf Weber, who returned it to Karl in
2006.

The \Webc{} system runs on Unix, 32-bit Windows systems, \MacOSX{}, and
other operating systems. It uses Knuth's original sources for \TeX{} and
other basic programs written in the \web{} literate programming system
and translates them into C source code.  The core \TeX{} programs
handled in this way are:

\begin{cmddescription}
\item[bibtex]    Maintaining bibliographies.
\item[dvicopy]   Expands virtual font references in \dvi{} files.
\item[dvitomp]   \dvi{} to MPX (MetaPost pictures).
\item[dvitype]   \dvi{} to human-readable text.
\item[gftodvi]   Generic font proofsheets.
\item[gftopk]    Generic to packed fonts.
\item[gftype]    GF to human-readable text.
\item[mf]        Creating typeface families.
\item[mft]       Prettyprinting \MF{} source.
\item[mpost]     Creating technical diagrams.
\item[patgen]    Creating hyphenation patterns.
\item[pktogf]    Packed to generic fonts.
\item[pktype]    PK to human-readable text.
\item[pltotf]    Plain text property list to TFM.
\item[pooltype]  Display \web{} pool files.
\item[tangle]    \web{} to Pascal.
\item[tex]       Typesetting.
\item[tftopl]    TFM to plain text property list.
\item[vftovp]    Virtual font to virtual property list.
\item[vptovf]    Virtual property list to virtual font.
\item[weave]     \web{} to \TeX.
\end{cmddescription}

\noindent The precise functions and syntax of these programs are
described in the documentation of the individual packages and of \Webc{}
itself.  However, knowing a few principles governing the whole family of
programs will help you take advantage of your \Webc{} installation.

All programs honor these standard \GNU options:
\begin{ttdescription}
\item[-{}-help] print basic usage summary.
\item[-{}-verbose] print detailed progress report.
\item[-{}-version] print version information, then exit.
\end{ttdescription}

For locating files the \Webc{} programs use the path searching library
\KPS{} (\url{http://tug.org/kpathsea}).  This library uses a combination
of environment variables and a configuration files to optimize searching
the (huge) collection of \TeX{} files.  \Webc{} can look at many
directory trees simultaneously, which is useful in maintaining \TeX's
standard distribution and local and personal extensions in distinct
trees.  To speed up file searches, the root of each tree has a file
\file{ls-R}, containing an entry showing the name and relative pathname
for all files under that root.


\subsection{Kpathsea path searching}
\label{sec:kpathsea}

Let us first describe the generic path searching mechanism of the \KPS{}
library.

We call a \emph{search path} a colon- or semicolon\hyph sepa\-rated list
of \emph{path elements}, which are basically directory names.  A
search path can come from (a combination of) many sources.  To look up
a file \samp{my-file} along a path \samp{.:/dir}, \KPS{} checks each
element of the path in turn: first \file{./my-file}, then
\file{/dir/my-file}, returning the first match (or possibly all
matches).

In order to adapt optimally to all operating systems' conventions, on
non-Unix systems \KPS{} can use filename separators different from
colon (\samp{:}) and slash (\samp{/}).

To check a particular path element \var{p}, \KPS{} first checks if a
prebuilt database (see ``Filename data\-base'' on
page~\pageref{sec:filename-database}) applies to \var{p}, i.e., if the
database is in a directory that is a prefix of \var{p}.  If so, the path
specification is matched against the contents of the database.

If the database does not exist, or does not apply to this path
element, or contains no matches, the filesystem is searched (if this
was not forbidden by a specification starting with \samp{!!}\ and if
the file being searched for must exist).  \KPS{} constructs the list
of directories that correspond to this path element, and then checks
in each for the file being sought.

The ``file must exist'' condition comes into play with \samp{.vf} files and
input files read by \TeX's \cs{openin} command.  Such files may not
exist (e.g., \file{cmr10.vf}), and so it would be wrong to search the
disk for them.  Therefore, if you fail to update \file{ls-R} when you
install a new \samp{.vf} file, it will never be found.
Each path element is checked in turn: first the database, then the
disk.  If a match is found, the search stops and the result is
returned.

Although the simplest and most common path element is a directory
name, \KPS{} supports additional features in search paths: layered
default values, environment variable names, config file values, users'
home directories, and recursive subdirectory searching.  Thus, we say
that \KPS{} \emph{expands} a path element, meaning it transforms all
the specifications into basic directory name or names.  This is
described in the following sections in the same order as it takes
place.

Note that if the filename being searched for is absolute or explicitly
relative, i.e., starts with \samp{/} or \samp{./} or \samp{../},
\KPS{} simply checks if that file exists.

\ifSingleColumn
\else
\begin{figure*}
\verbatiminput{examples/ex5.tex}
\setlength{\abovecaptionskip}{0pt}
  \caption{An illustrative configuration file sample}
  \label{fig:config-sample}
\end{figure*}
\fi

\subsubsection{Path sources}
\label{sec:path-sources}

A search path can come from many sources.  In the order in which
\KPS{} uses them:

\begin{enumerate}
\item
  A user-set environment variable, for instance, \envname{TEXINPUTS}\@.
  Environment variables with a period and a program name appended
  override; e.g., if \samp{latex} is the name of the program being run,
  then \envname{TEXINPUTS.latex} will override \envname{TEXINPUTS}.
\item
  A program-specific configuration file, for exam\-ple, a line
  \samp{S /a:/b} in \cmdname{dvips}'s \file{config.ps}.
\item   A \KPS{} configuration file \file{texmf.cnf}, containing a line like
  \samp{TEXINPUTS=/c:/d} (see below).
\item The compile-time default.
\end{enumerate}
\noindent You can see each of these values for a given search path by
using the debugging options (see ``Debugging actions'' on
page~\pageref{sec:debugging}).

\subsubsection{Config files}

\KPS{} reads \emph{runtime configuration files} named \file{texmf.cnf}
for search path and other definitions.  The search path used to look
for these files is named \envname{TEXMFCNF}, but we do not recommend
setting this (or any) environment variable.

Instead, normal installation results in a file
\file{.../2015/texmf.cnf}.  If you must make changes to the defaults
(not normally necessary), this is the place to put them.  The main
configuration file is in \file{.../2015/texmf-dist/web2c/texmf.cnf}.
You should not edit this latter file, as your changes will be lost when
the distributed version is updated.

\emph{All} \file{texmf.cnf} files in the search path will be read and
definitions in earlier files override those in later files.  For
example, with a search path of \verb|.:$TEXMF|, values from
\file{./texmf.cnf} override those from \verb|$TEXMF/texmf.cnf|.

\begin{itemize}
\item
  Comments start with \code{\%} and continue to the end of the line.
\item
  Blank lines are ignored.
\item
  A \bs{} at the end of a line acts as a continuation character,
  i.e., the next line is appended.  Whitespace at the beginning of
  continuation lines is not ignored.
\item
  Each remaining line has the form:
\begin{alltt}
  \var{variable}[.\var{progname}] [=] \var{value}
\end{alltt}%
  where the \samp{=} and surrounding whitespace are optional.
\item
  The \ttvar{variable} name may contain any character other
  than whitespace, \samp{=}, or \samp{.}, but sticking to
  \samp{A-Za-z\_} is safest.
\item
  If \samp{.\var{progname}} is present, the definition only
  applies if the program that is running is named
  \texttt{\var{progname}} or \texttt{\var{progname}.exe}.  This allows
  different flavors of \TeX{} to have different search paths, for
  example.
\item \var{value} may contain any characters except
  \code{\%} and \samp{@}.  The
  \code{\$\var{var}.\var{prog}} feature is not available on the
  right-hand side; instead, you must use an additional variable.  A
  \samp{;}\ in \var{value} is translated to \samp{:}\ if
  running under Unix; this is useful to be able to have a single
  \file{texmf.cnf} for Unix, \acro{MS-DOS} and Windows systems.
\item
  All definitions are read before anything is expanded, so
  variables can be referenced before they are defined.
\end{itemize}
A configuration file fragment illustrating most of these points is
\ifSingleColumn
shown below:

\verbatiminput{examples/ex5.tex}
\else
shown in Figure~\ref{fig:config-sample}.
\fi

\subsubsection{Path expansion}
\label{sec:path-expansion}


\KPS{} recognizes certain special characters and constructions in
search paths, similar to those available in Unix shells.  As a
general example, the  complex path,
\verb+~$USER/{foo,bar}//baz+, expands to all subdirectories under
directories \file{foo} and \file{bar} in \texttt{\$USER}'s home
directory that contain a directory or file \file{baz}.  These
expansions are explained in the sections below.
%$
\subsubsection{Default expansion}
\label{sec:default-expansion}

If the highest-priority search path (see ``Path sources'' on
page~\pageref{sec:path-sources}) contains an \emph{extra colon} (i.e.,
leading, trailing, or doubled), \KPS{} inserts at that point the
next-highest-prio\-rity search path that is defined.  If that inserted
path has an extra colon, the same happens with the next highest.  For
example, given an environment variable setting

\begin{alltt}
> \Ucom{setenv TEXINPUTS /home/karl:}
\end{alltt}
and a \code{TEXINPUTS} value from \file{texmf.cnf} of

\begin{alltt}
  .:\$TEXMF//tex
\end{alltt}
then the final value used for searching will be:

\begin{alltt}
  /home/karl:.:\$TEXMF//tex
\end{alltt}

Since it would be useless to insert the default value in more than one
place, \KPS{} changes only one extra \samp{:}\ and leaves any others in
place.  It checks first for a leading \samp{:}, then a trailing
\samp{:}, then a doubled \samp{:}.

\subsubsection{Brace expansion}
\label{sec:brace-expansion}

A useful feature is brace expansion, which means that, for instance,
\verb+v{a,b}w+ expands to \verb+vaw:vbw+. Nesting is allowed.
This is used to implement multiple \TeX{} hierarchies, by
assigning a brace list to \code{\$TEXMF}.
For example, in \file{texmf.cnf}, a definition like this
(simplified for this example) is made:
\begin{verbatim}
  TEXMF = {$TEXMFVAR,$TEXMFHOME,!!$TEXMFLOCAL,!!$TEXMFDIST}
\end{verbatim}
We can then use this to define, for example, the \TeX\ input path:
\begin{verbatim}
  TEXINPUTS = .;$TEXMF/tex//
\end{verbatim}
%$
which means that, after looking in the current directory, the
\code{\$TEXMFVAR/tex}, \code{\$TEXMFHOME/tex}, \code{\$TEXMFLOCAL/tex}
and \code{\$TEXMFDIST/tex} trees \emph{only}) will be searched (the
last two using \file{ls-R} data base files). It is a convenient
way for running two parallel \TeX{} structures, one ``frozen'' (on a
\CD, for instance) and the other being continuously updated with new
versions as they become available.  By using the \code{\$TEXMF}
variable in all definitions, one is sure to always search the
up-to-date tree first.

\subsubsection{Subdirectory expansion}
\label{sec:subdirectory-expansion}

Two or more consecutive slashes in a path element following a directory
\var{d} is replaced by all subdirectories of \var{d}: first those
subdirectories directly under \var{d}, then the subsubdirectories under
those, and so on.  At each level, the order in which the directories are
searched is \emph{unspecified}.

If you specify any filename components after the \samp{//}, only
subdirectories with matching components are included.  For example,
\samp{/a//b} expands into directories \file{/a/1/b}, \file{/a/2/b},
\file{/a/1/1/b}, and so on, but not \file{/a/b/c} or \file{/a/1}.

Multiple \samp{//} constructs in a path are possible, but
\samp{//} at the beginning of a path is ignored.

\subsubsection{List of special characters and their meaning: a summary}

The following list summarizes the special characters in \KPS{}
configuration files.

% need a wider space for the item labels here.
\newcommand{\CODE}[1]{\makebox[3em][l]{\code{#1}}}
\begin{ttdescription}
\item[\CODE{:}] Separator in path specification; at the beginning or
  the end of a path it substitutes the default path expansion.\par
\item[\CODE{;}] Separator on non-Unix systems (acts like \code{:}).
\item[\CODE{\$}] Variable expansion.
\item[\CODE{\string~}] Represents the user's home directory.
\item[\CODE{\char`\{...\char`\}}] Brace expansion.
\item[\CODE{//}] Subdirectory expansion (can occur anywhere in
    a path, except at its beginning).
\item[\CODE{\%}] Start of comment.
\item[\CODE{\bs}] Continuation character (allows multi-line entries).
\item[\CODE{!!}] Search \emph{only} database to locate file, \emph{do
    not} search the disk.
\end{ttdescription}


\subsection{Filename databases}
\label{sec:filename-database}

\KPS{} goes to some lengths to minimize disk accesses for searches.
Nevertheless, at installations with enough directories, searching each
possible directory for a given file can take an excessively long time
(this is especially true if many hundreds of font directories have to
be traversed.)  Therefore, \KPS{} can use an externally-built plain text
``database'' file named \file{ls-R} that maps files to directories,
thus avoiding the need to exhaustively search the disk.

A second database file \file{aliases} allows you to give additional
names to the files listed in \file{ls-R}.  This can be helpful to
confirm to \acro{DOS} 8.3 filename conventions in source files.

\subsubsection{The filename database}
\label{sec:ls-R}

As explained above, the name of the main filename database must be
\file{ls-R}.  You can put one at the root of each \TeX{} hierarchy in
your installation that you wish to be searched (\code{\$TEXMF} by
default).  \KPS{} looks for
\file{ls-R} files along the \code{TEXMFDBS} path.

The recommended way to create and maintain \samp{ls-R} is to run the
\code{mktexlsr} script included with the distribution. It is invoked
by the various \samp{mktex}\dots\ scripts.  In principle, this script
just runs the command
\begin{alltt}
cd \var{/your/texmf/root} && \path|\|ls -1LAR ./ >ls-R
\end{alltt}
presuming your system's \code{ls} produces the right output format
(\GNU \code{ls} is all right).  To ensure that the database is
always up-to-date, it is easiest to rebuild it regularly via
\code{cron}, so that it is automatically updated when the installed
files change, such as after installing or updating a \LaTeX{} package.

If a file is not found in the database, by default \KPS{} goes ahead
and searches the disk. If a particular path element begins with
\samp{!!}, however, \emph{only} the database will be searched for that
element, never the disk.


\subsubsection{kpsewhich: Standalone path searching}
\label{sec:invoking-kpsewhich}

The \texttt{kpsewhich} program exercises path searching independent of any
particular application.  This can be useful as a sort of \code{find}
program to locate files in \TeX{} hierarchies (this is used heavily in
the distributed \samp{mktex}\dots\ scripts).

\begin{alltt}
> \Ucom{kpsewhich \var{option}\dots{} \var{filename}\dots{}}
\end{alltt}
The options specified in \ttvar{option} start with either \samp{-}
or \samp{-{}-}, and any unambiguous abbreviation is accepted.

\KPS{} looks up each non-option argument on the command line as a
filename, and returns the first file found. There is no option to
return all the files with a particular name (you can run the Unix
\samp{find} utility for that).

The most common options are described next.

\begin{ttdescription}
\item[\texttt{-{}-dpi=\var{num}}]\mbox{}
  Set the resolution to \ttvar{num}; this only affects \samp{gf}
  and \samp{pk} lookups.  \samp{-D} is a synonym, for compatibility
  with \cmdname{dvips}.  Default is 600.

\item[\texttt{-{}-format=\var{name}}]\mbox{}\\
  Set the format for lookup to \ttvar{name}.  By default, the
  format is guessed from the filename. For formats which do not have
  an associated unambiguous suffix, such as \MP{} support files and
  \cmdname{dvips} configuration files, you have to specify the name as
  known to \KPS{}, such as \texttt{tex} or \texttt{enc files}.  Run
  \texttt{kpsewhich -{}-help} for a list.

\item[\texttt{-{}-mode=\var{string}}]\mbox{}\\
  Set the mode name to \ttvar{string}; this only affects \samp{gf}
  and \samp{pk} lookups.  No default: any mode will be found.
\item[\texttt{-{}-must-exist}]\mbox{}\\
  Do everything possible to find the files, notably including
  searching the disk.  By default, only the \file{ls-R} database is
  checked, in the interest of efficiency.
\item[\texttt{-{}-path=\var{string}}]\mbox{}\\
  Search along the path \ttvar{string} (colon-separated as usual),
  instead of guessing the search path from the filename.  \samp{//} and
  all the usual expansions are supported.  The options \samp{-{}-path}
  and \samp{-{}-format} are mutually exclusive.
\item[\texttt{-{}-progname=\var{name}}]\mbox{}\\
  Set the program name to \texttt{\var{name}}.
  This can affect the search paths via the \texttt{.\var{progname}}
  feature.
  The default is \cmdname{kpsewhich}.
\item[\texttt{-{}-show-path=\var{name}}]\mbox{}\\
  shows the path used for file lookups of file type \texttt{\var{name}}.
  Either a filename extension (\code{.pk}, \code{.vf}, etc.) or a
  name can be used, just as with \samp{-{}-format} option.
\item[\texttt{-{}-debug=\var{num}}]\mbox{}\\
  sets the debugging options to \texttt{\var{num}}.
\end{ttdescription}


\subsubsection{Examples of use}
\label{sec:examples-of-use}

Let us now have a look at \KPS{} in action.  Here's a straightforward search:

\begin{alltt}
> \Ucom{kpsewhich article.cls}
   /usr/local/texmf-dist/tex/latex/base/article.cls
\end{alltt}
We are looking for the file \file{article.cls}. Since the \samp{.cls}
suffix is unambiguous we do not need to specify that we want to look for a
file of type \optname{tex} (\TeX{} source file directories). We find it in
the subdirectory \file{tex/latex/base} below the \samp{texmf-dist} \TL\
directory.  Similarly, all of the following are found without problems
thanks to their unambiguous suffix.
\begin{alltt}
> \Ucom{kpsewhich array.sty}
   /usr/local/texmf-dist/tex/latex/tools/array.sty
> \Ucom{kpsewhich latin1.def}
   /usr/local/texmf-dist/tex/latex/base/latin1.def
> \Ucom{kpsewhich size10.clo}
   /usr/local/texmf-dist/tex/latex/base/size10.clo
> \Ucom{kpsewhich small2e.tex}
   /usr/local/texmf-dist/tex/latex/base/small2e.tex
> \Ucom{kpsewhich tugboat.bib}
   /usr/local/texmf-dist/bibtex/bib/beebe/tugboat.bib
\end{alltt}

By the way, that last is a \BibTeX{} bibliography database for
\textsl{TUGboat} articles.

\begin{alltt}
> \Ucom{kpsewhich cmr10.pk}
\end{alltt}
Font bitmap glyph files of type \file{.pk} are used by display
programs like \cmdname{dvips} and \cmdname{xdvi}.  Nothing is returned in
this case since there are no pre-generated Computer Modern \samp{.pk}
files in \TL{}\Dash the Type~1 variants are used by default.
\begin{alltt}
> \Ucom{kpsewhich wsuipa10.pk}
\ifSingleColumn   /usr/local/texmf-var/fonts/pk/ljfour/public/wsuipa/wsuipa10.600pk
\else /usr/local/texmf-var/fonts/pk/ljfour/public/
...                         wsuipa/wsuipa10.600pk
\fi\end{alltt}
For these fonts (a phonetic alphabet from the University of Washington)
we had to generate \samp{.pk} files, and since the default \MF{} mode on
our installation is \texttt{ljfour} with a base resolution of 600\dpi{}
(dots per inch), this instantiation is returned.
\begin{alltt}
> \Ucom{kpsewhich -dpi=300 wsuipa10.pk}
\end{alltt}
In this case, when specifying that we are interested in a resolution
of 300\dpi{} (\texttt{-dpi=300}) we see that no such font is available on
the system. A program like \cmdname{dvips} or \cmdname{xdvi} would
go off and actually build the required \texttt{.pk} files
using the script \cmdname{mktexpk}.

Next we turn our attention to \cmdname{dvips}'s header and configuration
files.  We first look at one of the commonly used files, the general
prologue \file{tex.pro} for \TeX{} support, before turning our attention
to the generic configuration file (\file{config.ps}) and the \PS{} font
map \file{psfonts.map}\Dash as of 2004, map and encoding files have
their own search paths and new location in \dirname{texmf} trees.  As
the \samp{.ps} suffix is ambiguous we have to specify explicitly which
type we are considering (\optname{dvips config}) for the file
\texttt{config.ps}.
\begin{alltt}
> \Ucom{kpsewhich tex.pro}
   /usr/local/texmf/dvips/base/tex.pro
> \Ucom{kpsewhich --format="dvips config" config.ps}
   /usr/local/texmf/dvips/config/config.ps
> \Ucom{kpsewhich psfonts.map}
   /usr/local/texmf/fonts/map/dvips/updmap/psfonts.map
\end{alltt}

We now take a closer look at the \acro{URW} Times \PS{} support
files.  The prefix for these in the standard font naming scheme is
\samp{utm}.  The first file we look at is the configuration file,
which contains the name of the map file:
\begin{alltt}
> \Ucom{kpsewhich --format="dvips config" config.utm}
   /usr/local/texmf-dist/dvips/psnfss/config.utm
\end{alltt}
The contents of that file is
\begin{alltt}
  p +utm.map
\end{alltt}
which points to the file \file{utm.map}, which we want to
locate next.
\begin{alltt}
> \Ucom{kpsewhich utm.map}
   /usr/local/texmf-dist/fonts/map/dvips/times/utm.map
\end{alltt}
This map file defines the file names of the Type~1 \PS{} fonts in
the URW collection.  Its contents look like (we only show part of the
lines):
\begin{alltt}
utmb8r  NimbusRomNo9L-Medi    ... <utmb8a.pfb
utmbi8r NimbusRomNo9L-MediItal... <utmbi8a.pfb
utmr8r  NimbusRomNo9L-Regu    ... <utmr8a.pfb
utmri8r NimbusRomNo9L-ReguItal... <utmri8a.pfb
utmbo8r NimbusRomNo9L-Medi    ... <utmb8a.pfb
utmro8r NimbusRomNo9L-Regu    ... <utmr8a.pfb
\end{alltt}
Let us, for instance, take the Times Roman instance
\file{utmr8a.pfb} and find its position in the \file{texmf} directory
tree with a search for Type~1 font files:
\begin{alltt}
> \Ucom{kpsewhich utmr8a.pfb}
\ifSingleColumn   /usr/local/texmf-dist/fonts/type1/urw/times/utmr8a.pfb
\else   /usr/local/texmf-dist/fonts/type1/
... urw/utm/utmr8a.pfb
\fi\end{alltt}

It should be evident from these examples how you can easily locate the
whereabouts of a given file. This is especially important if you suspect
that the wrong version of a file is picked up somehow, since
\cmdname{kpsewhich} will show you the first file encountered.

\subsubsection{Debugging actions}
\label{sec:debugging}

Sometimes it is necessary to investigate how a program resolves file
references. To make this practical, \KPS{} offers various levels of
debugging output:

\begin{ttdescription}
\item[\texttt{\ 1}] \texttt{stat} calls (disk lookups). When running
  with an up-to-date \file{ls-R} database this should almost give no
  output.
\item[\texttt{\ 2}] References to hash tables (such as \file{ls-R}
  databases, map files, configuration files).
\item[\texttt{\ 4}] File open and close operations.
\item[\texttt{\ 8}] General path information for file types
  searched by \KPS. This is useful to find out where a particular
  path for the file was defined.
\item[\texttt{16}] Directory list for each path element (only relevant
  for searches on disk).
\item[\texttt{32}] File searches.
\item[\texttt{64}] Variable values.
\end{ttdescription}
A value of \texttt{-1} will set all the above options; in practice,
this is usually the most convenient.

Similarly, with the \cmdname{dvips} program, by setting a combination of
debug switches, one can follow in detail where files are being picked up
from.  Alternatively, when a file is not found, the debug trace shows in
which directories the program looks for the given file, so that one can
get an indication what the problem~is.

Generally speaking, as most programs call the \KPS{} library
internally, one can select a debug option by using the
\envname{KPATHSEA\_DEBUG} environment variable, and setting it to (a
combination of) values as described in the above list.

(Note for Windows users: it is not easy to redirect
all messages to a file in this system. For diagnostic purposes
you can temporarily \texttt{SET KPATHSEA\_DEBUG\_OUTPUT=err.log}).

Let us consider, as an example, a small \LaTeX{} source file,
\file{hello-world.tex}, which contains the following input.
\begin{verbatim}
  \documentclass{article}
  \begin{document}
  Hello World!
  \end{document}
\end{verbatim}
This little file only uses the font \file{cmr10}, so let us look at
how \cmdname{dvips} prepares the \PS{} file (we want to use the Type~1
version of the Computer Modern fonts, hence the option \texttt{-Pcms}).
\begin{alltt}
> \Ucom{dvips -d4100 hello-world -Pcms -o}
\end{alltt}
In this case we have combined \cmdname{dvips}'s debug class 4 (font
paths) with \KPS's path element expansion (see the \cmdname{dvips}
reference manual).
The output (slightly rearranged) appears in
Figure~\ref{fig:dvipsdbga}.
\begin{figure*}[tp]
\centering
\begin{BVerbatim}[fontsize=\small]
debug:start search(file=texmf.cnf, must_exist=1, find_all=1,
  path=.:/usr/local/bin/texlive:/usr/local/bin:
       /usr/local/bin/texmf/web2c:/usr/local:
       /usr/local/texmf/web2c:/.:/./teTeX/TeX/texmf/web2c:).
kdebug:start search(file=ls-R, must_exist=1, find_all=1,
  path=~/tex:/usr/local/texmf).
kdebug:search(ls-R) =>/usr/local/texmf/ls-R
kdebug:start search(file=aliases, must_exist=1, find_all=1,
  path=~/tex:/usr/local/texmf).
kdebug:search(aliases) => /usr/local/texmf/aliases
kdebug:start search(file=config.ps, must_exist=0, find_all=0,
  path=.:~/tex:!!/usr/local/texmf/dvips//).
kdebug:search(config.ps) => /usr/local/texmf/dvips/config/config.ps
kdebug:start search(file=/root/.dvipsrc, must_exist=0, find_all=0,
  path=.:~/tex:!!/usr/local/texmf/dvips//).
search(file=/home/goossens/.dvipsrc, must_exist=1, find_all=0,
  path=.:~/tex/dvips//:!!/usr/local/texmf/dvips//).
kdebug:search($HOME/.dvipsrc) =>
kdebug:start search(file=config.cms, must_exist=0, find_all=0,
  path=.:~/tex/dvips//:!!/usr/local/texmf/dvips//).
kdebug:search(config.cms)
=>/usr/local/texmf/dvips/cms/config.cms
\end{BVerbatim}
\caption{Finding configuration files}\label{fig:dvipsdbga}

\bigskip

\begin{BVerbatim}[fontsize=\small]
kdebug:start search(file=texc.pro, must\_exist=0, find\_all=0,
  path=.:~/tex/dvips//:!!/usr/local/texmf/dvips//:
       ~/tex/fonts/type1//:!!/usr/local/texmf/fonts/type1//).
kdebug:search(texc.pro) => /usr/local/texmf/dvips/base/texc.pro
\end{BVerbatim}
\caption{Finding the prolog file}\label{fig:dvipsdbgb}

\bigskip

\begin{BVerbatim}[fontsize=\small]
kdebug:start search(file=cmr10.tfm, must\_exist=1, find\_all=0,
  path=.:~/tex/fonts/tfm//:!!/usr/local/texmf/fonts/tfm//:
       /var/tex/fonts/tfm//).
kdebug:search(cmr10.tfm) => /usr/local/texmf/fonts/tfm/public/cm/cmr10.tfm
kdebug:start search(file=texps.pro, must\_exist=0, find\_all=0,
   ...
<texps.pro>
kdebug:start search(file=cmr10.pfb, must\_exist=0, find\_all=0,
  path=.:~/tex/dvips//:!!/usr/local/texmf/dvips//:
       ~/tex/fonts/type1//:!!/usr/local/texmf/fonts/type1//).
kdebug:search(cmr10.pfb) => /usr/local/texmf/fonts/type1/public/cm/cmr10.pfb
<cmr10.pfb>[1]
\end{BVerbatim}
\caption{Finding the font file}\label{fig:dvipsdbgc}
\end{figure*}

\cmdname{dvips} starts by locating its working files. First,
\file{texmf.cnf} is found, which gives the definitions of the search
paths for the other files, then the file database \file{ls-R} (to
optimize file searching) and the file \file{aliases}, which makes it
possible to declare several names (e.g., a short \acro{DOS}-like 8.3 and
a more natural longer version) for the same file.  Then \cmdname{dvips}
goes on to find the generic configuration file \file{config.ps}
before looking for the customization file \file{.dvipsrc} (which, in
this case is \emph{not found}).  Finally, \cmdname{dvips} locates the
config file for the Computer Modern \PS{} fonts \file{config.cms}
(this was initiated with the \texttt{-Pcms} option on the \cmdname{dvips}
command).  This file contains the list of the map files which
define the relation between the \TeX{}, \PS{} and file system
names of the fonts.
\begin{alltt}
> \Ucom{more /usr/local/texmf/dvips/cms/config.cms}
   p +ams.map
   p +cms.map
   p +cmbkm.map
   p +amsbkm.map
\end{alltt}
\cmdname{dvips} thus goes on to find all these files, plus the generic
map file \file{psfonts.map}, which is always loaded (it contains
declarations for commonly used \PS{} fonts; see the last part of
Section \ref{sec:examples-of-use} for more details about \PS{} map
file handling).

At this point \cmdname{dvips} identifies itself to the user:
\begin{alltt}
This is dvips(k) 5.92b Copyright 2002 Radical Eye Software (www.radicaleye.com)
\end{alltt}
\ifSingleColumn
Then it goes on to look for the prolog file \file{texc.pro}:
\begin{alltt}\small
kdebug:start search(file=texc.pro, must\_exist=0, find\_all=0,
  path=.:~/tex/dvips//:!!/usr/local/texmf/dvips//:
       ~/tex/fonts/type1//:!!/usr/local/texmf/fonts/type1//).
kdebug:search(texc.pro) => /usr/local/texmf/dvips/base/texc.pro
\end{alltt}
\else
Then it goes on to look for the prolog file \file{texc.pro} (see
Figure~\ref{fig:dvipsdbgb}).
\fi

After having found the file in question, \cmdname{dvips} outputs
the date and time, and informs us that it will generate the
file \file{hello-world.ps}, then that it needs the font file
\file{cmr10}, and that the latter is declared as ``resident'' (no
bitmaps needed):
\begin{alltt}\small
TeX output 1998.02.26:1204' -> hello-world.ps
Defining font () cmr10 at 10.0pt
Font cmr10 <CMR10> is resident.
\end{alltt}
Now the search is on for the file \file{cmr10.tfm}, which is found,
then a few more prolog files (not shown) are referenced, and finally
the Type~1 instance \file{cmr10.pfb} of the font is located and
included in the output file (see last line).
\begin{alltt}\small
kdebug:start search(file=cmr10.tfm, must\_exist=1, find\_all=0,
  path=.:~/tex/fonts/tfm//:!!/usr/local/texmf/fonts/tfm//:
       /var/tex/fonts/tfm//).
kdebug:search(cmr10.tfm) => /usr/local/texmf/fonts/tfm/public/cm/cmr10.tfm
kdebug:start search(file=texps.pro, must\_exist=0, find\_all=0,
   ...
<texps.pro>
kdebug:start search(file=cmr10.pfb, must\_exist=0, find\_all=0,
  path=.:~/tex/dvips//:!!/usr/local/texmf/dvips//:
       ~/tex/fonts/type1//:!!/usr/local/texmf/fonts/type1//).
kdebug:search(cmr10.pfb) => /usr/local/texmf/fonts/type1/public/cm/cmr10.pfb
<cmr10.pfb>[1]
\end{alltt}

\subsection{Runtime options}

Another useful feature of \Webc{} is its possibility to control a number
of memory parameters (in particular, array sizes) via the runtime file
\file{texmf.cnf} read by \KPS{}.  The memory settings can be found in
Part~3 of that file in the \TL{} distribution. The more important
are:

\begin{ttdescription}
\item[\texttt{main\_memory}]
  Total words of memory available, for
  \TeX{}, \MF{} and \MP.  You must make a new format file for each
  different setting. For instance, you could generate a ``huge''
  version of \TeX{}, and call the format file \texttt{hugetex.fmt}.
  Using the standard way of specifying the program name used by \KPS{},
  the particular value of the \texttt{main\_memory} variable will then
  be read from \file{texmf.cnf}.
\item[\texttt{extra\_mem\_bot}]
  Extra space for ``large'' \TeX{} data structures:
  boxes, glue, breakpoints, etc.  Especially useful if you use
  \PiCTeX{}.
\item[\texttt{font\_mem\_size}]
  Number of words for font information available for \TeX. This
  is more or less the total size of all \acro{TFM} files read.
\item[\texttt{hash\_extra}]
  Additional space for the hash table of control sequence names.
  Only $\approx$10,000 control sequences can be stored in the main
  hash table; if you have a large book with numerous cross-references,
  this might not be enough.  The default value of
  \texttt{hash\_extra} is \texttt{50000}.
\end{ttdescription}

\noindent Of course, this facility is no substitute for truly dynamic
arrays and memory allocation, but since these are extremely difficult to
implement in the present \TeX{} source, these runtime parameters provide
a practical compromise allowing some flexibility.


\begin{comment}
\section{Building on a new Unix platform}

If you have a platform for which executables are not included, you will
need to compile \TeX{} and friends. This is not as hard as it
sounds. What you need is all in the directory \texttt{source} in the
distribution.

\subsection{Prerequisites}

You will need at least 100 megabytes of disk space to compile all of
\TeX{} and its support programs. You'll also need an \acro{ANSI} C
compiler, a \cmdname{make} utility, a lexical scanner, and a parser
generator. We recommend the \GNU version of these programs
(\cmdname{gcc}, \GNU \cmdname{make}, \cmdname{m4}, \cmdname{flex},
\cmdname{bison}).  You may be able to work with other C compilers and
\cmdname{make} programs, but you will need a good understanding of
building Unix programs to sort out problems.

Also, the command \texttt{uname} must return a sensible value.


\subsection{Configuration}

To begin, perform a normal installation of \TL{} to your disk (see
section~\ref{sec:install-disk} on
\p.\pageref{sec:install-disk}).  You may wish to skip installing
all of the prebuilt binaries.

Then, unpack the source from the compressed \texttt{tar} file in the
directory \dirname{source} to your disk and change directory to where
you placed it.

Next, run \cmdname{configure} with a command line like this:
\begin{alltt}
> \Ucom{sh configure -prefix=/usr/local/TeX}
\end{alltt}

The \optname{-prefix} directory is the one where you installed the
support tree; the directory layout will be as follows (where \$TEXDIR
stands for the directory you chose):

\noindent
\begin{tabular}{>{\ttfamily}ll@{}}
  \dirname{$TEXDIR/share/texmf}          & main tree with fonts,\\
                                         & \qquad macros, etc\\
  \dirname{$TEXDIR/man}                  & Unix manual pages\\
  \dirname{$TEXDIR/info}                 & \GNU style Info manuals\\
  \dirname{$TEXDIR/bin/$PLATFORM}        & binaries\\
\end{tabular}
%$

If you want to leave out the \dirname{$PLATFORM} directory level,
i.e., put the binaries directly into \dirname{$TEXDIR/bin}, specify
the \verb|--disable-multiplatform| option to \cmdname{configure}.

Have a look at the output of \verb|./configure --help| for more
options you can use.  For example, you can skip building of \OMEGA{} and
\eTeX{}.


\subsection{Running \cmdname{make}}

Make sure the shell variable or option \texttt{noclobber} is not set.
Then, run the main \cmdname{make} like this:
\begin{alltt}
> \Ucom{make world}
\end{alltt}
and relax\ldots

Alternatively, you want to log all the output, as in:
\begin{alltt}
> \Ucom{sh -c "make world >world.log 2>\&1" \&}
\end{alltt}

Before you believe that everything went ok, please check the log file
for errors: \GNU \cmdname{make} always uses the string \samp{***}
whenever a command fails.  Also, check if all the programs were built:

\begin{alltt}
> \Ucom{cd \var{TEXDIR}/bin/\var{archname}}
> \Ucom{ls | wc}
\end{alltt}
The result should be over 200 (you can check the exact number with the
\dirname{bin} directory contents in the distribution).

If you need special privileges for \texttt{make install}, you can
separate the \samp{make world} into two different runs, like this:
\begin{alltt}
> \Ucom{make all}
> \Ucom{su}
> \Ucom{make install strip}
\end{alltt}

After you've installed your new binaries, you should follow the normal
post-installation procedures, given in section~\ref{sec:postinstall}
(\p.\pageref{sec:postinstall}).

Also, if you'd like to make your binaries available to others, please
contact us.  We'll be happy to put them on the \TL\ web pages.
\end{comment}


\htmlanchor{ack}
\section{Acknowledgements}

\TL{} is a joint effort by virtually all of the \TeX{} user groups.
This edition of \TL{} was overseen by Karl Berry.  The other principal
contributors, past and present, are listed below.

\begin{itemize*}

\item The English, German, Dutch, and Polish \TeX{} user groups
(\acro{TUG}, \acro{DANTE} e.V., \acro{NTG}, and \acro{GUST},
respectively), which provide the necessary technical and administrative
infrastructure.  Please join the \TeX\ user group near you!  (See
\url{http://tug.org/usergroups.html}.)

\item The \acro{CTAN} team, notably Robin Fairbairns and Rainer
Sch\"opf, which distributes the \TL{} images and provides the common
infrastructure for package updates, upon which \TL{} depends.

\item Nelson Beebe, for making many platforms available to \TL\
developers, and his own comprehensive testing.

\item John Bowman, for making many changes to his advanced graphics
program Asymptote to make it work in \TL.

\item Peter Breitenlohner and the \eTeX\ team for the stable foundation
of future \TeX's, and Peter specifically for stellar help with \GNU\
autotools usage throughout \TL.

\item Jin-Hwan Cho and all of the DVIPDFM$x$ team, for their
excellent driver and responsiveness to configuration issues.

\item Thomas Esser, without whose marvelous \teTeX{} package \TL{}
would have never existed.

\item Michel Goossens, who co-authored the original documentation.

\item Eitan Gurari, whose \TeX4ht is used to create the \HTML{}
version of this documentation, and who worked tirelessly to improve it
at short notice, every year.  Eitan prematurely passed away in June
2009, and we dedicate this documentation to his memory.

\item Hans Hagen, for much testing and making his \ConTeXt\ format
(\url{http://pragma-ade.com}) work within \TL's framework.

\item \Thanh, Martin Schr\"oder, and the pdf\TeX\ team
(\url{http://pdftex.org}) for continuing enhancements of \TeX's
abilities.

\item Hartmut Henkel, for significant development contributions to
pdf\TeX\, Lua\TeX, and more.

\item Taco Hoekwater, for major renewed development efforts on MetaPost and
(Lua)\TeX\ (\url{http://luatex.org}) itself, incorporating
\ConTeXt\ into \TL, giving Kpathsea multi-threaded functionality, and
much more.

\item Pawe{\l} Jackowski, for the Windows installer \cmdname{tlpm},
and Tomasz {\L}uczak, for \cmdname{tlpmgui}, used in past releases.

\item Akira Kakuto, for providing the Windows
binaries from his \acro{W32TEX} distribution for Japanese \TeX\
(\url{http://w32tex.org}), and many other development contributions.

\item Jonathan Kew, for developing the remarkable \XeTeX{} engine and
taking the time and trouble to integrate it in \TL{}, as well as the
initial version of the Mac\TeX\ installer, and also for our recommended
front-end \TeX{}works.

\item Dick Koch, for maintaining Mac\TeX\ (\url{http://tug.org/mactex})
in very close tandem with \TL{}, and for his great good cheer in doing
so.

\item Reinhard Kotucha, for major contributions to the \TL{} 2008
infrastructure and installer, as well as Windows research efforts, the
\texttt{getnonfreefonts} script, and more.

\item Siep Kroonenberg, also for major contributions to the \TL{} 2008
infrastructure and installer, especially on Windows, and for the bulk of
work updating this manual describing those features.

\item Mojca Miklavec, for much help with \ConTeXt.

\item Heiko Oberdiek, for the \pkgname{epstopdf} package and many
others, compressing the huge \pkgname{pst-geo} data files so we could
include them, and most of all, for his remarkable work on
\pkgname{hyperref}.

\item Petr Ol\v{s}ak, who coordinated and checked all the Czech and Slovak
material very carefully.

\item Toshio Oshima, for his \cmdname{dviout} previewer for Windows.

\item Manuel P\'egouri\'e-Gonnard, for helping with package updates,
documentation improvements, and \cmdname{texdoc} development.

\item Fabrice Popineau, for the original Windows support in \TL{} and
work on the French documentation.

\item Norbert Preining, the principal architect of the \TL{} 2008
infrastructure and installer, and also for coordinating the Debian
version of \TL{} (together with Frank K\"uster), making many suggestions
along the way.

\item Sebastian Rahtz, for originally creating \TL{} and maintaining it
for many years.

\item Phil Taylor, for setting up the BitTorrent downloads.

\item Tomasz Trzeciak, for wide-ranging help with Windows.

\item Vladimir Volovich, for substantial help with porting and other
maintenance issues, and especially for making it feasible to include
\cmdname{xindy}.

\item Staszek Wawrykiewicz, the principal tester for all of \TL{},
and coordinator of the many major Polish contributions: fonts, Windows
installation, and more.

\item Olaf Weber, for his patient maintenance of \Webc.

\item Gerben Wierda, for creating and maintaining the original \MacOSX\
support.

\item Graham Williams, the originator of the \TeX\ Catalogue.

\end{itemize*}

Builders of the binaries:
Ettore Aldrovandi (\pkgname{i386-solaris}, \pkgname{x86\_64-solaris}),
Marc Baudoin (\pkgname{amd64-netbsd}, \pkgname{i386-netbsd}),
Karl Berry (\pkgname{i386-linux}),
Peter Breitenlohner (\pkgname{x86\_64-linux}),
Ken Brown (\pkgname{i386-cygwin}, \pkgname{x86\_64-cygwin}),
Simon Dales (\pkgname{armhf-linux}),
%Christian Mensing (\pkgname{mips-irix}),
Akira Kakuto (\pkgname{win32}),
Dick Koch (\pkgname{universal-darwin}, \pkgname{x86\_64-darwin}),
Nikola Le\v{c}i\'c (\pkgname{amd64-freebsd}, \pkgname{i386-freebsd}),
Mojca Miklavec (\pkgname{mipsel-linux}), \pkgname{sparc-solaris}),
Norbert Preining (\pkgname{alpha-linux}),
Thomas Schmitz (\pkgname{powerpc-linux}),
Boris Veytsman (\pkgname{armel-linux}).
For information on the \TL{} build process, see
\url{http://tug.org/texlive/build.html}.

Translators of this manual:
Boris Veytsman (Russian),
Jjgod Jiang, Jinsong Zhao, Yue Wang, \& Helin Gai (Chinese),
Uwe Ziegenhagen (German),
Manuel P\'egouri\'e-Gonnard (French),
Marco Pallante (Italian),
Nikola Le\v{c}i\'c (Serbian),
Petr Sojka \& Jan Busa (Czech\slash Slovak),
Staszek Wawrykiewicz (Polish).  The \TL{} documentation web page
is \url{http://tug.org/texlive/doc.html}.

Of course the most important acknowledgement must go to Donald Knuth,
first for inventing \TeX, and then for giving it to the world.


\section{Release history}
\label{sec:history}

\subsection{Past}

Discussion began in late 1993 when the Dutch \TeX{} Users Group was
starting work on its 4All\TeX{} \CD{} for \acro{MS-DOS} users, and it
was hoped at that time to issue a single, rational, \CD{} for all
systems. This was too ambitious a target for the time, but it did spawn
not only the very successful 4All\TeX{} \CD{}, but also the \acro{TUG}
Technical Council working group on a \emph{\TeX{} Directory Structure}
(\url{http://tug.org/tds}), which specified how to create consistent and
manageable collections of \TeX{} support files. A complete draft of the
\TDS{} was published in the December 1995 issue of \textsl{TUGboat}, and
it was clear from an early stage that one desirable product would be a
model structure on \CD{}. The distribution you now have is a very direct
result of the working group's deliberations. It was also clear that the
success of the 4All\TeX{} \CD{} showed that Unix users would benefit
from a similarly easy system, and this is the other main strand of
\TL.

We first undertook to make a new Unix-based \TDS{} \CD{} in the autumn
of 1995, and quickly identified Thomas Esser's \teTeX{} as the ideal
setup, as it already had multi-platform support and was built with
portability across file systems in mind. Thomas agreed to help, and work
began seriously at the start of 1996. The first edition was released in
May 1996. At the start of 1997, Karl Berry completed a major new release
of Web2c, which included nearly all the features which Thomas Esser had
added in \teTeX, and we decided to base the 2nd edition of the \CD{} on
the standard \Webc, with the addition of \teTeX's \texttt{texconfig}
script. The 3rd edition of the \CD{} was based on a major revision of
\Webc, 7.2, by Olaf Weber; at the same time, a new revision of \teTeX
was being made, and \TL{} included almost all of its features. The
4th edition followed the same pattern, using a new version of \teTeX,
and a new release of \Webc{} (7.3).  The system now included a complete
Windows setup.

For the 5th edition (March 2000) many parts of the \CD{} were revised
and checked, updating hundreds of packages. Package details were stored
in XML files. But the major change for \TeX\ Live 5 was that all
non-free software was removed. Everything in \TL{} is now intended
to be compatible with the Debian Free Software Guidelines
(\url{http://www.debian.org/intro/free}); we have done our best to check
the license conditions of all packages, but we would very much
appreciate hearing of any mistakes.

The 6th edition (July 2001) had much more material updated. The major
change was a new install concept: the user could select a more exact set
of needed collections. Language-related collections were completely
reorganized, so selecting any of them installs not only macros, fonts,
etc., but also prepares an appropriate \texttt{language.dat}.

The 7th edition of 2002 had the notable addition of \MacOSX{} support,
and the usual myriad of updates to all sorts of packages and
programs. An important goal was integration of the source back with
\teTeX, to correct the drift apart in versions~5 and~6.

\subsubsection{2003}

In 2003, with the continuing flood of updates and additions, we found
that \TL{} had grown so large it could no longer be contained on a
single \CD, so we split it into three different distributions (see
section~\ref{sec:tl-coll-dists}, \p.\pageref{sec:tl-coll-dists}).  In
addition:

\begin{itemize*}
\item At the request of the \LaTeX{} team, we changed the standard
      \cmdname{latex} and \cmdname{pdflatex} commands to now use \eTeX{} (see
      \p.\pageref{text:etex}).
\item The new Latin Modern fonts were included (and are recommended).
\item Support for Alpha \acro{OSF} was removed
      (\acro{HPUX} support was removed previously), since no one had (or
      volunteered) hardware available on which to compile new binaries.
\item Windows setup was substantially changed; for the first time
      an integrated environment based on XEmacs was introduced.
\item Important supplementary programs for Windows
      (Perl, Ghost\-script, Image\-Magick, Ispell) are now installed
      in the \TL{} installation directory.
\item Font map files used by \cmdname{dvips}, \cmdname{dvipdfm}
      and \cmdname{pdftex} are now generated by the new program
      \cmdname{updmap} and installed into \dirname{texmf/fonts/map}.
\item \TeX{}, \MF{}, and \MP{} now, by default, output most input
      characters (32 and above) as themselves in output (e.g.,
      \verb|\write|) files,
      log files, and the terminal, i.e., \emph{not} translated using the
      \verb|^^| notation.  In \TL{}~7, this translation was
      dependent on the system locale settings; now, locale settings do
      not influence the \TeX{} programs' behavior.  If for some reason
      you need the \verb|^^| output, rename the file
      \verb|texmf/web2c/cp8bit.tcx|.  (Future releases will have cleaner
      ways to control this.)
\item This documentation was substantially revised.
\item Finally, since the edition numbers had grown unwieldy,
      the version is now simply identified by the year: \TL{} 2003.
\end{itemize*}


\subsubsection{2004}

2004 saw many changes:

\begin{itemize}

\item If you have locally-installed fonts which use their own
\filename{.map} or (much less likely) \filename{.enc} support files, you
may need to move those support files.

\filename{.map} files are now searched for in subdirectories of
\dirname{fonts/map} only (in each \filename{texmf} tree), along the
\envname{TEXFONTMAPS} path.  Similarly, \filename{.enc} files are now
searched for in subdirectories of \dirname{fonts/enc} only, along the
\envname{ENCFONTS} path.  \cmdname{updmap} will attempt to warn about
problematic files.

For methods of handling this and other information, please see
\url{http://tug.org/texlive/mapenc.html}.

\item The \TK\ has been expanded with the addition of a \MIKTEX-based
installable \CD, for those who prefer that implementation to Web2C.
See section~\ref{sec:overview-tl} (\p.\pageref{sec:overview-tl}).

\item Within \TL, the single large \dirname{texmf} tree of previous
releases has been replaced by three: \dirname{texmf},
\dirname{texmf-dist}, and \dirname{texmf-doc}.  See
section~\ref{sec:tld} (\p.\pageref{sec:tld}), and the \filename{README}
files for each.

\item All \TeX-related input files are now collected in
the \dirname{tex} subdirectory of \dirname{texmf*} trees, rather than
having separate sibling directories \dirname{tex}, \dirname{etex},
\dirname{pdftex}, \dirname{pdfetex}, etc.  See
\CDref{texmf-dist/doc/generic/tds/tds.html\#Extensions}
{\texttt{texmf-dist/doc/generic/tds/tds.html\#Extensions}}.

\item Helper scripts (not meant to be invoked by users) are now located
in a new \dirname{scripts} subdirectory of \dirname{texmf*} trees, and
can be searched for via \verb|kpsewhich -format=texmfscripts|.  So if you have
programs which call such scripts, they'll need to be adjusted.  See
\CDref{texmf-dist/doc/generic/tds/tds.html\#Scripts}
{\texttt{texmf-dist/doc/generic/tds/tds.html\#Scripts}}.

\item Almost all formats leave most characters printable as
themselves via the ``translation file'' \filename{cp227.tcx}, instead of
translating them with the \verb|^^| notation.  Specifically, characters
at positions 32--256, plus tab, vertical tab, and form feed are
considered printable and not translated.  The exceptions are plain \TeX\
(only 32--126 printable), \ConTeXt\ (0--255 printable), and the
\OMEGA-related formats.  This default behavior is almost the same as in
\TL\,2003, but it's implemented more cleanly, with more possibilities
for customization.  See \CDref{texmf-dist/doc/web2c/web2c.html\#TCX-files}
{\texttt{texmf-dist/doc/web2c/web2c.html\#TCX-files}}.
(By the way, with Unicode input, \TeX\ may output partial character
sequences when showing error contexts, since it is byte-oriented.)

\item \textsf{pdfetex} is now the default engine for all formats
except (plain) \textsf{tex} itself.  (Of course it generates \acro{DVI}
when run as \textsf{latex}, etc.)  This means, among other things, that
the microtypographic features of \textsf{pdftex} are available in
\LaTeX, \ConTeXt, etc., as well as the \eTeX\ features
(\OnCD{texmf-dist/doc/etex/base/}).

It also means it's \emph{more important than ever} to use the
\pkgname{ifpdf} package (works with both plain and \LaTeX) or equivalent
code, because simply testing whether \cs{pdfoutput} or some other
primitive is defined is not a reliable way to determine if \acro{PDF}
output is being generated.  We made this backward compatible as best we
could this year, but next year, \cs{pdfoutput} may be defined even when
\acro{DVI} is being written.

\item pdf\TeX\ (\url{http://pdftex.org}) has many new features:

  \begin{itemize*}

  \item \cs{pdfmapfile} and \cs{pdfmapline} provide font map support
  from within a document.

  \item Microtypographic font expansion can be used more easily.\\
  \url{http://www.ntg.nl/pipermail/ntg-pdftex/2004-May/000504.html}

  \item All parameters previously set through the special configuration
  file \filename{pdftex.cfg} must now be set through primitives,
  typically in \filename{pdftexconfig.tex}; \filename{pdftex.cfg} is no
  longer supported.  Any extant \filename{.fmt} files must be redumped
  when \filename{pdftexconfig.tex} is changed.

  \item See the pdf\TeX\ manual for more: \OnCD{texmf-dist/doc/pdftex/manual/pdftex-a.pdf}.

  \end{itemize*}

\item The \cs{input} primitive in \cmdname{tex} (and \cmdname{mf} and
\cmdname{mpost}) now accepts double quotes containing spaces and other
special characters.  Typical examples:
\begin{verbatim}
\input "filename with spaces"   % plain
\input{"filename with spaces"}  % latex
\end{verbatim}
See the Web2C manual for more: \OnCD{texmf-dist/doc/web2c}.

\item enc\TeX\ support is now included within Web2C and consequently all
\TeX\ programs, via the \optname{-enc} option\Dash \emph{only when
formats are built}.  enc\TeX\ supports general re-encoding of input and
output, enabling full support of Unicode (in \acro{UTF}-8).  See
\OnCD{texmf-dist/doc/generic/enctex/} and
\url{http://www.olsak.net/enctex.html}.

\item Aleph, a new engine combining \eTeX\ and \OMEGA, is available.
A little information is available in \OnCD{texmf-dist/doc/aleph/base}
and \url{http://www.tex.ac.uk/cgi-bin/texfaq2html?label=aleph}.  The
\LaTeX-based format for Aleph is named \textsf{lamed}.

\item The latest \LaTeX\ release has a new version of the
\acro{LPPL}\Dash now officially a Debian-approved license.  Assorted
other updates, see the \filename{ltnews} files in
\OnCD{texmf-dist/doc/latex/base}.

\item \cmdname{dvipng}, a new program for converting \acro{DVI} to
\acro{PNG} image files, is included.  See
\url{http://www.ctan.org/pkg/dvipng}.

\item We reduced the \pkgname{cbgreek} package to a ``medium'' sized set
of fonts, with the assent and advice of the author (Claudio Beccari).
The excised fonts are the invisible, outline, and transparency ones,
which are relatively rarely used, and we needed the space.  The full set
is of course available from \acro{CTAN}
(\url{http://mirror.ctan.org/tex-archive/fonts/greek/cbfonts}).

\item \cmdname{oxdvi} has been removed; just use \cmdname{xdvi}.

\item The \cmdname{ini} and \cmdname{vir} commands (links) for
\cmdname{tex}, \cmdname{mf}, and \cmdname{mpost} are no longer created,
such as \cmdname{initex}.  The \cmdname{ini} functionality has been
available through the command-line option \optname{-ini} for years now.

\item \textsf{i386-openbsd} platform support was removed.  Since the
\pkgname{tetex} package in the \acro{BSD} Ports system is available, and
\acro{GNU/}Linux and Free\acro{BSD} binaries were available, it seemed
volunteer time could be better spent elsewhere.

\item On \textsf{sparc-solaris} (at least), you may have to set the
\envname{LD\_LIBRARY\_PATH} environment variable to run the
\pkgname{t1utils} programs.  This is because they are compiled with C++,
and there is no standard location for the runtime libraries.  (This is
not new in 2004, but wasn't previously documented.)  Similarly, on
\textsf{mips-irix}, the \acro{MIPS}pro 7.4 runtimes are required.

\end{itemize}

\subsubsection{2005}

2005 saw the usual huge number of updates to packages and programs.
The infrastructure stayed relatively stable from 2004, but inevitably
there were some changes there as well:

\begin{itemize}

\item New scripts \cmdname{texconfig-sys}, \cmdname{updmap-sys}, and
      \cmdname{fmtutil-sys} were introduced, which modify the
      configuration in the system trees.  The \cmdname{texconfig},
      \cmdname{updmap}, and \cmdname{fmtutil} scripts now modify
      user-specific files, under \dirname{$HOME/.texlive2005}.

\item Corresponding new variables \envname{TEXMFCONFIG} and
      \envname{TEXMFSYSCONFIG} to specify the trees where configuration
      files (user or system, respectively) are found.  Thus, you may
      need to move personal versions of \filename{fmtutil.cnf} and
      \filename{updmap.cfg} to these places; another option is to
      redefine \envname{TEXMFCONFIG} or \envname{TEXMFSYSCONFIG} in
      \filename{texmf.cnf}. In any case the real location of these files
      and the values of \envname{TEXMFCONFIG} and \envname{TEXMFSYSCONFIG}
      must agree.
      See section~\ref{sec:texmftrees}, \p.\pageref{sec:texmftrees}.

\item Last year, we kept \verb|\pdfoutput| and other primitives undefined
      for \dvi\ output, even though the \cmdname{pdfetex} program was
      being used.  This year, as promised, we undid that compatibility
      measure.  So if your document uses \verb|\ifx\pdfoutput\undefined|
      to test if PDF is being output, it will need to be changed.  You
      can use the package \pkgname{ifpdf.sty} (which works under both
      plain \TeX\ and \LaTeX) to do this, or steal its logic.

\item Last year, we changed most formats to output (8-bit) characters as
      themselves (see previous section).  The new TCX file
      \filename{empty.tcx} now provides an easier way to get the
      original \verb|^^| notation if you so desire, as in:
\begin{verbatim}
latex --translate-file=empty.tcx yourfile.tex
\end{verbatim}

\item The new program \cmdname{dvipdfmx} is included for translation of
      DVI to PDF; this is an actively maintained update of
      \cmdname{dvipdfm} (which is also still available for now, though
      no longer recommended).

\item The new programs \cmdname{pdfopen} and \cmdname{pdfclose} are included
      to allow reloading of pdf files in the Adobe Acrobat Reader without
      restarting the program.  (Other pdf readers, notably \cmdname{xpdf},
      \cmdname{gv}, and \cmdname{gsview}, have never suffered from this
      problem.)

\item For consistency, the variables \envname{HOMETEXMF} and
      \envname{VARTEXMF} have been renamed to \envname{TEXMFHOME} and
      \envname{TEXMFSYSVAR}, respectively.  There is also
      \envname{TEXMFVAR}, which is by default user-specific.  See the
      first point above.

\end{itemize}


\subsubsection{2006--2007}

In 2006--2007, the major new addition to \TL{} was the \XeTeX{} program,
available as the \texttt{xetex} and \texttt{xelatex} programs; see
\url{http://scripts.sil.org/xetex}.

MetaPost also received a notable update, with more planned for the
future (\url{http://tug.org/metapost/articles}), likewise pdf\TeX{}
(\url{http://tug.org/applications/pdftex}).

The \TeX\ \filename{.fmt} (high-speed format) and the similar files for
MetaPost and \MF\ are now stored in subdirectories of \dirname{texmf/web2c},
instead of in the directory itself (although the directory is still
searched, for the sake of existing \filename{.fmt}'s).  The
subdirectories are named for the `engine' in use, such as \filename{tex}
or \filename{pdftex} or \filename{xetex}.  This change should be
invisible in normal use.

The (plain) \texttt{tex} program no longer reads \texttt{\%\&} first
lines to determine what format to run; it is the pure Knuthian \TeX.
(\LaTeX\ and everything else do still read \texttt{\%\&} lines).

Of course the year also saw (the usual) hundreds of other updates to
packages and programs.  As usual, please check \acro{CTAN}
(\url{http://mirror.ctan.org}) for updates.

Internally, the source tree is now stored in Subversion, with a standard
web interface for viewing the tree, as linked from our home page.
Although not visible in the final distribution, we expect this will
provide a stable development foundation for future years.

Finally, in May 2006 Thomas Esser announced that he would no longer be
updating te\TeX{} (\url{http://tug.org/tetex}).  As a result, there was
been a surge of interest in \TL{}, especially among \GNU/Linux
distributors.  (There is a new \texttt{tetex} installation scheme in
\TL{}, which provides an approximate equivalent.)  We hope this will
eventually translate to improvements in the \TeX\ environment for
everyone.

\subsubsection{2008}

In 2008, the entire \TL{} infrastructure was redesigned and
reimplemented.  Complete information about an installation is now stored
in a plain text file \filename{tlpkg/texlive.tlpdb}.

Among other things, this finally makes possible upgrading a \TL{}
installation over the Internet after the initial installation, a feature
MiK\TeX\ has provided for many years.  We expect to regularly update new
packages as they are released to \CTAN.

The major new engine Lua\TeX\ (\url{http://luatex.org}) is included;
besides a new level of flexibility in typesetting, this provides an
excellent scripting language for use both inside and outside of \TeX\
documents.

Support among Windows and the Unix-based platforms is now much more
uniform.  In particular, most Perl and Lua scripts are now available on
Windows, using the Perl internally distributed with \TL.

The new \cmdname{tlmgr} script (section~\ref{sec:tlmgr}) is the
general interface for managing \TL{} after the initial installation.
It handles package updates and consequent regeneration of formats, map
files, and language files, optionally including local additions.

With the advent of \cmdname{tlmgr}, the \cmdname{texconfig} actions to
edit the format and hyphenation configuration files are now disabled.

The \cmdname{xindy} indexing program
(\url{http://xindy.sourceforge.net/}) is now included on most platforms.

The \cmdname{kpsewhich} tool can now report all matches for a given file
(option \optname{--all}) and limit matches to a given subdirectory
(option \optname{--subdir}).

The \cmdname{dvipdfmx} program now includes functionality to extract
bounding box information, via the command name \cmdname{extractbb}; this
was one of the last features provided by \cmdname{dvipdfm} not in
\cmdname{dvipdfmx}.

The font aliases \filename{Times-Roman}, \filename{Helvetica}, and so on
have been removed.  Different packages expected them to behave
differently (in particular, to have different encodings), and there was
no good way to resolve this.

The \pkgname{platex} format has been removed, to resolve a name conflict
with a completely different Japanese \pkgname{platex}; the
\pkgname{polski} package is now the main Polish support.

Internally, the \web\ string pool files are now compiled into the
binaries, to ease upgrades.

Finally, the changes made by Donald Knuth in his `\TeX\ tuneup of 2008'
are included in this release.  See
\url{http://tug.org/TUGboat/Articles/tb29-2/tb92knut.pdf}.

\subsubsection{2009}

In 2009, the default output format for Lua\AllTeX\ is now PDF, to take
advantage of Lua\TeX's OpenType support, et al.  New executables named
\code{dviluatex} and \code{dvilualatex} run Lua\TeX\ with DVI output.
The Lua\TeX\ home page is \url{http://luatex.org}.

The original Omega engine and Lambda format have been excised, after
discussions with the Omega authors.  The updated Aleph and Lamed remain,
as do the Omega utilities.

A new release of the AMS \TypeI\ fonts is included, including Computer
Modern: a few shape changes made over the years by Knuth in the Metafont
sources have been integrated, and the hinting has been updated.  The
Euler fonts have been thoroughly reshaped by Hermann Zapf (see
\url{http://tug.org/TUGboat/Articles/tb29-2/tb92hagen-euler.pdf}).  In
all cases, the metrics remain unchanged.  The AMS fonts home page is
\url{http://www.ams.org/tex/amsfonts.html}.

The new \GUI{} front end \TeX{}works is included for Windows, and also in
Mac\TeX.  For other platforms, and more information, see the \TeX{}works
home page, \url{http://tug.org/texworks}.  It is a cross-platform front
end inspired by the \MacOSX\ TeXShop editor, aiming at ease-of-use.

The graphics program Asymptote is included for several platforms.  This
implements a text-based graphics description language vaguely akin to
MetaPost, but with advanced 3D support and other features.  Its home
page is \url{http://asymptote.sourceforge.net}.

The separate \code{dvipdfm} program has been replaced by
\code{dvipdfmx}, which operates in a special compatibility mode under
that name.  \code{dvipdfmx} includes \acro{CJK} support and has
accumulated many other fixes over the years since the last
\code{dvipdfm} release.  The DVIPDFMx home page is
\url{http://project.ktug.or.kr/dvipdfmx}.

Executables for the \pkgname{cygwin} and \pkgname{i386-netbsd} platforms
are now included, while we were advised that Open\acro{BSD} users get
\TeX\ through their package systems, plus there were difficulties in
making binaries that have a chance of working on more than one version.

A miscellany of smaller changes: we now use \pkgname{xz} compression,
the stable replacement for \pkgname{lzma}
(\url{http://tukaani.org/xz/}); a literal |$| is allowed in filenames
when it does not introduce a known variable name; the Kpathsea library
is now multi-threaded (made use of in MetaPost); the entire \TL{} build
is now based on Automake.

Final note on the past: all releases of \TL{}, along with ancillary
material such as \CD\ labels, are available at
\url{ftp://tug.org/historic/systems/texlive}.


\subsubsection{2010}
\label{sec:2010news} % keep with 2010

In 2010, the default version for PDF output is now 1.5, enabling more
compression.  This applies to all the \TeX\ engines when used to produce
PDF and to \code{dvipdfmx}.  Loading the \pkgname{pdf14} \LaTeX\ package
changes back to PDF~1.4, or set |\pdfminorversion=4|.

pdf\AllTeX\ now \emph{automatically} converts a requested Encapsulated
PostScript (EPS) file to PDF, via the \pkgname{epstopdf} package, when
and if the \LaTeX\ \code{graphics.cfg} configuration file is loaded, and
PDF is being output.  The default options are intended to eliminate any
chance of hand-created PDF files being overwritten, but you can also
prevent \code{epstopdf} from being loaded at all by putting
|\newcommand{\DoNotLoadEpstopdf}{}| (or |\def...|) before the
\cs{documentclass} declaration.  It is also not loaded if the
\pkgname{pst-pdf} package is used.  For more details, see the
\pkgname{epstopdf} package documentation
(\url{http://ctan.org/pkg/epstopdf-pkg}).

A related change is that execution of a very few external commands from
\TeX, via the \cs{write18} feature, is now enabled by default.  These
are commands are \code{repstopdf}, \code{makeindex}, \code{kpsewhich},
\code{bibtex}, and \code{bibtex8}; the list is defined in
\code{texmf.cnf}.  Environments which must disallow all such external
commands can deselect this option in the installer (see
section~\ref{sec:options}), or override the value after installation by
running |tlmgr conf texmf shell_escape 0|.

Yet another related change is that \BibTeX\ and Makeindex now refuse to
write their output files to an arbitrary directory (like \TeX\ itself),
by default.  This is so they can now be enabled for use by the
restricted \cs{write18}.  To change this, the \envname{TEXMFOUTPUT}
environment variable can be set, or the |openout_any| setting changed.

\XeTeX\ now supports margin kerning along the same lines as pdf\TeX.
(Font expansion is not presently supported.)

By default, \prog{tlmgr} now saves one backup of each package updated
(\code{tlmgr option autobackup 1}), so broken packages updates can be
easily reverted with \code{tlmgr restore}.  If you do post-install
updates, and don't have the disk space for the backups, run \code{tlmgr
option autobackup 0}.

New programs included: the p\TeX\ engine and related utilities for
typesetting Japanese; the \BibTeX{}U program for Unicode-enabled
\BibTeX; the \prog{chktex} utility
(\url{http://baruch.ev-en.org/proj/chktex}) for checking \AllTeX\
documents; the \prog{dvisvgm} (\url{http://dvisvgm.sourceforge.net})
DVI-to-SVG translator.

Executables for these new platforms are now included: \code{amd64-freebsd},
\code{amd64-kfreebsd}, \code{i386-freebsd}, \code{i386-kfreebsd},
\code{x86\_64-darwin}, \code{x86\_64-solaris}.

A change in \TL{} 2009 that we failed to note: numerous \TeX4ht-related
executables (\url{http://tug.org/tex4ht}) were removed from the binary
directories.  The generic \code{mk4ht} program can be used to run any of
the various \code{tex4ht} combinations.

Finally, the \TL{} release on the \TK\ \DVD\ can no longer be run live
(oddly enough).  A single \DVD\ no longer has enough room.  One
beneficial side effect is that installation from the physical \DVD\ is
much faster.

\subsubsection{2011}

The \MacOSX\ binaries (\code{universal-darwin} and
\code{x86\_64-darwin}) now work only on Leopard or later; Panther and
Tiger are no longer supported.

The \code{biber} program for bibliography processing is included on
common platforms.  Its development is closely coupled with the
\code{biblatex} package, which completely reimplements the
bibliographical facilities provided by LaTeX.

The MetaPost (\code{mpost}) program no longer creates or uses
\code{.mem} files.  The needed files, such as \code{plain.mp}, are
simply read on every run.  This is related to supporting MetaPost as a
library, which is another significant though not user-visible change.

The \code{updmap} implementation in Perl, previously used only on
Windows, has been revamped and is now used on all platforms.  There
shouldn't be any user-visible changes as a result, except that it runs
much faster.

The \cmdname{initex} and \cmdname{inimf} programs were restored (but no
other \cmdname{ini*} variants).

\subsubsection{2012}

\code{tlmgr} supports updates from multiple network repositories.  The
section on multiple repositories in the \code{tlmgr help} output has
more.

The parameter \cs{XeTeXdashbreakstate} is set to~1 by default, for both
\code{xetex} and \code{xelatex}.  This allows line breaks after
em-dashes and en-dashes, which has always been the behavior of plain
\TeX, \LaTeX, Lua\TeX, etc.  Existing \XeTeX\ documents which must
retain perfect line-break compatibility will need to set
\cs{XeTeXdashbreakstate} to~0 explicitly.

The output files generated by \code{pdftex} and \code{dvips}, among
others, can now exceed 2gb.

The 35 standard PostScript fonts are included in the output of
\code{dvips} by default, since so many different versions of them are
extant.

In the restricted \cs{write18} execution mode, set by default,
\code{mpost} is now an allowed program.

A \code{texmf.cnf} file is also found in \filename{../texmf-local},
e.g., \filename{/usr/local/texlive/texmf-local/web2c/texmf.cnf}, if it
exists.

The \code{updmap} script reads a per-tree \code{updmap.cfg} instead of
one global config.  This change should be invisible, unless you edited
your updmap.cfg's directly.  The \verb|updmap --help| output has more.

Platforms: \pkgname{armel-linux} and \pkgname{mipsel-linux} added;
\pkgname{sparc-linux} and \pkgname{i386-netbsd} are no longer in the
main distribution.

\subsubsection{2013}

Distribution layout: the top-level \code{texmf/} directory has been
merged into \code{texmf-dist/}, for simplicity.  Both the
\code{TEXMFMAIN} and \code{TEXMFDIST} Kpathsea variables now point to
\code{texmf-dist}.

Many small language collections have been merged together, to simplify
installation.

\MP: native support for PNG output and floating-point (IEEE double) has
been added.

Lua\TeX: updated to Lua 5.2, and includes a new library
(\code{pdfscanner}) to process external PDF page content, among much
else (see its web pages).

\XeTeX\ (also see its web pages for more):
\begin{itemize}
\item The HarfBuzz library now used for font layout instead of
ICU.  (ICU is still used to support input encodings, bidirectionality,
and the optional Unicode line breaking.)
\item Graphite2 and HarfBuzz are used instead of SilGraphite for Graphite
layout.
\item On Macs, Core Text is used instead of the (deprecated) ATSUI.
\item Prefer TrueType/OpenType fonts to Type1 when the names are the same.
\item Fix occasional mismatch in font finding between \XeTeX\ and
\code{xdvipdfmx}.
\item Support OpenType math cut-ins.
\end{itemize}

\cmdname{xdvi}: now uses FreeType instead of \code{t1lib} for rendering.

\pkgname{microtype.sty}: some support for \XeTeX\ (protrusion) and
Lua\TeX\ (protrusion, font expansion, tracking), among other
enhancements.

\cmdname{tlmgr}: new \code{pinning} action to ease configuring multiple
repositories; that section in \verb|tlmgr --help| has more, online at
\url{http://tug.org/texlive/doc/tlmgr.html#MULTIPLE-REPOSITORIES}.

Platforms: \pkgname{armhf-linux}, \pkgname{mips-irix},
\pkgname{i386-netbsd}, and \pkgname{amd64-netbsd} added or revived;
\pkgname{powerpc-aix} removed.

% 
\subsubsection{2014}

2014 saw another \TeX\ tune-up from Knuth; this affected all engines,
but the only visible change likely is the restoration of the
\code{preloaded format} string on the banner line.  Per Knuth, this now
reflects the format that \emph{would} be loaded by default, rather than
an undumped format that is actually preloaded in the binary; it may be
overridden in various ways.

pdf\TeX: new warning-suppression parameter
\cs{pdfsuppresswarningpagegroup}; new primitives for fake interword
spaces to help with \acro{PDF} text reflowing: \cs{pdfinterwordspaceon},
\cs{pdfinterwordspaceoff}, \cs{pdffakespace}.

Lua\TeX: Notable changes and fixes were made to font loading and
hyphenation.  The biggest addition is a new engine variant,
\code{luajittex} (\url{http://foundry.supelec.fr/projects/luajittex})
and its siblings \code{texluajit} and \code{texluajitc}.  This uses a
just-in-time Lua compiler (detailed \textsl{TUGboat} article at
\url{http://tug.org/TUGboat/tb34-1/tb106scarso.pdf}).  \code{luajittex}
is still in development, is not available on all platforms, and is
considerably less stable than \code{luatex}.  Neither we nor its
developers recommend using it except for the specific purpose of
experimenting with jit on Lua code.

\XeTeX: The same image formats are now supported on all platforms
(including Mac); avoid Unicode compatibility decomposition fallback (but
not other variants); prefer OpenType to Graphite fonts, for
compatibility with previous \XeTeX\ versions.

\MP: A new numbersystem \code{decimal} is supported, along with a
companion internal \code{numberprecision}; a new definition of
\code{drawdot} in \filename{plain.mp}, per Knuth; bug fixes in
\acro{SVG} and \acro{PNG} output, among others.

The \cmdname{pstopdf} Con\TeX{}t utility will be removed as a standalone
command at some point after the release, due to conflicts with OS
utilities of the same name.  It can still (and now) be invoked as
\code{mtxrun --script pstopdf}.

\cmdname{psutils} has been substantially revised by a new maintainer.
As a result, several seldom-used utilities (\code{fix*}, \code{getafm},
\code{psmerge}, \code{showchar}) are now only in the \dirname{scripts/}
directory rather than being user-level executables (this can be reversed
if it turns out to be problematic).  A new script, \code{psjoin}, has
been added.

The Mac\TeX\ redistribution of \TeX\ Live (section~\ref{sec:macosx}) no
longer includes the optional Mac-only packages for the Latin Modern and
\TeX\ Gyre fonts, since it is easy enough for individual users to make
them available to the system.  The \cmdname{convert} program from
ImageMagick has also been excised, since \TeX4ht (specifically
\code{tex4ht.env}) now uses Ghostscript directly.

The \pkgname{langcjk} collection for Chinese, Japanese, and Korean
support has been split into individual language collections for the sake
of more moderate sizes.

Platforms: \pkgname{x86\_64-cygwin} added, \pkgname{mips-irix} removed;
Microsoft no longer supports Windows \acro{XP}, so our programs may
start failing there at any time.


\htmlanchor{news}
\subsection{Present---2015}
\label{sec:tlcurrent}

\LaTeXe\ now incorporates, by default, changes previously included only
by explicitly loading the \pkgname{fixltx2e} package, which is now a
no-op.  A new \pkgname{latexrelease} package and other mechanisms allow
for controlling what is done.  The included \LaTeX\ News \#22 and
``\LaTeX\ changes'' documents have details.  Incidentally, the
\pkgname{babel} and \pkgname{psnfss} packages, while core parts of
\LaTeX, are maintained separately and are not affected by these changes
(and should still work).

Internally, \LaTeXe\ now includes Unicode-related engine configuration
(what characters are letters, naming of primitives, etc.) which was
previously part of \TeX\ Live.  This change is intended to be invisible
to users; a few low-level internal control sequences have been renamed
or removed, but the behavior should be just the same.

pdf\TeX: Support \acro{JPEG} Exif as well as \acro{JFIF}; do not even
emit a warning if \cs{pdfinclusionerrorlevel} is negative; sync
with \prog{xpdf}~3.04.

Lua\TeX: New library \pkgname{newtokenlib} for scanning tokens; bug
fixes in the \code{normal} random number generator and other places.

\XeTeX: Image handling fixes; \prog{xdvipdfmx} binary looked for first
as a sibling to \prog{xetex}; internal \code{XDV} opcodes changed.

MetaPost: New numbersystem \code{binary}; new Japanese-enabled
\prog{upmpost} and \prog{updvitomp} programs, analogous to
\prog{up*tex}.

Mac\TeX: Updates to the included Ghostscript package for \acro{CJK}
support.  The \TeX\ Distribution Preference Pane now works in Yosemite
(\MacOSX~10.10).  Resource-fork font suitcases (generally without an
extension) are no longer supported by \XeTeX; data-fork suitcases
(\code{.dfont}) remain supported.

Infrastructure: The \prog{fmtutil} script has been reimplemented to read
\filename{fmtutil.cnf} on a per-tree basis, analogous to \prog{updmap}.
Web2C \prog{mktex*} scripts (including \prog{mktexlsr}, \prog{mktextfm},
\prog{mktexpk}) now prefer programs in their own directory, instead of
always using the existing \envname{PATH}.

Platforms: \pkgname{*-kfreebsd} removed, since \TeX\ Live is now easily
available through the system platform mechanisms.
Support for some additional platforms is available as custom binaries
(\url{http://tug.org/texlive/custom-bin.html}).  In addition, some
platforms are now omitted from the \DVD\ (simply to save space), but can
be installed normally over the net.

\subsection{Future}

\emph{\TL{} is not perfect!}  (And never will be.)  We intend to
continue to release new versions, and would like to provide more help
material, more utilities, more installation programs, and (of course) an
ever-improved and better-checked tree of macros and fonts. This work is
all done by volunteers in their spare time, and so there is always more
to do.  Please see \url{http://tug.org/texlive/contribute.html}.

Please send corrections, suggestions, and offers of help to:
\begin{quote}
\email{tex-live@tug.org} \\
\url{http://tug.org/texlive}
\end{quote}

\medskip
\noindent \textsl{Happy \TeX ing!}

\end{document}
