\documentclass[a4paper,11pt]{article}
\usepackage[T1]{fontenc}
\usepackage{lmodern}
\usepackage{miscdoc,url,booktabs,microtype}
\addtolength\textwidth{1cm}
\addtolength\oddsidemargin{-.5cm}
\addtolength\evensidemargin{-.5cm}
\raggedbottom
\setlength\parskip{4pt plus 2pt}

\begin{document}
\title{\texttt{url.sty} version 3.4}
\date{2013-09-16}
\author{Donald Arseneau%
  \thanks{Thanks to Robin Fairbairns for documentation conversion!}}
\maketitle

% url.sty  ver 3.4    16-Sep-2013   Donald Arseneau   asnd@triumf.ca
% Copyright 1996-2013 Donald Arseneau,  Vancouver, Canada.
% This program can be used, distributed, and modified under the terms
% of the LaTeX Project Public License, version 2 or later.

The package defines a form of \cs{verb} command that allows linebreaks
at certain characters or combinations of characters, accepts
reconfiguration, and can usually be used in the argument to another
command.  It is intended for formatting email addresses, hypertext links,
directories/paths, etc., which normally have no spaces.  The font used
may be selected using the \cs{urlstyle} command, and new url-like
commands may be defined using \cs{urldef}.
This package does not make hyper-links! For that purpose, see the hyperref
package (or some other deprecated ones).

\begin{center}
  \begin{tabular}{lp{3.5in}}
    \toprule 
    Usage &    Conditions \\
    \midrule
    \verb+\url{ }+ & The argument must not contain unbalanced braces.
                     If used in the argument to another command, the 
                     \cs{url} argument cannot contain any ``\verb+%+'',
                      ``\verb+#+'', or ``\verb+^^+'', or end with
                       ``\verb+\+''.\\
    \verb+\url|  |+ & where ``\texttt{|}'' is any character not used
                      in the argument and not ``\verb+{+'' or a space.  
                      The same restrictions apply as above except that 
                      the argument may contain unbalanced braces.\\ 
    \cs{xyz} & for the defined-url ``\cs{xyz}''; such a command can be
               used anywhere, no matter what characters it contains. \\
      \bottomrule
    \end{tabular}
  \end{center}

\noindent The ``\cs{url}'' command is fragile, and its argument is likely to be
very fragile, but a defined-url is robust.

\section{Package options}

\newitem Package Option:  \texttt{obeyspaces}

Ordinarily, all spaces are ignored in the url-text.  The
``\texttt{[obeyspaces]}'' option allows spaces, but may introduce
spurious spaces when a url containing ``\cs{}'' characters is given in
the argument to another command. 
So if you need to obey spaces you can say
``\cs{usepackage}\texttt{[obeyspaces]\char`\{url\char`\}}'', and if
you need both spaces and backslashes, use a defined-url.

\newitem Package Option:  \texttt{hyphens}

Ordinarily, breaks are not allowed after ``\texttt{-}'' characters
because this leads to confusion.  (Is the ``\texttt{-}'' part of the
address or just a hyphen?)
The package option ``\texttt{[hyphens]}'' allows breaks after explicit
hyphen characters.  The \cs{url} command will \textbf{never ever}
hyphenate words.

\newitem Package Option:  \texttt{spaces}

Likewise, given the ``\texttt{[obeyspaces]}'' option, breaks are not 
usually allowed after the spaces, but if you give the options
``\texttt{[obeyspaces,spaces]}'', \cs{url} will allow breaks at those
spaces.
\begin{quote}
  Note that it seems logical to allow the sole option
  ``\texttt{[spaces]}'' to let input spaces indicate break points, but
  not to display them in the output.  This would be easy to implement,
  but is left out to avoid(?)\ confusion.
\end{quote}

\newitem Package Option:  \texttt{lowtilde}

Normal treatment of the \verb+~+ character is to use the font's
``\cs{textasciitilde}'' character, if it has one (or claims to).
Otherwise, the character is faked using a mathematical ``\cs{sim}''.
The ``\texttt{[lowtilde]}'' option causes a faked character to be used
always (and a bit lower than usual).

\newitem Package Option:  \texttt{allowmove}

This option suppresses the test for \cs{url} being used in a so-called 
moving argument (check ``fragile command''). Using it will enable \cs{url}
to function in more contexts, but when it does fail, the error message
may be incomprehensible.


\section{Defining a defined-url}

Take for example the email address ``\url{myself%node@gateway.net}"
which could not be given (using ``\cs{url}'' or ``\cs{verb}'') in a
caption or parbox due to the percent sign.  This address can be
predefined with
\begin{quote}
  \verb|\urldef{\myself}\url{myself%node@gateway.net}| or\\
  \verb+\urldef{\myself}\url|myself%node@gateway.net|+
\end{quote}
and then you may use ``\cs{myself}'' instead of
``\verb+\url{myself%node@gateway.net}+''
in an argument, and even in a moving argument like a caption because a
defined-url is robust.

\section{Style}

You can switch the style of printing using ``\verb+\urlstyle{+$xx$\verb+}+'',
where ``$xx$'' can be any defined style.  The pre-defined styles are 
``\texttt{tt}'', ``\texttt{rm}'', ``\texttt{sf}'' and ``\texttt{same}'' 
which all allow the same linebreaks but use different fonts~--- the 
first three select a specific font and the ``\texttt{same}'' style 
uses the current text font.  You can define your own styles with 
different fonts and/or line-breaking by following the explanations 
below.  The ``\cs{url}'' command follows whatever the currently-set 
style dictates.

\section{Alternate commands}

It may be desireable to have different things treated differently, each
in a predefined style; e.g., if you want directory paths to always be
in typewriter and email addresses to be roman, then you would define new
url-like commands as follows:
\begin{quote}
   \verb+\DeclareUrlCommand+\meta{command}\verb+{+\meta{settings}\verb+}+\\
   \verb+\DeclareUrlCommand\email{\urlstyle{rm}}+\\
   \verb+\DeclareUrlCommand\directory{\urlstyle{tt}}+.
\end{quote}
In fact, this \cs{directory} example is exactly the \cs{path}
definition which might be pre-defined by the package.  Furthermore,
basic \cs{url} is defined with
\begin{quote}
   \verb+\DeclareUrlCommand\url{}+,
\end{quote}
without any \emph{settings}, so it uses whatever \cs{urlstyle} 
and other settings are already in effect.

You can make a defined-url for these other styles, using the usual
\cs{urldef} command as in this example:
\begin{quote}
\verb+\urldef{\myself}{\email}{myself%node.domain@gateway.net}+
\end{quote}
which makes \cs{myself} act like
\verb+\email{myself%node.domain@gateway.net}+,
if the \cs{email} command is defined as above.  The \cs{myself}
command would then be robust.

\section{Defining styles}

Before describing how to customize the printing style, it is best to
mention something about the unusual implementation of \cs{url}.  Although
the material is textual in nature, and the font specification required
is a text-font command, the text is actually typeset in \emph{math} mode.
This allows the context-sensitive linebreaking, but also accounts for
the default behavior of ignoring spaces.  (Maybe that underlying design 
will eventually change.) Now on to defining styles.

To change the font or the list of characters that allow linebreaks, you
could redefine the commands \cs{UrlFont}, \cs{UrlBreaks},
\cs{UrlSpecials}, etc., directly in the document, but it is better to
define a new `url-style' (following the example of \cs{url@ttstyle}
and \cs{url@rmstyle}) which defines all of \cs{UrlBigbreaks},
\cs{UrlNoBreaks}, \cs{UrlBreaks}, \cs{UrlSpecials}, and \cs{UrlFont}.

\subsection{Changing font}
The \cs{UrlFont} command selects the font.  The definition of
\cs{UrlFont} done by the pre-defined styles varies to cope with a
variety of \LaTeX{} font selection schemes, but it could be as simple
as \verb+\def\UrlFont{\tt}+.  Depending on the font selected, some
characters may need to be defined in the \cs{UrlSpecials} list because
many fonts don't contain all the standard input characters.

\subsection{Changing linebreaks}
The list of characters after which line-breaks are permitted is 
given by the two commands (list macros)
\cs{UrlBreaks} and \cs{UrlBigBreaks}. They consist of repeating
\cs{do}\cs{c} for each relevant character \texttt{c}.

The differences are that `BigBreaks' typically have a lower penalty (more
easily chosen) and do not break within a repeating sequence (e.g., 
``\verb+DEC::NODE+'').
(For gurus: `BigBreaks' are treated as mathrels while `Breaks' are mathbins;
see \textit{The TeXbook}, p.\,170.) The result is that a series of 
consecutive `BigBreak'
characters will break at the end and only at the end; a series of
`Break' characters will break after the first and after every following
\emph{pair}; there will be no break between a `Break' character and a
following `BigBreak' char; breaks are permitted when a `BigBreak' 
character is followed by `Break' or any other char.  In the case 
of \texttt{http://} it doesn't matter whether \texttt{:} is a 
`Break' or `BigBreak'~--- the breaks are the same in either case; but 
for (now ancient) \emph{DECnet} addresses using \texttt{::} it was
important to prevent breaks \emph{between} the colons, and that is why
colons are `BigBreaks'.  (The only other `BigBreak' character is,
optionally, the hyphen; slashes are regular `Break's.)

It is possible for characters to prevent breaks after the next
following character (this is used for parentheses).  Specify these in
\cs{UrlNoBreaks}.

You can allow some spacing around the breakable characters by assigning
\begin{quote}
\verb+\Urlmuskip = 0mu plus 1mu+
\end{quote}
(with \texttt{mu} units because of math mode).
You can change the penalties used for BigBreaks and Breaks by assigning
\begin{quote}
\verb+\mathchardef\UrlBreakPenalty=100+\\
\verb+\mathchardef\UrlBigBreakPenalty=100+
\end{quote}
The default penalties are \cs{binoppenalty} and \cs{relpenalty}.
These have such odd non-\LaTeX{} syntax because I don't expect people
to need to change them often. (The \verb+\mathchardef+ does not relate to 
math mode; it is only a way to store a number without consuming registers.)

\subsubsection{Arbitrary character actions}

You can do arbitrarily complex things with characters by specifying
their definition(s) in \cs{UrlSpecials}.  This makes them `active' in
math mode (mathcode \texttt{"8000}).  The format for setting
each special character \texttt{c} is:
\verb+\do\c{+\meta{definition}\verb+}+, but other definitions not
following this style can also be included.

Here is an example to make ``\texttt{!}''\ inside \cs{url} force a line break
instead of being treated verbatim (it uses \LaTeX's \cs{g@addto@macro}):

\begin{quote}
\verb+\makeatletter \g@addto@macro\UrlSpecials{\do\!{\newline}}+
\end{quote}

Here is another overly-complicated  example to put extra flexible
muglue around each ``\texttt{/}'' character, except when followed 
by another ``\texttt{/}'', as in ``\texttt{http://}'', where extra 
spacing looks poor.

\begin{quote}
\begin{verbatim}
% what we'll insert before and after each (lone) slash:
\newmuskip\Urlslashmuskip 
\Urlslashmuskip=2mu plus2mu minus2mu

% change what / does:
\g@addto@macro\UrlSpecials{\do\/{\Urlspaceyslash}}

% need to look ahead:
\def\Urlspaceyslash{\futurelet\Urlssnext\finishUrlspaceyslash}

\def\finishUrlspaceyslash{%
  \mskip\Urlslashmuskip  % extra space before
  \mathchar8239  % "202f, i.e., binary op, \fam, / char
  % if we see //, eliminate the extra space to taste:
  \ifx\Urlssnext/\mskip-\Urlslashmuskip
  \else\mskip\Urlslashmuskip \fi
 }
\end{verbatim}
\end{quote}

If this sounds confusing~\dots{} well, it is!  But I hope you
won't need to redefine breakpoints~--- the default assignments seem to
work well for a wide variety of applications.  If you do need to make
changes, you can test for breakpoints using regular math mode and the
characters ``\texttt{+=(a}''. 


\section{Yet more flexibility}
You can also customize the presentation of verbatim text by defining 
\cs{UrlRight} and/or \cs{UrlLeft}.  An example for ISO formatting of 
urls surrounded by \verb+<  >+ is
\begin{quote}
\begin{verbatim}
\DeclareUrlCommand\url{\def\UrlLeft{<url:\ }\def\UrlRight{>}%
    \urlstyle{tt}}
\end{verbatim}
\end{quote}
The meanings of \cs{UrlLeft} and \cs{UrlRight} are \emph{not}
reproduced verbatim.  This lets you use formatting commands there, but
you must be careful not to use \TeX's special characters
(\verb+\^_%~#$&{}+ etc.)\ improperly.  You can also define \cs{UrlLeft}
to reprocess the verbatim text, but the format of the definition is special:
\begin{quote}
\verb+\def\UrlLeft#1\UrlRight{+\,\dots\ do things with \verb+#1+ \dots\,\verb+}+
\end{quote}
Yes, that is \verb+#1+ followed by \cs{UrlRight} then the definition (a \TeX\
macro with delimited arguments).  For example, to produce a hyper\TeX\ hypertext 
link:
\begin{quote}
\verb+\def\UrlLeft#1\UrlRight{%+\\
\verb+      \special{html:<a href="#1">}#1\special{html:</a>}}+
\end{quote}
Using this technique, \path{url.sty} can provide a convenient
interface for performing various operations on verbatim text.  You
don't even need to print out the argument!  For greatest efficiency in
such obscure applications, you can define a null url-style where all
the lists like \cs{UrlBreaks} are empty.

Please note that this method is \emph{not} how the hyperref package
manages urls for its \cs{url} command, even though it makes use of
\path{url.sty}. Instead, hyperref's \cs{url} reads its argument in
a less-verbatim manner than described above, produces its hyperlink,
and invokes \cs{nolinkurl} to format the text.  \cs{nolinkurl} is 
the \cs{url} command descibed herein.

\end{document}


%%% Local Variables: 
%%% mode: latex
%%% TeX-master: t
%%% End: 
