%# -*- coding: utf-8 -*-
\ifx\epTeXinputencoding\undefined\else % defined in e-pTeX (> TL2016)
  \epTeXinputencoding utf8    % ensure utf-8 encoding for platex
\fi

\documentclass[a4paper]{jsarticle}
\begingroup
  \def\NeedsTeXFormat#1\ProvidesPackage#2[#3 #4 #5]{%
    \gdef\fileversion{#4}\gdef\filedate{#3}\endinput}
  %
% pxmulticol.sty
% written by Hironobu Yamashita (@aminophen)
%
% This package is part of the platex-tools bundle.
% https://github.com/aminophen/platex-tools
%
% This package is expected to be compatible with
%   * multicol.sty in latex-tools
%     (1995/10/19 v1.5l -- 2021/11/30 v1.9d)
%

\NeedsTeXFormat{pLaTeX2e}
\ProvidesPackage{pxmulticol}
  [2018/11/25 v0.2 multicol package for pLaTeX]
\RequirePackageWithOptions{multicol}

%% patching internal code for multicol.sty
%
\def\reinsert@footnotes{\ifvoid\footins\else
      \insert\footins{\iftbox\footins\tate\else\yoko\fi}\fi}
%
%% done

\endinput

\endgroup
\title{Package \textsf{pxmulticol} \fileversion}
\author{Hironobu Yamashita}
\date{\filedate}
\begin{document}

\maketitle

Package \textsf{multicol}, included in \textsf{latex-tools} bundle,
throws an error when \verb+\footnote+ appears in vertical writing on
(u)p\LaTeX. The package \textsf{pxmulticol} provides a patch to
solve this problem.

This package is part of \textsf{platex-tools} bundle:
\begin{verbatim}
  https://github.com/aminophen/platex-tools
\end{verbatim}

The rest of this document is written in Japanese, and includes some
examples of usage.

\bigskip

\textsf{latex-tools}バンドルに含まれる\textsf{multicol}パッケージは、
縦組で脚注 (\verb+\footnote+) を使うとエラーを出してしまいます。
この\textsf{pxmulticol}パッケージを読み込むことで、縦組クラスでも
\textsf{multicols}環境内で脚注を使うことができるようになります。

\section{使いかた}

\textsf{multicol}パッケージの代わりに、あるいは追加で
\textsf{pxmulticol}パッケージを読み込みます。

\begin{verbatim}
  \documentclass[a4paper]{tarticle}
  \usepackage{pxmulticol}% multicol の代わり
  \begin{document}
  \begin{multicols}{2}
  縦組で段組をする場合も、脚注\footnote{こんな感じです。}を使う
  ことができます。
  \end{multicols}
  \end{document}
\end{verbatim}

\section*{References}

\begin{itemize}
\item 縦書き二段組における段移動\\
  \texttt{https://oku.edu.mie-u.ac.jp/tex/mod/forum/discuss.php?d=2535}
\end{itemize}

\end{document}
