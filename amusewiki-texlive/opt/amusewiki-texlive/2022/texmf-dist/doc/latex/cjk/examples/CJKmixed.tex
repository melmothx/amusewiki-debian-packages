% This is the file CJKmixed.tex of the CJK package
%   for testing CJK text written vertically.
%
% written by Werner Lemberg <wl@gnu.org>
%
% Version 4.8.5 (16-Oct-2021)
%
% Vietnamese support comes with the VnTeX package.

\documentclass[12pt]{article}

\usepackage{CJKutf8}
\usepackage{CJKvert}
\usepackage{CJKulem}

\usepackage{inputenc}               % load it without argument
                                    % to avoid Babel warnings

\usepackage[vietnamese,             % T5 font encoding
            USenglish]{babel}

\newenvironment{TChinese}{%
  \CJKfamily{bsmi}%
  \CJKtilde
  \CJKnospace}{}


\begin{document}

\begin{CJK}{UTF8}{}

\CJKhorz

This is a test how CJK scripts can be typeset horizontally and
vertically at the same time.  It is not too difficult to achieve,
nevertheless it is not trivial.  Most importantly, you need a
\texttt{.fdx} file which corresponds to your CJK font, and which
defines how to set up the font for vertical typesetting.

\begin{center}
  \rotatebox[origin=c]{-90}{%
    \begin{minipage}[c]{8cm}
      \CJKvert
      \begin{TChinese}
        本常問問答集~(FAQ list)~是從一些經常被問到的問題及其適當的解答
        中,以方便的形式摘要而出的。\uline{跟上一版不同的是,其編排結構
          已徹底改變。} \textbf{有關新結構的細節,可參考「如何閱讀本問
          答集及了解其編排結構」該項中的說明。}
      \end{TChinese}
    \end{minipage}%
  }%
  \hspace{0.5cm}%
  \begin{minipage}[c]{8cm}
    \begin{otherlanguage}{vietnamese}
      Phần ``Những câu hỏi và giải đáp thường gặp'' (viết tắt từ tiếng
      Anh là FAQ) được nêu ra ở đây nhằm mục đích thu thập những câu
      hỏi thường gặp trong thực tế và những lời giải đáp thích hợp
      nhất của nó.  \uline{Từ lần ấn bản cuối cùng đến nay, đã có
        những sự thay đổi khá lớn trong cấu trúc của FAQ.}  \textbf{Để
        hiểu rõ hơn bạn nên xem lại chương ``Làm sao tôi có thể đọc
        đuợc FAQ''.}
    \end{otherlanguage}
  \end{minipage}
\end{center}

\begin{TChinese}
  本常問問答集~(FAQ list)~是從一些經常被問到的問題及其適當的解答中,以
  方便的形式摘要而出的。\uline{跟上一版不同的是,其編排結構已徹底改變。}
  \textbf{有關新結構的細節,可參考「如何閱讀本問答集及了解其編排結構」
    該項中的說明。}
\end{TChinese}

\end{CJK}

\end{document}

%%% Local Variables:
%%% mode: latex
%%% coding: utf-8
%%% TeX-master: t
%%% End:
