% This is the file CJKutf8.tex of the CJK package
%   for testing the CJKutf8 style file.
%
% written by Werner Lemberg <wl@gnu.org>
%
% Version 4.8.5 (16-Oct-2021)

% Vietnamese support comes with the VnTeX package.
% UTF-8 support for Greek comes with the greek-inputenc package.
%
% - T2A encoding for Computer Modern Cyrillic requires LH fonts
% - uses CB Greek fonts, CM-Super fonts
% - requires cyrillic, greek-fontenc, greek-inputenc, vntex
%
% Read CJKutf8.txt for more details.

\documentclass[12pt]{article}

\usepackage{cmap}

\usepackage[10pt]{type1ec}          % use only 10pt fonts
\usepackage[T1]{fontenc}

\usepackage{CJKutf8}                % CJK wrapper of [utf8]{inputenc}
\usepackage[german,
            russian,                % T2A font encoding
            vietnamese,             % T5 font encoding
            greek,                  % LGR font encoding
            USenglish]{babel}

\usepackage{textalpha}              % improved LGR support

\usepackage[overlap, CJK]{ruby}
\usepackage{CJKulem}                % CJK wrapper of {ulem}

\renewcommand{\rubysep}{-0.2ex}

\newenvironment{SChinese}{%
  \CJKfamily{gbsn}%
  \CJKtilde
  \CJKnospace}{}
\newenvironment{TChinese}{%
  \CJKfamily{bsmi}%
  \CJKtilde
  \CJKnospace}{}
\newenvironment{Japanese}{%
  \CJKfamily{min}%
  \CJKtilde
  \CJKnospace}{}
\newenvironment{Korean}{%
  \CJKfamily{mj}}{}


\begin{document}

\parskip 3ex
\parindent 0pt

\begin{CJK}{UTF8}{}

\begin{Korean}
  이 FAQ 은 자주 반복되는 질문과 그에 대한 대답을 간단명료한 양식으로
  모아 엮어졌습니다. \uline{이 FAQ 의 구조는 지난 판에 비하여
    획기적으로 변경되었습니다.}  \textbf{상세한 것은 ``이 FAQ 을 어떻게
    읽을 것인가'' 라는 대목을 참조하시기 바랍니다.}
\end{Korean}

\begin{otherlanguage}{german}
  Dieses FAQ wurde erstellt, um häufig gestellte Fragen und ihre
  Antworten in einer gefälligen Form zusammenzufassen.  \uline{Die
    Struktur dieses FAQ wurde drastisch geändert seit der letzten
    Version.}  \textbf{Für Details siehe den Abschnitt "`Wie lese ich
    dieses FAQ"'.}
\end{otherlanguage}

\begin{SChinese}
  本常问问答集~(FAQ list)~是从一些经常被问到的问题及其适当的解答中,以
  方便的形式摘要而出的。\uline{跟上一版不同的是,其编排结构已彻底改变。}
  \textbf{有关新结构的细节,可参考「如何阅读本问答集及了解其编排结构」
    该项中的说明。}
\end{SChinese}

\begin{TChinese}
  本常問問答集~(FAQ list)~是從一些經常被問到的問題及其適當的解答中,以
  方便的形式摘要而出的。\uline{跟上一版不同的是,其編排結構已徹底改變。}
  \textbf{有關新結構的細節,可參考「如何閱讀本問答集及了解其編排結構」
    該項中的說明。}
\end{TChinese}

This FAQ list was made to summarize some frequently asked questions
and their answers in a convenient form.  \uline{The structure of this
  FAQ list has drastically changed since the last version.}
\textbf{For details of the new structure, see the entry of `How to
  read this FAQ and its structure'.}

\begin{Japanese}
  この~FAQ~リストは、よくある質問とその答を集め、役に立つようにしたもの
  です。\uline{この~FAQ~リストの構造は、以前のものと比べて大幅に変更さ
    れています。}\textbf{\ruby{新}{あたら}しい構造に関しては、「こ
    の~FAQ~ の読み方とその構造」の項目を\ruby{参}{さん}\ruby{照}{しょ
      う}して下さい。}
\end{Japanese}

\begin{otherlanguage}{vietnamese}
  Phần ``Những câu hỏi và giải đáp thường gặp'' (viết tắt từ tiếng Anh
  là FAQ) được nêu ra ở đây nhằm mục đích thu thập những câu hỏi
  thường gặp trong thực tế và những lời giải đáp thích hợp nhất của
  nó.  \uline{Từ lần ấn bản cuối cùng đến nay, đã có những sự thay đổi
    khá lớn trong cấu trúc của FAQ.}  \textbf{Để hiểu rõ hơn bạn nên
    xem lại chương ``Làm sao tôi có thể đọc đuợc FAQ''.}
\end{otherlanguage}

\begin{otherlanguage}{russian}
  Этот список был составлен для суммирования некоторых часто
  задаваемых вопросов (FAQ), вместе с ответами на них, в удобной
  форме.  \uline{Структура этого FAQ кардинально изменилась по
    сравнению с послед\-ней версией.}  \textbf{В разделе `Как читать
    этот FAQ и его структура' объяснены детали этой новой структуры.}
\end{otherlanguage}

\begin{otherlanguage}{greek}
  Η λίστα αυτή ΣΤΕ (συχνά τιθεμένων ερωτήσεων) έχει σαν σκοπό να
  συμπεριλάβει σε εύχρηστη μορφή κάποιες σημαντικές ερωτήσεις και τις
  απαντήσεις τους.  \uline{Η οργάνωση αυτής της λίστας άλλαξε
    σημαντικά από την τελευταία έκδοσή της και μετά.}  \textbf{Για
    λεπτομέρειες πάνω στη νέα οργάνωση, βλέπε το λήμμα <<Πώς να
    διαβάσετε αυτή την ΣΤΕ και πώς είναι οργανωμένη>>.}
\end{otherlanguage}

\end{CJK}

\end{document}


%%% Local Variables:
%%% coding: utf-8
%%% mode: latex
%%% TeX-master: t
%%% End:
