% This is the file `arphic-sampler.tex' which shows the shapes
% of the Arphic fonts for simplified and traditional Chinese.
%
% written by Werner Lemberg <wl@gnu.org> 17-Jul-2005

\documentclass[12pt]{article}

\usepackage{CJK}


\begin{document}

\begin{center}
\Large Arphic Font Sampler
\end{center}

\begin{CJK*}{UTF8}{}

\CJKtilde

\CJKfamily{gbsn}
\noindent AR PL SungtiL GB:

\noindent 本常问问答集~(FAQ list)~是从一些经常被问到的问题及其适当的解
答中,以方便的形式摘要而出的。跟上一版不同的是,其编排结构已彻底改变。
\textbf{有关新结构的细节,可参考「如何阅读本问答集及了解其编排结构」该
  项中的说明。}

\vspace{1cm}

\CJKfamily{gkai}
\noindent AR PL KaitiM GB:

\noindent 本常问问答集~(FAQ list)~是从一些经常被问到的问题及其适当的解
答中,以方便的形式摘要而出的。跟上一版不同的是,其编排结构已彻底改变。
\textbf{有关新结构的细节,可参考「如何阅读本问答集及了解其编排结构」该
  项中的说明。}

\vspace{1cm}

\CJKfamily{bsmi}
\noindent AR PL Mingti2L Big5:

\noindent 本常問問答集~(FAQ list)~是從一些經常被問到的問題及其適當的解
答中,以方便的形式摘要而出的。跟上一版不同的是,其編排結構已徹底改變。
\textbf{有關新結構的細節,可參考「如何閱讀本問答集及了解其編排結構」該
  項中的說明。}

\vspace{1cm}

\CJKfamily{bkai}
\noindent AR PL KaitiM Big5:

\noindent 本常問問答集~(FAQ list)~是從一些經常被問到的問題及其適當的解
答中,以方便的形式摘要而出的。跟上一版不同的是,其編排結構已徹底改變。
\textbf{有關新結構的細節,可參考「如何閱讀本問答集及了解其編排結構」該
  項中的說明。}

\end{CJK*}

\end{document}


%%% Local Variables:
%%% coding: utf-8
%%% mode: latex
%%% TeX-master: t
%%% End:
