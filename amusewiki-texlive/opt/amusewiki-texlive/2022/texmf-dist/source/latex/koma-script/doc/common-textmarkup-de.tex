% ======================================================================
% common-textmarkup-de.tex
% Copyright (c) Markus Kohm, 2001-2022
%
% This file is part of the LaTeX2e KOMA-Script bundle.
%
% This work may be distributed and/or modified under the conditions of
% the LaTeX Project Public License, version 1.3c of the license.
% The latest version of this license is in
%   http://www.latex-project.org/lppl.txt
% and version 1.3c or later is part of all distributions of LaTeX 
% version 2005/12/01 or later and of this work.
%
% This work has the LPPL maintenance status "author-maintained".
%
% The Current Maintainer and author of this work is Markus Kohm.
%
% This work consists of all files listed in MANIFEST.md.
% ======================================================================
%
% Paragraphs that are common for several chapters of the KOMA-Script guide
% Maintained by Markus Kohm
%
% ======================================================================

\KOMAProvidesFile{common-textmarkup-de.tex}
                 [$Date: 2022-06-05 12:40:11 +0200 (So, 05. Jun 2022) $
                  KOMA-Script guide (common paragraphs)]

\section{Textauszeichnungen}
\seclabel{textmarkup}%
\BeginIndexGroup
\BeginIndex{}{Text>Auszeichnung}%
\BeginIndex{}{Schrift>Art}%
%\BeginIndex{}{Element=\UseIndex {gen}\protect\GuideFontElement}%

\IfThisCommonFirstRun{}{%
  Es gilt sinngemäß, was in \autoref{sec:\ThisCommonFirstLabelBase.textmarkup}
  geschrieben wurde. Falls Sie also
  \autoref{sec:\ThisCommonFirstLabelBase.textmarkup} bereits gelesen und
  verstanden haben, können Sie
  \IfThisCommonLabelBaseOneOf{scrextend,scrlayer-notecolumn}{%
  }{%
    \IfThisCommonLabelBase{scrjura}{% Umbruchkorrektur
      % unter Beachtung von \autoref{tab:\ThisCommonLabelBase.fontelements}%
    }{%
      sich auf \autoref{tab:\ThisCommonLabelBase.fontelements},
      \autopageref{tab:\ThisCommonLabelBase.fontelements} beschränken und
      ansonsten %
    }%
  }%
  auf \autopageref{sec:\ThisCommonLabelBase.textmarkup.next} mit
  \autoref{sec:\ThisCommonLabelBase.textmarkup.next} fortfahren.%
  \IfThisCommonLabelBase{scrextend}{\ Es sei jedoch darauf
    hingewiesen\textnote{Einschränkung} dass aus
    \autoref{tab:maincls.fontelements}, \autopageref{tab:maincls.fontelements}
    nur die Elemente für den Dokumenttitel, den schlauen Spruch, die Fußnoten
    und die \DescRef{maincls.env.labeling}-Umgebung unterstützt werden. Das
    Element \DescRef{maincls.fontelement.disposition} ist zwar auch vorhanden,
    wird jedoch nur für den Dokumenttitel verwendet.%
  }{}%
}

% Umbruchkorrektur
\IfThisCommonLabelBaseOneOf{scrlayer-scrpage,scrextend,scrjura}{}{%
  \LaTeX{} verfügt über eine ganze Reihe von Anweisungen zur
  Textauszeichnung. %
  \IfThisCommonLabelBaseOneOf{scrlttr2}{}{%
    Neben der Wahl der Schriftart gehören dazu auch Befehle zur Wahl einer
    Textgröße oder der Textausrichtung. %
  }%
  Näheres zu den normalerweise definierten Möglichkeiten ist \cite{l2kurz},
  \cite{latex:usrguide} und \cite{latex:fntguide} zu entnehmen.%
}%

% Die beiden Anweisungen sind schon so lange in LaTeX, dass ich sie hier nicht
% mehr dokumentiere, obwohl KOMA-Script ggf. noch immer \textsubscript selbst
% definiert.
%\IfThisCommonLabelBaseOneOf{scrlayer-scrpage,scrjura,scrlayer-notecolumn}{%
  \iffalse%
%}{%
%  \csname iftrue\endcsname}%
  \begin{Declaration}
    \Macro{textsuperscript}\Parameter{Text}
    \Macro{textsubscript}\Parameter{Text}
  \end{Declaration}
  Im \LaTeX-Kern ist bereits die Anweisung
  \Macro{textsuperscript}\IndexCmd{textsuperscript} definiert, mit der
  \PName{Text} höher gestellt werden kann.  Eine
  entsprechende Anweisung, um Text tief\Index{Tiefstellung} statt
  hoch\Index{Hochstellung} zu stellen, bietet
  \LaTeX{}\textnote{\LaTeX~2015/01/01} erst seit Version
  2015/01/01. Für ältere \LaTeX-Versionen definiert \KOMAScript{} daher
  \Macro{textsubscript}. %
  \ifthiscommonfirst
    \begin{Example}
      \phantomsection
      \xmpllabel{cmd.textsubscript}%
      Sie schreiben einen Text über den menschlichen Stoffwechsel. Darin
      kommen hin und wieder einfache chemische Summenformeln vor. Dabei sind
      einzelne Ziffern tief zu stellen. Im Sinne des logischen Markups
      definieren Sie zunächst in der Dokumentpräambel oder einem eigenen
      Paket:
\begin{lstcode}
  \newcommand*{\Molek}[2]{#1\textsubscript{#2}}
\end{lstcode}
      \newcommand*{\Molek}[2]{#1\textsubscript{#2}}%
      Damit schreiben Sie dann:
\begin{lstcode}
  Die Zelle bezieht ihre Energie unter anderem aus
  der Reaktion von \Molek C6\Molek H{12}\Molek O6 
  und \Molek O2 zu \Molek H2\Molek O{} und 
  \Molek C{}\Molek O2. Arsen (\Molek{As}{}) wirkt
  sich auf den Stoffwechsel sehr nachteilig aus.
\end{lstcode}
      Das Ergebnis sieht daraufhin so aus:
      \begin{ShowOutput}
        Die Zelle bezieht ihre Energie unter anderem
        aus der Reaktion von 
        \Molek C6\Molek H{12}\Molek O6 und \Molek O2 zu
        \Molek H2\Molek O{} und \Molek C{}\Molek O2.  
        Arsen (\Molek{As}{}) wirkt sich auf
        den Stoffwechsel sehr nachteilig aus.
      \end{ShowOutput}

      Etwas später entscheiden Sie, dass Summenformeln grundsätzlich
      serifenlos geschrieben werden sollen. Nun zeigt sich, wie gut die
      Entscheidung für konsequentes logisches Markup war. Sie müssen nur die
      \Macro{Molek}-Anweisung umdefinieren:
\begin{lstcode}
  \newcommand*{\Molek}[2]{%
    \textsf{#1\textsubscript{#2}}%
  }
\end{lstcode}
      \renewcommand*{\Molek}[2]{\textsf{#1\textsubscript{#2}}}%
      Schon ändert sich die Ausgabe im gesamten Dokument:
      \begin{ShowOutput}
        Die Zelle bezieht ihr Energie unter anderem aus der Reaktion von
        \Molek C6\Molek H{12}\Molek O6 und \Molek O2 zu \Molek H2\Molek
        O{} und \Molek C{}\Molek O2.  Arsen (\Molek{As}{}) wirkt sich
        auf den Stoffwechsel sehr nachteilig aus.
      \end{ShowOutput}
    \end{Example}
    \iffalse % vielleicht in einer späteren Auf-lage
      Für Experten ist in \autoref{sec:experts.knowhow},
      \DescPageRef{experts.macroargs} dokumentiert, warum das Beispiel
      funktioniert, obwohl teilweise die Argumente von \Macro{Molek} nicht in
      geschweifte Klammern gesetzt wurden.%
    \fi%
  \else%
    Ein Anwendungsbeispiel finden Sie in
    \autoref{sec:\ThisCommonFirstLabelBase.textmarkup},
    \PageRefxmpl{\ThisCommonFirstLabelBase.cmd.textsubscript}.%
  \fi%
  \EndIndexGroup%
\fi


\begin{Declaration}
  \Macro{setkomafont}\Parameter{Element}\Parameter{Befehle}%
  \Macro{addtokomafont}\Parameter{Element}\Parameter{Befehle}%
  \Macro{usekomafont}\Parameter{Element}
\end{Declaration}%
Mit%
\IfThisCommonLabelBase{maincls}{%
  \ChangedAt{v2.8p}{\Class{scrbook}\and \Class{scrreprt}\and
    \Class{scrartcl}}%
}{} Hilfe der Anweisungen \Macro{setkomafont} und \Macro{addtokomafont} ist es
möglich, die \PName{Befehle} festzulegen, mit denen die Schrift eines
bestimmten \PName{Element}s umgeschaltet wird. Theoretisch könnten als
\PName{Befehle} alle möglichen Anweisungen einschließlich Textausgaben
verwendet werden.  Sie\textnote{Achtung!}  sollten sich jedoch unbedingt auf
solche Anweisungen beschränken, mit denen wirklich nur Schriftattribute
umgeschaltet werden. In der Regel werden dies Befehle wie \Macro{rmfamily},
\Macro{sffamily}, \Macro{ttfamily}, \Macro{upshape}, \Macro{itshape},
\Macro{slshape}, \Macro{scshape}, \Macro{mdseries}, \Macro{bfseries},
\Macro{normalfont} oder einer der Befehle \Macro{Huge}, \Macro{huge},
\Macro{LARGE}, \Macro{Large}, \Macro{large}, \Macro{normalsize},
\Macro{small}, \Macro{footnotesize}, \Macro{scriptsize} und \Macro{tiny}
sein. Die Erklärung zu diesen Befehlen entnehmen Sie bitte \cite{l2kurz},
\cite{latex:usrguide} oder \cite{latex:fntguide}. Auch Farbumschaltungen wie
\Macro{normalcolor} sind möglich (siehe \cite{package:graphics} und
\cite{package:xcolor}).%
\iffalse % Umbruchkorrekturtext
\ Das Verhalten bei Verwendung anderer Anweisungen, inbesondere solcher, die
zu Umdefinierungen führen oder Ausgaben tätigen, ist nicht
definiert. Seltsames Verhalten ist möglich und stellt keinen Fehler dar.
\else%
\ Die Verwendung anderer Anweisungen, inbesondere solcher, die Umdefinierungen
vornehmen oder zu Ausgaben führen, ist nicht vorgesehen.  Seltsames Verhalten
ist in diesen Fällen möglich und stellt keinen Fehler dar.%
\fi

Mit \Macro{setkomafont}\important{\Macro{setkomafont}} wird die
Schriftumschaltung eines Elements mit einer völlig neuen Definition
versehen. Demgegenüber wird mit
\Macro{addtokomafont}\important{\Macro{addtokomafont}} die existierende
Definition lediglich erweitert.  Es wird empfohlen, beide Anweisungen nicht
innerhalb des Dokuments, sondern nur in der Dokumentpräambel zu
verwenden. Beispiele für die Verwendung entnehmen Sie bitte den Abschnitten zu
den jeweiligen Elementen.%
\IfThisCommonLabelBase{scrlayer-notecolumn}{}{%
  \ Namen und Bedeutung der einzelnen Elemente
  \IfThisCommonLabelBaseOneOf{scrlayer-scrpage,scrjura}{und deren
    Voreinstellungen }{}%
  sind in %
  \IfThisCommonLabelBase{scrextend}{%
    \autoref{tab:maincls.fontelements}, \autopageref{tab:maincls.fontelements}
  }{%
    \autoref{tab:\ThisCommonLabelBase.fontelements} %
  }%
  aufgelistet.%
  \IfThisCommonLabelBase{scrextend}{ %
    Allerdings werden davon in \Package{scrextend} nur\textnote{Einschränkung}
    die Elemente für den Dokumenttitel, den schlauen Spruch, die Fußnoten und
    die \DescRef{maincls.env.labeling}-Umgebung behandelt. Das Element
    \DescRef{maincls.fontelement.disposition} ist zwar auch verfügbar, wird
    jedoch von \Package{scrextend} ebenfalls nur für den Dokumenttitel
    verwendet.%
  }{%
    \IfThisCommonLabelBase{scrlayer-scrpage}{ %
      Die angegebenen Voreinstellungen gelten nur, wenn das jeweilige Element
      beim Laden von \Package{scrlayer-scrpage} nicht bereits definiert
      ist. Beispielsweise definieren die \KOMAScript-Klassen
      \DescRef{maincls.fontelement.pageheadfoot} und es gilt dann die von
      \Package{scrlayer-scrpage} vorgefundene Einstellung.%
    }{%
      \IfThisCommonLabelBase{scrjura}{}{ %
        Die Voreinstellungen sind den jeweiligen Abschnitten zu entnehmen.%
      }%
    }%
  }%
}%

\IfThisCommonLabelBaseOneOf{scrlttr2,scrextend,scrlayer-notecolumn}{% Umbruchvarianten
  \IfThisCommonLabelBase{scrlayer-notecolumn}{\pagebreak}{}% Umbruchkorrektur
  Mit der Anweisung \Macro{usekomafont}\important{\Macro{usekomafont}} kann
  die aktuelle Schriftart auf die für das angegebene \PName{Element}
  umgeschaltet werden.%
}{%
  Mit der Anweisung \Macro{usekomafont}\important{\Macro{usekomafont}} kann
  die aktuelle Schriftart auf diejenige umgeschaltet werden, die für das
  angegebene \PName{Element} definiert ist.%
}
\IfThisCommonLabelBase{maincls}{\iftrue}{\csname iffalse\endcsname}
  \begin{Example}
    \phantomsection\xmpllabel{cmd.setkomafont}%
    Angenommen, für das Element
    \DescRef{\ThisCommonLabelBase.fontelement.captionlabel} soll dieselbe
    Schriftart wie für
    \DescRef{\ThisCommonLabelBase.fontelement.descriptionlabel} verwendet
    werden. Das erreichen Sie einfach mit:
\begin{lstcode}
  \setkomafont{captionlabel}{%
    \usekomafont{descriptionlabel}%
  }
\end{lstcode}
    Weitere Beispiele finden Sie in den Abschnitten zu den jeweiligen
    Elementen.
  \end{Example}

  \begin{desclist}
    \desccaption{%
      Elemente, deren Schrift bei \Class{scrbook}, \Class{scrreprt} oder
      \Class{scrartcl} mit \Macro{setkomafont} und \Macro{addtokomafont}
      verändert werden kann%
      \label{tab:maincls.fontelements}%
      \label{tab:scrextend.fontelements}%
    }{%
      Elemente, deren Schrift verändert werden kann (\emph{Fortsetzung})%
    }%
    \feentry{author}{%
      \ChangedAt{v3.12}{\Class{scrbook}\and \Class{scrreprt}\and
        \Class{scrartcl}\and \Package{scrextend}}%
      Autorangaben im Haupttitel des Dokuments mit
      \DescRef{\ThisCommonLabelBase.cmd.maketitle}, also das Argument von
      \DescRef{\ThisCommonLabelBase.cmd.author} (siehe
      \autoref{sec:maincls.titlepage}, \DescPageRef{maincls.cmd.author})}%
    \feentry{caption}{Text einer Abbildungs- oder Tabellenunter- oder
      "~überschrift (siehe \autoref{sec:maincls.floats},
      \DescPageRef{maincls.cmd.caption})}%
    \feentry{captionlabel}{%
      Label einer Abbildungs- oder Tabellenunter- oder "~überschrift;
      Anwendung erfolgt nach dem Element
      \DescRef{\ThisCommonLabelBase.fontelement.caption} (siehe
      \autoref{sec:maincls.floats}, \DescPageRef{maincls.cmd.caption})}%
    \feentry{chapter}{%
      Überschrift der Ebene \DescRef{\ThisCommonLabelBase.cmd.chapter} (siehe
      \autoref{sec:maincls.structure}, \DescPageRef{maincls.cmd.chapter})}%
    \feentry{chapterentry}{%
      Inhaltsverzeichniseintrag der Ebene
      \DescRef{\ThisCommonLabelBase.cmd.chapter} (siehe
      \autoref{sec:maincls.toc}, \DescPageRef{maincls.cmd.tableofcontents})}%
    \feentry{chapterentrydots}{%
      \ChangedAt{v3.15}{\Class{scrbook}\and \Class{scrreprt}}%
      optionale Verbindungspunkte in Inhaltsverzeichniseinträgen der Ebene
      \DescRef{\ThisCommonLabelBase.cmd.chapter} abweichend von Element
      \DescRef{\ThisCommonLabelBase.fontelement.chapterentry},
      \Macro{normalfont} und \Macro{normalsize} (siehe
      \autoref{sec:maincls.toc}, \DescPageRef{maincls.cmd.tableofcontents})}%
    \feentry{chapterentrypagenumber}{%
      Seitenzahl des Inhaltsverzeichniseintrags der Ebene
      \DescRef{\ThisCommonLabelBase.cmd.chapter} abweichend vom Element
      \DescRef{\ThisCommonLabelBase.fontelement.chapterentry} (siehe
      \autoref{sec:maincls.toc}, \DescPageRef{maincls.cmd.tableofcontents})}%
    \feentry{chapterprefix}{%
      Kapitelnummernzeile sowohl bei Einstellung
      \OptionValueRef{maincls}{chapterprefix}{true} als auch
      \OptionValueRef{maincls}{appendixprefix}{true} (siehe
      \autoref{sec:maincls.structure},
      \DescPageRef{maincls.option.chapterprefix})}%
    \feentry{date}{%
      \ChangedAt{v3.12}{\Class{scrbook}\and \Class{scrreprt}\and
        \Class{scrartcl}\and \Package{scrextend}}%
      Datum im Haupttitel des Dokuments mit
      \DescRef{\ThisCommonLabelBase.cmd.maketitle}, also das Argument von
      \DescRef{\ThisCommonLabelBase.cmd.date} (siehe
      \autoref{sec:maincls.titlepage}, \DescPageRef{maincls.cmd.date})}%
    \feentry{dedication}{%
      \ChangedAt{v3.12}{\Class{scrbook}\and \Class{scrreprt}\and
        \Class{scrartcl}\and \Package{scrextend}}%
      Widmung nach dem Haupttitel des Dokuments mit
      \DescRef{\ThisCommonLabelBase.cmd.maketitle}, also das Argument von
      \DescRef{\ThisCommonLabelBase.cmd.dedication} (siehe
      \autoref{sec:maincls.titlepage}, \DescPageRef{maincls.cmd.dedication})}%
    \feentry{descriptionlabel}{%
      Label, also das optionale Argument der
      \DescRef{\ThisCommonLabelBase.cmd.item.description}-Anweisung, in einer
      \DescRef{\ThisCommonLabelBase.env.description}-Umgebung (siehe
      \autoref{sec:maincls.lists}, \DescPageRef{maincls.env.description})}%
    \feentry{dictum}{%
      mit \DescRef{\ThisCommonLabelBase.cmd.dictum} gesetzter schlauer Spruch
      (siehe \autoref{sec:maincls.dictum}, \DescPageRef{maincls.cmd.dictum})}%
    \feentry{dictumauthor}{%
      Urheber eines schlauen Spruchs; Anwendung erfolgt nach dem Element
      \DescRef{\ThisCommonLabelBase.fontelement.dictum} (siehe
      \autoref{sec:maincls.dictum}, \DescPageRef{maincls.cmd.dictum})}%
    \feentry{dictumtext}{%
      alternative Bezeichnung für
      \DescRef{\ThisCommonLabelBase.fontelement.dictum}}%
    \feentry{disposition}{%
      alle Gliederungsüberschriften, also die Argumente von
      \DescRef{\ThisCommonLabelBase.cmd.part} bis
      \DescRef{\ThisCommonLabelBase.cmd.subparagraph} und
      \DescRef{\ThisCommonLabelBase.cmd.minisec} sowie die Überschrift der
      Zusammenfassung; die Anwendung erfolgt vor dem Element der jeweiligen
      Gliederungsebene (siehe \autoref{sec:maincls.structure} ab
      \autopageref{sec:maincls.structure})}%
    \feentry{footnote}{%
      Marke und Text einer Fußnote (siehe \autoref{sec:maincls.footnotes},
      \DescPageRef{maincls.cmd.footnote})}%
    \feentry{footnotelabel}{%
      Marke einer Fußnote; Anwendung erfolgt nach dem Element
      \DescRef{\ThisCommonLabelBase.fontelement.footnote} (siehe
      \autoref{sec:maincls.footnotes}, \DescPageRef{maincls.cmd.footnote})}%
    \feentry{footnotereference}{%
      Referenzierung der Fußnotenmarke im Text (siehe
      \autoref{sec:maincls.footnotes}, \DescPageRef{maincls.cmd.footnote})}%
    \feentry{footnoterule}{%
      Linie\ChangedAt{v3.07}{\Class{scrbook}\and \Class{scrreprt}\and
        \Class{scrartcl}} über dem Fußnotenapparat (siehe
      \autoref{sec:maincls.footnotes},
      \DescPageRef{maincls.cmd.setfootnoterule})}%
    \feentry{itemizelabel}{%
      \ChangedAt{v3.33}{\Class{scrbook}\and \Class{scrreprt}\and
        \Class{scrartcl}}%
      Grundeinstellung für die voreingestellten Aufzählungszeichen der
      Umgebung \DescRef{\ThisCommonLabelBase.env.itemize} (siehe
      \autoref{sec:maincls.lists}, \DescPageRef{maincls.env.itemize})}%
    \feentry{labelinglabel}{%
      Label, also das optionale Argument der
      \DescRef{\ThisCommonLabelBase.cmd.item.labeling}-Anweisung, und
      Trennzeichen, also das optionale Argument der
      \DescRef{\ThisCommonLabelBase.env.labeling}-Umgebung, in einer
      \DescRef{\ThisCommonLabelBase.env.labeling}-Umgebung (siehe
      \autoref{sec:maincls.lists}, \DescPageRef{maincls.env.labeling})}%
    \feentry{labelingseparator}{%
      Trennzeichen, also das optionale Argument der
      \DescRef{\ThisCommonLabelBase.env.labeling}-Umgebung, in einer
      \DescRef{\ThisCommonLabelBase.env.labeling}-Umgebung; Anwendung erfolgt
      nach dem Element
      \DescRef{\ThisCommonLabelBase.fontelement.labelinglabel} (siehe
      \autoref{sec:maincls.lists}, \DescPageRef{maincls.env.labeling})}%
    \feentry{labelitemi}{%
      \ChangedAt{v3.33}{\Class{scrbook}\and \Class{scrreprt}\and
        \Class{scrartcl}}%
      Schriftart für die Verwendung in der Definition des Aufzählungszeichens
      \DescRef{\ThisCommonLabelBase.cmd.labelitemi} (siehe
      \autoref{sec:maincls.lists}, \DescPageRef{maincls.env.itemize})}%
    \feentry{labelitemii}{%
      \ChangedAt{v3.33}{\Class{scrbook}\and \Class{scrreprt}\and
        \Class{scrartcl}}%
      Schriftart für die Verwendung in der Definition des Aufzählungszeichens
      \DescRef{\ThisCommonLabelBase.cmd.labelitemii} (siehe
      \autoref{sec:maincls.lists}, \DescPageRef{maincls.env.itemize})}%
    \feentry{labelitemiii}{%
      \ChangedAt{v3.33}{\Class{scrbook}\and \Class{scrreprt}\and
        \Class{scrartcl}}%
      Schriftart für die Verwendung in der Definition des Aufzählungszeichens
      \DescRef{\ThisCommonLabelBase.cmd.labelitemiii} (siehe
      \autoref{sec:maincls.lists}, \DescPageRef{maincls.env.itemize})}%
    \feentry{labelitemiv}{%
      \ChangedAt{v3.33}{\Class{scrbook}\and \Class{scrreprt}\and
        \Class{scrartcl}}%
      Schriftart für die Verwendung in der Definition des Aufzählungszeichens
      \DescRef{\ThisCommonLabelBase.cmd.labelitemiv} (siehe
      \autoref{sec:maincls.lists}, \DescPageRef{maincls.env.itemize})}%
    \feentry{minisec}{%
      mit \DescRef{\ThisCommonLabelBase.cmd.minisec} gesetzte Überschrift
      (siehe \autoref{sec:maincls.structure} ab
      \DescPageRef{maincls.cmd.minisec})}%
    \feentry{pagefoot}{%
      wird nur verwendet, wenn das Paket \Package{scrlayer-scrpage} geladen
      ist (siehe \autoref{cha:scrlayer-scrpage},
      \DescPageRef{scrlayer-scrpage.fontelement.pagefoot})}%
    \feentry{pagehead}{%
      alternative Bezeichnung für
      \DescRef{\ThisCommonLabelBase.fontelement.pageheadfoot}, solange
      \hyperref[cha:scrlayer-scrpage]{\Package{scrlayer-scrpage}} nicht
      geladen ist (siehe auch
      \autoref{sec:scrlayer-scrpage.predefined.pagestyles},
      \DescPageRef{scrlayer-scrpage.fontelement.pageheadfoot})}%
    \feentry{pageheadfoot}{%
      Seitenkopf und Seitenfuß bei allen von \KOMAScript{} definierten
      Seitenstilen (siehe \autoref{sec:maincls.pagestyle} ab
      \autopageref{sec:maincls.pagestyle})}%
    \feentry{pagenumber}{%
      Seitenzahl im Kopf oder Fuß der Seite (siehe
      \autoref{sec:maincls.pagestyle},
      \DescPageRef{\LabelBase.fontelement.pagenumber})}%
    \feentry{pagination}{%
      alternative Bezeichnung für
      \DescRef{\ThisCommonLabelBase.fontelement.pagenumber}}%
    \feentry{paragraph}{%
      Überschrift der Ebene \DescRef{\ThisCommonLabelBase.cmd.paragraph}
      (siehe \autoref{sec:maincls.structure},
      \DescPageRef{maincls.cmd.paragraph})}%
    \feentry{part}{%
      Überschrift der Ebene \DescRef{\ThisCommonLabelBase.cmd.part}, jedoch
      ohne die Zeile mit der Nummer des Teils (siehe
      \autoref{sec:maincls.structure}, \DescPageRef{maincls.cmd.part})}%
    \feentry{partentry}{%
      Inhaltsverzeichniseintrag der Ebene
      \DescRef{\ThisCommonLabelBase.cmd.part} (siehe
      \autoref{sec:maincls.toc}, \DescPageRef{maincls.cmd.tableofcontents})}%
    \feentry{partentrypagenumber}{%
      Seitenzahl des Inhaltsverzeichniseintrags der Ebene
      \DescRef{\ThisCommonLabelBase.cmd.part} abweichend vom Element
      \DescRef{\ThisCommonLabelBase.fontelement.partentry} (siehe
      \autoref{sec:maincls.toc}, \DescPageRef{maincls.cmd.tableofcontents})}%
    \feentry{partnumber}{%
      Zeile mit der Nummer des Teils in Überschrift der Ebene
      \DescRef{\ThisCommonLabelBase.cmd.part} (siehe
      \autoref{sec:maincls.structure}, \DescPageRef{maincls.cmd.part})}%
    \feentry{publishers}{%
      \ChangedAt{v3.12}{\Class{scrbook}\and \Class{scrreprt}\and
        \Class{scrartcl}\and \Package{scrextend}}%
      Verlagsangabe im Haupttitel des Dokuments mit
      \DescRef{\ThisCommonLabelBase.cmd.maketitle}, also das Argument von
      \DescRef{\ThisCommonLabelBase.cmd.publishers} (siehe
      \autoref{sec:maincls.titlepage}, \DescPageRef{maincls.cmd.publishers})}%
    \feentry{section}{%
      Überschrift der Ebene \DescRef{\ThisCommonLabelBase.cmd.section} (siehe
      \autoref{sec:maincls.structure}, \DescPageRef{maincls.cmd.section})}%
    \feentry{sectionentry}{%
      Inhaltsverzeichniseintrag der Ebene
      \DescRef{\ThisCommonLabelBase.cmd.section} (nur bei \Class{scrartcl}
      verfügbar, siehe \autoref{sec:maincls.toc},
      \DescPageRef{maincls.cmd.tableofcontents})}%
    \feentry{sectionentrydots}{%
      \ChangedAt{v3.15}{\Class{scrartcl}}%
      optionale Verbindungspunkte in Inhaltsverzeichniseinträgen der Ebene
      \DescRef{\ThisCommonLabelBase.cmd.section} abweichend vom Element
      \DescRef{\ThisCommonLabelBase.fontelement.sectionentry},
      \Macro{normalfont} und \Macro{normalsize} (nur bei \Class{scrartcl}
      verfügbar, siehe \autoref{sec:maincls.toc},
      \DescPageRef{maincls.cmd.tableofcontents})}%
    \feentry{sectionentrypagenumber}{%
      Seitenzahl des Inhaltsverzeichniseintrags der Ebene
      \DescRef{\ThisCommonLabelBase.cmd.section} abweichend vom Element
      \DescRef{\ThisCommonLabelBase.fontelement.sectionentry} (nur bei
      \Class{scrartcl} verfügbar, siehe \autoref{sec:maincls.toc},
      \DescPageRef{maincls.cmd.tableofcontents})}%
    \feentry{sectioning}{%
      alternative Bezeichnung für
      \DescRef{\ThisCommonLabelBase.fontelement.disposition}}%
    \feentry{subject}{%
      Typisierung des Dokuments, also das Argument von
      \DescRef{\ThisCommonLabelBase.cmd.subject} auf der Haupttitelseite mit
      \DescRef{\ThisCommonLabelBase.cmd.maketitle} (siehe
      \autoref{sec:maincls.titlepage}, \DescPageRef{maincls.cmd.subject})}%
    \feentry{subparagraph}{%
      Überschrift der Ebene \DescRef{\ThisCommonLabelBase.cmd.subparagraph}
      (siehe \autoref{sec:maincls.structure},
      \DescPageRef{maincls.cmd.subparagraph})}%
    \feentry{subsection}{%
      Überschrift der Ebene \DescRef{\ThisCommonLabelBase.cmd.subsection}
      (siehe \autoref{sec:maincls.structure},
      \DescPageRef{maincls.cmd.subsection})}%
    \feentry{subsubsection}{%
      Überschrift der Ebene \DescRef{\ThisCommonLabelBase.cmd.subsubsection}
      (siehe \autoref{sec:maincls.structure},
      \DescPageRef{maincls.cmd.subsubsection})}%
    \feentry{subtitle}{%
      Untertitel des Dokuments, also das Argument von
      \DescRef{\ThisCommonLabelBase.cmd.subtitle} auf der Haupttitelseite mit
      \DescRef{\ThisCommonLabelBase.cmd.maketitle} (siehe
      \autoref{sec:maincls.titlepage}, \DescPageRef{maincls.cmd.title})}%
    \feentry{title}{%
      Haupttitel des Dokuments, also das Argument von
      \DescRef{\ThisCommonLabelBase.cmd.title} bei Verwendung von
      \DescRef{\ThisCommonLabelBase.cmd.maketitle} (bezüglich der Größe des
      Haupttitels siehe die ergänzenden Bemerkungen im Text von
      \autoref{sec:maincls.titlepage} ab \DescPageRef{maincls.cmd.title})}%
    \feentry{titlehead}{%
      \ChangedAt{v3.12}{\Class{scrbook}\and \Class{scrreprt}\and
        \Class{scrartcl}\and \Package{scrextend}}%
      Kopf über dem Haupttitel des Dokuments, also das Argument von
      \DescRef{\ThisCommonLabelBase.cmd.titlehead} mit
      \DescRef{\ThisCommonLabelBase.cmd.maketitle} (siehe
      \autoref{sec:maincls.titlepage}, \DescPageRef{maincls.cmd.titlehead})}%
  \end{desclist}
\else
  \IfThisCommonLabelBase{scrextend}{\iftrue}{\csname iffalse\endcsname}
    \begin{Example}
      Angenommen, Sie wollen, dass der Titel in Serifenschrift und rot gesetzt
      wird. Das erreichen Sie einfach mit:
      \iffalse% Umbruchkorrektur
\begin{lstcode}[moretexcs=color]
  \setkomafont{title}{%
    \color{red}%
  }
\end{lstcode}
      \else%
\begin{lstcode}[moretexcs=color]
  \setkomafont{title}{\color{red}}
\end{lstcode}
      \fi%
      Für die Anweisung \Macro{color}\PParameter{red} wird das Paket
      \Package{color}\IndexPackage{color} oder
      \Package{xcolor}\IndexPackage{xcolor} benötigt. %
      \iffalse% Umbruchkorrektur
      Die zusätzliche Angabe von \Macro{normalfont} ist in diesem Beispiel
      deshalb nicht notwendig, weil diese Anweisung bereits in der Definition
      des Titels enthalten ist. %
      \fi %
      Das\textnote{Achtung!} Beispiel setzt voraus, dass
      Option \OptionValueRef{scrextend}{extendedfeature}{title} gesetzt ist
      (siehe \autoref{sec:scrextend.optionalFeatures},
      \DescPageRef{scrextend.option.extendedfeature}).
    \end{Example}
  \else
    \IfThisCommonLabelBase{scrlttr2}{%
      Ein allgemeines Beispiel für die Anwendung sowohl von
      \Macro{setkomafont} als auch
      \Macro{usekomafont} finden Sie in \autoref{sec:maincls.textmarkup},
      \PageRefxmpl{maincls.cmd.setkomafont}.

      \begin{desclist}
        \desccaption[{%
          Elemente, deren Schrift bei Briefen mit \Macro{setkomafont} und
          \Macro{addtokomafont} verändert werden kann%
        }]%
        {%
          Elemente, deren Schrift bei der Klasse \Class{scrlttr2} oder dem
          Paket \Package{scrletter} mit \Macro{setkomafont} und
          \Macro{addtokomafont} verändert werden
          kann\label{tab:scrlttr2.fontelements}%
        }{%
          Elemente, deren Schrift verändert werden kann (\emph{Fortsetzung})%
        }%
        \feentry{addressee}{Name und Anschrift im Anschriftfenster
          (\autoref{sec:scrlttr2.firstpage},
          \DescPageRef{scrlttr2.option.addrfield})}%
        \feentry{backaddress}{Rücksendeadresse für einen Fensterbriefumschlag
          (\autoref{sec:scrlttr2.firstpage},
          \DescPageRef{scrlttr2.option.backaddress})}%
        \feentry{descriptionlabel}{Label, also das optionale Argument von
          \DescRef{\ThisCommonLabelBase.cmd.item.description}, in einer
          \DescRef{\ThisCommonLabelBase.env.description}-Umgebung
          (\autoref{sec:scrlttr2.lists},
          \DescPageRef{scrlttr2.env.description})}%
        \feentry{foldmark}{Faltmarke auf dem Briefpapier; ermöglicht Änderung
          der Linienfarbe (\autoref{sec:scrlttr2.firstpage},
          \DescPageRef{scrlttr2.option.foldmarks})}%
        \feentry{footnote}{%
          Marke und Text einer Fußnote (\autoref{sec:scrlttr2.footnotes},
          \DescPageRef{scrlttr2.cmd.footnote})}%
        \feentry{footnotelabel}{%
          Marke einer Fußnote; Anwendung erfolgt nach dem Element
          \DescRef{\ThisCommonLabelBase.fontelement.footnote}
          (\autoref{sec:scrlttr2.footnotes},
          \DescPageRef{scrlttr2.cmd.footnote})}%
        \feentry{footnotereference}{%
          Referenzierung der Fußnotenmarke im Text
          (\autoref{sec:scrlttr2.footnotes},
          \DescPageRef{scrlttr2.cmd.footnote})}%
        \feentry{footnoterule}{%
          Linie\ChangedAt{v3.07}{\Class{scrlttr2}} über dem Fußnotenapparat
          (\autoref{sec:scrlttr2.footnotes},
          \DescPageRef{scrlttr2.cmd.setfootnoterule})}%
        \feentry{fromaddress}{Absenderadresse im Briefkopf
          (\autoref{sec:scrlttr2.firstpage},
          \DescPageRef{scrlttr2.variable.fromaddress})}%
        \feentry{fromname}{Name des Absenders im Briefkopf abweichend von
          \PValue{fromaddress} (\autoref{sec:scrlttr2.firstpage},
          \DescPageRef{scrlttr2.variable.fromname})}%
        \feentry{fromrule}{Linie im Absender im Briefkopf; gedacht für
          Farbänderungen (\autoref{sec:scrlttr2.firstpage},
          \DescPageRef{scrlttr2.option.fromrule})}%
        \feentry{itemizelabel}{%
          \ChangedAt{v3.33}{\Class{scrlttr2}}%
          Grundeinstellung für die voreingestellten Aufzählungszeichen der
          Umgebung \DescRef{\ThisCommonLabelBase.env.itemize} (siehe
          \autoref{sec:scrlttr2.lists}, \DescPageRef{scrlttr2.env.itemize})}%
        \feentry{labelinglabel}{%
          Label, also das optionale Argument der
          \DescRef{\ThisCommonLabelBase.cmd.item.labeling}-Anweisung, und
          Trennzeichen, also das optionale Argument der
          \DescRef{\ThisCommonLabelBase.env.labeling}-Umgebung, in einer
          \DescRef{\ThisCommonLabelBase.env.labeling}-Umgebung
          (\autoref{sec:scrlttr2.lists},
          \DescPageRef{scrlttr2.env.labeling})}%
        \feentry{labelingseparator}{%
          Trennzeichen, also das optionale Argument der
          \DescRef{\ThisCommonLabelBase.env.labeling}-Umgebung, in einer
          \DescRef{\ThisCommonLabelBase.env.labeling}-Umgebung; Anwendung
          erfolgt nach dem Element
          \DescRef{\ThisCommonLabelBase.fontelement.labelinglabel}
          (\autoref{sec:scrlttr2.lists},
          \DescPageRef{scrlttr2.env.labeling})}%
        \feentry{labelitemi}{%
          \ChangedAt{v3.33}{\Class{scrlttr2}}%
          Schriftart für die Verwendung in der Definition des
          Aufzählungszeichens \DescRef{\ThisCommonLabelBase.cmd.labelitemi}
          (siehe \autoref{sec:scrlttr2.lists},
          \DescPageRef{scrlttr2.env.itemize})}%
        \feentry{labelitemii}{%
          \ChangedAt{v3.33}{\Class{scrlttr2}}%
          Schriftart für die Verwendung in der Definition des
          Aufzählungszeichens \DescRef{\ThisCommonLabelBase.cmd.labelitemii}
          (siehe \autoref{sec:scrlttr2.lists},
          \DescPageRef{scrlttr2.env.itemize})}%
        \feentry{labelitemiii}{%
          \ChangedAt{v3.33}{\Class{scrlttr2}}%
          Schriftart für die Verwendung in der Definition des
          Aufzählungszeichens \DescRef{\ThisCommonLabelBase.cmd.labelitemiii}
          (siehe \autoref{sec:scrlttr2.lists},
          \DescPageRef{scrlttr2.env.itemize})}%
        \feentry{labelitemiv}{%
          \ChangedAt{v3.33}{\Class{scrlttr2}}%
          Schriftart für die Verwendung in der Definition des
          Aufzählungszeichens \DescRef{\ThisCommonLabelBase.cmd.labelitemiv}
          (siehe \autoref{sec:scrlttr2.lists},
          \DescPageRef{scrlttr2.env.itemize})}%
        \feentry{pagefoot}{%
          wird je nach Seitenstil nach
          \DescRef{\ThisCommonLabelBase.fontelement.pageheadfoot} auf den
          Seitenfuß angewendet (\autoref{sec:\LabelBase.pagestyle},
          \DescPageRef{\LabelBase.fontelement.pagefoot})}%
        \feentry{pagehead}{%
          wird je nach Seitenstil nach
          \DescRef{\ThisCommonLabelBase.fontelement.pageheadfoot} auf den
          Seitenkopf angewendet (\autoref{sec:\LabelBase.pagestyle},
          \DescPageRef{\LabelBase.fontelement.pagehead})}%
        \feentry{pageheadfoot}{%
          Seitenkopf und Seitenfuß bei allen von \KOMAScript{} definierten
          Seitenstilen (\autoref{sec:\LabelBase.pagestyle},
          \DescPageRef{\ThisCommonLabelBase.fontelement.pageheadfoot})}%
        \feentry{pagenumber}{%
          Seitenzahl im Kopf oder Fuß der Seite
          (\autoref{sec:\LabelBase.pagestyle},
          \DescPageRef{\ThisCommonLabelBase.fontelement.pagenumber})}%
        \feentry{pagination}{%
          alternative Bezeichnung für
          \DescRef{\ThisCommonLabelBase.fontelement.pagenumber}}%
        \feentry{placeanddate}{%
          \ChangedAt{v3.12}{\Class{scrlttr2}}%
          Ort und Datum, falls statt einer Geschäftszeile nur eine Datumszeile
          verwendet wird (\autoref{sec:scrlttr2.firstpage},
          \DescPageRef{scrlttr2.variable.placeseparator})}%
        \feentry{refname}{%
          Bezeichnung der Felder in der Geschäftszeile
          (\autoref{sec:scrlttr2.firstpage},
          \DescPageRef{scrlttr2.variable.yourref})}%
        \feentry{refvalue}{%
          Werte der Felder in der Geschäftszeile
          (\autoref{sec:scrlttr2.firstpage},
          \DescPageRef{scrlttr2.variable.yourref})}%
        \feentry{specialmail}{Versandart im Anschriftfenster
          (\autoref{sec:scrlttr2.firstpage},
          \DescPageRef{scrlttr2.variable.specialmail})}%
        \feentry{lettersubject}{%
          \ChangedAt{v3.17}{\Class{scrlttr2}\and \Package{scrletter}}%
          Betreff in der Brieferöffnung (\autoref{sec:scrlttr2.firstpage},
          \DescPageRef{scrlttr2.variable.subject})}%
        \feentry{lettertitle}{%
          \ChangedAt{v3.17}{\Class{scrlttr2}\and \Package{scrletter}}%
          Titel in der Brieferöffnung (\autoref{sec:scrlttr2.firstpage},
          \DescPageRef{scrlttr2.variable.title})}%
        \feentry{toaddress}{Abweichung vom Element
          \DescRef{\ThisCommonLabelBase.fontelement.addressee} für die
          Anschrift (ohne Name) des Empfängers im Anschriftfeld
          (\autoref{sec:scrlttr2.firstpage},
          \DescPageRef{scrlttr2.variable.toaddress})}%
        \feentry{toname}{Abweichung vom Element
          \DescRef{\ThisCommonLabelBase.fontelement.addressee} für den Namen
          des Empfängers im Anschriftfeld (\autoref{sec:scrlttr2.firstpage},
          \DescPageRef{scrlttr2.variable.toname})}%
      \end{desclist}
    }{%
      \IfThisCommonLabelBase{scrlayer-scrpage}{%
        \begin{desclist}
          \desccaption[{Elemente, deren Schrift bei \Package{scrlayer-scrpage}
            mit \Macro{setkomafont} und \Macro{addtokomafont} verändert werden
            kann, einschließlich der jeweiligen Voreinstellung}]%
          {Elemente, deren Schrift bei \Package{scrlayer-scrpage} mit
            \Macro{setkomafont} und \Macro{addtokomafont} verändert werden
            kann, einschließlich der jeweiligen Voreinstellung, falls die
            Elemente beim Laden von \Package{scrlayer-scrpage} nicht bereits
            definiert sind%
            \label{tab:scrlayer-scrpage.fontelements}%
          }%
          {Elemente, deren Schrift verändert werden kann
            (\emph{Fortsetzung})}%
          \feentry{footbotline}{%
            Linie unter dem Fuß eines mit \Package{scrlayer-scrpage}
            definierten Seitenstils (siehe
            \autoref{sec:scrlayer-scrpage.pagestyle.content},
            \DescPageRef{scrlayer-scrpage.fontelement.footbotline}). Das
            Element wird nach \Macro{normalfont}\IndexCmd{normalfont} und nach
            den Elementen
            \DescRef{\ThisCommonLabelBase.fontelement.pageheadfoot}%
            \IndexFontElement{pageheadfoot} und
            \DescRef{\ThisCommonLabelBase.fontelement.pagefoot}%
            \IndexFontElement{pagefoot} angewandt. Es wird empfohlen, dieses
            Element lediglich für Farbänderungen zu verwenden.\par
            \mbox{Voreinstellung: \emph{leer}}%
          }%
          \feentry{footsepline}{%
            Linie über dem Fuß eines mit \Package{scrlayer-scrpage}
            definierten Seitenstils (siehe
            \autoref{sec:scrlayer-scrpage.pagestyle.content},
            \DescPageRef{scrlayer-scrpage.fontelement.footsepline}). Das
            Element wird nach \Macro{normalfont}\IndexCmd{normalfont} und nach
            den Elementen
            \DescRef{\ThisCommonLabelBase.fontelement.pageheadfoot}%
            \IndexFontElement{pageheadfoot} und
            \DescRef{\ThisCommonLabelBase.fontelement.pagefoot}%
            \IndexFontElement{pagefoot} angewandt. Es wird empfohlen, dieses
            Element lediglich für Farbänderungen zu verwenden.\par
            \mbox{Voreinstellung: \emph{leer}}%
          }%
          \feentry{headsepline}{%
            Linie unter dem Kopf eines mit \Package{scrlayer-scrpage}
            definierten Seitenstils (siehe
            \autoref{sec:scrlayer-scrpage.pagestyle.content},
            \DescPageRef{scrlayer-scrpage.fontelement.headsepline}). Das
            Element wird nach \Macro{normalfont}\IndexCmd{normalfont} und nach
            den Elementen
            \DescRef{\ThisCommonLabelBase.fontelement.pageheadfoot}%
            \IndexFontElement{pageheadfoot} und
            \DescRef{scrlayer-scrpage.fontelement.pagehead}%
            \IndexFontElement{pagehead} angewandt. Es wird empfohlen, dieses
            Element lediglich für Farbänderungen zu verwenden.\par
            \mbox{Voreinstellung: \emph{leer}}%
          }%
          \feentry{headtopline}{%
            Linie über dem Kopf eines mit \Package{scrlayer-scrpage}
            definierten Seitenstils (siehe
            \autoref{sec:scrlayer-scrpage.pagestyle.content},
            \DescPageRef{scrlayer-scrpage.fontelement.headtopline}). Das
            Element wird nach \Macro{normalfont}\IndexCmd{normalfont} und nach
            den Elementen
            \DescRef{\ThisCommonLabelBase.fontelement.pageheadfoot}%
            \IndexFontElement{pageheadfoot} und
            \DescRef{scrlayer-scrpage.fontelement.pagehead}%
            \IndexFontElement{pagehead} angewandt. Es wird empfohlen, dieses
            Element lediglich für Farbänderungen zu verwenden.\par
            \mbox{Voreinstellung: \emph{leer}}%
          }%
          \feentry{pagefoot}{%
            Inhalt des Fußes eines mit \Package{scrlayer-scrpage}
            definierten Seitenstils (siehe
            \autoref{sec:scrlayer-scrpage.predefined.pagestyles},
            \DescPageRef{scrlayer-scrpage.fontelement.pagefoot}). Das Element
            wird nach \Macro{normalfont}\IndexCmd{normalfont} und nach dem
            Element \DescRef{\ThisCommonLabelBase.fontelement.pageheadfoot}%
            \IndexFontElement{pageheadfoot} angewandt.\par
            \mbox{Voreinstellung: \emph{leer}}%
          }%
          \feentry{pagehead}{%
            Inhalt des Kopfes eines mit \Package{scrlayer-scrpage}
            definierten Seitenstils (siehe
            \autoref{sec:scrlayer-scrpage.predefined.pagestyles},
            \DescPageRef{scrlayer-scrpage.fontelement.pagehead}). Das Element
            wird nach \Macro{normalfont}\IndexCmd{normalfont} und nach Element
            \DescRef{\ThisCommonLabelBase.fontelement.pageheadfoot}%
            \IndexFontElement{pageheadfoot} angewandt.\par
            \mbox{Voreinstellung: \emph{leer}}%
          }%
          \feentry{pageheadfoot}{%
            Inhalt des Kopfes oder Fußes eines mit
            \Package{scrlayer-scrpage} definierten Seitenstils (siehe
            \autoref{sec:scrlayer-scrpage.predefined.pagestyles},
            \DescPageRef{scrlayer-scrpage.fontelement.pageheadfoot}). Das
            Element wird nach \Macro{normalfont}\IndexCmd{normalfont}
            angewandt.\par
            \mbox{Voreinstellung:
              \Macro{normalcolor}\Macro{slshape}}%
          }%
          \feentry{pagenumber}{%
            Die mit \DescRef{\ThisCommonLabelBase.cmd.pagemark} gesetzte
            Paginierung (siehe
            \autoref{sec:scrlayer-scrpage.predefined.pagestyles},
            \DescPageRef{scrlayer-scrpage.fontelement.pagenumber})%
            \iftrue % Umbruchkorrektur
            . Bei einer
            etwaigen Umdefinierung von
            \DescRef{\ThisCommonLabelBase.cmd.pagemark} ist dafür zu sorgen,
            dass die Umdefinierung auch ein
            \Macro{usekomafont}\PParameter{pagenumber} enthält!%
            \else %
            , solange \DescRef{\ThisCommonLabelBase.cmd.pagemark} nicht
            unsachgemäß umdefiniert wird.%
            \fi %
            \par
            \mbox{Voreinstellung: \Macro{normalfont}}%
          }%
        \end{desclist}
      }{%
        \IfThisCommonLabelBase{scrjura}{%
          \par Ein allgemeines Beispiel für die Anwendung von
          \Macro{setkomafont} und \Macro{usekomafont} finden Sie in
          \autoref{sec:maincls.textmarkup},
          \PageRefxmpl{maincls.cmd.setkomafont}.%
          \begin{table}
            \caption{Elemente, deren Schrift bei \Package{scrjura} mit
              \Macro{setkomafont} und \Macro{addtokomafont} verändert werden
              kann, einschließlich der jeweiligen Voreinstellung}%
            \label{tab:scrjura.fontelements}%
            \begin{desctabular}
              \feentry{Clause}{%
                Alias für \FontElement{\PName{Umgebungsname}.Clause}
                innerhalb einer Vertragsumgebung, beispielsweise
                \FontElement{contract.Clause} innerhalb von
                \DescRef{\ThisCommonLabelBase.env.contract}; ist kein
                entsprechendes Element definiert, so wird auf
                \FontElement{contract.Clause} zurückgegriffen%
              }\\[-1.7ex]
              \feentry{contract.Clause}{%
                Überschrift eines Paragraphen innerhalb von
                \DescRef{\ThisCommonLabelBase.env.contract} (siehe
                \autoref{sec:\ThisCommonLabelBase.contract},
                \DescPageRef{\ThisCommonLabelBase.fontelement.contract.Clause});
                \par
                \mbox{Voreinstellung:
                  \Macro{sffamily}\Macro{bfseries}\Macro{large}}%
              }\\[-1.7ex]
              \entry{\DescRef{\ThisCommonLabelBase.fontelement./Name/.Clause}}{%
                \IndexFontElement[indexmain]{\PName{Name}.Clause}%
                Überschrift eines Paragraphen innerhalb einer Umgebung
                \PName{Name}, die mit
                \DescRef{\ThisCommonLabelBase.cmd.DeclareNewJuraEnvironment}
                definiert wurde, soweit bei der Definition mit
                \Option{ClauseFont} eine Einstellung vorgenommen wurde oder
                das Element nachträglich definiert wurde (siehe
                \autoref{sec:\ThisCommonLabelBase.newenv},
                \DescPageRef{\ThisCommonLabelBase.fontelement./Name/.Clause});
                \par
                \mbox{Voreinstellung: \emph{keine}}%
              }\\[-1.7ex]
              \feentry{parnumber}{%
                Absatznummern innerhalb einer Vertragsumgebung (siehe
                \autoref{sec:\ThisCommonLabelBase.par},
                \DescPageRef{\ThisCommonLabelBase.fontelement.parnumber});\par
                \mbox{Voreinstellung: \emph{leer}}%
              }\\[-1.7ex]
              \feentry{sentencenumber}{%
                \ChangedAt{v3.26}{\Package{scrjura}}%
                Satznummer der Anweisung
                \DescRef{\ThisCommonLabelBase.cmd.Sentence} (siehe
                \autoref{sec:\ThisCommonLabelBase.sentence}, \DescPageRef{%
                  \ThisCommonLabelBase.fontelement.sentencenumber});\par
                \mbox{Voreinstellung: \emph{leer}}%
              }%
            \end{desctabular}
          \end{table}
        }{%
          \IfThisCommonLabelBase{scrlayer-notecolumn}{}{%
            \InternalCommonFileUsageError%
          }%
        }%
      }%
    }%
  \fi%
\fi
\EndIndexGroup
\IfThisCommonLabelBase{scrextend}{\ExampleEndFix}{}%


\begin{Declaration}
  \Macro{usefontofkomafont}\Parameter{Element}%
  \Macro{useencodingofkomafont}\Parameter{Element}%
  \Macro{usesizeofkomafont}\Parameter{Element}%
  \Macro{usefamilyofkomafont}\Parameter{Element}%
  \Macro{useseriesofkomafont}\Parameter{Element}%
  \Macro{useshapeofkomafont}\Parameter{Element}%
\end{Declaration}
Manchmal\ChangedAt{v3.12}{\Class{scrbook}\and \Class{scrreprt}\and
  \Class{scrartcl}\and \Package{scrextend}} werden in der Schrifteinstellung
eines Elements auch Dinge vorgenommen, die mit der Schrift eigentlich gar
nichts zu tun haben, obwohl dies ausdrücklich nicht empfohlen wird. Soll dann
nur die Schrifteinstellung, aber keine dieser zusätzlichen Einstellungen
ausgeführt werden, so kann statt
\DescRef{\ThisCommonLabelBase.cmd.usekomafont} die Anweisung
\Macro{usefontofkomafont} verwendet werden. Diese Anweisung übernimmt nur die
Schriftgröße und den Grundlinienabstand, die Codierung
(engl. \emph{encoding}), die Familie (engl. \emph{family}), die Strichstärke
oder Ausprägung (engl. \emph{font series}) und die Form oder Ausrichtung
(engl. \emph{font shape}).

Mit den übrigen Anweisungen können auch einzelne Schriftattribute
übernommen werden. Dabei übernimmt \Macro{usesizeofkomafont} sowohl die
Schriftgröße als auch den Grundlinienabstand.%
%
\IfThisCommonLabelBase{scrextend}{% Umbruchvariante!
}{%
  \IfThisCommonLabelBaseOneOf{scrjura,scrlttr2,scrlayer-scrpage}{%
    \par%
    Vor dem Missbrauch der Schrifteinstellungen wird dennoch dringend gewarnt
    (siehe \autoref{sec:maincls-experts.fonts},
    \DescPageRef{maincls-experts.cmd.addtokomafontrelaxlist})!%
  }{%
    \par%
    Diese Befehle sollten jedoch nicht als Legitimation dafür verstanden
    werden, in die Schrifteinstellungen der Elemente beliebige Anweisungen
    einzufügen. Das kann nämlich sehr schnell zu Fehlern führen (siehe
    \autoref{sec:maincls-experts.fonts},
    \DescPageRef{maincls-experts.cmd.addtokomafontrelaxlist}).%
  }%
}%
\EndIndexGroup
%
\EndIndexGroup

%%% Local Variables: 
%%% mode: latex
%%% TeX-master: "scrguide-de.tex"
%%% coding: utf-8
%%% ispell-local-dictionary: "de_DE"
%%% eval: (flyspell-mode 1)
%%% End: 
