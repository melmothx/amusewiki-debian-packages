% ======================================================================
% common-footnotes-en.tex
% Copyright (c) Markus Kohm, 2001-2022
%
% This file is part of the LaTeX2e KOMA-Script bundle.
%
% This work may be distributed and/or modified under the conditions of
% the LaTeX Project Public License, version 1.3c of the license.
% The latest version of this license is in
%   http://www.latex-project.org/lppl.txt
% and version 1.3c or later is part of all distributions of LaTeX 
% version 2005/12/01 or later and of this work.
%
% This work has the LPPL maintenance status "author-maintained".
%
% The Current Maintainer and author of this work is Markus Kohm.
%
% This work consists of all files listed in MANIFEST.md.
% ======================================================================
%
% Paragraphs that are common for several chapters of the KOMA-Script guide
% Maintained by Markus Kohm
%
% ======================================================================

\KOMAProvidesFile{common-footnotes-en.tex}
                 [$Date: 2022-06-05 12:40:11 +0200 (So, 05. Jun 2022) $
                  KOMA-Script guide (common paragraphs: Footnotes)]
\translator{Markus Kohm\and Krickette Murabayashi\and Karl Hagen}

\section{Footnotes}
\seclabel{footnotes}%
\BeginIndexGroup
\BeginIndex{}{footnotes}%

\IfThisCommonFirstRun{}{%
  The information in \autoref{sec:\ThisCommonFirstLabelBase.footnotes} applies
  equally to this chapter. So if you have already read and understood
  \autoref{sec:\ThisCommonFirstLabelBase.footnotes}, you can skip ahead to
  \autopageref{sec:\ThisCommonLabelBase.footnotes.next},
  \autopageref{sec:\ThisCommonLabelBase.footnotes.next}.%
  \IfThisCommonLabelBase{scrlttr2}{ %
    If you do not use a \KOMAScript{} class, \Package{scrletter}%
    \OnlyAt{\Package{scrletter}} relies on the
    \hyperref[cha:scrextend]{\Package{scrextend}}\IndexPackage{scrextend}%
    \important{\hyperref[cha:scrextend]{\Package{scrextend}}} package.
    Therefore, see also \autoref{sec:scrextend.footnotes},
    \autopageref{sec:scrextend.footnotes} when using \Package{scrletter}.%
    \iffalse% Umbruchkorrekturtext
    \ Note in particular that in this case some typical \KOMAScript{}
    extensions are not active by default\textnote{default}. Instead, the
    footnotes make use of the class used, or the \LaTeX{} kernel.%
    \fi%
  }{}%
}

\IfThisCommonLabelBase{maincls}{%
  Unlike\textnote{\KOMAScript{} vs. standard classes} the standard classes,
  \KOMAScript{} offers the ability to configure the format of the footnote
  block.%
}{%
  \IfThisCommonLabelBase{scrlttr2}{%
    You can find the basic commands to set footnotes in any introduction to
    \LaTeX, for example \cite{lshort}. \KOMAScript{}%
    \textnote{\KOMAScript{} vs. standard classes} provides additional features
    to change the format of the footnote block. %
    \iffalse % Umbruchoptimierung
	
      Whether footnotes should be allowed for letters depends very much on the
      type of letter and its layout. For example, you should not allow
      footnotes to overlap visually with the letterhead footer or be confused
      with the courtesy-copy list. Doing so is the responsibility of the
      user.%

      Since footnotes are rarely used in letters, examples in this section have
      been omitted. If you need examples, you can find them in
      \autoref{sec:\ThisCommonFirstLabelBase.footnotes}, 
      \autopageref{sec:\ThisCommonFirstLabelBase.footnotes}.%
    \fi%
  }{%
    \IfThisCommonLabelBase{scrextend}{%
      The footnote capabilities of the \KOMAScript{} classes are also
      provided by \Package{scrextend}. By default, the formatting of
      footnotes is left to the class used. This changes as soon as you
      issue the \DescRef{\ThisCommonLabelBase.cmd.deffootnote} command,
      which is explained in detail on
      \DescPageRef{\ThisCommonLabelBase.cmd.deffootnote}.
      
      The options for adjusting the dividing line above footnotes, however,
      are not provided by \Package{scrextend}.%
    }{\InternalCommonFileUsageError}%
  }%
}%

\begin{Declaration}
  \OptionVName{footnotes}{setting}
  \Macro{multfootsep}
\end{Declaration}
\IfThisCommonLabelBase{scrextend}{Many classes mark footnotes }{%
  Footnotes %
  \IfThisCommonLabelBase{maincls}{%
    \ChangedAt{v3.00}{\Class{scrbook}\and \Class{scrreprt}\and
      \Class{scrartcl}}%
  }{%
    \IfThisCommonLabelBase{scrlttr2}{%
      \ChangedAt{v3.00}{\Class{scrlttr2}}%
    }{}%
  }%
  are marked %
}%
by default in the text with a small superscript number. If several footnotes
appear in succession at the same point, it gives the impression that there is
one footnote with a large number rather than multiple footnotes (e.\,g.
footnote 12 instead of footnotes 1 and 2).
With\important{\OptionValue{footnotes}{multiple}}
\OptionValue{footnotes}{multiple}\IndexOption{footnotes=~multiple}, footnotes
that follow each other directly are separated with a delimiter instead. The
default delimiter in \Macro{multfootsep}\important{\Macro{multfootsep}} is
defined as a comma without a space:
\begin{lstcode}
  \newcommand*{\multfootsep}{,}
\end{lstcode}
This can be redefined.

The whole mechanism is compatible with the 
\Package{footmisc}\IndexPackage{footmisc}\important{\Package{footmisc}} 
package, version~5.3d to 5.5b (see \cite{package:footmisc}). It affects
footnote markers placed using 
\DescRef{\ThisCommonLabelBase.cmd.footnote}\IndexCmd{footnote}, as well as 
those placed directly with 
\DescRef{\ThisCommonLabelBase.cmd.footnotemark}\IndexCmd{footnotemark}.

You can switch back to the default 
\OptionValue{footnotes}{nomultiple} at any time using the 
\DescRef{\ThisCommonLabelBase.cmd.KOMAoptions} or
\DescRef{\ThisCommonLabelBase.cmd.KOMAoption} command. However, if you
encounter any problems using another package that alters the footnotes, you
should not use this option, nor should you change the \PName{setting} anywhere
inside the document.

A summary of the available \PName{setting} values of \Option{footnotes} can
be found in \autoref{tab:\ThisCommonFirstLabelBase.footnotes}%
\IfThisCommonFirstRun{%
  .
  \begin{table}
    \caption[{Available values for the \Option{footnotes} option}]
    {Available values for the \Option{footnotes} option to configure footnotes}
    \label{tab:\ThisCommonLabelBase.footnotes}
    \begin{desctabular}
      \pventry{multiple}{%
        Consecutive footnote marks will be separated by
        \DescRef{\ThisCommonLabelBase.cmd.multfootsep}\IndexCmd{multfootsep}.%
        \IndexOption{footnotes~=\textKValue{multiple}}}%
      \pventry{nomultiple}{%
        Consecutive footnote marks will be handled like single footnotes
		and not separated from each other.%
        \IndexOption{footnotes~=\textKValue{nomultiple}}}%
    \end{desctabular}
  \end{table}%
}{,
  \autopageref{tab:\ThisCommonFirstLabelBase.footnotes}.%
}%
%
\EndIndexGroup


\begin{Declaration}
  \Macro{footnote}\OParameter{number}\Parameter{text}%
  \Macro{footnotemark}\OParameter{number}%
  \Macro{footnotetext}\OParameter{number}\Parameter{text}%
  \Macro{multiplefootnoteseparator}%
\end{Declaration}%
Footnotes in \KOMAScript{} are produced, as they are in the standard classes,
with the \Macro{footnote} command, or alternatively the pair of commands
\Macro{footnotemark} and \Macro{footnotetext}. As in the standard classes,
it is possible for a page break to occur within a footnote. Normally this
happens if the footnote mark is placed so near the bottom of a page as to
leave \LaTeX{} no choice but to move the footnote to the next page.
Unlike\textnote{\KOMAScript{} vs. standard classes}
\IfThisCommonLabelBase{maincls}{%
  \ChangedAt{v3.00}{\Class{scrbook}\and \Class{scrreprt}\and
    \Class{scrartcl}}%
}{%
  \IfThisCommonLabelBase{scrlttr2}{%
    \ChangedAt{v3.00}{\Class{scrlttr2}}%
  }{}%
} %
the standard classes, \KOMAScript{} can recognize and separate consecutive
footnotes automatically. 
See\important{\DescRef{\ThisCommonLabelBase.option.footnotes}} the previously
documented option \DescRef{\ThisCommonLabelBase.option.footnotes}.

If instead you want to place this delimiter manually, you can do so by calling
\Macro{multiplefootnoteseparator}. However, users should not redefine this
command, as it contains not only the delimiter but also the delimiter's
formatting, for example the font size selection and the superscript. The
delimiter itself is stored in the previously described
\DescRef{\ThisCommonLabelBase.cmd.multfootsep}%
\important{\DescRef{\ThisCommonLabelBase.cmd.multfootsep}}%
\IndexCmd{multfootsep} command.

\IfThisCommonFirstRun{\iftrue}{%
  You can find examples and additional hints in
  \autoref{sec:\ThisCommonFirstLabelBase.footnotes} from
  \PageRefxmpl{\ThisCommonFirstLabelBase.cmd.footnote}.%
  \csname iffalse\endcsname }%
  \begin{Example}
    \phantomsection\xmpllabel{cmd.footnote}%
    Suppose you want to put two footnotes after a single word. First you try
\begin{lstcode}
  Word\footnote{1st footnote}\footnote{2nd footnote}
\end{lstcode}
    Let's assume that the footnotes are numbered 1 and 2. Since the two
    footnote numbers follow each other directly, it creates the impression
    that the word has only one footnote numbered 12. You can change this
    behaviour by using
\begin{lstcode}
  \KOMAoptions{footnotes=multiple}
\end{lstcode}
    to enable the automatic recognition of footnote sequences. Alternatively,
    you can use
\begin{lstcode}
  word\footnote{Footnote 1}%
  \multiplefootnoteseparator
  \footnote{Footnote 2}
\end{lstcode}
    This should give you the desired result even if automatic detection
    fails or cannot be used for some reason.
	
    Now suppose you also want the footnote numbers to be separated not just by
    a comma, but by a comma and a space. In this case, write
\begin{lstcode}
  \renewcommand*{\multfootsep}{,\nobreakspace}
\end{lstcode}
    in the preamble of your document.
    \Macro{nobreakspace}\IndexCmd{nobreakspace} was used here instead of a
    normal space to avoid paragraph or page breaks within the sequence of
    footnotes.
  \end{Example}%
\fi%
%
\EndIndexGroup


\begin{Declaration}
  \Macro{footref}\Parameter{reference}
\end{Declaration}
Sometimes\IfThisCommonLabelBase{maincls}{%
  \ChangedAt{v3.00}{\Class{scrbook}\and \Class{scrreprt}\and
    \Class{scrartcl}}%
}{%
  \IfThisCommonLabelBase{scrlttr2}{%
    \ChangedAt{v3.00}{\Class{scrlttr2}}%
  }{}} you have a footnote in a document to which there are several references
in the text. An inconvenient way to typeset this would be to use
\DescRef{\ThisCommonLabelBase.cmd.footnotemark} to set the number directly.
The disadvantage of this method is that you need to know the number and
manually set every \DescRef{\ThisCommonLabelBase.cmd.footnotemark} command.
And if the number changes because you add or remove an earlier footnote, you
will have to change each \DescRef{\ThisCommonLabelBase.cmd.footnotemark}.
\KOMAScript{} thefore offers the \Macro{label}\IndexCmd{label}%
\important{\Macro{label}} mechanism to handle such cases. After placing a
\Macro{label} inside the footnote, you can use \Macro{footref} to set all the
other marks for this footnote in the text.
\IfThisCommonFirstRun{\iftrue}{\csname iffalse\endcsname}%
  \begin{Example}
    \phantomsection\xmpllabel{cmd.footref}%
    You are writing a text in which you must create a footnote each time a
    brand name occurs, indicating that it is a registered trademark. You can
    write, for example,
\begin{lstcode}
  Company SplishSplash\footnote{This is a registered trade name.
    All rights are reserved.\label{refnote}}
  produces not only SplishPlump\footref{refnote}
  but also SplishPlash\footref{refnote}.
\end{lstcode}
    This will produce the same footnote mark three times, but only one
    footnote text. The first footnote mark is produced by
    \DescRef{\ThisCommonLabelBase.cmd.footnote} itself, and the following two
    footnote marks are produced by the additional \Macro{footref}
    commands. The footnote text will be produced by
    \DescRef{\ThisCommonLabelBase.cmd.footnote}. 
  \end{Example}
\fi%
When setting footnote marks with the \Macro{label} mechanism, any
changes to the footnote numbers will require at least two \LaTeX{} runs to
ensure correct numbers for all \Macro{footref} marks.%
\IfThisCommonLabelBaseOneOf{scrlttr2,scrextend}{\par%
  You can find an example of how to use \Macro{footref} in
  \autoref{sec:\ThisCommonFirstLabelBase.footnotes} on
  \PageRefxmpl{\ThisCommonFirstLabelBase.cmd.footref}. %
}{}%
\IfThisCommonLabelBase{scrlttr2}{}{%
  \par
  Note\textnote{Attention!} that statements like \Macro{ref}\IndexCmd{ref}
  or \Macro{pageref}\IndexCmd{pageref} are fragile and therefore you should
  put \Macro{protect}\IndexCmd{protect} in front of them if they appear in
  moving arguments such as headings. %
}%
By the way, from\IfThisCommonLabelBase{maincls}{%
  \ChangedAt{v3.33}{\Class{scrbook}\and \Class{scrreprt}\and
    \Class{scrartcl}\and \Package{scrextend}}%
}{%
  \IfThisCommonLabelBase{scrlttr2}{%
    \ChangedAt{v3.33}{\Class{scrlttr2}}%
  }{}%
} %
\LaTeX{} 2021-05-01 on, the command is provided by \LaTeX{} itself.%
\EndIndexGroup


\begin{Declaration}
  \Macro{deffootnote}\OParameter{mark width}\Parameter{indent}%
                     \Parameter{parindent}\Parameter{definition}%
  \Macro{deffootnotemark}\Parameter{definition}%
  \Macro{thefootnotemark}
\end{Declaration}%
\IfThisCommonLabelBase{maincls}{The \KOMAScript{} classes set}{\KOMAScript{}
  sets}\textnote{\KOMAScript{} vs. standard classes} footnotes slightly
differently than the standard classes do. As in the standard classes, the
footnote mark in the text is rendered with small, superscript numbers. The
same formatting is used in the footnote itself. The mark in the footnote is
typeset right-justified in a box with a width of \PName{mark width}. The first
line of the footnote follows directly.

All subsequent lines will be indented by the length of \PName{indent}. If the
optional parameter \PName{mark width} is not specified, it defaults to
\PName{indent}. If the footnote consists of more than one paragraph, the first
line of each paragraph is indented by the value of \PName{parindent}.

\autoref{fig:\ThisCommonFirstLabelBase.deffootnote} %
\IfThisCommonFirstRun{}{on
  \autopageref{fig:\ThisCommonFirstLabelBase.deffootnote} }{}%
shows the different parameters%
\IfThisCommonLabelBase{maincls}{ again}{}%
. The default configuration of the \KOMAScript{} classes is as follows:
\IfThisCommonLabelBase{scrextend}{\iftrue}{\csname iffalse\endcsname}%
\begin{lstcode}
  \deffootnote[1em]{1.5em}{1em}{%
    \textsuperscript{\thefootnotemark}}
\end{lstcode}
\else
\begin{lstcode}
  \deffootnote[1em]{1.5em}{1em}{%
    \textsuperscript{\thefootnotemark}%
  }
\end{lstcode}
\fi%
\Macro{textsuperscript} controls both the
superscript and the smaller font size. The command \Macro{thefootnotemark}
contains the current footnote mark without any formatting.%
\IfThisCommonLabelBase{scrextend}{ %
  The \Package{scrextend} package, by contrast, does not change the default
  footnote settings of the class you are using. Simply loading the package,
  therefore, should not lead to any changes in the formatting of footnote
  marks or footnote text. To use the default settings of the \KOMAScript{}
  classes with \Package{scrextend}, you must change the settings above
  yourself. For example, you can insert the line of code above immediately
  after loading the \Package{scrextend} package.%
}{}%

\IfThisCommonLabelBase{maincls}{%
  \begin{figure}
%  \centering
    \KOMAoption{captions}{bottombeside}
    \setcapindent{0pt}%
    \begin{captionbeside}
      [{Parameters that control the footnote layout}]%
      {\label{fig:\ThisCommonLabelBase.deffootnote}\hspace{0pt plus 1ex}%
        Parameters that control the footnote layout}%
      [l]
      \setlength{\unitlength}{1mm}
      \begin{picture}(100,22)
        \thinlines
        % frame of following paragraph
        \put(5,0){\line(1,0){90}}
        \put(5,0){\line(0,1){5}}
        \put(10,5){\line(0,1){5}}\put(5,5){\line(1,0){5}}
        \put(95,0){\line(0,1){10}}
        \put(10,10){\line(1,0){85}}
        % frame of first paragraph
        \put(5,11){\line(1,0){90}}
        \put(5,11){\line(0,1){5}}
        \put(15,16){\line(0,1){5}}\put(5,16){\line(1,0){10}}
        \put(95,11){\line(0,1){10}}
        \put(15,21){\line(1,0){80}}
        % box of the footnote mark
        \put(0,16.5){\framebox(14.5,4.5){\mbox{}}}
        % description of paragraphs
        \put(45,16){\makebox(0,0)[l]{\textsf{first paragraph of a footnote}}}
        \put(45,5){\makebox(0,0)[l]{\textsf{next paragraph of a footnote}}}
        % help lines
        \thicklines
        \multiput(0,0)(0,3){7}{\line(0,1){2}}
        \multiput(5,0)(0,3){3}{\line(0,1){2}}
        % parameters
        \put(2,7){\vector(1,0){3}}
        \put(5,7){\line(1,0){5}}
        \put(15,7){\vector(-1,0){5}}
        \put(15,7){\makebox(0,0)[l]{\small\PName{parindent}}}
        % 
        \put(-3,13){\vector(1,0){3}}
        \put(0,13){\line(1,0){5}}
        \put(10,13){\vector(-1,0){5}}
        \put(10,13){\makebox(0,0)[l]{\small\PName{indent}}}
        % 
        \put(-3,19){\vector(1,0){3}}
        \put(0,19){\line(1,0){14.5}}
        \put(19.5,19){\vector(-1,0){5}}
        \put(19.5,19){\makebox(0,0)[l]{\small\PName{mark width}}}
      \end{picture}
    \end{captionbeside}
  \end{figure}}

\BeginIndexGroup
\BeginIndex{FontElement}{footnote}\LabelFontElement{footnote}%
\BeginIndex{FontElement}{footnotelabel}\LabelFontElement{footnotelabel}%
The footnote\IfThisCommonLabelBase{maincls}{%
  \ChangedAt{v2.8q}{\Class{scrbook}\and \Class{scrreprt}\and
    \Class{scrartcl}%
}}{}, including the footnote mark, uses the font specified in the
\FontElement{footnote}\important{\FontElement{footnote}} element. You can
change the font of the footnote mark separately using the 
\DescRef{\ThisCommonLabelBase.cmd.setkomafont} and
\DescRef{\ThisCommonLabelBase.cmd.addtokomafont} commands (see
\autoref{sec:\ThisCommonLabelBase.textmarkup},
\DescPageRef{\ThisCommonLabelBase.cmd.setkomafont})
for the \FontElement{footnotelabel}\important{\FontElement{footnotelabel}}
element. See also \autoref{tab:\ThisCommonLabelBase.fontelements},
\autopageref{tab:\ThisCommonLabelBase.fontelements}.
The default setting is no change to the font.%
\IfThisCommonLabelBase{scrextend}{ %
  However, with \Package{scrextend} these elements will only change the fonts 
  if footnotes are handled by the package, that is, after using
  \Macro{deffootnote}.%
}{} Please don't misuse this element for other purposes, for example to set
the footnotes ragged right (see also \DescRef{\LabelBase.cmd.raggedfootnote},
\DescPageRef{\LabelBase.cmd.raggedfootnote}).

\BeginIndex{FontElement}{footnotereference}%
\LabelFontElement{footnotereference}%
The footnote mark in the text is defined separately from the mark in
front of the actual footnote. This is done with
\Macro{deffootnotemark}. The default setting is:
\begin{lstcode}
  \deffootnotemark{%
    \textsuperscript{\thefootnotemark}}
\end{lstcode}
With\IfThisCommonLabelBase{maincls}{%
  \ChangedAt{v2.8q}{\Class{scrbook}\and \Class{scrreprt}\and
    \Class{scrartcl}}%
}{} this default, the font for the 
\FontElement{footnotereference}\important{\FontElement{footnotereference}}
element is used (see \autoref{tab:\ThisCommonLabelBase.fontelements},
\autopageref{tab:\ThisCommonLabelBase.fontelements}). Thus, the footnote marks
in the text and in the footnote itself are identical. You can change the font
with the commands \DescRef{\ThisCommonLabelBase.cmd.setkomafont} and
\DescRef{\ThisCommonLabelBase.cmd.addtokomafont} (see
\autoref{sec:\ThisCommonLabelBase.textmarkup},
\DescPageRef{\ThisCommonLabelBase.cmd.setkomafont}).

\IfThisCommonFirstRun{\iftrue}{\csname iffalse\endcsname}%
  \begin{Example}
    \phantomsection
    \xmpllabel{cmd.deffootnote}%
    One\textnote{Hint!} feature that is often requested is footnote marks
    which are neither in superscript nor in a smaller font. They should not
    touch the footnote text but be separated by a small space. You can
    accomplish this as follows:
\begin{lstcode}
  \deffootnote{1em}{1em}{\thefootnotemark\ }
\end{lstcode}
    This will set the footnote mark and subsequent space right-aligned in a 
    box of width 1\Unit{em}. The lines of the footnote text that follow are 
    also indented by 1\Unit{em} from the left margin.

    Another\textnote{Hint!} layout that is often requested is footnote marks
    that are left-aligned. You can obtain them with the following
    definition:
\begin{lstcode}
  \deffootnote{1.5em}{1em}{%
      \makebox[1.5em][l]{\thefootnotemark}}
\end{lstcode}
  
    If, however you want to change the font for all footnotes, for example
    to sans serif, this can easily be done with the commands
    \DescRef{\ThisCommonLabelBase.cmd.setkomafont} and
    \DescRef{\ThisCommonLabelBase.cmd.addtokomafont} (see
    \autoref{sec:\ThisCommonLabelBase.textmarkup},
    \DescPageRef{\ThisCommonLabelBase.cmd.setkomafont}):
\begin{lstcode}
  \setkomafont{footnote}{\sffamily}
\end{lstcode}
  \end{Example}%
  \IfThisCommonLabelBase{scrextend}{}{%
    As the examples show, {\KOMAScript} allows a wide variety of different
    footnote formats with this simple user interface.%
  }%
\fi%
\IfThisCommonFirstRun{}{%
  For examples, see \autoref{sec:\ThisCommonFirstLabelBase.footnotes},
  \PageRefxmpl{\ThisCommonFirstLabelBase.cmd.deffootnote}.%
}{}%
%
\EndIndexGroup
\EndIndexGroup

\IfThisCommonLabelBase{scrextend}{\iffalse}{\csname iftrue\endcsname}
\begin{Declaration}
  \Macro{setfootnoterule}\OParameter{thickness}\Parameter{length}%
\end{Declaration}%
Generally,\IfThisCommonLabelBase{maincls}{%
  \ChangedAt{v3.06}{\Class{scrbook}\and \Class{scrreprt}\and
    \Class{scrartcl}}%
}{%
  \IfThisCommonLabelBase{scrlttr2}{%
    \ChangedAt{v3.06}{\Class{scrlttr2}}%
  }{%
    \IfThisCommonLabelBase{scrextend}{%
      \ChangedAt{v3.06}{\Package{scrextend}}%
    }{}%
  }%
} a horizontal rule is set between the text area and the footnote area, but
normally this rule does not extend the full width of the type area. With
\Macro{setfootnoterule}, you can set the exact thickness and length of the
rule. In this case, the parameters \PName{thickness} and \PName{length} are
only evaluated when setting the rule itself. If the optional argument
\PName{thickness} has been omitted, the thickness of the rule will not be
changed. Empty arguments for \PName{thickness} or \PName{length} are also
allowed and do not change the corresponding parameters. Using absurd values
will result in warning messages both when setting and when using the
parameters.

\BeginIndexGroup
\BeginIndex{FontElement}{footnoterule}\LabelFontElement{footnoterule}%
You can%
\IfThisCommonLabelBase{maincls}{%
  \ChangedAt{v3.07}{\Class{scrbook}\and \Class{scrreprt}\and
    \Class{scrartcl}}%
}{%
  \IfThisCommonLabelBase{scrlttr2}{%
    \ChangedAt{v3.07}{\Class{scrlttr2}}%
  }{%
    \IfThisCommonLabelBase{scrextend}{%
      \ChangedAt{v3.07}{\Package{scrextend}}%
    }{}%
  }%
} %
change the colour of the rule with the
\FontElement{footnoterule}\important{\FontElement{footnoterule}} element using
the \DescRef{\ThisCommonLabelBase.cmd.setkomafont} and
\DescRef{\ThisCommonLabelBase.cmd.addtokomafont} commands (see
\autoref{sec:\ThisCommonLabelBase.textmarkup},
\DescPageRef{\ThisCommonLabelBase.cmd.setkomafont}). The default is no change
of font or colour. In order to change the colour, you must also load a colour
package like
\Package{xcolor}\IndexPackage{xcolor}\important{\Package{xcolor}}.%
\EndIndexGroup
\EndIndexGroup
\fi

\begin{Declaration}
  \Macro{raggedfootnote}
\end{Declaration}
By default%
\IfThisCommonLabelBase{maincls}{%
  \ChangedAt{v3.23}{\Class{scrbook}\and \Class{scrreprt}\and
    \Class{scrartcl}}%
}{%
  \IfThisCommonLabelBase{scrlttr2}{%
    \ChangedAt{v3.23}{\Class{scrlttr2}}%
  }{%
    \IfThisCommonLabelBase{scrextend}{%
      \ChangedAt{v3.23}{\Package{scrextend}}%
    }{}%
  }%
} %
\KOMAScript{} justifies footnotes just as in the standard classes.
But\IfThisCommonLabelBase{scrextend}{%
  \ if you use \DescRef{\LabelBase.cmd.deffootnote}%
  \important{\DescRef{\LabelBase.cmd.deffootnote}}%
  \IndexCmd{deffootnote}%
}{%
  \textnote{\KOMAScript{} vs. standard classes}%
} you can also change the justification separately from the rest of the
document by redefining \Macro{raggedfootnote}. Valid definitions are
\Macro{raggedright}\IndexCmd{raggedright},
\Macro{raggedleft}\IndexCmd{raggedleft},
\Macro{centering}\IndexCmd{centering}, \Macro{relax}\IndexCmd{relax} or an
empty definition, which is the default. The alignment commands of the
\Package{ragged2e}\IndexPackage{ragged2e} package are also valid (see
\cite{package:ragged2e}).  \IfThisCommonLabelBase{scrextend}{%
  You can find a suitable example in
  \autoref{sec:\ThisCommonFirstLabelBase.footnotes},
  \PageRefxmpl{\ThisCommonFirstLabelBase.cmd.raggedfootnote}.%
  \iffalse }{\csname iftrue\endcsname}%
  \begin{Example}
    \phantomsection\xmpllabel{cmd.raggedfootnote}%
    Suppose you are using footnotes only to provide references to very long
    links, where line breaks would produce poor results if justified. You can
    use
\begin{lstcode}
  \let\raggedfootnote\raggedright    
\end{lstcode}
    in your document's preamble to switch to ragged-right footnotes.
  \end{Example}%
\fi
\EndIndexGroup

\begin{Declaration}
  \DoHook{footnote/text/begin}%
  \DoHook{footnote/text/end}%
\end{Declaration}
\BeginIndex{}{hook}%
For\ChangedAt{v3.36}{\Class{scrbook}\and \Class{scrreprt}\and
  \Class{scrartcl}\and \Package{scrextend}} experts there are also two hooks
of type \emph{do-hook} (see \autoref{sec:scrbase.hooks} from
\autopageref{sec:scrbase.hooks}). The first of these is used at the very
beginning of \Macro{@makefntext} before
\DescRef{\LabelBase.cmd.raggedfootnote} is executed. The second one at the end
before the paragraph is finished. Currently neither hook is used by
\KOMAScript{} itself.%
\EndIndexGroup
%
\EndIndexGroup


%%% Local Variables:
%%% mode: latex
%%% coding: utf-8
%%% ispell-local-dictionary: "en_GB"
%%% eval: (flyspell-mode 1)
%%% TeX-master: "../guide"
%%% End:
