\documentclass[a4paper]{article}
\usepackage[colorlinks=true,linkcolor=blue]{hyperref}
\usepackage{parskip}

\usepackage[greek,english]{babel}
% \languageattribute{greek}{polutoniko}
% \languageattribute{greek}{ancient}

\ifdefined \UnicodeEncodingName
  \usepackage{fontspec}
  \setmainfont{DejaVu Serif}
\fi

\begin{document}

\title{Greek support for Babel}
\author{Günter Milde}
\date{2020/11/10}
\maketitle

The babel option «greek» activates the support for the Greek language
defined in the file «greek.ldf» (source «greek.dtx»).

Typesetting Greek texts requires a font containing Greek letters. With the
XeTeX or LuaTeX engines, the user must ensure that the selected font
contains the required glyphs (the default Latin Modern fonts miss most of
them).

\section{Language Switch}

The declaration \verb|\selectlanguage| switches between languages.

\begin{quote}
  \selectlanguage{greek}
  Τί φήις; Ἱδὼν ἐνθέδε παῖδ’ ἐλευθέραν
  τὰς πλησίον Νύμφας στεφανοῦσαν, Σώστρατε,
  ἐρῶν άπῆλθες εὐθύς;
\end{quote}

The command \verb|\foreignlanguage| sets its second argument in the language
specified as first argument. This is intended for short text parts like
\foreignlanguage{greek}{Βιβλιοθήκη}.

\section{Font Encoding}

Every language switch to \texttt{greek} calls the \verb|\extrasgreek|
command which in turn calls \verb|\greekscript| to ensure a Greek-supporting
font encoding (LGR or TU).  At this point, the «greekfontencoding»
is \texttt{\greekfontencoding}. 


LGR has Greek characters in the slots reserved in a TeX \emph{standard text
font encoding}. Therefore, the Babel core defines the declaration
\verb|\latintext| and the command \verb|\textlatin| to switch to the TU, T1
or OT1 font encoding or typeset the argument using this encoding. At this
point, the «latinencoding» is \texttt{\latinencoding}.

With the Unicode font encoding \texttt{TU},
Latin characters can be used in Greek text parts and
input via the «LGR Latin transcription» is not possible.

The following quote mixes Latin transcription and Greek literal characters:

\begin{quote}
  \greekscript Φίλων τοῦ \textlatin{TeX} (ΕΦΤ) --
  \latintext Friends (\ensuregreek{F\'ilwn}) of TeX.%
\end{quote}

\section{LICR Macros}

Babel defines macros for several autogenerated strings so that they may
appear in the choosen language. \emph{babel-greek} uses LICR%
\footnote{LaTeX internal character representation} macros in
order to let the string macros work independent of the font encoding.

If \verb|\greekfontencoding| is TU, \emph{babel-greek} loads Greek LICR
definitions from the file \texttt{tuenc-greek.def} provided by
\href{http://www.ctan.org/pkg/greek-fontenc}{greek-fontenc}
since version~0.14 (2020-02-28).

With this setup, it is also possible to use accent macros instead of
pre-composed Unicode characters for letters with diacritics%
\ifdefined \UnicodeEncodingName
  : Τ\'ι φ\'ηις;, \`<ορα = \accdasiavaria{ο}ρα
\fi
.

\subsection{Captions}

\selectlanguage{greek}
\prefacename,
\refname,
\abstractname,
\bibname,
\chaptername,
\appendixname,
\contentsname,
\listfigurename ,
\listtablename,
\indexname,
\figurename,
\tablename,
\partname,
\enclname,
\ccname,
\headtoname,
\pagename,
\seename,
\alsoname,
\proofname,
\glossaryname
\selectlanguage{english}

Test correct upcasing (dropping of accents):

\selectlanguage{greek}
\MakeUppercase{
\prefacename,
\refname,
\abstractname,
\bibname,
\chaptername,
\appendixname,
\contentsname,
\listfigurename,
\listtablename,
\indexname,
\figurename,
\tablename,
\partname,
\enclname,
\ccname,
\headtoname,
\pagename,
\seename,
\alsoname,
\proofname,
\glossaryname
}
\selectlanguage{english}


\subsection{Months}

\selectlanguage{greek}
\newcounter{foo}
\stepcounter{foo} \month=\value{foo} \today \\
\stepcounter{foo} \month=\value{foo} \today \\
\stepcounter{foo} \month=\value{foo} \today \\
\stepcounter{foo} \month=\value{foo} \today \\
\stepcounter{foo} \month=\value{foo} \today \\
\stepcounter{foo} \month=\value{foo} \today \\
\stepcounter{foo} \month=\value{foo} \today \\
\stepcounter{foo} \month=\value{foo} \today \\
\stepcounter{foo} \month=\value{foo} \today \\
\stepcounter{foo} \month=\value{foo} \today \\
\stepcounter{foo} \month=\value{foo} \today \\
\stepcounter{foo} \month=\value{foo} \today \\
\selectlanguage{english}

\section{Greek Numerals (\greeknumeral{1} to \Greeknumeral{999999})}

See greek.pdf for the formation rules of Greek numerals.
Some examples:

\selectlanguage{greek}

\greeknumeral{1},
\greeknumeral{2},
\greeknumeral{3},
\greeknumeral{4},
\greeknumeral{5},
\greeknumeral{6},
\greeknumeral{7},
\greeknumeral{8},
\greeknumeral{9},
\greeknumeral{10},
\greeknumeral{11},
\greeknumeral{12},
\greeknumeral{20},
\greeknumeral{345},
\greeknumeral{500},
\greeknumeral{1997},
\greeknumeral{2013}

\Greeknumeral{1},
\Greeknumeral{2},
\Greeknumeral{3},
\Greeknumeral{4},
\Greeknumeral{5},
\Greeknumeral{6},
\Greeknumeral{7},
\Greeknumeral{8},
\Greeknumeral{9},
\Greeknumeral{10},
\Greeknumeral{11},
\Greeknumeral{12},
\Greeknumeral{20},
\Greeknumeral{345},
\Greeknumeral{500},
\Greeknumeral{1997},
\Greeknumeral{2013}
\selectlanguage{english}


Enumerated lists use Greek characters/numerals in the second and fourth level:

\selectlanguage{greek}
\begin{enumerate}
  \item item 1
  \begin{enumerate}
    \item item 1.1
    \begin{enumerate}
      \item item 1.1.1
       \begin{enumerate}
         \item item 1.1.1.1
         \item item 1.1.1.2
       \end{enumerate}
      \item item 1.1.2
    \end{enumerate}
  \end{enumerate}
\end{enumerate}
\selectlanguage{english}


This may be problematic with fonts that only partially support Greek and
miss the numeral signs (dexiakeraia and aristerikeraia).

You may redefine the commands \verb+\textdexiakeraia+ and
\verb+\textaristerikeraia+ to some substitute characters.
Or, if you prefer the ``normal'' enumeration, write in the preamble after
loading babel:

\begin{verbatim}
  \makeatletter
  \addto\extrasgreek{\let\@alph\latin@alph
  		     \let\@Alph\latin@Alph}
  \makeatother
\end{verbatim}

\end{document}
