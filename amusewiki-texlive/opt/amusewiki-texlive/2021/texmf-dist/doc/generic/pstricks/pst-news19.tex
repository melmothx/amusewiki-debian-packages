%% $Id: pst-news19.tex 4 2020-06-09 08:32:19Z herbert $
\PassOptionsToPackage{useCalc}{pstricks}
\documentclass[11pt,english,BCOR=10mm,DIV=12,bibliography=totoc,parskip=false,headings=small,
    headinclude=false,footinclude=false,twoside]{pst-doc}
\listfiles
\let\Lfile\LFile
\usepackage{pstricks}
\let\pstnodeFV\fileversion
\let\pstnodeFD\filedate
\usepackage{pst-node}
\usepackage{pst-plot}
\usepackage{pst-solides3d}
\usepackage{pst-node}
\usepackage{pst-calculate}
\usepackage{pstricks-add}
\usepackage{xkvview}
\renewcommand\bgImage{\psscalebox{15}{\color{blue!20}\the\year}}
\def\textat{\char064}
\usepackage{dtk-logos}
\usepackage{biblatex}
\addbibresource{PSTricks.bib}

\lstset{explpreset={pos=l,width=-99pt,overhang=0pt,hsep=\columnsep,vsep=\bigskipamount,rframe={}},
    escapechar=?}
\begin{document}

%\psset{PstDebug=1}
\title{\texttt{News -- \the\year}\\ \Large new macros and bugfixes for the
basic package \nxLFile{pstricks}}
\author{Herbert Voß}
\date{\today}

\maketitle

\clearpage
\tableofcontents

\clearpage
\part{\texttt{pstricks} -- package}

%--------------------------------------------------------------------------------------
\section{\texttt{pstricks.sty} -- \texttt{pstricks-pdf.sty}}
%--------------------------------------------------------------------------------------

There is now a new optional argument for the package: \Loption{ckeckengine}, which will
be used in later versions. 
 
%--------------------------------------------------------------------------------------
\section{\texttt{pstricks-tex.tex}}
%--------------------------------------------------------------------------------------
This package collects all additional latex macros which must be definied
when running PSTricks with tex.  They all moved from the base \texttt{pstricks.tex} into
this new file.


%--------------------------------------------------------------------------------------
\section{\texttt{pstricks.tex} (v. 2.98 -- 2019/05/16)}
%--------------------------------------------------------------------------------------

\subsection{Plotting multilines}

There is a new macro called \Lcs{psLine} (uppercase L!) which can have
only one or two pairs of coordinates. Remember that \Lcs{psline} can have more.
With \Lcs{psLine} and an defined arrow you'll get nothing for something
like \verb|\psLine{->}{1,1)(1,1)|, whereas \Lcs{psline} will output the arrowtip!
This behaviour of \Lcs{psLine} maybe helpful for animations where a speedvector
or a vector
is  shown with its components and one do not really knows the coordinates.

\begin{BDef}
\Lcs{psLine}\OptArgs\OptArg{\Largr{$x_0,y_0$}}\Largr{$x_1,y_1$}\\
\Lcs{psLineSegments}\OptArgs\Largr{$x_0,y_0$}\Largr{$x_1,y_1$}\ldots\Largr{$x_{n-1},y_{n-1}$}\Largr{$x_n,y_n$}
\end{BDef}

If $P_0$ is missing for \Lcs{psLine} then $(0,0)$ as first point is assumed. \Lcs{psLineSegments} expects always
pairwise coordinates.


\begin{LTXexample}[width=7cm]
\psset{unit=2cm}
\begin{pspicture}(-1,-1)(1,1)
\degrees[12]
 \pscircle{1}
  \multido{\iA=0+1,\iB=3+1}{12}{%
    \pnode(1;\iA){R}
    \psLine[linecolor=gray,linestyle=dashed,
       linewidth=0.4pt](R)
    \psset{arrows=->,arrowscale=1.5,linewidth=1.5pt}
    \rput(R){%
      \psLine[linecolor=blue]({1;\iB}|0,0)
      \psLine[linecolor=red](0,0|{1;\iB})
      \psLine(1;\iB)}%
}
\end{pspicture}
\end{LTXexample}


\begin{LTXexample}[width=4cm]
\psset{unit=0.5cm}
\begin{pspicture}(6,6)
\psLineSegments[linecolor=blue]{->}(0,0)(0,1)(0,2)(0,3)(0,4)(1,4)%
     (3,4)(4,4)(4,3)(4,1)(4,0)(2,0)
\end{pspicture}
\end{LTXexample}


\iffalse
There is a new experimental feature, which allows algebraic notation for
coordinates. It needs

\begin{verbatim}
\usepackage[useCalc]{pstricks}
\end{verbatim}

\begin{LTXexample}[width=5.5cm]
\begin{pspicture}[showgrid](5,5)
\psline[linecolor=blue]{->}%
    (0,4/3*0.5)(0.6*sqrt{33}, sqrt{20+0.3})
\end{pspicture}
\end{LTXexample}

\fi
%$

\subsection{Relative coordinates -- experimental}

All objects which can have at least two pairs of coordinates
can use the »+«-operator to mark relative coordinates:

\begin{BDef}
\Lcs{psline}\OptArgs\OptArg{\Largr{$x_0,y_0$}}\Largr{$x_1,y_1$}+\Largr{$x_2,y_2$}\ldots\\
\end{BDef}

\begin{LTXexample}[width=5.5cm]
\begin{pspicture}[showgrid](5,5)
\psline[linecolor=blue]{->}%
    (1,1)+(3,4)+(-2,-3)
\end{pspicture}
\end{LTXexample}


\begin{LTXexample}[width=5.5cm]
\begin{pspicture}[showgrid](-1,-2)(4,3)
\psaxes[labels=none]{->}(0,0)(-1,-2)(4,3)
\psbezier%
  (!0.8 -95 PtoC 1 sub)+(8.5;85)+(5;-100)(2,1)%
 +(1;-70)+(0.5;-110)(3,2)%
 +(1;-80)+(0.5;95)(3.4,-0.8)
\end{pspicture}
\end{LTXexample}




\subsection{Error message}

Using PSTricks with \Lprog{pdflatex} will work only when using package
\LPack{auto-pst-pdf} and running the \TeX-file with

\begin{verbatim}
pdflatex -shell-escape <file>
\end{verbatim}

otherwise you'll get an error message which was misleading in the past:

\begin{verbatim}
[...]
! Undefined control sequence.
<recently read> \c@lor@to@ps 
\end{verbatim}

This changes now to 


\begin{verbatim}
[...]
! Undefined control sequence.
\c@lor@to@ps ->\PSTricks 
                         _Not_Configured_For_This_Format
\end{verbatim}


\subsection{Optional argument \texttt{xetex}}
The output driver \Lprog{xdvipdfmx} for using \XeTeX\  or  \XeLaTeX\ is not fully
compatible to \Lprog{dvips}. Especially some node operations will not work. If the
\LaTeX\  package detects a programm run with \XeLaTeX\ it automatically loads the file
\Lfile{pstricks-xetex.def} which defines some macros with a new name to keep the existing
ones. By now there is only
\Lcs{NCput}, which is the same as \Lcs{ncput}, but works with \XeLaTeX.

If someone wants to use these macros though he/she runs not \XeLaTeX\ then these macros are
available too by using the optional argument \Loption{xetex}:

\begin{verbatim}
\usepackage[xetex]{pstricks}
\end{verbatim}






%--------------------------------------------------------------------------------------
\section{\texttt{pstricks.pro}}
%--------------------------------------------------------------------------------------



\clearpage
\nocite{*}
\printbibliography

\printindex


\end{document}

