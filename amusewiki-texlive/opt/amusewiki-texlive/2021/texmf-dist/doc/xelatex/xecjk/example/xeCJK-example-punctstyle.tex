%%
%% This is file `xeCJK-example-punctstyle.tex',
%% generated with the docstrip utility.
%%
%% The original source files were:
%%
%% xeCJK.dtx  (with options: `ex-punctstyle')
%% 
\documentclass{article}
\usepackage[draft,columns=1,margin=false,headings=false]{typogrid}
\usepackage{xeCJK}
\usepackage{xeCJKfntef}
\setCJKmainfont{SimSun}
\xeCJKsetup{AllowBreakBetweenPuncts}
\xeCJKDeclarePunctStyle { mine }
  {
    fixed-punct-ratio       = nan ,
    fixed-margin-width      = 0 pt ,
    mixed-margin-width      = \maxdimen ,
    mixed-margin-ratio      = 0.5 ,
    middle-margin-width     = \maxdimen ,
    middle-margin-ratio     = 0.5 ,
    add-min-bound-to-margin = true ,
    min-bound-to-kerning    = true ,
    kerning-margin-minimum  = 0.1 em ,
    bound-punct-width       = 0 em ,
    enabled-hanging         = true
  }
\xeCJKDeclarePunctStyle { JIS } { bound-punct-ratio = .5 }

\begin{document}
\setlength\parindent{2em}
\newcommand\showexample[1]{%
  \texttt{------ #1 ------}\par\xeCJKsetup{PunctStyle=#1}\showtexts\par}

\newcommand\showtexts{%
《第一回\quad 甄士隐梦幻识通灵\quad 贾雨村风尘怀闺秀》\\
《第一回\quad 甄士隐梦幻识通灵\quad 贾雨村风尘怀闺秀》

列位看官:你道此书从何而来?说起根由,虽近荒唐,细按则深有趣味。待在下将此来历注明,方使阅者了然不惑。

\CJKunderline*{原来女娲氏炼石补天之时,于大荒山无稽崖炼成高经十二丈、方经二十四丈顽石三万六千五百零一块。娲皇氏只用了三万六千五百块,只单单剩了一块未用,便弃在此山青埂峰下。谁知此石自经煆炼之后,灵性已通,因见众石俱得补天,独自己无材不堪入选,遂自怨自叹,日夜悲号惭愧。}

一日,正当嗟悼之际,俄见一僧一道远远而来,生得骨格不凡,丰神迥别,说说笑笑,来至峰下,坐于石边,高谈快论:先是说些云山雾海、神仙玄幻之事,后便说到红尘中荣华富贵。此石听了,不觉打动凡心,也想要到人间去享一享这荣华富贵,但自恨粗蠢,不得已,便口吐人言,向那僧道说道:“大师,弟子蠢物,不能见礼了!适闻二位谈那人世间荣耀繁华,心切慕之。弟子质虽粗蠢,性却稍通,况见二师仙形道体,定非凡品,必有补天济世之材,利物济人之德。如蒙发一点慈心,携带弟子得入红尘,在那富贵场中,温柔乡里受享几年,自当永佩洪恩,万劫不忘也!”二仙师听毕,齐憨笑道:“善哉,善哉!那红尘中有却有些乐事,但不能永远依恃;况又有‘美中不足,好事多磨’八个字紧相连属,瞬息间则又乐极悲生,人非物换,究竟是到头一梦,万境归空,倒不如不去的好。”这石凡心已炽,那里听得进这话去,乃复苦求再四。二仙知不可强制,乃叹道:“此亦静极思动,无中生有之数也!既如此,我们便携你去受享受享,只是到不得意时,切莫后悔!”石道:“自然,自然。”那僧又道:“若说你性灵,却又如此质蠢,并更无奇贵之处。如此也只好踮脚而已。也罢!我如今大施佛法,助你助,待劫终之日,复还本质,以了此案。你道好否?”石头听了,感谢不尽。那僧便念咒书符,大展幻术,将一块大石登时变成一块鲜明莹洁的美玉,且又缩成扇坠大小的可佩可拿。那僧托于掌上,笑道:“形体倒也是个宝物了!还只没有实在的好处,须得再镌上数字,使人一见便知是奇物方妙。然后好携你到那昌明隆盛之邦、诗礼簪缨之族、花柳繁华地、温柔富贵乡去安身乐业。”石头听了,喜不能禁,乃问:“不知赐了弟子那哪几件奇处?又不知携了弟子到何地方?望乞明示,使弟子不惑。”那僧笑道:“你且莫问,日后自然明白的。”说着,便袖了这石,同那道人飘然而去,竟不知投奔何方何舍。

\hfill
——曹雪芹《红楼梦》\unskip\unkern}

\showexample{mine}
\showexample{quanjiao}
\showexample{JIS}
\showexample{banjiao}
\showexample{CCT}
\showexample{kaiming}
\showexample{hangmobanjiao}
\showexample{plain}

\end{document}
%% 
%%
%% End of file `xeCJK-example-punctstyle.tex'.
