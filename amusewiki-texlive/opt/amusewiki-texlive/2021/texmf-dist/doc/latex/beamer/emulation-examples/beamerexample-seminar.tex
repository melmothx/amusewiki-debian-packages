% $Header$

% This file is a demonstration on how a seminar file should be
% changed to make it work with beamer.


% Copyright notice:

% Except for the changes indicated by CHANGED, this file is the original
% file semsamp2.tex, which is part of the examples of the seminar
% package. 


%% BEGIN semsamp2.tex
% This is a sample document for seminar.sty, v0.93 (and maybe later).
%
% This file contains both landscape and portrait mode slides.
% Choose one of the following to print them out:
%  - If using PSTricks, try the semcolor style option.
%  - If using Rokicki's dvips, try the semrot style option.
%  - To print the landscape slides, put \landscapeonly in the preamble.
%    To print the portrait slides, include the portrait style option and
%    put \portraitonly in the preamble.
%
%

% CHANGED: commented the following:
%\documentstyle[%
%  slidesonly,%  Try notes or notesonly instead.
%  %notes,%      Use instead of slidesonly to typeset the notes.
%  %notesonly,%  Use instead of slidesonly to typeset notes and slides.
%  %semcolor,%   Try me if using PSTricks.
%  %semrot,%     Try me if using Rokicki's dvips.
%  %semhelv,%    Try me if using a PostScript printer.
%  %article,%    Try me.
%  %portrait,%   Try me.
%  %sem-a4,%     Try me if using A4 paper.
%  semlayer%     This must be included, but you need the semcolor option to
%  ]{seminar}                                  % actually see the overlays.
%
%\slidesmag{5}
%\articlemag{1}
%
%%\twoup                     % Try me for twoup printing.
%
%%\portraitonly              % To print only portrait slides
%%\landscapeonly             % To print only landscape slides
%
%%\notslides{\ref{questions}-7,1}   %Try me: The slides are omitted.
%%\onlyslides{\ref{questions}-7,1}  %Try me: Only these slides are included.
%%\onlynotestoo                     %Try me: For selecting notes as well.
%
%\colorlayers{red,blue}      % Try deleting this if using the semcolor option,
%                            % to get \blue and \red to use PostScript color.
%
%%\overlaysfalse             % Suppress overlays with semcolor option.
%%\layersfalse               % Suppress color layers with semcolor option.
%
%\rotateheaderstrue          % Try this out if using rotation macros.

% CHANGED: Added following three lines:
\documentclass[ignorenonframetext]{beamer}
\usepackage[accumulated]{beamerseminar}
                                % remove ``accumulated'' option
                                % for original behaviour
\usepackage{beamerthemeclassic}

\title{Example for seminar.sty}
\author{Policarpa Salabarrieta}
\date{July 21, 1991}

\newcommand{\sref}[1]{SLIDE \ref{#1}}

% CHANGED: different definition of \heading
%\newcommand{\heading}[1]{\begin{center}\large\bf #1\end{center}}
\let\heading=\frametitle

% CHANGED: Commented:
%\newpagestyle{MH}%
%  {University of Guaduas, March 13, 1998\hfil\thepage}{}
%\pagestyle{MH}

\begin{document}

% CHANGED: Added \frame
\frame{
\maketitle          % This won't show up when \onlynotestoo is in effect.
}

% CHANGED: Commented
%\begin{slide}
%  \ifslidesonly              % Title slide only for slidesonly selection.
%    \maketitle
%    \addtocounter{slide}{-1}
%    \slidepagestyle{empty}
%  \fi
%\end{slide}


This is a lot of gobbledy-gook intended only to illustrate some of the
features of seminar.sty.

 The phrase information overload rings a bell with just about anyone.
Certainly you all receive more working papers or more applications for
graduate school than you can readily read. Nevertheless, the term information
overload is ill-defined. (\sref{too_much}, top)

 A message like this when you check your email conjures up the notion of
information overload. More generally, information overload always means too
much information, in some sense or another. But what does ``too much'' mean?
(\sref{too_much}, bottom) It might just mean that people cannot process all
the information they receive. That is certainly true for everyone. A claim
that is much stronger, and that is implicit when people complain about
information overload, is that people {\em should} receive less information, by
some criterion.

% CHANGED: Added \frame
\frame{
\begin{slide}\label{too_much}%
\begin{center}
  \large\bf
 Information overload = ``Too much'' information
\end{center}
\smallskip

\begin{verse} \bf\tt
  You have 134 unread messages:\\
  Do you want to read them now?
\end{verse}

\begin{enumerate}
  {\overlay2
  \item People {\overlay1 cannot process all} the information they receive.}
  \item People {\em should} receive less information.
 \end{enumerate}
\end{slide}
}

  In this paper, I use the term ``information overload'' in both senses.
(\sref{overload}, bottom). Specifically, I say that an {\em individual} is
overloaded with information if she receives more information than she can
process. But I say that there is information overload in a {\em network} if
there is some mechanism that  makes the senders and/or receivers better off by
restricting the flow of  information. This latter notion of information
overload is an equilibrium property, and it depends on what we mean by
``better off.''

% CHANGED: Added \frame
\frame{
\begin{slide*}\label{overload}
\ptsize{12}

\begin{itemize}{\overlay1
  \item There is information {\overlay0 overload in a network if} there is
some mechanism that, compared to the {\em status quo}, makes the senders
and/or receivers better off by restricting the flow} of information.'

  \item There is information overload in a network if there is some mechanism
that, compared to the {\em status quo}, makes the senders and/or receivers
better off by restricting the flow of information.
\end{itemize}

\end{slide*}
}

(\sref{questions})
  The purpose of my paper is to show why there can be information overload in
a network and what kind of mechanisms can make the receivers and/or senders
better off. Since the cost of communication is one factor that restricts
communication, I am thus also going to look at how the welfare of the senders
and receivers depends on the cost of communication.

  Most messages don't become jumbled and we can choose which ones to process. 
But some of us may have a bias towards choosing to process more information
than we should, like the graduate student who feels compelled to read every
article on the usual lengthy reading list, and just ends up getting confused
and ruffling through the papers.

% CHANGED: Added \frame
\frame{
\begin{slide}[7.3in,5.5in] \label{questions}
\heading{Questions}

\begin{itemize}
  {\overlay1\item When could {\blue there be overload} in networks?}
  \item What mechanism make the receivers and senders better off?
  \item How does the welfare {\red of the senders} and receivers depend on the
cost of communication?
\end{itemize}
\end{slide}
}

However, experiments in consumer research and psychology have failed to find
that such a bias is prevalent. This is in spite of the fact that it is common
for stress and cognitive strain to increase with information load. We may
incur such stress and strain because the information we choose to process is
valuable to us.

More commonly, then, we can and do choose to process roughly as much
information as we can handle efficiently. This is called screening. But when
we choose which messages to begin to process, we're ignorant of their
contents, since otherwise there would be no reason to process them in the
first place. Therefore, if we receive more junk mail, then some of the
important mail gets crowded out, and we are effectively less informed.


% CHANGED: Added \frame
\frame{
\begin{slide} \label{informed}
\begin{center}
  {\bf Being more informed} \par
    \smallskip
  is always better,\par
    \medskip
  \overlay1{but it's not the same as \par
    \smallskip
  {\bf receiving more information}}
\end{center}
\end{slide}
}

  Why would the senders communicate too many messages in the first place? If I
present too much material in this seminar, you have to choose which parts to
ignore and I would rather make that decision myself, since I know what I most
want to get across. Thus, it is in my interest not to overload you with
information. Generally, whenever there is a single sender of messages, that
sender will prefer to screen rather than have the receiver screen, because the
sender has an interest in which messages the receiver processes.
But when there are more senders, one sender's  messages tend to crowd out the
messages of the other senders, as in this example here. If the senders don't
take this external cost into account when sending messages, they may
collectively overload the receiver. (\sref{akbar})

There are several reasons that our scarcity of attention, that is, our limited
capacity to process information, can mean that we become less informed when we
receive  more information. I have a cartoon here to illustrate these reasons.
(\sref{akbar})

% CHANGED: Added \frame
\frame{
\begin{slide}\label{akbar}\def\slidefuzz{15pt}
  {\large A tax $\tau$ on communication is said to support
$\tilde{\cal{X}}(c)$ if $\tilde{\cal{X}}(c)$ is an equilibrium for
$\Gamma(c+\tau)$.}
\medskip

 {\bf Proposition 6.} {\em Assume $\tilde{\cal{X}}(c)$ is not an equilibrium
for $\Gamma(c)$.\vspace{-3pt}
\begin{enumerate}
 \item If $\mbox{supp}(\gamma)=[0,1]^n$, there is no tax that supports
$\tilde{\cal{X}}(c)$.
 \item If $\mbox{supp}(\gamma)=S^{n-1}$, there is a tax that supports
$\tilde{\cal{X}}(c)$ if and only if $m=1$, $p_j>c\, \forall j$, and
   \begin{enumerate}
     \item $n=2$; or
     \item $n=3$ and $p_i^{-1}+p_j^{-1}\geq p_k^{-1}$ for all distinct
$i,j,k$; or
     \item $n=4$ and $p_1=p_2=p_3=p_4$.
\end{enumerate}
\end{enumerate}}
\end{slide}
}
  If, by restricting communication, we eliminate the less relevant messages,
then we can become more informed. But how can we achieve this? Restricting the
flow of information shifts the task of screening messages from the receivers
to the senders. Unlike the receivers, the senders do know the contents of the
messages they originate. If the senders' interests coincide with those of the
receiver and if the senders have sufficient knowledge about the receivers,
then the senders will choose the messages which are most relevant to the
receivers. This may make the receivers, and even the senders, better off.

  The network in Slide \ref{architectures} attains the minimal delay $c(8,24)
= 6$ using 8 processors. It is an example of the efficient one-shot networks
described by Foo. We will focus on a class of networks that are similar to the
Foo networks but that may differ slightly. For $q$, $c$ and $n$ such that $1
\leq q \leq \lfloor n/2 \rfloor$ and $c(q,n) \leq c \leq n$, let $R_{nqc}$ be
the class of essential networks for adding $n$ items using  $q$ processors in
$c$ cycles that have the following  properties:

% CHANGED: Added \frame
\frame{
\begin{slide*}\label{architectures}
\heading{Architecture}

\begin{center}
\setlength{\unitlength}{1.65in}
\begin{picture}(1.1,1.6)(3.5,5.0)
\put(4.0,6.5){\circle*{.04}}
\put(4.1,6.5){1}
\put(4.0,6.0){\circle*{.04}}
\put(4.1,6.0){2}
\put(4.0,6.1){\vector(0,1){.3}}
\put(3.5,6.0){\circle*{.04}}
\put(3.6,6.0){3}
\put(3.6,6.1){\vector(1,1){.3}}
\put(4.0,5.5){\circle*{.04}}
\put(4.1,5.5){4}
\put(4.0,5.6){\vector(0,1){.3}}
\put(4.5,6.0){\circle*{.04}}
\put(4.6,6.0){5}
\put(4.4,6.1){\vector(-1,1){.3}}
\put(4.5,5.5){\circle*{.04}}
\put(4.6,5.5){6}
\put(4.4,5.6){\vector(-1,1){.3}}
\put(3.5,5.5){\circle*{.04}}
\put(3.6,5.5){7}
\put(3.5,5.6){\vector(0,1){.3}}
\put(4.0,5.0){\circle*{.04}}
\put(4.1,5.0){8}
\put(4.0,5.1){\vector(0,1){.3}}
\end{picture}
\end{center}
\end{slide*}
}

Why would the senders communicate too many messages in the first place? If I
present too much material in this seminar, you have to choose which parts to
ignore and I would rather make that decision myself, since I know what I most
want to get across. Thus, it is in my interest not to overload you with
information.

Generally, whenever there is a single sender of messages, that sender will
prefer to screen rather than have the receiver screen, because the sender has
an interest in which messages the receiver processes. But when there are more
senders, one sender's  messages tend to crowd out the messages of the other
senders, as in this example here. If the senders don't take this external cost
into account when sending messages, they may collectively overload the
receiver. (\sref{architectures})

\end{document}
%% END semsamp2.tex
