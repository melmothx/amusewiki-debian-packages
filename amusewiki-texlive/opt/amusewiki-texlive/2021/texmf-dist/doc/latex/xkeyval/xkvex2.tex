%%
%% This is file `xkvex2.tex',
%% generated with the docstrip utility.
%%
%% The original source files were:
%%
%% xkeyval.dtx  (with options: `xkvex2')
%% 
%% ---------------------------------------
%% Copyright (C) 2004-2020 Hendri Adriaens
%% ---------------------------------------
%%
%% This work may be distributed and/or modified under the
%% conditions of the LaTeX Project Public License, either version 1.3
%% of this license or (at your option) any later version.
%% The latest version of this license is in
%%   http://www.latex-project.org/lppl.txt
%% and version 1.3 or later is part of all distributions of LaTeX
%% version 2003/12/01 or later.
%%
%% This work has the LPPL maintenance status "maintained".
%%
%% This Current Maintainer of this work is Hendri Adriaens.
%%
%% This work consists of the file xkeyval.dtx and derived files
%% keyval.tex, xkvtxhdr.tex, xkeyval.sty, xkeyval.tex, xkvview.sty,
%% xkvltxp.sty, xkvutils.tex, pst-xkey.tex, pst-xkey.sty, xkveca.cls,
%% xkvecb.cls, xkvesa.sty, xkvesb.sty, xkvesc.sty, xkvex1.tex,
%% xkvex2.tex, xkvex3.tex and xkvex4.tex.
%%
%% The following files constitute the xkeyval bundle and must be
%% distributed as a whole: readme, xkeyval.pdf, keyval.tex,
%% pst-xkey.sty, pst-xkey.tex, xkeyval.sty, xkeyval.tex, xkvview.sty,
%% xkvltxp.sty, xkvtxhdr.tex, xkvutils.tex, pst-xkey.dtx and xkeyval.dtx.
%%
\documentclass{article}
\usepackage{xkeyval}
\usepackage{xcolor,calc}

%% Example of setting keyval options using xkeyval
\usepackage{hyperref}
\usepackage{helvet}
\setkeys{Hyp,Hel}{colorlinks,urlcolor=green,scaled=.85}

\title{\textsf{xkeyval} package\\\normalsize\emph{Example 2}}
\author{Hendri Adriaens}
\setlength{\parindent}{0cm}

\makeatletter

\begin{document}
\maketitle

This example demonstrates the use of the new macros for setting
keys of the \textsf{xkeyval} package. Below each piece of verbatim
code, you can find its output.

\section{Simple operations}

Define keys.
\begin{verbatim}
\define@key{fami}{keyi}{key 1: #1\space}
\define@key{famii}{keyii}{key 2: #1\space}
\define@key{famiii}{keyiii}[test 2]{key 3: #1\space}
\define@key{famiv}{keyi}{key 4: #1\space}
\end{verbatim}
\define@key{fami}{keyi}{key 1: #1\space}
\define@key{famii}{keyii}{key 2: #1\space}
\define@key{famiii}{keyiii}[test 2]{key 3: #1\space}
\define@key{famiv}{keyi}{key 4: #1\space}

Set keys from multiple families.
\begin{verbatim}
\setkeys{fami,famii,famiii}{keyi=test 1,keyiii}
\end{verbatim}
\begin{itemize}
\item[$\Rightarrow$]
\setkeys{fami,famii,famiii}{keyi=test 1,keyiii}
\end{itemize}

Multiplely defined keys will be taken from the first family on the
list that defines the key. When \verb|+| is used, all families
defining the key will set it.
\begin{verbatim}
\setkeys{fami,famiv}{keyi=test 3}
\setkeys{famiv,fami}{keyi=test 4}
\setkeys+{fami,famiv}{keyi=test 5}
\end{verbatim}
\begin{itemize}
\item[$\Rightarrow$]
\setkeys{fami,famiv}{keyi=test 3}\par
\setkeys{famiv,fami}{keyi=test 4}\par
\setkeys+{fami,famiv}{keyi=test 5}
\end{itemize}

Starred form sets keys that it can find in the families specified.
Keys that cannot be located will be put onto a list for later use
by \verb+\setrmkeys+.
\begin{verbatim}
\setkeys*{famiv}{keyi=test 9,keyii=\url{http://www.test10.com},
  keyiii=this key has been set on page \thepage}
\end{verbatim}
\begin{itemize}
\item[$\Rightarrow$]
\setkeys*{famiv}{keyi=test 9,keyii=\url{http://www.test10.com},
  keyiii=this key has been set on page \thepage}
\end{itemize}

Try to set the remaining keys from the previous step. The starred
version defines a new list in case there are still keys that
cannot be located in the given families. The command below does
not do anything since the \verb+keyi+ key has already been taken
from family \verb+famiv+ above.
\begin{verbatim}
\setrmkeys*{fami}
\end{verbatim}
\setrmkeys*{fami}\par

The normal version will complain (as \verb+\setkeys+) when it
cannot locate a key which was previously submitted to
\verb+\setkeys+ and hasn't been set yet by previous commands.
\textsf{xkeyval} commands can also be used to set \textsf{keyval}
options as is done to the color of the url created with the
\textsf{hyperref} package and the scaling of the font used for
`\textsf{xkeyval}' with the \textsf{helvet} package. (See the
preamble of the example for the code).
\begin{verbatim}
\setrmkeys{famii,famiii}
\end{verbatim}
\begin{itemize}
\item[$\Rightarrow$]
\setrmkeys{famii,famiii}
\end{itemize}

Set keys, excluding some keys.
\begin{verbatim}
\setkeys{fami,famii,famiii}[keyi,keyiii]{keyi=test 6,keyii=test 7,keyiii=test 8}
\end{verbatim}
\begin{itemize}
\item[$\Rightarrow$]
\setkeys{fami,famii,famiii}[keyi,keyiii]{keyi=test 6,keyii=test 7,keyiii=test 8}
\end{itemize}

Test whether a key is defined in a family from the list.
\begin{verbatim}
\key@ifundefined{fami,famii}{keyiii}
{keyiii is not defined in family `fami' or `famii'}
{keyiii is defined in family `fami' or `famii'}
\end{verbatim}
\begin{itemize}
\item[$\Rightarrow$]
\key@ifundefined{fami,famii}{keyiii}
{keyiii is not defined in family `fami' or `famii'}
{keyiii is defined in family `fami' or `famii'}
\end{itemize}

Define some keys and macros.
\begin{verbatim}
\define@cmdkey[my]{fama}[my@]{keya}[none]{}
\define@choicekey[my]{fama}{keyb}{a,b,c}[b]{correct input: #1}
\define@boolkey[my]{famb}{keyc}[true]{}
\def\testa{\setkeys[my]{fama}}
\def\testb{\setkeys[my]{famb}}
\def\testset{\setkeys[my]{fama,famb}}
\end{verbatim}
\define@cmdkey[my]{fama}[my@]{keya}[none]{}
\define@choicekey[my]{fama}{keyb}{a,b,c}[b]{correct input: #1}
\define@boolkey[my]{famb}{keyc}[true]{}
\def\testa{\setkeys[my]{fama}}
\def\testb{\setkeys[my]{famb}}
\def\testset{\setkeys[my]{fama,famb}}
Now \verb+\testa+ can only set \verb+keya+, \verb+\testb+ can only
set \verb+keyb+ and \verb+\testset+ can set both.
\begin{verbatim}
\testa{keya}
  \my@keya\
\testa{keyb}
\testb{keyc}
  \ifmy@famb@keyc true \else false \fi
\testset{keya=test 11,keyb=c,keyc=false}
  \my@keya\space
  \ifmy@famb@keyc true \else false \fi
\end{verbatim}
\begin{itemize}
\item[$\Rightarrow$]
\testa{keya}
  \my@keya\
\testa{keyb}
\testb{keyc}
  \ifmy@famb@keyc true \else false \fi
\testset{keya=test 11,keyb=c,keyc=false}
  \my@keya\space
  \ifmy@famb@keyc true \else false \fi
\end{itemize}

\section{Pointers and values}

First we (re)define some keys.
\begin{verbatim}
\define@key{fam}{keyi}{key 1: #1 }
\define@key{fam}{keyii}{key 2: #1 }
\define@key{fam}{keyiii}[test 2]{key 3: #1 }
\end{verbatim}
\define@key{fam}{keyi}{key 1: #1 }
\define@key{fam}{keyii}{key 2: #1 }
\define@key{fam}{keyiii}[test 2]{key 3: #1 }

Pointers can be used to copy the value of other keys in the same
family. This can be done in two ways. The first example shows `on
the spot' value saving.
\begin{verbatim}
\setkeys{fam}{\savevalue{keyi}=test 12,\savevalue{keyii}=\usevalue{keyi}}
\setkeys{fam}{keyiii=\usevalue{keyii}}
\end{verbatim}
\begin{itemize}
\item[$\Rightarrow$]
\setkeys{fam}{\savevalue{keyi}=test 12,\savevalue{keyii}=\usevalue{keyi}}
\setkeys{fam}{keyiii=\usevalue{keyii}}
\end{itemize}

However, you can also predefine the keys to be saved for the entire
document with a single command. Note that we force keys to be saved
globally. This is necessary here since the commands in the source of
this document appear in an environment.
\begin{verbatim}
\savekeys{fam}{\global{keyi},\global{keyii}}
\setkeys{fam}{keyi=test 13,keyii=\usevalue{keyi}}
\setkeys{fam}{keyiii=\usevalue{keyii}}
\end{verbatim}
\begin{itemize}
\item[$\Rightarrow$]
\savekeys{fam}{\global{keyi},\global{keyii}}
\setkeys{fam}{keyi=test 13,keyii=\usevalue{keyi}}
\setkeys{fam}{keyiii=\usevalue{keyii}}
\end{itemize}

Pointers can be used in default values as well. Remember that the
previous example used \verb+\savekeys+ for \verb+keyi+ and \verb+keyii+.
\begin{verbatim}
\define@key{fam}{keyi}{keyi: #1 }
\define@key{fam}{keyii}[\usevalue{keyi}]{keyii: #1 }
\define@key{fam}{keyiii}[\usevalue{keyii}]{keyiii: #1 }
\setkeys{fam}{keyi=test 14}
\setkeys{fam}{keyii}
\setkeys{fam}{keyiii}
\end{verbatim}
\begin{itemize}
\item[$\Rightarrow$]
\define@key{fam}{keyi}{keyi: #1 }
\define@key{fam}{keyii}[\usevalue{keyi}]{keyii: #1 }
\define@key{fam}{keyiii}[\usevalue{keyii}]{keyiii: #1 }
\setkeys{fam}{keyi=test 14}
\setkeys{fam}{keyii}
\setkeys{fam}{keyiii}
\end{itemize}

\section{Presetting keys}

Simple example of the preset system.
\begin{verbatim}
\define@key{fam}{keyi}{keyi: #1 }
\define@key{fam}{keyii}{keyii: #1 }
\presetkeys{fam}{keyi=blue}{keyii=\usevalue{keyi}}
\setkeys{fam}{}\par
\setkeys{fam}{keyi=red}\par
\setkeys{fam}{keyii=red}
\presetkeys{fam}{keyi=red}{}
\setkeys{fam}{keyii=\usevalue{keyi}}
\end{verbatim}
\begin{itemize}
\item[$\Rightarrow$]
\define@key{fam}{keyi}{keyi: #1 }
\define@key{fam}{keyii}{keyii: #1 }
\presetkeys{fam}{keyi=blue}{keyii=\usevalue{keyi}}
\setkeys{fam}{}\par
\setkeys{fam}{keyi=red}\par
\setkeys{fam}{keyii=red}
\presetkeys{fam}{keyi=red}{}\par
\setkeys{fam}{keyii=\usevalue{keyi}}
\end{itemize}

\section{Application of pointers, values and presets}

\setkeys{Hyp}{urlcolor=black}
More complicated example of the preset
system and pointer system. See for more information about this
example the
\textsf{xkeyval} article on\par
\url{http://www.tug.org/TUGboat/tb25-2/tb81adriaens.pdf}.
\begin{verbatim}
\newdimen\shadowsize
\define@boolkey{Fbox}{frame}[true]{}
\define@boolkey{Fbox}{shadow}[true]{}
\define@key{Fbox}{framecolor}{\def\Fboxframecolor{#1}}
\define@key{Fbox}{shadowcolor}{\def\Fboxshadowcolor{#1}}
\define@key{Fbox}{framesize}{\setlength\fboxrule{#1}}
\define@key{Fbox}{shadowsize}{\setlength\shadowsize{#1}}
\presetkeys{Fbox}{frame,framecolor=red,framesize=0.5pt}%
 {shadow=\usevalue{frame},shadowcolor=\usevalue{framecolor}!40,%
  shadowsize=\usevalue{framesize}*4}
\savekeys{Fbox}{frame,framecolor,framesize}
\newcommand*\Fbox[2][]{%
 \setkeys{Fbox}{#1}%
 {\ifKV@Fbox@frame\else\fboxrule0pt\fi
  \ifKV@Fbox@shadow\else\shadowsize0pt\fi
  \sbox0{\fcolorbox{\Fboxframecolor}{white}{#2}}%
  \hskip\shadowsize
  \color{\Fboxshadowcolor}%
  \rule[-\dp0]{\wd0}{\ht0+\dp0}%
  \llap{\raisebox{\shadowsize}%
   {\box0\hskip\shadowsize}}}%
}
\Fbox{demo1}
\Fbox[framecolor=blue]{demo2}
\Fbox[shadow=false]{demo3}
\Fbox[framesize=1pt]{demo4}
\Fbox[frame=false,shadow]{demo5}
\end{verbatim}
\newdimen\shadowsize
\define@boolkey{Fbox}{frame}[true]{}
\define@boolkey{Fbox}{shadow}[true]{}
\define@key{Fbox}{framecolor}%
 {\def\Fboxframecolor{#1}}
\define@key{Fbox}{shadowcolor}%
 {\def\Fboxshadowcolor{#1}}
\define@key{Fbox}{framesize}%
 {\setlength\fboxrule{#1}}
\define@key{Fbox}{shadowsize}%
 {\setlength\shadowsize{#1}}
\presetkeys{Fbox}%
 {frame,framecolor=red,framesize=0.5pt}%
 {shadow=\usevalue{frame},
  shadowcolor=\usevalue{framecolor}!40,
  shadowsize=\usevalue{framesize}*4}
\savekeys{Fbox}{frame,framecolor,framesize}
\newcommand*\Fbox[2][]{%
 \setkeys{Fbox}{#1}%
 {\ifKV@Fbox@frame\else\fboxrule0pt\fi
  \ifKV@Fbox@shadow\else\shadowsize0pt\fi
  \sbox0{\fcolorbox{\Fboxframecolor}{white}{#2}}%
  \hskip\shadowsize
  \color{\Fboxshadowcolor}%
  \rule[-\dp0]{\wd0}{\ht0+\dp0}%
  \llap{\raisebox{\shadowsize}%
   {\box0\hskip\shadowsize}}}%
}
\Fbox{demo1}
\Fbox[framecolor=blue]{demo2}
\Fbox[shadow=false]{demo3}
\Fbox[framesize=1pt]{demo4}
\Fbox[frame=false,shadow]{demo5}

\end{document}
\endinput
%%
%% End of file `xkvex2.tex'.
