% tipaman2.tex
% Copyright 2002 FUKUI Rei
%
% This program may be distributed and/or modified under the
% conditions of the LaTeX Project Public License, either version 1.2
% of this license or (at your option) any later version.
% The latest version of this license is in
%   http://www.latex-project.org/lppl.txt
% and version 1.2 or later is part of all distributions of LaTeX 
% version 1999/12/01 or later.
%
% This program consists of all files listed in Manifest.txt.
%

\appendix
\addcontentsline{toc}{chapter}{Appendix}

\begingroup
\raggedbottom

\chapter{Annotated List of TIPA Symbols}

For each symbol, a large scale image of the symbol is displayed with a
frame. Within the frame, horizontal lines that indicate
\verb|x_height| and baseline are also shown. At the top left corner of
a frame, a number indicating the octal code
of the symbol is shown. In the case of a symbol from
\texttt{tipx} fonts, the code number is underlined.

Next, the following information is shown at the right of each symbol
in this order: (1) the name of the symbol, (2) explanation on its
usage with some examples (for non-IPA usages, an asterisk is put at the
beginning), (3) input method in typewriter style, and finally (4)
sources or references.

Sometimes the input method is displayed in the form of \textit{Input1:
  xxx, Input2: yyy\/}. In such cases \textit{Input1} indicates the one
used in the normal text environment and \textit{Input2}, the one used
in the IPA environment.

The following abbreviations are used in the examples of usage and
explanations in the footnote.

\vspace{1cm}

\emph{ExtIPA} = \emph{ExtIPA Symbols for Disordered Speech}

\emph{VoQS} = \emph{Voice Quality Symbols}

\PSG{} = \emph{Phonetic Symbol Guide} \citep{PSG:II}

\emph{Handbook} = \emph{Handbook of the International Phonetic
  Association} \citep{Handbook}

\emph{Principles} = \emph{Principles of the International Phonetic
  Association} \citep{Principles}

\emph{JIPA} = \emph{Journal of the International Phonetic Association}

\emph{IE} Indo-European

\emph{OHG} Old High German

\emph{OCS} Old Church Slavic


\newpage
\section{Vowels and Consonants}\label{list:symbols}

\begingroup
\setlength\parindent{0pt}

\medskip

\ipaitem{a}{Lower-case A}%
  {open front unrounded vowel}%
  {a}{}{\ipaall}{'141}

\ipxitem{\textrhooka}{Right-hook A}%
  {}%
  {\tbs textrhooka}{}{\PSG}{'040}

\ipaitem{\textturna}{Turned A}%
  {near-open central vowel}%
  {\tbs textturna}{5}{\ipaall}{'065}

\ipaitem{\textscripta}{Script A}%
  {open back unrounded vowel}%
  {\tbs textscripta}{A}{\ipaall}{'101}

\ipaitem{\textturnscripta}{Turned script A}%
  {open back rounded vowel}%
  {\tbs textturnscripta}{6}{\ipaall}{'066}

\ipxitem{\textinvscripta}{Inverted script A}%
  {}%
  {\tbs textinvscripta}{}{\PSG}{'041}

\ipaitem{\ae}{Ash}%
  {near-open front unrounded vowel}%
  {\tbs ae}{}{\ipaall}{'346}

\ipxitem{\textaolig}{A-O ligature }%
  {}%
  {\tbs textaolig}{}{\PSG}{'042}

\ipaitem{\textsca}{Small capital A\footnotemark}%
  {*open central unrounded vowel}%
  {\tbs textsca}{\tbs;A}{\PSG}{'300}%
  \footnotetext{This symbol is fairly common among Chinese
    phoneticians.}

\ipxitem{\textlhookfour}{Left-hook four }%
  {}%
  {\tbs textlhookfour}{}{\PSG}{'043}

\ipxitem{\textinvsca}{Inverted small capital A }%
  {}%
  {\tbs textinvsca}{}{\PSG}{'160}

\ipxitem{\textscaolig}{Small capital A-O ligature }%
  {}%
  {\tbs textscaolig}{}{\PSG}{'161}

\ipaitem{\textturnv}{Turned V\footnotemark}%
  {open-mid back unrounded vowel}%
  {\tbs textturnv}{2}{\ipaall}{'062}%
  \footnotetext{In a previous version of \PSG{} this symbol was
    called `Inverted V' but it was apparently a mistake.}

\ipxitem{\textscdelta}{Small capital delta }%
  {}%
  {\tbs textscdelta}{}{\PSG}{'162}

\ipaitem{b}{Lower-case B}%
  {voiced bilabial plosive}%
  {b}{}{\ipaall}{'142}

\ipaitem{\textcrb}{Crossed B}%
  {}%
  {\tbs textcrb}{}{\PSG}{'240}

\ipaitem{\textbarb}{Barred B}%
  {}%
  {\tbs textbarb}{}{\PSG}{Macro}

\ipaitem{\textsoftsign}{Soft sign}%
  {*as in \emph{OCS} ogn\textsoftsign{} `fire'.}%
  {\tbs textsoftsign}{}{\PSG}{'272}

\ipaitem{\texthardsign}{Hard sign}%
  {*as in \emph{OCS} grad\texthardsign{} `town'.}%
  {\tbs texthardsign}{}{\PSG}{'273}

\ipaitem{\texthtb}{Hooktop B}%
  {voiced bilabial implosive}%
  {\tbs texthtb}{\tbs!b}{\ipaall}{'341}

\ipaitem{\textscb}{Small capital B}%
  {voiced bilabial trill}%
  {\tbs textscb}{\tbs;B}{\ipanew}{'340}

\ipaitem{\textbeta}{Beta}%
  {voiced bilabial fricative}%
  {\tbs textbeta}{B}{\ipaall}{'102}

\ipaitem{c}{Lower-case C}%
  {voiceless palatal plosive}%
  {c}{}{\ipaall}{'143}

\ipaitem{\textbarc}{Barred C}%
  {}%
  {\tbs textbarc}{}{\PSG}{Macro}

\ipaitem{\v{c}}{Wedge C}%
  {*equivalent to IPA \textipa{tS}}%
  {\TD{v}{c}}{}{\PSG}{Macro}

\ipaitem{\c{c}}{C Cedilla}%
  {voiceless palatal fricative}%
  {\TD{c}{c}}{}{\ipaall}{'347}

\ipaitem{\texthtc}{Hooktop C}%
  {voiceless palatal implosive}%
  {\tbs texthtc}{}{IPA '89}{'301}

\ipaitem{\textctc}{Curly-tail C}%
  {voiceless alveolo-palatal fricative}%
  {\tbs textctc}{C}{\ipaall}{'103}

\ipaitem{\textstretchc}{Stretched C\footnotemark}%
  {postalveolar click}%
  {\tbs textstretchc}{}{\ipaold}{'302}%
  \footnotetext{The shape of this symbol differs according to the
    sources. In \PSG{} and recent articles in \emph{JIPA}, it is
    `stretched' toward both the ascender and descender regions and the
    whole shape looks like a thick staple. In the old days, however,
    it was stretched only toward the descender and the whole shape
    looked more like a stretched c, as is shown in the next item
    (original form).}

\ipxitem{\textstretchcvar}{Stretched C (original form) }%
  {}%
  {\tbs textstretchcvar}{}{\cite{Hottentot}}{'044}

\ipxitem{\textctstretchc}{Curly-tail stretched C }%
  {}%
  {\tbs textctstretchc}{}{\PSG}{'045}

\ipxitem{\textctstretchcvar}{Curly-tail stretched C (original form) }%
  {}%
  {\tbs textctstretchcvar}{}{\cite{Hottentot}}{'046}

\ipaitem{d}{Lower-case D}%
  {voiced dental or alveolar plosive}%
  {d}{}{\ipaall}{'144}

\ipaitem{\textcrd}{Crossed D}%
  {}%
  {\tbs textcrd}{}{\PSG}{'241}

\ipaitem{\textbard}{Barred D}%
  {}%
  {\tbs textbard}{}{\PSG}{Macro}

\ipxitem{\textfrhookd}{Front-hook D }%
  {}%
  {\tbs textfrhookd}{}{\PSG}{'047}

\ipxitem{\textfrhookdvar}{Front-hook D (Original)\footnotemark}%
  {}%
  {\tbs textfrhookdvar}{}{}{'050}%
  \footnotetext{This shape is used by \cite{Jones:Phoneme}.}

\ipaitem{\texthtd}{Hooktop D}%
  {voiced dental or alveolar implosive}%
  {\tbs texthtd}{\tbs!d}{\ipaall}{'342}

\ipaitem{\textrtaild}{Right-tail D}%
  {voiced retroflex plosive}%
  {\tbs textrtaild}{\tbs:d}{\ipaall}{'343}

\ipaitem{\texthtrtaild}{Hooktop right-tail D }%
  {voiced retroflex implosive}%
  {\tbs texthtrtaild}{}{\PSG, \Handbook}{'243}

\ipaitem{\textctd}{Curly-tail D}%
  {*voiced alveolo-palatal plosive}%
  {\tbs textctd}{}{}{'242}

\ipxitem{\textdblig}{D-B ligature }%
  {}%
  {\tbs textdblig}{}{\PSG}{'051}

\ipaitem{\textdzlig}{D-Z ligature}%
  {}%
  {\tbs textdzlig}{}{\PSG}{Macro}

\ipaitem{\textdctzlig}{D-Curly-tail Z ligature}%
  {}%
  {\tbs textdctzlig}{}{}{Macro}

\ipaitem{\textdyoghlig}{D-Yogh ligature}%
  {voiced postalveolar affricate}%
  {\tbs textdyoghlig}{}{\ipaall}{'303}

\ipaitem{\textctdctzlig}{Curly-tail D-Curly-tail Z ligature}%
  {}%
  {\tbs textctdctzlig}{}{}{Macro}

\ipaitem{\dh}{Eth}%
  {voiced dental fricative}%
  {\tbs dh}{D}{\ipaall}{'104}

\ipaitem{e}{Lower-case E}%
  {close-mid front unrounded vowel}%
  {e}{}{\ipaall}{'145}

\ipxitem{\textrhooke}{Right-hook E }%
  {}%
  {\tbs textrhooke}{}{\PSG}{'052}

\ipaitem{\textschwa}{Schwa}%
  {mid central vowel}%
  {\tbs textschwa}{@}{\ipaall}{'100}

\ipaitem{\textrhookschwa}{Right-hook schwa}%
  {r-colored \textschwa}%
  {\tbs textrhookschwa}{}{\ipaold}{'304}

\ipaitem{\textreve}{Reversed E}%
  {close-mid central unrounded vowel}%
  {\tbs textreve}{9}{\ipaall}{'071}

\ipaitem{\textsce}{Small capital E}%
  {}%
  {\tbs textsce}{\tbs;E}{\PSG}{'244}

\ipaitem{\textepsilon}{Epsilon}%
  {open-mid front unrounded vowel}%
  {\tbs textepsilon}{E}{\ipaall}{'105}

\ipxitem{\textrhookepsilon}{Right-hook epsilon }%
  {}%
  {\tbs textrhookepsilon}{}{\PSG}{'053}

\ipaitem{\textcloseepsilon}{Closed epsilon\footnotemark}%
  {(obsolete) open-mid central rounded vowel}%
  {\tbs textcloseepsilon}{}{IPA '93}{'305}%
  \footnotetext{In the 1993 version of IPA, this symbol was used as
  the symbol for the open-mid central rounded vowel. However, in the
  1996 version, this symbol was replaced by Closed reversed epsilon,
  i.e., \textcloserevepsilon. In fact, it was a typographical error, as 
  was anounced in \citet[p.\ 48]{IPA:Preview}.}

\ipaitem{\textrevepsilon}{Reversed epsilon}%
  {open-mid central unrounded vowel}%
  {\tbs textrevepsilon}{3}{\ipaall}{'063}

\ipaitem{\textrhookrevepsilon}{Right-hook reversed epsilon}%
  {r colored \textrevepsilon}%
  {\tbs textrhookrevepsilon}{}{\PSG}{'307}

\ipaitem{\textcloserevepsilon}{Closed reversed epsilon\footnotemark}%
  {open-mid central rounded vowel}%
  {\tbs textcloserevepsilon}{}{\Handbook}{'306}%
  \footnotetext{See the footnote above.}

\ipaitem{f}{Lower-case F}%
  {voiceless labiodental fricative}%
  {f}{}{\ipaall}{'146}

\ipxitem{\textscf}{Small capital F }%
  {}%
  {\tbs textscf}{}{\PSG}{'163}

\ipaitem{g}{Lower-case G}%
  {voiced velar plosive}%
  {\tbs textscriptg}{g}{\ipaall}{'147}

\ipaitem{\textbarg}{Barred G}%
  {}%
  {\tbs textbarg}{}{\PSG}{Macro}

\ipaitem{\textcrg}{Crossed G}%
  {}%
  {\tbs textcrg}{}{\PSG}{Macro}

\ipaitem{\texthtg}{Hooktop G}%
  {voiced velar implosive}%
  {\tbs texthtg}{\tbs!g}{\ipaall}{'344}

\ipaitem{\textg}{Looptail G}%
  {equivalent to \textscriptg}%
  {g}{\tbs textg}{}{'245}

\ipaitem{\textscg}{Small capital G}%
  {voiced uvular plosive}%
  {\tbs textscg}{\tbs;G}{\ipaall}{'345}

\ipaitem{\texthtscg}{Hooktop small capital G}%
  {voiced uvular implosive}%
  {\tbs texthtscg}{\tbs!G}{\ipanew}{'311}

\ipaitem{\textgamma}{Gamma}%
  {voiced velar fricative}%
  {\tbs textgamma}{G}{\ipaall}{'107}

\ipxitem{\textgrgamma}{Greek gamma\footnotemark}%
  {}%
  {\tbs textgrgamma}{}{\PSG}{'054}%
  \footnotetext{It is not my intention to include all the Greek letters 
  appearing in \PSG. The reason for including this symbol is to
  assure typographical consistency with the next two symbols derived from
  Greek gamma.}

\ipxitem{\textfrtailgamma}{Front-tail gamma }%
  {}%
  {\tbs textfrtailgamma}{}{\PSG}{'055}

\ipxitem{\textbktailgamma}{Back-tail gamma }%
  {}%
  {\tbs textbktailgamma}{}{\PSG}{'056}

\ipaitem{\textbabygamma}{Baby gamma}%
  {(obsolete) close-mid back unrounded vowel}%
  {\tbs textbabygamma}{}{\ipaold}{'310}

\ipaitem{\textramshorns}{Ram's horns}%
  {close-mid back unrounded vowel}%
  {\tbs textramshorns}{7}{\ipanew}{'067}

\ipaitem{h}{Lower-case H}%
  {voiceless glottal fricative}%
  {h}{}{\ipaall}{'150}

\ipaitem{\texthvlig}{H-V ligature}%
  {*as in \emph{Gothic} \texthvlig{}as `what'.}%
  {\tbs texthvlig}{}{\PSG}{'377}

\ipaitem{\textcrh}{Crossed H\footnotemark}%
  {voiceless pharyngeal fricative}%
  {\tbs textcrh}{}{\ipaall}{'350}%
  \footnotetext{In \Handbook, this symbol is called `Barred H'.}

\ipaitem{\texthth}{Hooktop H}%
  {voiced glottal fricative}%
  {\tbs texthth}{H}{\ipaall}{'110}

\ipxitem{\textrtailhth}{Right-tail hooktop H }%
  {}%
  {\tbs textrtailhth}{}{\PSG}{'057}

\ipxitem{\textheng}{Heng }%
  {}%
  {\tbs textheng}{}{\PSG}{'060}

\ipaitem{\texththeng}{Hooktop heng}%
  {simultaneous \textesh\ and x}%
  {\tbs texththeng}{}{\ipaall}{'312}

\ipaitem{\textturnh}{Turned H}%
  {voiced labial-palatal approximant}%
  {\tbs textturnh}{4}{\ipaall}{'064}

\ipaitem{\textsch}{Small capital H}%
  {voiceless epiglottal fricative}%
  {\tbs textsch}{\tbs;H}{\ipanew}{'313}

\ipaitem{i}{Lower-case I}%
  {close front unrounded vowel}%
  {i}{}{\ipaall}{'151}

\ipaitem{\i}{Undotted I}%
  {*used in Turkish orthography}%
  {\tbs i}{}{\PSG}{'031}

\ipaitem{\textbari}{Barred I}%
  {close central unrounded vowel}%
  {\tbs textbari}{1}{\ipaall}{'061}

\ipaitem{\textsci}{Small capital I}%
  {near-close near-front unrounded vowel}%
  {\tbs textsci}{I}{\ipanew}{'111}

\ipaitem{\textiota}{Iota}%
  {(obsolete) near-close near-front unrounded vowel}%
  {\tbs textiota}{}{\ipaold}{'314}

\ipxitem{\textlhti}{Left-hooktop I\footnotemark}%
  {}%
  {\tbs textlhti}{}{}{'061}%
  \footnotetext{This symbol is sometimes found instead of
    \textlhtlongi\ (next item) in textbooks of Chinese in Japan.}

\ipaitem{\textlhtlongi}{Left-hooktop Long I\footnotemark}%
  {}%
  {\tbs textlhtlongi}{}{\PSG}{'246}%
  \footnotetext{The two symbols \textlhtlongi{} and \textvibyi{}
    are mainly used among Chinese linguists. These
    symbols are based on ``det svenska landsm\aa{}lsalfabetet'' and
    introduced to China by Bernhard Karlgren. The original shapes of
    these symbols were in italic as was always the case with 
    ``det svenska landsm\aa{}lsalfabetet''. It seems that the Chinese
    linguists who wanted to continue to use these symbols in IPA
    changed their shapes upright. \PSG's descriptions to the origin of
    these symbols are inaccurate.}

\ipaitem{\textvibyi}{Viby I\footnotemark}%
  {}%
  {\tbs textvibyi}{}{\PSG}{'247}%
  \footnotetext{I call this symbol `Viby I', based on the
    following description by Bernhard Karlgren: ``Une voyelle tr\`es
    analogue \`a \textvibyi{} se rencontre dans certains dial.\ su\'edois;
    on l'appelle `i de Viby'.'' \citep[p.\ 295]{Karlgren}\label{vibyi}}

\ipaitem{\textraisevibyi}{Raised Viby I}%
  {}%
  {\tbs textraisevibyi}{}{}{Macro}

\ipaitem{j}{Lower-case J}%
  {voiced palatal approximant}%
  {j}{}{\ipaall}{'152}

\ipaitem{\j}{Undotted J}%
  {}%
  {\tbs j}{}{}{'032}

\ipaitem{\textctj}{Curly-tail J\footnotemark}%
  {voiced palatal fricative}%
  {\tbs textctj}{J}{\ipanew}{'112}%
  \footnotetext{In the official IPA charts of '89 through '96, this symbol 
    has a dish serif on top of the stem, rather than the normal sloped 
    serif found in the letter j. I found no reason why it should have
    a dish serif here, so I changed it to a normal sloped serif.
    The official (?) IPA shape can be used by the \texttt{\tbs textctjvar}
    command. (\textctjvar)}

\ipxitem{\textctjvar}{Curly-tail J (a variety found in 1996 IPA) }%
  {same as the above}%
  {\tbs textctjvar}{}{\ipanew}{'062}

\ipaitem{\v{\j}}{Wedge J}%
  {*equivalent to IPA \textipa{dZ}}%
  {\TD{v}{\tbs j}}{}{\PSG}{Macro}

\ipaitem{\textbardotlessj}{Barred dotless J}%
  {voiced palatal plosive}%
  {\tbs textbardotlessj}{}{\ipanew}{'351}

\ipaitem{\textObardotlessj}{Old barred dotless J}%
  {voiced palatal plosive}%
  {\tbs textObardotlessj}{}{\ipaold}{'315}

\ipaitem{\texthtbardotlessj}{Hooktop barred dotless J\footnotemark}%
  {voiced palatal implosive}%
  {\tbs texthtbardotlessj}{\tbs!j}{\Handbook}{'352}%
  \footnotetext{In \PSG{} the shape of this symbol slightly
    differs. Here I followed the shape found in IPA '89--'96.}

\ipxitem{\texthtbardotlessjvar}{Hooktop barred dotless J (a variety) }%
  {same as the above}%
  {\tbs texthtbardotlessjvar}{}{IPA '89--'93, \PSG}{'063}

\ipaitem{\textscj}{Small capital J}%
  {}%
  {\tbs textscj}{\tbs;J}{\PSG}{'250}

\ipaitem{k}{Lower-case K}%
  {voiceless velar plosive}%
  {k}{}{\ipaall}{'153}

\ipaitem{\texthtk}{Hooktop K}%
  {voiceless velar implosive}%
  {\tbs texthtk}{}{IPA '89}{'316}

\ipaitem{\textturnk}{Turned K}%
  {}%
  {\tbs textturnk}{\tbs*k}{\PSG}{'251}

\ipxitem{\textsck}{Small capital K }%
  {}%
  {\tbs textsck}{}{\PSG}{'164}

\ipxitem{\textturnsck}{Turned small capital K }%
  {}%
  {\tbs textturnsck}{}{\PSG}{'165}

\ipaitem{l}{Lower-case L}%
  {alveolar lateral approximant}%
  {l}{}{\ipaall}{'154}

\ipaitem{\textltilde}{L with tilde}%
  {}%
  {\tbs textltilde}{\tbs|\ttilde l}{\ipaall}{'353}

\ipaitem{\textbarl}{Barred L}%
  {}%
  {\tbs textbarl}{}{\PSG}{'252}

\ipaitem{\textbeltl}{Belted L}%
  {voiceless dental or alveolar lateral fricative}%
  {\tbs textbeltl}{}{\ipaall}{'354}

\ipaitem{\textrtaill}{Right-tail L}%
  {retroflex lateral approximant}%
  {\tbs textrtaill}{\tbs:l}{\ipaall}{'355}

\ipaitem{\textlyoghlig}{L-Yogh ligature}%
  {voiced alveolar lateral fricative}%
  {\tbs textlyoghlig}{}{\ipanew}{'320}

\ipaitem{\textOlyoghlig}{Old L-Yogh ligature}%
  {voiced alveolar lateral fricative}%
  {\tbs textOlyoghlig}{}{\ipaold}{'255}

\ipxitem{\textlfishhookrlig}{L-Fish-hook R ligature}%
  {alveolar lateral flap}%
  {\tbs textlfishhookrlig}{}{}{'111}

\ipaitem{\textscl}{Small capital L}%
  {velar lateral approximant}%
  {\tbs textscl}{\tbs;L}{\ipanew}{'317}

\ipxitem{\textrevscl}{Reversed small capital L }%
  {}%
  {\tbs textrevscl}{}{\PSG}{'166}

\ipaitem{\textlambda}{Lambda}%
  {}%
  {\tbs textlambda}{}{\PSG}{'253}

\ipaitem{\textcrlambda}{Crossed lambda}%
  {}%
  {\tbs textcrlambda}{}{\PSG}{'254}

\ipaitem{m}{Lower-case M}%
  {bilabial nasal}%
  {m}{}{\ipaall}{'155}

\ipaitem{\textltailm}{Left-tail M (at right)\footnotemark}%
  {labiodental nasal}%
  {\tbs textltailm}{M}{\ipaall}{'115}%
  \footnotetext{\PSG\ calls this symbol `Meng'.}

\ipxitem{\texthmlig}{H-M ligature }%
  {}%
  {\tbs texthmlig}{}{\PSG}{'064}

\ipaitem{\textturnm}{Turned M}%
  {close back unrounded vowel}%
  {\tbs textturnm}{W}{\ipaall}{'127}

\ipaitem{\textturnmrleg}{Turned M, right leg}%
  {voiced velar approximant}%
  {\tbs textturnmrleg}{}{IPA '79--'93}{'356}

\ipxitem{\textscm}{Small capital M }%
  {}%
  {\tbs textscm}{}{\PSG}{'167}

\ipaitem{n}{Lower-case N}%
  {dental or alveolar nasal}%
  {n}{}{\ipaall}{'156}

\ipxitem{\textfrbarn}{Front-bar N\footnotemark}%
  {}%
  {\tbs textfrbarn}{}{\PSG}{'065}%
  \footnotetext{This shape is based on \PSG\ (p.~119). However, its
  original shape looks a little different. Here I simply followed 
  the shape found in \PSG\ because in its source \citep{Trager} the
  shape of this symbol is unclear (typewritten, modified by handwriting).}

\ipxitem{\textnrleg}{N, right leg\footnotemark}%
  {}%
  {\tbs textnrleg}{}{IPA '49}{'066}%
  \footnotetext{In \PSG, this symbol is called `Long-Leg N'.}

\ipaitem{\~n}{N with tilde}%
  {}%
  {\tbs\ttilde n}{}{\PSG}{Macro}

\ipaitem{\textltailn}{Left-tail N (at left)}%
  {palatal nasal}%
  {\tbs textltailn}{}{\ipaall}{'361}

\ipaitem{\ng}{Eng}%
  {velar nasal}%
  {\tbs ng}{N}{\ipaall}{'116}

\ipaitem{\textrtailn}{Right-tail N}%
  {retroflex nasal}%
  {\tbs textrtailn}{\tbs:n}{\ipaall}{'357}

\ipaitem{\textctn}{Curly-tail N}%
  {*alveolo-palatal nasal}%
  {\tbs textctn}{}{}{'256}

\ipaitem{\textscn}{Small capital N}%
  {uvular nasal}%
  {\tbs textscn}{\tbs;N}{\ipaall}{'360}

\ipaitem{o}{Lower-case O}%
  {close-mid back rounded vowel}%
  {o}{}{\ipaall}{'157}

\ipxitem{\textfemale}{Female sign }%
  {}%
  {\tbs textfemale}{}{\PSG}{'067}

\ipxitem{\textuncrfemale}{Uncrossed female sign }%
  {}%
  {\tbs textuncrfemale}{}{\PSG}{'070}

\ipaitem{\textbullseye}{Bull's eye\footnotemark}%
  {bilabial click}%
  {\tbs textbullseye}{\tbs!o}{IPA '93, '96}{'362}%
  \footnotetext{In \PSG\ this name is spelled `Bullseye'.}

\ipxitem{\textObullseye}{Bull's eye (an old version) }%
  {bilabial click}%
  {\tbs textObullseye}{}{IPA '79, '89}{'071}

\ipaitem{\textbaro}{Barred O}%
  {close-mid central rounded vowel}%
  {\tbs textbaro}{8}{\ipaall}{'070}

\ipaitem{\o}{Slashed O}%
  {close-mid front rounded vowel}%
  {\tbs o}{}{\ipaall}{'370}

\ipaitem{\oe}{O-E ligature}%
  {open-mid front rounded vowel}%
  {\tbs oe}{}{\ipaall}{'367}

\ipaitem{\textscoelig}{Small capital O-E ligature}%
  {open front rounded vowel}%
  {\tbs textscoelig}{\tbs OE}{IPA '79--'96}{'327}

\ipaitem{\textopeno}{Open O}%
  {open-mid back rounded vowel}%
  {\tbs textopeno}{O}{\ipaall}{'117}

\ipxitem{\textrhookopeno}{Right-hook open O }%
  {}%
  {\tbs textrhookopeno}{}{\PSG}{'072}

\ipaitem{\textturncelig}{Turned C (Open O)-E ligature}%
  {}%
  {\tbs textturncelig}{}{\PSG}{'257}

\ipaitem{\textomega}{Omega}%
  {}%
  {\tbs textomega}{}{\PSG}{'260}

\ipxitem{\textinvomega}{Inverted omega }%
  {}%
  {\tbs textinvomega}{}{\PSG}{'073}

\ipaitem{\textcloseomega}{Closed omega}%
  {(obsolete) near-close near-back rounded vowel}%
  {\tbs textcloseomega}{}{\ipaold}{'321}

\ipaitem{\textscomega}{Small capital omega}%
  {}%
  {\tbs textscomega}{}{\PSG}{'261}

\ipaitem{p}{Lower-case P}%
  {voiceless bilabial plosive}%
  {p}{}{\ipaall}{'160}

\ipaitem{\texthtp}{Hooktop P}%
  {voiceless bilabial implosive}%
  {\tbs texthtp}{}{IPA '89}{'322}

\ipxitem{\textlhookp}{Left-hook P }%
  {}%
  {\tbs textlhookp}{}{\PSG}{'074}

\ipxitem{\textscp}{Small capital P }%
  {}%
  {\tbs textscp}{}{\PSG}{'170}

\ipaitem{\textwynn}{Wynn}%
  {*labiovelar approximant}%
  {\tbs textwynn}{}{Old English}{'337}

\ipaitem{\textthorn}{Thorn}%
  {*interdental fricative}%
  {\tbs textthorn}{\tbs th}{Old English}{'376}

\ipxitem{\textthornvari}{A variety of thorn (1) }%
  {}%
  {\tbs textthornvari}{}{\PSG}{'120}

\ipxitem{\textthornvarii}{A variety of thorn (2) }%
  {}%
  {\tbs textthornvarii}{}{\PSG}{'121}

\ipxitem{\textthornvariii}{A variety of thorn (3) }%
  {}%
  {\tbs textthornvariii}{}{\PSG}{'122}

\ipxitem{\textthornvariv}{A variety of thorn (4) }%
  {}%
  {\tbs textthornvariv}{}{\PSG}{'123}

\ipaitem{\textphi}{Phi}%
  {voiceless bilabial fricative}%
  {\tbs textphi}{F}{\ipaall}{'106}

\ipaitem{q}{Lower-case Q}%
  {voiceless uvular plosive}%
  {q}{}{\ipaall}{'161}

\ipaitem{\texthtq}{Hooktop Q}%
  {voiceless uvular implosive}%
  {\tbs texthtq}{}{IPA '89}{'323}

\ipxitem{\textqplig}{Q-P ligature }%
  {}%
  {\tbs textqplig}{}{\PSG}{'075}

\ipxitem{\textscq}{Small capital Q\footnotemark}%
  {*voiceless pharyngeal plosive}%
  {\tbs textscq}{\tbs;Q}{}{'171}%
  \footnotetext{Suggested by Prof S. Tsuchida for Austronesian
    languages in Taiwan. In \PSG{} `Female Sign' and `Uncrossed Female
    Sign'(pp.~110--111) are noted for pharyngeal stops, as proposed by
    Trager (1964). Also, I'm not sure about the difference between an
    epiglottal plosive and a pharyngeal stop.}

\ipaitem{r}{Lower-case R}%
  {alveolar trill}%
  {r}{}{\ipaall}{'162}

\ipaitem{\textfishhookr}{Fish-hook R}%
  {alveolar tap or flap}%
  {\tbs textfishhookr}{R}{\ipaall}{'122}

\ipaitem{\textlonglegr}{Long-leg R}%
  {alveolar fricative trill}%
  {\tbs textlonglegr}{}{\ipaold}{'324}

\ipaitem{\textrtailr}{Right-tail R}%
  {retroflex tap or flap}%
  {\tbs textrtailr}{\tbs:r}{\ipaall}{'363}

\ipaitem{\textturnr}{Turned R}%
  {alveolar approximant}%
  {\tbs textturnr}{\tbs*r}{\ipaall}{'364}

\ipaitem{\textturnrrtail}{Turned R, right tail}%
  {retroflex approximant}%
  {\tbs textturnrrtail}{\tbs:R}{\ipaall}{'365}

\ipaitem{\textturnlonglegr}{Turned long-leg R}%
  {alveolar lateral flap}%
  {\tbs textturnlonglegr}{}{\ipaall}{'325}

\ipaitem{\textscr}{Small capital R}%
  {uvular trill}%
  {\tbs textscr}{\tbs;R}{\ipaall}{'366}

\ipxitem{\textrevscr}{Reversed small capital R }%
  {}%
  {\tbs textrevscr}{}{\PSG}{'172}

\ipaitem{\textinvscr}{Inverted small capital R}%
  {voiced uvular fricative}%
  {\tbs textinvscr}{K}{\ipaall}{'113}

\ipaitem{s}{Lower-case S}%
  {voiceless alveolar fricative}%
  {s}{}{\ipaall}{'163}

\ipaitem{\v{s}}{Wedge S}%
  {*equivalent to IPA \textesh}%
  {\TD{v}{s}}{}{\PSG}{Macro}

\ipaitem{\textrtails}{Right-tail S (at left)}%
  {voiceless retroflex fricative}%
  {\tbs textrtails}{\tbs:s}{\ipaall}{'371}

\ipaitem{\textesh}{Esh}%
  {voiceless postalveolar fricative}%
  {\tbs textesh}{S}{\ipaall}{'123}

\ipaitem{\textdoublebaresh}{Double-barred esh}%
  {}%
  {\tbs textdoublebaresh}{}{\cite{Hottentot}, \PSG}{Macro}

\ipxitem{\textlooptoprevesh}{Reversed esh with top loop }%
  {}%
  {\tbs textlooptoprevesh}{}{IPA '49}{'076}

\ipaitem{\textctesh}{Curly-tail esh}%
  {palatalized \textesh}%
  {\tbs textctesh}{}{\ipaold}{'262}

\ipaitem{t}{Lower-case T}%
  {voiceless dental or alveolar plosive}%
  {t}{}{\ipaall}{'164}

\ipxitem{\textfrhookt}{Front-hook T }%
  {}%
  {\tbs textfrhookt}{}{\PSG}{'077}

\ipaitem{\textlhookt}{Left-hook T}%
  {palatalized t}%
  {\tbs textlhookt}{}{\PSG}{'263}

\ipaitem{\textrtailt}{Right-tail T}%
  {voiceless retroflex plosive}%
  {\tbs textrtailt}{\tbs:t}{\ipaall}{'372}

\ipaitem{\texthtt}{Hooktop T}%
  {voiceless dental or alveolar implosive}%
  {\tbs texthtt}{}{IPA '89}{'326}

\ipaitem{\textturnt}{Turned T}%
  {dental click}%
  {\tbs textturnt}{\tbs*t}{\ipaold}{'330}

\ipxitem{\textctturnt}{Curly-tail turned T }%
  {}%
  {\tbs textctturnt}{}{\cite{Hottentot}, \PSG}{'100}

\ipaitem{\textctt}{Curly-tail T}%
  {*voiceless alveolo-palatal plosive}%
  {\tbs textctt}{}{}{'264}

\ipaitem{\texttctclig}{T-Curly-tail C ligature}%
  {}%
  {\tbs texttctclig}{}{}{Macro}

\ipaitem{\textcttctclig}{Curly-tail T-Curly-tail C ligature}%
  {}%
  {\tbs textcttctclig}{}{}{Macro}

\ipaitem{\texttslig}{T-S ligature}%
  {}%
  {\tbs texttslig}{}{\ipaold}{'265}

\ipaitem{\textteshlig}{T-Esh ligature}%
  {voiceless postalveolar affricate}%
  {\tbs textteshlig}{}{\ipaall}{'331}

\ipaitem{\texttheta}{Theta}%
  {voiceless dental fricative}%
  {\tbs texttheta}{T}{\ipaall}{'124}

\ipaitem{u}{Lower-case U}%
  {close back rounded vowel}%
  {u}{}{\ipaall}{'165}

\ipaitem{\textbaru}{Barred U}%
  {close central rounded vowel}%
  {\tbs textbaru}{0}{\ipaall}{'060}

\ipaitem{\textupsilon}{Upsilon}%
  {near-close near-back rounded vowel}%
  {\tbs textupsilon}{U}{\ipanew}{'125}

\ipaitem{\textscu}{Small capital U}%
  {*equivalent to IPA \textupsilon}%
  {\tbs textscu}{\tbs;U}{\ipaall}{'366}

\ipxitem{\textturnscu}{Turned small capital U }%
  {}%
  {\tbs textturnscu}{}{\PSG}{'173}

\ipaitem{v}{Lower-case V}%
  {voiced labiodental fricative}%
  {v}{}{\ipaall}{'166}

\ipaitem{\textscriptv}{Script V\footnotemark}%
  {voiced labiodental approximant}%
  {\tbs textscriptv}{V}{\ipaall}{'126}%
  \footnotetext{In \Handbook, this symbols is called `Cursive V'.}

\ipaitem{w}{Lower-case W}%
  {voiced labio-velar approximant}%
  {w}{}{\ipaall}{'167}

\ipaitem{\textturnw}{Turned W}%
  {voiceless labio-velar fricative}%
  {\tbs textturnw}{\tbs*w}{\ipaall}{'373}

\ipaitem{x}{Lower-case X}%
  {voiceless velar fricative}%
  {x}{}{\ipaall}{'170}

\ipaitem{\textchi}{Chi}%
  {voiceless uvular fricative}%
  {\tbs textchi}{X}{\ipaall}{'130}

\ipaitem{y}{Lower-case Y}%
  {close front rounded vowel}%
  {y}{}{\ipaall}{'171}

\ipaitem{\textturny}{Turned Y}%
  {palatal lateral approximant}%
  {\tbs textturny}{L}{\ipaall}{'114}

\ipaitem{\textscy}{Small capital Y}%
  {near-close near-front rounded vowel}%
  {\tbs textscy}{Y}{\ipaall}{'131}

\ipaitem{\textlhtlongy}{Left-hooktop long Y\footnotemark\ }%
  {}%
  {\tbs textlhtlongy}{}{\PSG}{'266}%
  \footnotetext{See explanations in footnote~\ref{vibyi}.}

\ipaitem{\textvibyy}{Viby Y\footnotemark}%
  {}%
  {\tbs textvibyy}{}{\PSG}{'267}%
  \footnotetext{See explanations in footnote~\ref{vibyi}.}

\ipaitem{z}{Lower-case Z}%
  {voiced alveolar fricative}%
  {z}{}{\ipaall}{'172}

\ipaitem{\textcommatailz}{Comma-tail Z}%
  {*as in \emph{OHG} \"e\textcommatailz\textcommatailz an `to eat'.}%
  {\tbs textcommatailz}{}{OHG, \PSG}{'336}

\ipaitem{\v{z}}{Wedge Z}%
  {*equivalent to IPA \textyogh}%
  {\TD{v}{z}}{}{\PSG}{Macro}

\ipaitem{\textctz}{Curly-tail Z}%
  {voiced alveolo-palatal fricative}%
  {\tbs textctz}{}{\ipaall}{'375}

\ipaitem{\textrtailz}{Right-tail Z}%
  {voiced retroflex fricative}%
  {\tbs textrtailz}{\tbs:z}{\ipaall}{'374}

\ipaitem{\textcrtwo}{Crossed two}%
  {}%
  {\tbs textcrtwo}{}{IPA '49}{Macro}

\ipxitem{\textturntwo}{Turned two }%
  {}%
  {\tbs textturntwo}{}{IPA '49}{'101}

\ipaitem{\textyogh}{Yogh\footnotemark}%
  {voiced postalveolar fricative}%
  {\tbs textyogh}{Z}{\ipaall}{'132}%
  \footnotetext{In \Handbook, this symbols is called `Ezh'.}

\ipxitem{\textbenttailyogh}{Bent-tail yogh }%
  {}%
  {\tbs textbenttailyogh}{}{IPA '49}{'102}

\ipaitem{\textctyogh}{Curly-tail yogh}%
  {palatalized \textyogh}%
  {\tbs textctyogh}{}{\ipaold}{'270}

\ipaitem{\textrevyogh}{Reversed yogh}%
  {}%
  {\tbs textrevyogh}{}{\PSG}{'271}

\ipxitem{\textturnthree}{Turned three }%
  {}%
  {\tbs textturnthree}{}{IPA '49}{'103}

\ipaitem{\textglotstop}{Glottal stop}%
  {glottal plosive}%
  {\tbs textglotstop}{P}{\ipaall}{'120}

\ipxitem{\textglotstopvari}{A variety of glottal stop (1) }%
  {}%
  {\tbs textglotstopvari}{}{\PSG}{'124}

\ipxitem{\textglotstopvarii}{A variety of glottal stop (2) }%
  {}%
  {\tbs textglotstopvarii}{}{\PSG}{'125}

\ipxitem{\textglotstopvariii}{A variety of glottal stop (3) }%
  {}%
  {\tbs textglotstopvariii}{}{\PSG}{'126}

\ipaitem{\textraiseglotstop}{Superscript glottal stop}%
  {}%
  {\tbs textraiseglotstop}{}{}{'274}

\ipaitem{\textbarglotstop}{Barred glottal stop}%
  {epiglottal plosive}%
  {\tbs textbarglotstop}{}{\ipanew}{'334}

\ipaitem{\textinvglotstop}{Inverted glottal stop}%
  {alveolar lateral click}%
  {\tbs textinvglotstop}{}{\ipaold}{'333}

\ipaitem{\textcrinvglotstop}{Crossed inverted glottal stop}%
  {}%
  {\tbs textcrinvglotstop}{}{IPA '49}{Macro}

\ipxitem{\textctinvglotstop}{Curly-tail inverted glottal stop }%
  {}%
  {\tbs textctinvglotstop}{}{\cite{Hottentot}, \PSG}{'104}

\ipxitem{\textturnglotstop}{Turned glottal stop (PSG 1996:211) }%
  {}%
  {\tbs textturnglotstop}{}{\PSG}{'105}

\ipaitem{\textrevglotstop}{Reversed glottal stop}%
  {voiced pharyngeal fricative}%
  {\tbs textrevglotstop}{Q}{\ipaall}{'121}

\ipaitem{\textbarrevglotstop}{Barred reversed glottal stop}%
  {voiced epiglottal fricative}%
  {\tbs textbarrevglotstop}{}{\ipanew}{'335}

\ipaitem{\textpipe}{Pipe}%
  {dental click}%
  {\tbs textpipe}{|}{\ipanew}{'174}

\ipxitem{\textpipevar}{Pipe (a variety with no descender) }%
  {dental click}%
  {\tbs textpipevar}{}{\PSG}{'106}

\ipaitem{\textdoublebarpipe}{Double-barred pipe}%
  {palatoalveolar click}%
  {\tbs textdoublebarpipe}{}{\ipanew}{'175}

\ipxitem{\textdoublebarpipevar}{Double-barred pipe (a variety with no descender) }%
  {same as the above}%
  {\tbs textdoublebarpipevar}{}{\PSG}{'110}

\ipaitem{\textdoublebarslash}{Double-barred slash}%
  {*a variant of \textdoublebarpipe}%
  {\tbs textdoublebarslash}{}{\PSG}{Macro}

\ipaitem{\textdoublepipe}{Double pipe}%
  {alveolar lateral click}%
  {\tbs textdoublepipe}{||}{\ipanew}{'177}

\ipxitem{\textdoublepipevar}{Double pipe (a variety with no descender) }%
  {same as the above}%
  {\tbs textdoublepipevar}{}{\PSG}{'107}

\ipaitem{!}{Exclamation point}%
  {(post)alveolar click}%
  {!}{}{\ipanew}{'041}


\section{Suprasegmentals}

\ipaitem{\textprimstress}{Vertical stroke (Superior)}%
  {primary stress}%
  {\tbs textprimstress}{"}{\ipaall}{'042}

\ipaitem{\textsecstress}{Vertical stroke (Inferior)}%
  {secondary stress}%
  {\tbs textsecstress}{""}{\ipaall}{'177}

\ipaitem{\textlengthmark}{Length mark}%
  {long}%
  {\tbs textlengthmark}{:}{\ipaall}{'072}

\ipaitem{\texthalflength}{Half-length mark}%
  {half-long}%
  {\tbs texthalflength}{;}{\ipaall}{'073}

\ipaitem{\textvertline}{Vertical line}%
  {minor (foot) group}%
  {\tbs textvertline}{}{\ipanew}{'222}

\ipaitem{\textdoublevertline}{Double vertical line}%
  {major (intonation) group}%
  {\tbs textdoublevertline}{}{\ipanew}{'223}

\ipaitem{\textbottomtiebar{  }}{Bottom tie bar}%
  {linking (absence of a break)}%
  {\tbs textbottomtiebar}{\tbs t*\TT{}}{\ipanew}{'074}
  \label{bottomtiebar}

\ipaitem{\textdownstep}{Down arrow\footnotemark}%
  {downstep}%
  {\tbs textdownstep}{}{\ipanew}{'224}%
  \footnotetext{The shapes of \tbs\texttt{textdownstep} and
    \tbs\texttt{textupstep} differ according to sources. Here I followed
    the shapes found in the recent IPA charts.}

\ipaitem{\textupstep}{Up arrow}%
  {upstep}%
  {\tbs textupstep}{}{\ipanew}{'225}

\ipaitem{\textglobfall}{Downward diagonal arrow}%
  {global fall}%
  {\tbs textglobfall}{}{\ipanew}{'226}

\ipaitem{\textglobrise}{Upward diagonal arrow}%
  {global rise}%
  {\tbs textglobrise}{}{\ipanew}{'227}

\ipxitem{\textspleftarrow}{Superscript left arrow }%
  {}%
  {\tbs textspleftarrow}{}{\PSG, p.~243}{'005}

\ipxitem{\textdownfullarrow}{Down full arrow }%
  {ingressive airflow}%
  {\tbs textdownfullarrow}{}{ExtIPA, \Handbook}{'007}

\ipxitem{\textupfullarrow}{Up full arrow }%
  {egressive airflow}%
  {\tbs textupfullarrow}{}{ExtIPA, \Handbook}{'010}

\ipxitem{\textsubrightarrow}{Subscript right arrow }%
  {sliding articulation}%
  {\tbs textsubrightarrow}{}{ExtIPA}{'011}

\ipxitem{\textsubdoublearrow}{Subscript double arrow }%
  {labial spreading}%
  {\tbs textsubdoublearrow}{}{ExtIPA}{'012}


\subsection{Tone letters}

The tones illustrated here are only a representative sample of what is
possible.  For more details see section~\ref{sec:tone}.

\bigskip

\ipaitem{\tone{55}}{Extra high tone}%
  {}%
  {\TD{tone}{55}}{}{\ipanew}{Macro}

\ipaitem{\tone{44}}{High tone}%
  {}%
  {\TD{tone}{44}}{}{\ipanew}{Macro}

\ipaitem{\tone{33}}{Mid tone}%
  {}%
  {\TD{tone}{33}}{}{\ipanew}{Macro}

\ipaitem{\tone{22}}{Low tone}%
  {}%
  {\TD{tone}{22}}{}{\ipanew}{Macro}

\ipaitem{\tone{11}}{Extra low tone}%
  {}%
  {\TD{tone}{11}}{}{\ipanew}{Macro}

\ipaitem{\tone{51}}{Falling tone}%
  {}%
  {\TD{tone}{51}}{}{\ipanew}{Macro}

\ipaitem{\tone{15}}{Rising tone}%
  {}%
  {\TD{tone}{15}}{}{\ipanew}{Macro}

\ipaitem{\tone{45}}{High rising tone}%
  {}%
  {\TD{tone}{45}}{}{\ipanew}{Macro}

\ipaitem{\tone{12}}{Low rising tone}%
  {}%
  {\TD{tone}{12}}{}{\ipanew}{Macro}

\ipaitem{\tone{454}}{High rising falling tone}%
  {}%
  {\TD{tone}{454}}{}{\ipanew}{Macro}


\subsection{Diacritical Tone Marks}

Some symbols included in the next section are also used as diacritical 
tone marks.

\bigskip

\ipaitem{\texthighrise{a}}{Macron plus acute accent}%
  {high rising tone}%
  {\TD{texthighrise}{a}}{}{\ipanew}{'230}

\ipaitem{\textlowrise{a}}{Grave accent plus macron}%
  {low rising tone}%
  {\TD{textlowrise}{a}}{}{\ipanew}{'231}

\ipaitem{\textrisefall{a}}{Grave plus acute plus grave accent}%
  {rising-falling tone}%
  {\TD{textrisefall}{a}}{}{\ipanew}{'232}

\ipaitem{\textfallrise{a}}{Acute plus grave plus acute accent}%
  {falling-rising tone}%
  {\TD{textfallrise}{a}}{}{}{'233}


\section{Accents and Diacritics}\label{list:diacritics}

\ipaitem{\`e}{Grave accent}%
  {low tone}%
  {\tbs`e}{}{\ipaall}{'000}

\ipaitem{\'e}{Acute accent}%
  {high tone}%
  {\tbs'e}{}{\ipaall}{'001}

\ipaitem{\^e}{Circumflex accent}%
  {falling tone}%
  {\tbs\tcircum e}{}{\ipaall}{'002}

\ipaitem{\~e}{Tilde}%
  {nasalized}%
  {\tbs\ttilde e}{}{\ipaall}{'003}

\ipaitem{\"e}{Umlaut}%
  {centralized}%
  {\tbs"e}{}{\ipaall}{'004}

\ipaitem{\H{e}}{Double acute accent}%
  {extra high tone}%
  {\TD{H}{e}}{}{\ipanew}{'005}

\ipaitem{\r{e}}{Ring}%
  {}%
  {\TD{r}{e}}{}{}{'006}

\ipaitem{\v{e}}{Wedge}%
  {rising tone}%
  {\TD{v}{e}}{}{\ipaall}{'007}

\ipaitem{\u{e}}{Breve}%
  {extra short}%
  {\TD{u}{e}}{}{\ipaall}{'010}

\ipaitem{\=e}{Macron}%
  {mid tone}%
  {\tbs=e}{}{}{'011}

\ipaitem{\.e}{Dot}%
  {}%
  {\tbs.e}{}{}{'012}

\ipaitem{\c{e}}{Cedilla}%
  {}%
  {\TD{c}{e}}{}{}{'013}

\ipaitem{\textpolhook{e}}{Polish hook (Ogonek accent)}%
  {}%
  {\TD{textpolhook}{e}}{\TD{k}{e}}{}{'014}

\ipxitem{\textrevpolhook{o}}{Reversed Polish hook }%
  {}%
  {\TD{textrevpolhook}{o}}{}{\PSG, p.~129}{'000}

\ipaitem{\textdoublegrave{e}}{Double grave accent}%
  {extra low tone}%
  {\TD{textdoublegrave}{e}}{\tbs H*e}{\ipanew}{'015}

\ipaitem{\textsubgrave{e}}{Subscript grave accent}%
  {low falling tone}%
  {\TD{textsubgrave}{e}}{\tbs`*e}{\ipaold}{'016}

\ipaitem{\textsubacute{e}}{Subscript acute accent}%
  {low rising tone}%
  {\TD{textsubacute}{e}}{\tbs'*e}{\ipaold}{'017}

\ipaitem{\textsubcircum{e}}{Subscript circumflex accent}%
  {}%
  {\TD{textsubcircum}{e}}{\tbs\tcircum*e}{}{Macro}

\ipaitem{\textroundcap{g}}{Round cap}%
  {}%
  {\TD{textroundcap}{g}}{\tbs|c\TT{g}}{}{'020}

\ipaitem{\textacutemacron{a}}{Acute accent with macron}%
  {}%
  {\tbs textacutemacron\TT{a}}{\tbs'=a}{}{Macro}

\ipaitem{\textgravemacron{a}}{Grave accent with macron}%
  {}%
  {\tbs textgravemacron\TT{a}}{}{}{Macro}

\ipaitem{\textvbaraccent{a}}{Vertical bar accent}%
  {}%
  {\tbs textvbaraccent\TT{a}}{}{}{'234}

\ipaitem{\textdoublevbaraccent{a}}{Double vertical bar accent}%
  {}%
  {\tbs textdoublevbaraccent\TT{a}}{}{}{'235}

\ipaitem{\textgravedot{e}}{Grave dot accent}%
  {}%
  {\TD{textgravedot}{e}}{\tbs`.e}{}{'236}

\ipaitem{\textdotacute{e}}{Dot acute accent}%
  {}%
  {\TD{textdotacute}{e}}{\tbs'.e}{}{'237}

\ipaitem{\textcircumdot{a}}{Circumflex dot accent}%
  {}%
  {\TD{textcircumdot}{a}}{\tbs\tcircum.a}{}{Macro}

\ipaitem{\texttildedot{a}}{Tilde dot accent}%
  {}%
  {\TD{texttildedot}{a}}{\tbs \ttilde.a}{}{Macro}

\ipaitem{\textbrevemacron{a}}{Breve macron accent}%
  {}%
  {\TD{textbrevemacron}{a}}{\tbs u=a}{}{Macro}

\ipaitem{\textringmacron{a}}{Ring macron accent}%
  {}%
  {\TD{textringmacron}{a}}{\tbs r=a}{}{Macro}

\ipaitem{\textacutewedge{s}}{Acute wedge accent}%
  {}%
  {\TD{textacutewedge}{s}}{\tbs v's}{}{Macro}

\ipaitem{\textdotbreve{a}}{Dot breve accent}%
  {}%
  {\TD{textdotbreve}{a}}{}{}{Macro}

\ipaitem{\textsubbridge{t}}{Subscript bridge}%
  {dental}%
  {\TD{textsubbridge}{t}}{\tbs|[t}{\ipaall}{'021}

\ipaitem{\textinvsubbridge{d}}{Inverted subscript bridge}%
  {apical}%
  {\TD{textinvsubbridge}{d}}{\tbs|]t}{\ipanew}{'022}

\ipaitem{\textsubsquare{n}}{Subscript square}%
  {laminal}%
  {\TD{textsubsquare}{n}}{}{\ipanew}{'023}

\ipaitem{\textsubrhalfring{o}}{Subscript right half-ring\footnotemark}%
  {more rounded}%
  {\TD{textsubrhalfring}{o}}{\tbs|)o}{\ipaall}{'024}%
  \footnotetext{Diacritics {\tt\tbs textsubrhalfring} and
    {\tt\tbs textsublhalfring} can be placed after a symbol by inputting,
    for example, {\tt[e\tbs textsubrhalfring\tbi\tbii]}
    \textipa{[e\textsubrhalfring{}]}.}

\ipaitem{\textsublhalfring{o}}{Subscript left half-ring}%
  {less rounded}%
  {\TD{textsublhalfring}{o}}{\tbs|(o}{\ipaall}{'025}

\ipaitem{\textsubw{k}}{Subscript W}%
  {labialized}%
  {\TD{textsubw}{k}}{\tbs|w\TT{k}}{IPA '79}{'026}

\ipaitem{\textoverw{g}}{Over W}%
  {*labialized}%
  {\TD{textoverw}{g}}{}{}{'026}

\ipaitem{\textseagull{t}}{Subscript seagull}%
  {linguolabial}%
  {\TD{textseagull}{t}}{\tbs|m\TT{t}}{\ipanew}{'027}

\ipaitem{\textovercross{e}}{Over-cross}%
  {mid-centralized}%
  {\TD{textovercross}{e}}{\tbs|x\TT{e}}{\ipaall}{'030}

\ipaitem{\textsubplus{\textopeno}}{Subscript plus\footnotemark}%
  {advanced}%
  {\TD{textsubplus}{\tbs textopeno}}{\tbs|+O}{\ipaall}{'033}%
  \footnotetext{The diacritics such as
    {\tt\tbs textsubplus}, {\tt\tbs textraising}, {\tt\tbs textlowering}
    {\tt\tbs textadvancing} and {\tt\tbs textretracting}
    can be placed after a symbol  by inputting
    {\tt[e\tbs textsubplus\tbi\tbii]} \textipa{[e\textsubplus{}]},
    for example. }

\ipaitem{\textraising{\textepsilon}}{Raising sign}%
  {raised}%
  {\TD{textraising}{\tbs textepsilon}}{\tbs|'E}{\ipaall}{'034}

\ipaitem{\textlowering{e}}{Lowering sign}%
  {lowered}%
  {\TD{textlowering}{e}}{\tbs|`e}{\ipaall}{'035}

\ipaitem{\textadvancing{u}}{Advancing sign}%
  {advanced tongue root}%
  {\TD{textadvancing}{u}}{\tbs|<u}{\ipaall}{'036}

\ipaitem{\textretracting{\textschwa}}{Retracting sign}%
  {retracted tongue root}%
  {\TD{textretracting}{\tbs textschwa}}{\tbs|>@}{\ipaall}{'037}

\ipaitem{\textsubtilde{e}}{Subscript tilde}%
  {creaky voiced}%
  {\TD{textsubtilde}{e}}{\tbs\ttilde*e}{\ipanew}{'003}

\ipaitem{\textsubumlaut{e}}{Subscript umlaut}%
  {breathy voiced}%
  {\TD{textsubumlaut}{e}}{\tbs"*e}{IPA '79, '89, '93}{'004}

\ipaitem{\textsubring{u}}{Subscript ring}%
  {voiceless}%
  {\TD{textsubring}{u}}{\tbs r*u}{\ipaall}{'006}

\ipaitem{\textsubwedge{e}}{Subscript wedge}%
  {voiced}%
  {\TD{textsubwedge}{e}}{\tbs v*e}{\ipaall}{'007}

\ipaitem{\textsubbar{e}}{Subscript bar}%
  {retracted}%
  {\TD{textsubbar}{e}}{\tbs=*e}{\ipaall}{'011}

\ipaitem{\textsubdot{e}}{Subscript dot}%
  {*retroflex}%
  {\TD{textsubdot}{e}}{\tbs.*e}{}{'012}

\ipaitem{\textsubarch{e}}{Subscript arch}%
  {non-syllabic}%
  {\TD{textsubarch}{e}}{}{}{'020}

\ipaitem{\textsyllabic{m}}{Syllabicity mark}%
  {syllabic}%
  {\TD{textsyllabic}{m}}{\TD{s}{m}}{\ipaall}{'042}

\ipaitem{\textsuperimposetilde{t}}{Superimposed tilde}%
  {velarized or pharyngealized}%
  {\TD{textsuperimposetilde}%
  {t}}{\tbs|\ttilde\TT{t}}{\ipaall}{'046}

\ipaitem{t\textcorner}{Corner}%
  {no audible release}%
  {t\tbs textcorner}{}{\ipanew}{'136}

\ipaitem{t\textopencorner}{Open corner}%
  {*release/burst}%
  {t\tbs textopencorner}{}{}{'137}

\ipaitem{\textschwa\textrhoticity}{Rhoticity}%
  {rhoticity}%
  {\tbs textschwa\tbs textrhoticity}{}{\ipanew}{'176}

\ipaitem{b\textceltpal}{Celtic palatalization mark}%
  {*as in \emph{Irish} b\textceltpal an `woman'.}%
  {b\tbs textceltpal}{}{}{'040}

\ipaitem{k\textlptr}{Left pointer}%
  {}%
  {k\tbs textlptr}{}{}{'275}

\ipaitem{k\textrptr}{Right pointer}%
  {}%
  {k\tbs textrptr}{}{}{'276}

\ipxitem{p\textrectangle}{Rectangle\footnotemark}%
  {*equivalent to IPA \textcorner\ (Corner)}%
  {p\tbs textrectangle}{}{}{'004}%
  \footnotetext{This symbol is used among Japanese linguists as a
    diacritical symbol indicating no audible release (IPA \textcorner),
    because the symbol \textcorner{} is used to indicate pitch accent in
    Japanese.}

\ipxitem{\textretractingvar}{Retracting sign (a variety) }%
  {}%
  {\tbs textretractingvar}{}{IPA '49}{'006}

\ipaitem{\texttoptiebar{gb}}{Top tie bar}%
  {affricates and double articulations}%
  {\TD{texttoptiebar}{gb}}{\TD{t}{gb}}{}{'076}

\hspace*{2em}\emph{See} page~\pageref{bottomtiebar} for `Bottom tie bar'.
\par\bigskip

\ipaitem{'}{Apostrophe}%
  {ejective}%
  {'}{}{\ipaall}{'047}

\ipaitem{\textrevapostrophe}{Reversed apostrophe}%
  {(obsolete) week aspiration}%
  {\tbs textrevapostrophe}{}{\ipaold}{'134}

\ipaitem{.}{Period}%
  {syllable break as in [\textturnr{}i.\ae{}kt]}%
  {.}{}{\ipanew}{'056}

\ipaitem{\texthooktop}{Hooktop}%
  {}%
  {\tbs texthooktop}{}{}{'043}

\ipaitem{\textrthook}{Right hook}%
  {}%
  {\tbs textrthook}{}{}{'044}

\ipxitem{\textrthooklong}{Right hook (long) }%
  {}%
  {\tbs textrthooklong}{}{}{'001}

\ipaitem{\textpalhook}{Palatalization hook}%
  {}%
  {\tbs textpalhook}{}{}{'045}

\ipxitem{\textpalhooklong}{Palatalization hook (long) }%
  {}%
  {\tbs textpalhooklong}{}{}{'002}

\ipxitem{\textpalhookvar}{Palatalization hook (a variety) }%
  {}%
  {\tbs textpalhookvar}{}{}{'003}

\ipaitem{p\super{h}}{Superscript H}%
  {aspirated}%
  {p\TD{textsuperscript}{h}}{p\tbs super h}{\ipaall}{Macro}

\ipaitem{k\super{w}}{Superscript W}%
  {labialized}%
  {k\TD{textsuperscript}{w}}{k\tbs super w}{\ipaall}{Macro}

\ipaitem{t\super{j}}{Superscript J}%
  {palatalized}%
  {t\TD{textsuperscript}{j}}{t\tbs super j}{\ipaall}{Macro}

\ipaitem{t\super{\textgamma}}{Superscript gamma}%
  {velarized}%
  {t\TD{textsuperscript}{\tbs textgamma}}{t\tbs super G}{\ipanew}{Macro}

\ipaitem{d\super{\textrevglotstop}}{Superscript reversed glottal stop}%
  {pharyngealized}%
  {d\TD{textsuperscript}{\tbs textrevglotstop}}{d\tbs super Q}{\ipanew}{Macro}

\ipaitem{d\super{n}}{Superscript N}%
  {nasal release}%
  {d\TD{textsuperscript}{n}}{d\tbs super n}{\ipanew}{Macro}

\ipaitem{d\super{l}}{Superscript L}%
  {lateral release}%
  {d\TD{textsuperscript}{l}}{d\tbs super l}{\ipanew}{Macro}



\section{Diacritics for ExtIPA, VoQS}

In order to use diacritics listed in this section, it is necessary to
specify the option `\texttt{extra}' at the preamble (See the section
entitled ``Other options'' on section~\ref{sec:otheroptions}). Note
also that some of the diacritics are defined by using symbols from
fonts other than \tipa{} so that they may not look quite satisfactory
and/or may not be slanted (e.g. \verb|\whistle{s}| \whistle{s}).

\bigskip

\ipxitem{\spreadlips{s}}{Subscript double arrow}%
  {*labial spreading}%
  {\TD{spreadlips}{s}}{}{ExtIPA '94}{'011}

\ipaitem{\overbridge{v}}{Overbridge}%
  {*dentolabial}%
  {\TD{overbridge}{v}}{}{ExtIPA '94}{Macro}

\ipaitem{\bibridge{n}}{Bibridge}%
  {*interdental/bidental}%
  {\TD{bibridge}{n}}{}{ExtIPA '94}{Macro}

\ipaitem{\subdoublebar{t}}{Subscript double bar}%
  {*alveolar}%
  {\TD{subdoublebar}{t}}{}{ExtIPA '94}{Macro}

\ipaitem{\subdoublevert{f}}{Subscript double vertical line}%
  {*strong articulation}%
  {\TD{subdoublevert}{f}}{}{ExtIPA '94}{Macro}

\ipaitem{\subcorner{v}}{Subscript corner}%
  {*weak articulation}%
  {\TD{subcorner}{v}}{}{ExtIPA '94}{Macro}

\ipaitem{\whistle{s}}{Up arrow}%
  {*whistled articulation}%
  {\TD{whistle}{s}}{}{ExtIPA '94}{Macro}

\ipxitem{\sliding{Ts}}{Subscript right arrow}%
  {*sliding articulation}%
  {\TD{sliding}{\TD{textipa}{Ts}}}{}{ExtIPA '94}{'012}

\ipaitem{\crtilde{m}}{Crossed tilde}%
  {*denasal}%
  {\TD{crtilde}{m}}{}{ExtIPA '94}{Macro}

\ipaitem{\dottedtilde{a}}{Dotted tilde}%
  {*nasal escape}%
  {\TD{dottedtilde}{a}}{}{ExtIPA '94}{Macro}

\ipaitem{\doubletilde{s}}{Double tilde}%
  {*velopharyngeal friction}%
  {\TD{doubletilde}{s}}{}{ExtIPA '94}{Macro}

\ipaitem{\partvoiceless{n}}{Parenthesis plus ring}%
  {*partial voiceless}%
  {\TD{partvoiceless}{n}}{}{ExtIPA '94}{Macro}

\ipaitem{\inipartvoiceless{n}}{Parenthesis plus ring}%
  {*initial partial voiceless}%
  {\TD{inipartvoiceless}{n}}{}{ExtIPA '94}{Macro}

\ipaitem{\finpartvoiceless{n}}{Parenthesis plus ring}%
  {*final partial voiceless}%
  {\TD{finpartvoiceless}{n}}{}{ExtIPA '94}{Macro}

\ipaitem{\partvoice{s}}{Parenthesis plus subwedge}%
  {*partial voicing}%
  {\TD{partvoice}{s}}{}{ExtIPA '94}{Macro}

\ipaitem{\inipartvoice{s}}{Parenthesis plus subwedge}%
  {*initial partial voicing}%
  {\TD{inipartvoice}{s}}{}{ExtIPA '94}{Macro}

\ipaitem{\finpartvoice{s}}{Parenthesis plus subwedge}%
  {*final partial voicing}%
  {\TD{finpartvoice}{s}}{}{ExtIPA '94}{Macro}

\ipaitem{\sublptr{\*J}}{Subscript left pointer}%
  {*right offset jaw voice}%
  {\TD{sublptr}{J}}{}{VoQS '94}{'275}

\ipaitem{\subrptr{\*J}}{Subscript right pointer}%
  {*left offset jaw voice}%
  {\TD{subrptr}{J}}{}{VoQS '94}{'276}

\endgroup % end of \setlength\parindent{0pt}

\endgroup % end of \raggedbottom

\endinput

%%% Local Variables: 
%%% mode: latex
%%% TeX-master: "tipaman"
%%% End: 
