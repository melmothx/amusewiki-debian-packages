% \iffalse meta-comment
% ======================================================================
% scrkernel-notepaper.dtx
% Copyright (c) Markus Kohm, 2006-2021
%
% This file is part of the LaTeX2e KOMA-Script bundle.
%
% This work may be distributed and/or modified under the conditions of
% the LaTeX Project Public License, version 1.3c of the license.
% The latest version of this license is in
%   http://www.latex-project.org/lppl.txt
% and version 1.3c or later is part of all distributions of LaTeX 
% version 2005/12/01 or later and of this work.
%
% This work has the LPPL maintenance status "author-maintained".
%
% The Current Maintainer and author of this work is Markus Kohm.
%
% This work consists of all files listed in manifest.txt.
% ----------------------------------------------------------------------
% scrkernel-notepaper.dtx
% Copyright (c) Markus Kohm, 2006-2021
%
% Dieses Werk darf nach den Bedingungen der LaTeX Project Public Lizenz,
% Version 1.3c, verteilt und/oder veraendert werden.
% Die neuste Version dieser Lizenz ist
%   http://www.latex-project.org/lppl.txt
% und Version 1.3c ist Teil aller Verteilungen von LaTeX
% Version 2005/12/01 oder spaeter und dieses Werks.
%
% Dieses Werk hat den LPPL-Verwaltungs-Status "author-maintained"
% (allein durch den Autor verwaltet).
%
% Der Aktuelle Verwalter und Autor dieses Werkes ist Markus Kohm.
% 
% Dieses Werk besteht aus den in manifest.txt aufgefuehrten Dateien.
% ======================================================================
% \fi
%
% \CharacterTable
%  {Upper-case    \A\B\C\D\E\F\G\H\I\J\K\L\M\N\O\P\Q\R\S\T\U\V\W\X\Y\Z
%   Lower-case    \a\b\c\d\e\f\g\h\i\j\k\l\m\n\o\p\q\r\s\t\u\v\w\x\y\z
%   Digits        \0\1\2\3\4\5\6\7\8\9
%   Exclamation   \!     Double quote  \"     Hash (number) \#
%   Dollar        \$     Percent       \%     Ampersand     \&
%   Acute accent  \'     Left paren    \(     Right paren   \)
%   Asterisk      \*     Plus          \+     Comma         \,
%   Minus         \-     Point         \.     Solidus       \/
%   Colon         \:     Semicolon     \;     Less than     \<
%   Equals        \=     Greater than  \>     Question mark \?
%   Commercial at \@     Left bracket  \[     Backslash     \\
%   Right bracket \]     Circumflex    \^     Underscore    \_
%   Grave accent  \`     Left brace    \{     Vertical bar  \|
%   Right brace   \}     Tilde         \~}
%
% \iffalse
%%% From File: $Id: scrkernel-notepaper.dtx 3577 2021-03-05 10:17:58Z kohm $
%<option>%%%            (run: option)
%<body>%%%            (run: body)
%<*dtx>
% \fi
\ifx\ProvidesFile\undefined\def\ProvidesFile#1[#2]{}\fi
\begingroup
  \def\filedate$#1: #2-#3-#4 #5${\gdef\filedate{#2/#3/#4}}
  \filedate$Date: 2021-03-05 11:17:58 +0100 (Fri, 05 Mar 2021) $
  \def\filerevision$#1: #2 ${\gdef\filerevision{r#2}}
  \filerevision$Revision: 1767 $
  \edef\reserved@a{%
    \noexpand\endgroup
    \noexpand\ProvidesFile{scrkernel-notepaper.dtx}
                          [\filedate\space\filerevision\space
                           KOMA-Script source
                           (note-paper)]
  }%
\reserved@a
% \iffalse
\documentclass{scrdoc}
\usepackage[english,ngerman]{babel}
\CodelineIndex
\RecordChanges
\GetFileInfo{scrkernel-notepaper.dtx}
\title{\KOMAScript{} \partname\ \texttt{\filename}%
  \footnote{Dies ist Version \fileversion\ von Datei \texttt{\filename}.}}
\date{\filedate}
\author{Markus Kohm}

\begin{document}
  \maketitle
  \tableofcontents
  \DocInput{\filename}
\end{document}
%</dtx>
% \fi
%
% \selectlanguage{ngerman}
%
% \changes{v2.95}{2006/03/16}{%
%   erste Version aus der Aufteilung von \textsf{scrclass.dtx}}
%
% \section{Briefbogen und Briefpapier}
%
% Bei Briefen kommt dem Briefbogen, also dem ersten Blatt des Briefes eine
% besondere Bedeutung zu. Dieser ist bei \texttt{scrlttr2} durch diverse
% Optionen, Variablen, Pseudo-Längen und Anweisungen bestimmt. Die weiteren
% Briefseiten werden hier verinfacht als das Briefpapier bezeichnet. Natürlich
% ist genau genommen auch der Briefbogen Briefpapier, aber eben ein besonderes.
%
% \StopEventually{\PrintIndex\PrintChanges}
%
% \iffalse
%<*option>
% \fi
%
% \iffalse
%<*letter>
% \fi
%
% \subsection{Option für den Briefbogen}
%
% Gerade der Briefbogen ist durch eine ganze Anzahl von Optionen
% konfigurierbar. Daher wird hier weiter thematisch untergliedert.
%
%
% \subsubsection{Optionen für den Briefkopf}
%
% Die Gestaltung des Briefkopfes war bei der alten \texttt{scrlettr}
% Klasse eng begrenzt. Deshalb hatte ich für mich selbst das
% \texttt{briefkopf} Paket geschrieben und ohne Support
% veröffentlicht. Die Funktionalität ist nun direkt verfügbar. Dazu
% werden verschiedene Optionen benötigt.
%
% \begin{option}{firsthead}
% \changes{v2.97e}{2007/11/22}{Neu}%^^A
% Der Briefkopf kann auch einfach abgeschaltet werden.
% \begin{macro}{\if@firsthead}
% \changes{v2.8q}{2002/01/25}{Neuer Schalter (intern)}%^^A
% Gespeichert wird die Information in einem einfachen Schalter.
% Voreingestellt ist die Verwendung des Briefkopfes.
%    \begin{macrocode}
\KOMA@ifkey{firsthead}{@firsthead}
\KOMAExecuteOptions{firsthead=true}
%    \end{macrocode}
% \end{macro}
% \end{option}
%
% \begin{option}{fromalign}
% \changes{v2.8q}{2002/01/25}{Neu}%^^A
% \changes{v3.12}{2013/03/05}{Verwendung der Status-Signalisierung mit
%     \cs{FamilyKeyState}}%^^A
% \changes{v3.17}{2015/03/16}{interne Speicherung der Werte}%^^A
% Diese Option legt fest, ob der Absender linksbündig, mittig oder
% rechtsbündig gesetzt werden soll.
% \begin{macro}{\@alignfrom}
% \changes{v2.8q}{2002/01/25}{Neu}%^^A
% \changes{v2.97e}{2007/11/13}{neue Werte für Verwendung der
%     Absenderergänzung}%^^A
% Die Einstellung wird in einem nummerisch auszuwertenden Makro
% gespeichert. Dabei gelten folgende Werte:
% \begin{center}
%   \begin{tabular}{rl}
%     Wert & Ausrichtung \\\hline
%     0    & linksbündig \\
%     1    & zentriert \\
%     2    & rechtsbündig \\
%     3    & linksbündig in der Absenderergänzung ohne Kopf \\
%     4    & rechtbündig in der Absenderergänzung ohne Kopf \\
%     10000 & nicht antasten \\
%   \end{tabular}
% \end{center}
% Voreingestellt ist "`nicht antasten"'.
%    \begin{macrocode}
\newcommand*{\@alignfrom}{0}
\KOMA@key{fromalign}{%
  \KOMA@set@ncmdkey{fromalign}{@alignfrom}{%
    {left}{0},%
    {center}{1},{centered}{1},{middle}{1},%
    {right}{2},%
    {locationleft}{3},{leftlocation}{3},%
    {locationright}{4},{rightlocation}{4},{location}{4},%
    {no}{10000},{false}{10000},{off}{10000}%
  }{#1}%
  \ifx\FamilyKeyState\FamilyKeyStateProcessed
    \KOMA@kav@xreplacevalue{.%
%<class>      \KOMAClassFileName
%<package>      scrletter.\scr@pkgextension
    }{fromalign}{#1}%
    \ifcase\@alignfrom\or\or\or\@firstheadfalse\or\@firstheadfalse\fi
  \fi
}
\KOMA@kav@add{.%
%<class>  \KOMAClassFileName
%<package>  scrletter.\scr@pkgextension
}{fromalign}{left}
%    \end{macrocode}
% \end{macro}
% \end{option}
%
% \begin{option}{fromrule}
% \changes{v2.8q}{2002/01/25}{Neu}%^^A
% \changes{v3.17}{2015/03/16}{interne Speicherung der Werte}%^^A
% Es kann gewählt werden, ob im Absender nach dem Namen oder nach dem
% gesamten Absender eine Linie gesetzt werden soll.
% \begin{macro}{\@rulefrom}
% \changes{v2.8q}{2002/01/25}{Neu}%^^A
% Die Einstellung wird in einem nummerisch auszuwertenden Makro
% gespeichert. Dabei gelten folgende Werte:
% \begin{center}
%   \begin{tabular}{rl}
%     Wert & Absenderlinie \\\hline
%     0    & keine \\
%     1    & unter dem Name \\
%     2    & unter der Adresse \\
%   \end{tabular}
% \end{center}
% Voreingestellt ist keine Linie.
%    \begin{macrocode}
\KOMA@ncmdkey{fromrule}[below]{@rulefrom}{%
  {false}{0},{no}{0},{off}{0},{aftername}{1},%
  {below}{2},{afteraddress}{2},{true}{2},{yes}{2},{on}{2}%
}
\KOMA@kav@add{.%
%<class>  \KOMAClassFileName
%<package>  scrletter.\scr@pkgextension
}{fromrule}{false}
%    \end{macrocode}
% \end{macro}
% \end{option}
%
% \begin{option}{fromphone}
% \changes{v2.8q}{2002/01/25}{Neu}%^^A
% Ob die Telefonnummer im Absender gesetzt werden soll oder nicht,
% kann hier angegeben werden.
% \begin{macro}{\if@phone}
% \changes{v2.8q}{2002/01/25}{Neu}%^^A
% \begin{macro}{\@phonetrue}
% \changes{v2.8q}{2002/01/25}{Neu}%^^A
% \begin{macro}{\@phonefalse}
% \changes{v2.8q}{2002/01/25}{Neu}%^^A
% Gespeichert wird die Information in einem einfachen Schalter.
% Voreingestellt ist keine Telefonnummer im Absender.
%    \begin{macrocode}
\KOMA@ifkey{fromphone}{@phone}
%    \end{macrocode}
% \end{macro}
% \end{macro}
% \end{macro}
% \end{option}
%
% \begin{option}{frommobilephone}
% \changes{v3.12}{2013/05/29}{Neu}%^^A
% Ob die Handynummer im Absender gesetzt werden soll oder nicht,
% kann hier angegeben werden.
% \begin{macro}{\if@mobilephone}
% \changes{v3.12}{2013/05/29}{Neu}%^^A
% \begin{macro}{\@mobilephonetrue}
% \changes{v3.12}{2013/05/29}{Neu}%^^A
% \begin{macro}{\@mobilephonefalse}
% \changes{v3.12}{2013/05/29}{Neu}%^^A
% Gespeichert wird die Information in einem einfachen Schalter.
% Voreingestellt ist keine Handynummer im Absender.
%    \begin{macrocode}
\KOMA@ifkey{frommobilephone}{@mobilephone}
%    \end{macrocode}
% \end{macro}
% \end{macro}
% \end{macro}
% \end{option}
%
% \begin{option}{fromfax}
% \changes{v2.8q}{2002/01/25}{Neu}%^^A
% Ob die Faxnummer im Absender gesetzt werden soll oder nicht,
% kann hier angegeben werden.
% \begin{macro}{\if@fax}
% \changes{v2.8q}{2002/01/25}{Neu}%^^A
% \begin{macro}{\@faxtrue}
% \changes{v2.8q}{2002/01/25}{Neu}%^^A
% \begin{macro}{\@faxfalse}
% \changes{v2.8q}{2002/01/25}{Neu}%^^A
% Gespeichert wird die Information in einem einfachen Schalter.
% Voreingestellt ist keine Faxnummer im Absender.
%    \begin{macrocode}
\KOMA@ifkey{fromfax}{@fax}
%    \end{macrocode}
% \end{macro}
% \end{macro}
% \end{macro}
% \end{option}
%
% \begin{option}{fromemail}
% \changes{v2.8q}{2002/01/25}{Neu}%^^A
% Ob die E-Mail-Adresse im Absender gesetzt werden soll oder nicht,
% kann hier angegeben werden.
% \begin{macro}{\if@email}
% \changes{v2.8q}{2002/01/25}{Neu}%^^A
% \begin{macro}{\@emailtrue}
% \changes{v2.8q}{2002/01/25}{Neu}%^^A
% \begin{macro}{\@emailfalse}
% \changes{v2.8q}{2002/01/25}{Neu}%^^A
% Gespeichert wird die Information in einem einfachen Schalter.
% Voreingestellt ist keine E-Mail-Adresse im Absender.
%    \begin{macrocode}
\KOMA@ifkey{fromemail}{@email}
%    \end{macrocode}
% \end{macro}
% \end{macro}
% \end{macro}
% \end{option}
%
% \begin{option}{fromurl}
% \changes{v2.8q}{2002/01/25}{Neu}%^^A
% Ob die URL auf eine Homepage im Absender gesetzt werden soll oder
% nicht, kann hier angegeben werden.
% \begin{macro}{\if@www}
% \changes{v2.8q}{2002/01/25}{Neu}%^^A
% \begin{macro}{\@wwwtrue}
% \changes{v2.8q}{2002/01/25}{Neu}%^^A
% \begin{macro}{\@wwwfalse}
% \changes{v2.8q}{2002/01/25}{Neu}%^^A
% Gespeichert wird die Information in einem einfachen Schalter.
% Voreingestellt ist keine URL im Absender.
%    \begin{macrocode}
\KOMA@ifkey{fromurl}{@www}
%    \end{macrocode}
% \end{macro}
% \end{macro}
% \end{macro}
% \end{option}
%
% \begin{option}{fromlogo}
% \changes{v2.8q}{2002/01/25}{Neu}%^^A
% Ob ein Logo in den Briefkopf gesetzt werden soll oder
% nicht, kann hier angegeben werden.
% \begin{macro}{\if@logo}
% \changes{v2.8q}{2002/01/25}{Neu}%^^A
% \begin{macro}{\@logotrue}
% \changes{v2.8q}{2002/01/25}{Neu}%^^A
% \begin{macro}{\@logofalse}
% \changes{v2.8q}{2002/01/25}{Neu}%^^A
% Gespeichert wird die Information in einem einfachen Schalter.
% Voreingestellt ist kein Logo im Briefkopf.
%    \begin{macrocode}
\KOMA@ifkey{fromlogo}{@logo}
%    \end{macrocode}
% \end{macro}
% \end{macro}
% \end{macro}
% \end{option}
%
%
% \subsubsection{Optionen für Anschrift und Absenderergänzung}
%
% \begin{option}{addrfield}
% \changes{v2.8q}{2001/10/09}{Neu}%^^A
% \changes{v3.03}{2009/03/04}{neue Werte}%^^A
% \changes{v3.17}{2015/03/16}{interne Speicherung der Werte}%^^A
% Das Adressfeld für Fensterumschläge kann mit \texttt{addrfield} ein- und
% ausgeschaltet werden. Darüber hinaus kann die Art des Adressfeldes gewählt
% werden.
% \begin{macro}{\if@afield}
% \changes{v3.03}{2009/03/04}{ersetzt durch \cs{@afieldmode}}
% \begin{macro}{\@afieldtrue}
% \changes{v3.03}{2009/03/04}{ersetzt durch \cs{@afieldmode}}
% \begin{macro}{\@afieldfalse}
% \changes{v3.03}{2009/03/04}{ersetzt durch \cs{@afieldmode}}
% \begin{macro}{\@afieldmode}
% \changes{v3.03}{2009/03/04}{neu (intern)}%^^A
% Die Speicherung erfolgt wiederum in einem Makro.
%    \begin{macrocode}
%<*class>
\newcommand*{\if@afield}{%
  \ClassWarning{\KOMAClassName}{%
    Usage of \string\if@afield\space deprecated.\MessageBreak
    Internal macro \string\if@afield\space was removed at\MessageBreak
    KOMA-Script version 3.03.  I'll replace the usage of\MessageBreak
    \string\if@afield\space by \string\ifnum
    \string\@afieldmode=\string\@ne.\MessageBreak
    This may result in errors.  A better solution would be\MessageBreak
    to change the code}%
  \ifnum \@afieldmode=\@ne}
\newcommand*{\@afieldtrue}{%
  \ClassWarning{\KOMAClassName}{%
    Usage of \string\if@afield\space deprecated.\MessageBreak
    Internal macro \string\if@afield\space was removed at\MessageBreak
    KOMA-Script version 3.03.  I'll replace the usage of\MessageBreak
    \string\@afieldtrue\space by
    \string\KOMAoptions{addrfield=true}.\MessageBreak
    This may result in errors.  A better solution would be\MessageBreak
    to change the code}%
  \KOMAoptions{addrfield=true}}
\newcommand*{\@afieldfalse}{%
  \ClassWarning{\KOMAClassName}{%
    Usage of \string\if@afield\space deprecated.\MessageBreak
    Internal macro \string\if@afield\space was removed at\MessageBreak
    KOMA-Script version 3.03.  I'll replace the usage of\MessageBreak
    \string\@afieldfalse\space by
    \string\KOMAoptions{addrfield=false}.\MessageBreak
    This may result in errors.  A better solution would be\MessageBreak
    to change the code}%
  \KOMAoptions{addrfield=false}}
%</class>
\newcommand*{\@afieldmode}{\@ne}
\KOMA@ncmdkey{addrfield}[true]{@afieldmode}{%
  {false}{0},{off}{0},{no}{0},%
  {true}{1},{on}{1},{yes}{1},%
  {PP}{2},{pp}{2},%
  {PPexplicite}{2},{PPExplicite}{2},%
  {ppexplicite}{2},{ppExplicite}{2},%
  {backgroundimage}{3},%
  {PPbackgroundimage}{3},{PPBackgroundImage}{3},{PPBackGroundImage}{3},%
  {ppbackgroundimage}{3},{ppBackgroundImage}{3},{ppBackGroundImage}{3},%
  {image}{4},{Image}{4},%
  {PPimage}{4},{PPImage}{4},%
  {ppimage}{4},{ppImage}{4},%
  {topaligned}{5},{alignedtop}{5}%
}
\KOMA@kav@add{.%
%<class>  \KOMAClassFileName
%<package>  scrletter.\scr@pkgextension
}{addrfield}{true}
%    \end{macrocode}
% \end{macro}
% \end{macro}
% \end{macro}
% \end{macro}
% \end{option}
%
% \begin{option}{priority}
% \changes{v3.03}{2009/03/04}{neu}%^^A
% \changes{v3.17}{2015/03/16}{interne Speicherung der Werte}%^^A
% Im internationalen Briefverkehr sind zwei Prioritätseinstellungen
% bekannt. Dafür gibt es dann so etwas wie eine Versandart nicht mehr.
% \begin{macro}{\ltr@priority}
% \changes{v3.03}{2009/03/04}{neu (intern)}%^^A
% Abgelegt wird die Art der Priorität in einem Macro.
%    \begin{macrocode}
\KOMA@ncmdkey{priority}[A]{ltr@priority}{%
  {no}{0},{off}{0},{false}{0},{manual}{1},%
  {B}{1},{b}{1},%
  {economy}{1},{Economy}{1},{ECONOMY}{1},%
  {B-ECONOMY}{1},{B-Economy}{1},{b-economy}{1},%
  {A}{2},{a}{2},%
  {priority}{2},{Priority}{2},{PRIORITY}{2},%
  {A-PRIORITY}{2},{A-Priority}{2},{a-priority}{2}%
}%
\KOMA@kav@add{.%
%<class>  \KOMAClassFileName
%<package>  scrletter.\scr@pkgextension
}{priority}{false}
%    \end{macrocode}
% \end{macro}
% \end{option}
%
% \begin{option}{backaddress}
% \changes{v2.8q}{2002/01/22}{Neu}%^^A
% \changes{v3.12}{2013/03/05}{Verwendung der Status-Signalisierung mit
%     \cs{FamilyKeyState}}%^^A
% \changes{v3.17}{3015/03/16}{interne Speicherung der Werte}%^^A
% Im Adressfeld kann die Rücksendeadresse gesetzt werden. Dies geschieht nun
% nicht mehr allein durch Definition einer Rücksendeadresse, sondern auch
% durch eine Option.
% \begin{macro}{\if@baddr}
% \begin{macro}{\@baddrtrue}
% \begin{macro}{\@baddrfalse}
% \begin{macro}{\backaddr@format}
% \changes{v2.96}{2006/07/31}{neu (intern)}%^^A
% Dieses Makro hat genau ein Argument und bestimmt die Form der
% Rücksendeadresse. Voreingestellt ist Unterstreichung. Dies entspricht der
% vor Version~2.96 festen Voreinstellung.
%    \begin{macrocode}
\newcommand{\backaddr@format}[1]{}
\let\backaddr@format\underline
%    \end{macrocode}
% \end{macro}
% Die Einstellung true/false der Option wird in einem Schalter gespeichert.
%    \begin{macrocode}
\newif\if@baddr\@baddrtrue
\KOMA@key{backaddress}[true]{%
  \begingroup
    \KOMA@set@ncmdkey{backaddress}{@tempa}{%
      {false}{0},{off}{0},{no}{0},%
      {true}{1},{on}{1},{yes}{1},%
      {underlined}{2},%
      {plain}{3}%
    }{#1}%
    \ifx\FamilyKeyState\FamilyKeyStateProcessed
      \aftergroup\FamilyKeyStateProcessed
      \ifcase\@tempa
        \endgroup
        \@baddrfalse
      \or
        \endgroup
        \@baddrtrue
      \or
        \endgroup
        \@baddrtrue
        \let\backaddr@format\underline
      \or
        \endgroup
        \@baddrtrue
        \let\backaddr@format\@firstofone
      \else
        \endgroup
      \fi
      \if@baddr
        \ifx\backaddr@format\underline
          \KOMA@kav@replacevalue{.%
%<class>            \KOMAClassFileName
%<package>            scrletter.\scr@pkgextension
          }{backaddress}{underlined}%
        \else
          \ifx\backaddr@format\@firstofone
            \KOMA@kav@replacevalue{.%
%<class>              \KOMAClassFileName
%<package>              scrletter.\scr@pkgextension
            }{backaddress}{plain}%
          \else
            \KOMA@kav@replacevalue{.%
%<class>              \KOMAClassFileName
%<package>              scrletter.\scr@pkgextension
            }{backaddress}{true}%
          \fi
        \fi
      \else
        \KOMA@kav@replacevalue{.%
%<class>          \KOMAClassFileName
%<package>          scrletter.\scr@pkgextension
        }{backaddress}{false}%
      \fi
  \else
      \endgroup
      \FamilyKeyStateUnknownValue
  \fi
}
%    \end{macrocode}
% \end{macro}
% \end{macro}
% \end{macro}
% \end{option}
%
% \begin{option}{locfield}
% \changes{v2.8q}{2001/10/09}{Neu}%^^A
% \changes{v2.97c}{2007/03/30}{vertauschte Bedeutung korrigiert}%^^A
% \changes{v3.12}{2013/03/05}{Verwendung der Status-Signalisierung mit
%     \cs{FamilyKeyState}}%^^A
% \changes{v3.17}{2015/03/16}{interne Speicherung der Werte}%^^A
% Mit Hilfe der Optione \texttt{locfield} kann der Platz neben dem Adressfeld
% groß oder klein gewählt werden.
% \begin{macro}{\if@bigloc}
% \begin{macro}{\@bigloctrue}
% \begin{macro}{\@biglocfalse}
% Die Einstellung wird in einem Schalter gespeichert.
%    \begin{macrocode}
\newif\if@bigloc
%    \end{macrocode}
% \end{macro}
% \end{macro}
% \end{macro}
%    \begin{macrocode}
\KOMA@key{locfield}[wide]{%
  \begingroup
    \KOMA@set@ncmdkey{locfield}{@tempa}{{wide}{0},{narrow}{1}}{#1}%
    \ifx\FamilyKeyState\FamilyKeyStateProcessed
      \ifcase\@tempa
        \endgroup
        \@bigloctrue
        \KOMA@kav@replacevalue{.%
%<class>          \KOMAClassFileName
%<package>          scrletter.\scr@pkgextension
        }{locfield}{wide}%
      \else
        \endgroup
        \@biglocfalse
        \KOMA@kav@replacevalue{.%
%<class>          \KOMAClassFileName
%<package>          scrletter.\scr@pkgextension
        }{locfield}{narrow}%
      \fi
    \else
      \endgroup
      \FamilyKeyStateUnknownValue
  \fi
}
\KOMA@kav@add{.%
%<class>  \KOMAClassFileName
%<package>  scrletter.\scr@pkgextension
}{locfield}{narrow}
%    \end{macrocode}
% \begin{option}{wlocfield}
% \changes{v3.01a}{2008/11/20}{deprecated}%^^A
% \begin{option}{slocfield}
% \changes{v3.01a}{2008/11/20}{deprecated}%^^A
%    \begin{macrocode}
\KOMA@DeclareDeprecatedOption{wlocfield}{locfield=wide}
\KOMA@DeclareDeprecatedOption{slocfield}{locfield=narrow}
%    \end{macrocode}
% \end{option}
% \end{option}
% \end{option}
%
%
% \subsubsection{Optionen für Faltmarken}
%
% \begin{option}{foldmarks}
% \changes{v2.8q}{2001/09/10}{Neu}%^^A
% \changes{v3.00}{2008/06/06}{es muss \cs{FamilyUnkownKeyVal} statt
%     \cs{KOMA@unkown@keyval} lokal umdefiniert werden}%^^A
% \changes{v3.12}{2013/03/05}{Verwendung der Status-Signalisierung mit
%     \cs{FamilyKeyState}}%^^A
% \changes{v3.17}{2015/03/16}{interne Speicherung der Werte}%^^A
% Die Faltmarken können mit \texttt{foldmarks} aktiviert und deaktiviert
% werden.
% \begin{macro}{\if@fold}
% \begin{macro}{\@foldtrue}
% \begin{macro}{\@foldfalse}
% Ob Faltmarken gesetzt werden sollen oder nicht, wird in einem
% Schalter gespeichert.
%    \begin{macrocode}
\newif\if@fold\@foldtrue
%    \end{macrocode}
% \begin{macro}{\@usefolds}
% Welche Faltmarken gesetzt werden sollen, wird in einem Makro gespeichert.
% \changes{v2.97e}{2007/11/08}{Neu (intern)}%^^A
%    \begin{macrocode}
\newcommand*{\@usefolds}{TMBPL}
\KOMA@key{foldmarks}[true]{%
  \begingroup
    \KOMA@set@ifkey\@empty{@fold}{#1}%
    \ifx\FamilyKeyState\FamilyKeyStateProcessed
      \if@fold\aftergroup\@foldtrue\else\aftergroup\@foldfalse\fi
      \aftergroup\FamilyKeyStateProcessed
    \else
      \aftergroup\FamilyKeyStateProcessed
      \def\sc@nf@ldch@rs##1{%
        \ifx ##1\@nil\else
          \if ##1T\g@addto@macro\@usefolds{##1}\else
            \if ##1M\g@addto@macro\@usefolds{##1}\else
              \if ##1B\g@addto@macro\@usefolds{##1}\else
                \if ##1P\g@addto@macro\@usefolds{##1}\else
                  \if ##1L\g@addto@macro\@usefolds{##1}\else
                    \if ##1V\g@addto@macro\@usefolds{L}\else
                      \if ##1H\g@addto@macro\@usefolds{TMBP}\else
                        \if ##1t\scr@removechar\@usefolds{T}\else
                          \if ##1m\scr@removechar\@usefolds{M}\else
                            \if ##1b\scr@removechar\@usefolds{B}\else
                              \if ##1p\scr@removechar\@usefolds{P}\else
                                \if ##1l\scr@removechar\@usefolds{L}\else
                                  \if ##1v%
                                    \scr@removechar\@usefolds{L}%
                                  \else
                                    \if ##1h%
                                      \scr@removechar\@usefolds{T}%
                                      \scr@removechar\@usefolds{M}%
                                      \scr@removechar\@usefolds{B}%
                                    \else
                                      \aftergroup\FamilyKeyStateUnknownValue
                                    \fi
                                  \fi
                                \fi
                              \fi
                            \fi
                          \fi
                        \fi
                      \fi
                    \fi
                  \fi
                \fi
              \fi
            \fi
          \fi
          \expandafter\sc@nf@ldch@rs
        \fi
      }%
      \expandafter\sc@nf@ldch@rs#1\@nil
    \fi
  \endgroup
  \ifx\FamilyKeyState\FamilyKeyStateProcessed
    \KOMA@kav@xreplacevalue{.%
%<class>      \KOMAClassFileName
%<package>      scrletter.\scr@pkgextension
    }{foldmarks}{\@usefolds}%
    \KOMA@kav@xadd{.%
%<class>      \KOMAClassFileName
%<package>      scrletter.\scr@pkgextension
    }{foldmarks}{\if@fold true\else false\fi}%
  \fi
}
\KOMA@kav@xadd{.%
%<class>  \KOMAClassFileName
%<package>  scrletter.\scr@pkgextension
}{foldmarks}{\@usefolds}%
\KOMA@kav@add{.%
%<class>  \KOMAClassFileName
%<package>  scrletter.\scr@pkgextension
}{foldmarks}{true}%
%    \end{macrocode}
% \begin{macro}{\scr@removechar}
% \changes{v2.97e}{2007/11/08}{Neu (intern)}%^^A
%    \begin{macrocode}
\newcommand*{\scr@removechar}[2]{%
  \def\in@@##1#2##2##3\in@@{%
    \ifx\in@##2\else
      \def\in@@####1#2\in@{\gdef#1{##1####1}}%
      \in@@##2##3%
    \fi}%
  \expandafter\in@@#1#2\in@\in@@
}
%    \end{macrocode}
% \end{macro}
% \end{macro}
% \end{macro}
% \end{macro}
% \end{macro}
% \end{option}
%
%
% \subsubsection{Optionen für die Geschäftszeile}
%
% \begin{option}{refline}
% \changes{v2.8n}{2001/09/08}{Neue Option}%^^A
% \changes{v2.96a}{2006/12/07}{vertauschte Bedeutung korrigiert}%^^A
% \changes{v3.12}{2013/03/05}{Verwendung der Status-Signalisierung mit
%     \cs{FamilyKeyState}}%^^A
% \changes{v3.17}{2015/03/16}{interne Speicherung der Werte}%^^A
% Kopf und Fuß der ersten Seite sind normalweise so breit, dass sie
% zur Adresse des Empfängers passen. Für die Geschäftszeile gibt es
% zwei Alternativen, eine breite und eine, deren Breite der Textbreite
% entspricht. Voreingestellt ist die Textbreite.
% \begin{macro}{\if@refwide}
% \changes{v2.8n}{2001/09/08}{Neuer Schalter}%^^A
% \changes{v3.09}{2011/03/07}{Neue Werte: \texttt{dateleft},
%     \texttt{dateright}, \texttt{nodate}}
% \begin{macro}{\@refwidetrue}
% \begin{macro}{\@refwidefalse}
% Gespeichert wird die Information in einem Schalter.
% \begin{macro}{\@datepos}
% \changes{v3.09}{2011/03/07}{Neu (intern)}%^^A
% Das Datum kann links oder rechts oder gar nicht gesetzt werden.
% Wo die Speicherung erfolgt wird in einem nummerischen Makro
% abglegt. Voreingestellt ist rechts.
% \begin{macro}{\if@refline}
% \selectlanguage{english}%
% \changes{v3.27}{2019/05/03}{new (internal) boolean}%^^A
% \changes{v3.27}{2019/05/03}{new option \texttt{refline!=false}}%^^A
% This boolean is used to switch of the whole reference
% line. \texttt{refline=false} is a tri-state switch to
% \texttt{refline=narrow} and \texttt{refline=wide}. But this is available
% only for the package.%
% \selectlanguage{ngerman}%
%    \begin{macrocode}
\newif\if@refwide
%<package>\newif\if@refline\@reflinetrue
\newcommand*{\@datepos}{1}
\KOMA@key{refline}[wide]{%
  \begingroup
    \KOMA@set@ncmdkey{refline}{@tempa}{{wide}{0},{narrow}{1},%
      {dateleft}{2},{leftdate}{2},%
      {dateright}{3},{rightdate}{3},%
      {nodate}{4}%
%<package>      ,{no}{5},{off}{5},{false}{5}%
    }{#1}%
    \ifx\FamilyKeyState\FamilyKeyStateProcessed
      \aftergroup\FamilyKeyStateProcessed
      \ifcase\@tempa% wide
        \endgroup
        \@refwidetrue
%<*package>
        \@reflinetrue
        \KOMA@kav@remove{.scrletter.\scr@pkgextension}{refline}{false}%
%</package>
        \KOMA@kav@remove{.%
%<class>          \KOMAClassFileName
%<package>          scrletter.\scr@pkgextension
        }{refline}{wide}%
        \KOMA@kav@remove{.%
%<class>          \KOMAClassFileName
%<package>          scrletter.\scr@pkgextension
        }{refline}{narrow}%
        \KOMA@kav@add{.%
%<class>          \KOMAClassFileName
%<package>          scrletter.\scr@pkgextension
        }{refline}{wide}%
      \or% narrow
        \endgroup
        \@refwidefalse
%<*package>
        \@reflinetrue
        \KOMA@kav@remove{.scrletter.\scr@pkgextension}{refline}{false}%
%</package>
        \KOMA@kav@remove{.%
%<class>          \KOMAClassFileName
%<package>          scrletter.\scr@pkgextension
        }{refline}{wide}%
        \KOMA@kav@remove{.%
%<class>          \KOMAClassFileName
%<package>          scrletter.\scr@pkgextension
        }{refline}{narrow}%
        \KOMA@kav@add{.%
%<class>          \KOMAClassFileName
%<package>          scrletter.\scr@pkgextension
        }{refline}{narrow}%
      \or% dateleft
        \endgroup
        \def\@datepos{2}%
        \KOMA@kav@remove{.%
%<class>          \KOMAClassFileName
%<package>          scrletter.\scr@pkgextension
        }{refline}{dateleft}%
        \KOMA@kav@remove{.%
%<class>          \KOMAClassFileName
%<package>          scrletter.\scr@pkgextension
        }{refline}{dateright}%
        \KOMA@kav@remove{.%
%<class>          \KOMAClassFileName
%<package>          scrletter.\scr@pkgextension
        }{refline}{nodate}%
        \KOMA@kav@add{.%
%<class>          \KOMAClassFileName
%<package>          scrletter.\scr@pkgextension
        }{refline}{dateleft}%
      \or% dateright
        \endgroup
        \def\@datepos{1}%
        \KOMA@kav@remove{.%
%<class>          \KOMAClassFileName
%<package>          scrletter.\scr@pkgextension
        }{refline}{dateleft}%
        \KOMA@kav@remove{.%
%<class>          \KOMAClassFileName
%<package>          scrletter.\scr@pkgextension
        }{refline}{dateright}%
        \KOMA@kav@remove{.%
%<class>          \KOMAClassFileName
%<package>          scrletter.\scr@pkgextension
        }{refline}{nodate}%
        \KOMA@kav@add{.%
%<class>          \KOMAClassFileName
%<package>          scrletter.\scr@pkgextension
        }{refline}{dateright}%
      \or% nodate
        \endgroup
        \def\@datepos{0}%
        \KOMA@kav@remove{.%
%<class>          \KOMAClassFileName
%<package>          scrletter.\scr@pkgextension
        }{refline}{dateleft}%
        \KOMA@kav@remove{.%
%<class>          \KOMAClassFileName
%<package>          scrletter.\scr@pkgextension
        }{refline}{dateright}%
        \KOMA@kav@remove{.%
%<class>          \KOMAClassFileName
%<package>          scrletter.\scr@pkgextension
        }{refline}{nodate}%
        \KOMA@kav@add{.%
%<class>          \KOMAClassFileName
%<package>          scrletter.\scr@pkgextension
        }{refline}{nodate}%
%<*package>
      \or% false
        \endgroup
        \@reflinefalse
        \KOMA@kav@remove{.scrletter.\scr@pkgextension}{refline}{wide}%
        \KOMA@kav@remove{.scrletter.\scr@pkgextension}{reline}{narrow}%
        \KOMA@kav@add{.scrletter.\scr@pkgextension}{refline}{false}%
%</package>
      \fi
    \else
      \endgroup
      \FamilyKeyStateUnknownValue
  \fi
}
\KOMA@kav@add{.%
%<class>  \KOMAClassFileName
%<package>  scrletter.\scr@pkgextension
}{refline}{dateright}%
\KOMA@kav@add{.%
%<class>  \KOMAClassFileName
%<package>  scrletter.\scr@pkgextension
}{refline}{narrow}%
%    \end{macrocode}
% \end{macro}
% \end{macro}
% \end{macro}
% \end{macro}
% \end{macro}
% \end{option}
%
%
% \subsubsection{Optionen für den Betreff}
%
% \begin{option}{subject}
% \changes{v2.8q}{2001/10/09}{Neu}%^^A
% \changes{v3.12}{2013/03/05}{Verwendung der Status-Signalisierung mit
%     \cs{FamilyKeyState}}%^^A
% \changes{v3.17}{2015/03/16}{interne Speicherung der Werte}%^^A
% Ein Betreff-Titel kann mit \texttt{subject} ein- und abgeschaltet
% werden. In England und Frankreich scheint es teilweise üblich zu
% sein, den Betreff erst nach der Anrede und dafür zentriert zu
% setzen\footnote{Ich kann mir nur mühsahm eine Bemerkung dazu
%   verkneifen.}. 
% Deshalb kann alternativ zum Einschalten des Betreff-Titels hier auch
% dies aktiviert werden.
% \begin{macro}{\subject@format}
% \changes{v2.97c}{2007/07/06}{neu (intern)}%^^A
%   Das Format des Betreffs kann geändert werden.
%    \begin{macrocode}
\newcommand{\subject@format}[1]{#1}
%    \end{macrocode}
% \end{macro}
% \begin{macro}{\raggedsubject}
% \changes{v2.97c}{2007/07/06}{neu (intern)}%^^A
%    \begin{macrocode}
\newcommand*{\raggedsubject}{}
%    \end{macrocode}
% \end{macro}
% \begin{macro}{\if@subj}
% \begin{macro}{\@subjtrue}
% \begin{macro}{\@subjfalse}
% \begin{macro}{\if@subjafter}
% \begin{macro}{\@subjaftertrue}
% \begin{macro}{\@subjafterfalse}
% Die Speicherung erfolgt in zwei Schaltern.
%    \begin{macrocode}
\newif\if@subj
\newif\if@subjafter
\KOMA@key{subject}{%
  \begingroup
    \KOMA@set@ncmdkey{subject}{@tempa}{%
      {titled}{0},%
      {untitled}{1},%
      {afteropening}{2},%
      {beforeopening}{3},%
      {underlined}{4},%
      {left}{5},%
      {right}{6},%
      {centered}{7}%
    }{#1}%
    \ifx\FamilyKeyState\FamilyKeyStateProcessed
      \aftergroup\FamilyKeyStateProcessed
      \ifcase\@tempa
        \aftergroup\@subjtrue
      \or
        \aftergroup\@subjfalse
      \or
        \aftergroup\@subjaftertrue
      \or
        \aftergroup\@subjafterfalse
      \or
        \aftergroup\let\aftergroup\subject@format\aftergroup\underline
      \or
        \aftergroup\let\aftergroup\raggedsubject\aftergroup\raggedright
      \or
        \aftergroup\let\aftergroup\raggedsubject\aftergroup\raggedleft
      \or
        \aftergroup\let\aftergroup\raggedsubject\aftergroup\centering
      \fi
    \else
      \aftergroup\FamilyKeyStateUnknownValue
    \fi
  \endgroup
  \ifx\FamilyKeyState\FamilyKeyStateProcessed
    \KOMA@kav@xreplacevalue{.%
%<class>      \KOMAClassFileName
%<package>      scrletter.\scr@pkgextension
    }{subject}{\if@subj titled\else untitled\fi}%
    \KOMA@kav@xadd{.%
%<class>      \KOMAClassFileName
%<package>      scrletter.\scr@pkgextension
    }{subject}{\if@subjafter after\else before\fi opening}%
    \ifx\subject@format\underline
      \KOMA@kav@add{.%
%<class>        \KOMAClassFileName
%<package>        scrletter.\scr@pkgextension
      }{subject}{underlined}%
    \fi
    \ifx\raggedsubject\raggedright
      \KOMA@kav@add{.%
%<class>        \KOMAClassFileName
%<package>        scrletter.\scr@pkgextension
      }{subject}{left}%
    \else
      \ifx\raggedsubject\raggedleft
        \KOMA@kav@add{.%
%<class>          \KOMAClassFileName
%<package>          scrletter.\scr@pkgextension
        }{subject}{right}%
      \else
        \ifx\raggedsubject\centering
          \KOMA@kav@add{.%
%<class>            \KOMAClassFileName
%<package>            scrletter.\scr@pkgextension
          }{subject}{centered}%
        \fi
      \fi
    \fi
  \fi
}
\KOMA@kav@add{.%
%<class>  \KOMAClassFileName
%<package>  scrletter.\scr@pkgextension
}{subject}{untitled}%
\KOMA@kav@add{.%
%<class>  \KOMAClassFileName
%<package>  scrletter.\scr@pkgextension
}{subject}{beforeopening}%
%    \end{macrocode}
% \end{macro}
% \end{macro}
% \end{macro}
% \end{macro}
% \end{macro}
% \end{macro}
% \end{option}
%
%
% \subsubsection{Optionen für den Satzspiegel}
%
% \begin{option}{enlargefirstpage}
% \changes{v2.8o}{2001/09/18}{Neue Option}%^^A
% Normalerweise wird der Text des Briefboges entsprechend dem
% normalen Satzspiegel gesetzt. Ist allerdings der Fuß der ersten
% Seite zu hoch, dann wird der Satzspiegel vertikal entsprechend
% verkürzt. Mit der Option \texttt{firstpageenlarge} kann erzwungen
% werden, dass im umgekehrten Fall, der Satzspiegel vertikal
% verlängert wird.
% \begin{macro}{\if@enlargefp}
% \begin{macro}{\@enlargefptrue}
% \begin{macro}{\@enlargefpfalse}
% Gespeichert wird die Information über die Optionen in einem
% Schalter.
%    \begin{macrocode}
\KOMA@ifkey{enlargefirstpage}{@enlargefp}
%    \end{macrocode}
% \end{macro}
% \end{macro}
% \end{macro}
% \end{option}
%
%
% \subsubsection{Optionen für den Briefbogenfuß}
%
% \begin{option}{firstfoot}
% \changes{v2.97e}{2007/11/22}{Neu}%^^A
% Der Fuß des Briefbogens kann auch einfach abgeschaltet werden.
% \begin{macro}{\if@firstfoot}
% \changes{v2.8q}{2002/01/25}{Neuer Schalter (intern)}%^^A
% Gespeichert wird die Information in einem einfachen Schalter.
% Voreingestellt ist die Verwendung des Fußes.
%    \begin{macrocode}
\KOMA@ifkey{firstfoot}{@firstfoot}\@firstfoottrue
%    \end{macrocode}
% \end{macro}
% \end{option}
%
%
% \subsection{Optionen für den Kopf des weiteren Briefpapiers}
%
% Diese Optionen haben zwar nichts mit dem eigentlichen Briefbogen zu tun. Das
% Briefpapier schließt sich aber an, weshalb die Optionen hier zu finden sind.
%
% \begin{option}{pagenumber}
% \changes{v2.8q}{2002/01/31}{Neu}%^^A
% \changes{v3.12}{2013/03/05}{Verwendung der Status-Signalisierung mit
%     \cs{FamilyKeyState}}
% Bei Briefen kann gewählt werden, ob auf Folgeseiten die Seitenzahl
% im Kopf oder im Fuß stehen soll. Dabei kann zwischen Seitenzahl
% rechts, links oder zentriert gewählt werden. Voreingestellt ist der
% Fuß und mittig.
% \begin{macro}{\@pageat}
% \changes{v2.8q}{2002/01/31}{Neu}%^^A
% In dieser nummerisch auswertbaren Variablen wird gespeichert, wo die
% Seitenzahl stehen soll. Dabei gilt:
% \begin{center}\begin{tabular}{rl}
%   Wert & Bedeutung für die Seitenzahl\\\hline
%   -1   & keine \\
%   0    & im Kopf links \\
%   1    & im Kopf mittig \\
%   2    & im Kopf rechts \\
%   3    & im Fuß links \\
%   4    & im Fuß mittig \\
%   5    & im Fuß rechts \\
% \end{tabular}\end{center}
%    \begin{macrocode}
\newcommand*{\@pageat}{4}
\KOMA@key{pagenumber}{%
  \begingroup
    \KOMA@set@ncmdkey{pagenumber}{@tempa}{%
      {no}{-1},{off}{-1},{false}{-1},%
      {topleft}{0},{headleft}{0},%
      {topmiddle}{1},{topcenter}{1},{topcentered}{1},%
      {headmiddle}{1},{headcenter}{1},{headcentered}{1},%
      {topright}{2},{headright}{2},%
      {botleft}{3},{footleft}{3},%
      {botmiddle}{4},{botcenter}{4},{botcentered}{4},%
      {footmiddle}{4},{footcenter}{4},{footcentered}{4},%
      {botright}{5},{footright}{5},%
      {top}{10},{head}{10},{bot}{11},{foot}{11},%
      {left}{20},{middle}{21},{center}{21},{centered}{21},{right}{22}%
    }{#1}%
    \ifx\FamilyKeyState\FamilyKeyStateProcessed
      \aftergroup\FamilyKeyStateProcessed
      \ifnum\@tempa<10
        \edef\@tempa{\endgroup\edef\noexpand\@pageat{\@tempa}}%
      \else
        \ifnum\@tempa<20
          \@tempcnta\@pageat\relax
          \ifnum\@tempcnta>2
            \advance\@tempcnta by -3
          \fi
          \ifnum\@tempa=11
            \advance\@tempcnta by 3
          \fi
          \edef\@tempa{\endgroup\edef\noexpand\@pageat{\the\@tempcnta}}%
        \else
          \ifnum\@pageat>2
            \edef\@pageat{3}%
          \else
            \edef\@pageat{0}%
          \fi
          \@tempcnta\@tempa
          \advance\@tempcnta by -20
          \advance\@tempcnta by\@pageat
          \edef\@tempa{\endgroup\edef\noexpand\@pageat{\the\@tempcnta}}%
        \fi
      \fi
    \else
      \def\@tempa{\endgroup\FamilyKeyStateUnknownValue}%
    \fi
  \@tempa
  \KOMA@kav@xreplacevalue{.%
%<class>    \KOMAClassFileName
%<package>    scrletter.\scr@pkgextension
  }{pagenumber}{%
    \ifcase \@pageat
      topleft%
    \or
      topmiddle%
    \or
      topright%
    \or
      botleft%
    \or
      botmiddle%
    \or
      botright%
    \else
      false%
    \fi
  }%
}
\KOMA@kav@add{.%
%<class>  \KOMAClassFileName
%<package>  scrletter.\scr@pkgextension
}{pagenumber}{botmiddle}
%    \end{macrocode}
% \end{macro}
% \end{option}
%
%
% \iffalse
%</letter>
%</option>
%<*body>
%<*letter>
% \fi
%
% \subsection{Anweisungen für den Briefbogen}
%
% Auch bei den Anweisungen zunächst alles, was mit dem Briefbogen zu tun hat.
% 
% \subsubsection{Der Kopf des Briefbogens}
%
% \begin{macro}{\firsthead}
% \changes{v2.8q}{2002/01/28}{\cs{@alignfrom} auf 10000 gesetzt}
% \changes{v3.08}{2010/10/28}{der Kopf des Briefbogens wird über eine
%   Variable realisiert}%^^A
% \begin{macro}{\@firsthead}
% \changes{v3.08}{2010/10/28}{der Kopf des Briefbogens wird über eine
%   Variable realisiert}%^^A
% \begin{macro}{\@gen@firsthead}
% \changes{v2.8q}{2002/01/28}{Neu}%^^A
% \changes{v2.97c}{2007/05/08}{Dicke der Linie konfigurierbar}%^^A
% \changes{v2.97c}{2007/05/08}{Schrift (Farbe) der Linie konfigurierbar}%^^A
% \changes{v3.06a}{2010/07/29}{Verwendung des Font-Elements
%   \texttt{fromrule} eingefügt}
% \changes{v3.08}{2010/10/28}{der Kopf des Briefbogens wird über eine
%   Variable realisiert}%^^A
% \begin{KOMAvar}{firsthead}
% \changes{v3.08}{2010/10/28}{Neu}%^^A
% \changes{v3.13b}{2014/10/04}{neuer \emph{hook} nur für
%   \texttt{DINmtext.lco}}%^^A
% \changes{v3.27}{2019/04/02}{\emph{hook} entfernt}%^^A
% \changes{v3.28}{2019/11/15}{\cs{ifkomavarempty} durch \cs{Ifkomavarempty}
%   ersetzt}%^^A
% Bei \texttt{scrlttr2} gibt es die üblichen Seitenstile, allerdings
% ist deren Verhalten etwas anders als gewohnt. Dies ist in den
% Besonderheiten für Briefen begründet. Zunächst die Voreinstellung für
% Briefbögen.
%
% TODO: Diese ganze Prozedur gehört (für \textsf{scrletter}) komplett
% überabeitet, so dass \cs{@gen@firsthead} überflüssig wird und das Setzen von
% \texttt{fromalign} auch gleich \texttt{firsthead} umdefiniert. 
%    \begin{macrocode}
%<*class>
\newcommand*{\@firsthead}{%
  \ClassWarning{scrlttr2}{%
    Deprecated usage of \string\@firsthead.\MessageBreak
    \string\usekomavar{firsthead} should be used\MessageBreak
    instead of \string\@firsthead
  }%
  \usekomavar{firsthead}%
}
%</class>
\newkomavar{firsthead}
\setkomavar{firsthead}{%
  \parbox[b]{\useplength{firstheadwidth}}%
  {\centering\usekomafont{fromaddress}{%
      \Ifkomavarempty{fromname}{}{%
        {\usekomafont{fromname}%
          {\ignorespaces\usekomavar{fromname}}\\[-8pt]}
        {\usekomafont{fromrule}{\rule{\useplength{firstheadwidth}}{.4pt}}}\\
      }%
      \Ifkomavarempty{fromaddress}{}{%
        \ignorespaces\usekomavar{fromaddress}%
      }%
    }%
  }%
}
%    \end{macrocode}
% \end{KOMAvar}
% \begin{macro}{\scr@firsthead@postsetvar}
% \changes{v3.08}{2010/10/28}{Neu (intern)}%^^A
% \changes{v3.17}{2015/03/16}{internes Setzen von Option
%     \texttt{fromalign!=no}}%^^A
% Über dieses Makro wird jeweils nach dem Setzen der Variable die Ausrichtung
% auf manuell umgeschaltet. Das darf deshalb auch erst nach dem erstmaligen
% Setzen definiert werden!
%    \begin{macrocode}
\newcommand*{\scr@firsthead@postsetvar}{%
  \KOMAExecuteOptions[.%
%<class>    \KOMAClassFileName
%<package>    scrletter.\scr@pkgextension
  ]{fromalign=no}%
}
%    \end{macrocode}
% \end{macro}
%    \begin{macrocode}
\newcommand*{\@gen@firsthead}{%
%    \end{macrocode}
% Damit wird ggf. die Kopfzeile der ersten Seite umdefiniert, nämlich dann,
% wenn dies per Option aktiviert wurde.
%    \begin{macrocode}
  \ifcase\@alignfrom
%    \end{macrocode}
% Wir befinden uns nun in dem Bereich für \verb|\@alignfrom|=0.
% Entsprechend unserer Festlegung wird daher als Erstes der
% linksbündige Absender definiert. Wir verwenden dafür eine
% \emph{parbox}, in der der Text linksbündig angeordnet wird. Diese
% Box soll über die gesamte Breite gehen dürfen:
%    \begin{macrocode}
    \setkomavar{firsthead}{%
      \parbox[b]{\textwidth}{\raggedright
%    \end{macrocode}
% Zuerst wird der Name gesetzt. Dabei soll eine eigene Schriftart für
% den Namen verwendet werden.
%    \begin{macrocode}
        {\usekomafont{fromname}\strut\ignorespaces
          \usekomavar{fromname}%
%    \end{macrocode}
% Hat \verb|\@rulefrom| den nummerischen Wert 1, so soll direkt auf den
% Absender eine Linie folgen. Die Länge dieser Linie kann vom Anwender
% bestimmt oder automatisch gewählt werden. Es werden dabei die
% Pseudo-Längen \texttt{fromrulewidth} und \texttt{fromrulethickness}
% verwendet. Ist \texttt{fromrulewidth} negativ, so wird sie automatisch
% bestimmt. Bei der automatischen Längenbestimmung wird dafür gesorgt, dass
% die Linie nicht bis in das Logo ragt, das eventuell rechtsbündig neben dem
% Absender steht:
%    \begin{macrocode}
          {%
            \ifnum\@rulefrom=1\\[-.5\baselineskip]%
              \usekomafont{fromrule}{%
                \setlength{\@tempdima}{\useplength{fromrulewidth}}%
                \ifdim\@tempdima=\z@
                  \setlength{\@tempdima}{\textwidth}%
                  \if@logo
                    \settowidth{\@tempdimb}{\usekomavar{fromlogo}}%
                    \addtolength{\@tempdima}{-\@tempdimb}%
                  \fi
                \fi
                \expandafter\rule\expandafter{\the\@tempdima}{%
                  \useplength{fromrulethickness}}}%
            \fi
          }%
          \\
        }%
%    \end{macrocode}
% Als Nächstes wird die Absenderadresse einschließlich aller
% optionalen Felder gesetzt. Dabei werden optionale Felder nur
% gesetzt, wenn der entsprechende Schalter \emph{true} ist. Die
% Adresse wird ebenfalls innerhalb einer Gruppe in einer eigenen
% Schriftart gesetzt.%
% \changes{v3.12}{2013/05/29}{Variable \texttt{frommobilephone} wird
%     berücksichtigt}%
%    \begin{macrocode}
        {\usekomafont{fromaddress}\strut\ignorespaces
          \usekomavar{fromaddress}%
          \if@phone\\\usekomavar*{fromphone}\usekomavar{fromphone}\fi%
          \if@mobilephone
            \\\usekomavar*{frommobilephone}\usekomavar{frommobilephone}%
          \fi%
          \if@fax\\\usekomavar*{fromfax}\usekomavar{fromfax}\fi%
          \if@email\\\usekomavar*{fromemail}\usekomavar{fromemail}\fi%
          \if@www\\\usekomavar*{fromurl}\usekomavar{fromurl}\fi%
%    \end{macrocode}
% Wie weiter oben erwähnt, kann auch an dieser Stelle eine Linie
% gesetzt  werden. Wir gehen davon aus, dass die Linie automatisch
% nicht in das Logo ragt, da es in diesem Fall über der Linie zu
% stehen hat. Die entsprechende Berechnung kann also entfallen.
%    \begin{macrocode}
          \ifnum\@rulefrom=2\\[-.5\baselineskip]%
            \usekomafont{fromrule}{%
              \setlength{\@tempdima}{\useplength{fromrulewidth}}%
              \ifdim\@tempdima=\z@
                \setlength{\@tempdima}{\textwidth}%
              \fi
              \expandafter\rule\expandafter{\the\@tempdima}{%
                \useplength{fromrulethickness}}}%
          \fi
          \\
        }%
      }%
%    \end{macrocode}
% Die parbox ist damit ebenfalls beendet. Es muss nun ggf. noch das
% Logo gesetz werden. Es wird in diesem Fall von der aktuellen
% Position am rechten Rand nach links ausgegeben.
%    \begin{macrocode}
      \if@logo
        \llap{\usekomavar{fromlogo}}%
      \fi
    }%
    \renewcommand*{\@alignfrom}{0}%
%    \end{macrocode}
% Damit ist die Definition des linksbündigen Absenders abgeschlossen.
% Es geht weiter mit dem zentrierten Absender. Dieser hat als
% Besonderheit, dass das Logo nicht in den Briefkopf, sondern in den
% Bereich rechts neben dem Empfänger gesetzt wird. Dort ist bereits
% das location-Feld, daher wird dieser Fall auch dort behandelt.%
% \changes{v3.12}{2013/05/29}{Variable \texttt{frommobilephone} wird
%     berücksichtigt}%
%    \begin{macrocode}
  \or
    \setkomavar{firsthead}{%
      \parbox[b]{\textwidth}{\centering
        {\usekomafont{fromname}\strut\ignorespaces
          \usekomavar{fromname}%
          {%
            \ifnum\@rulefrom=1\\[-.5\baselineskip]%
              \usekomafont{fromrule}{%
                \setlength{\@tempdima}{\useplength{fromrulewidth}}%
                \ifdim\@tempdima=\z@
                  \setlength{\@tempdima}{\textwidth}%
                \fi
                \expandafter\rule\expandafter{\the\@tempdima}{%
                  \useplength{fromrulethickness}}}%
            \fi
          }%
          \\
        }%
        {\usekomafont{fromaddress}\strut\ignorespaces
          \usekomavar{fromaddress}%
          \if@phone\\\usekomavar*{fromphone}\usekomavar{fromphone}\fi%
          \if@mobilephone
            \\\usekomavar*{frommobilephone}\usekomavar{frommobilephone}%
          \fi%
          \if@fax\\\usekomavar*{fromfax}\usekomavar{fromfax}\fi%
          \if@email\\\usekomavar*{fromemail}\usekomavar{fromemail}\fi%
          \if@www\\\usekomavar*{fromurl}\usekomavar{fromurl}\fi%
          \ifnum\@rulefrom=2\\[-.5\baselineskip]%
            \usekomafont{fromrule}{%
              \setlength{\@tempdima}{\useplength{fromrulewidth}}%
              \ifdim\@tempdima=\z@
                \setlength{\@tempdima}{\textwidth}%
              \fi
              \expandafter\rule\expandafter{\the\@tempdima}{%
                \useplength{fromrulethickness}}}%
          \fi
          \\
        }%
      }%
    }%
    \renewcommand*{\@alignfrom}{1}%
%    \end{macrocode}
% Damit ist die Definition des zentrierten Absenders abgeschlossen. Es
% geht weiter mit dem rechtsbündigen Absender. Das ganze wird einfach
% beim linksbündigen Absender abgeschaut.%
% \changes{v3.12}{2013/05/29}{Variable \texttt{frommobilephone} wird
%     berücksichtigt}%
%    \begin{macrocode}
  \or
    \setkomavar{firsthead}{%
      \if@logo
        \rlap{\usekomavar{fromlogo}}%
      \fi
      \parbox[b]{\textwidth}{\raggedleft
        {\usekomafont{fromname}\strut\ignorespaces
          \usekomavar{fromname}%
          {%
            \ifnum\@rulefrom=1\\[-.5\baselineskip]%
              \usekomafont{fromrule}{%
                \setlength{\@tempdima}{\useplength{fromrulewidth}}%
                \ifdim\@tempdima=\z@
                  \setlength{\@tempdima}{\textwidth}%
                  \if@logo
                    \settowidth{\@tempdimb}{\usekomavar{fromlogo}}%
                    \addtolength{\@tempdima}{-\@tempdimb}%
                  \fi
                \fi
                \expandafter\rule\expandafter{\the\@tempdima}{%
                  \useplength{fromrulethickness}}}%
            \fi
          }%
          \\
        }%
        {\usekomafont{fromaddress}\strut\ignorespaces
          \usekomavar{fromaddress}%
          \if@phone\\\usekomavar*{fromphone}\usekomavar{fromphone}\fi%
          \if@mobilephone
            \\\usekomavar*{frommobilephone}\usekomavar{frommobilephone}%
          \fi%
          \if@fax\\\usekomavar*{fromfax}\usekomavar{fromfax}\fi%
          \if@email\\\usekomavar*{fromemail}\usekomavar{fromemail}\fi%
          \if@www\\\usekomavar*{fromurl}\usekomavar{fromurl}\fi%
          \ifnum\@rulefrom=2\\[-.5\baselineskip]%
            \usekomafont{fromrule}{%
              \setlength{\@tempdima}{\useplength{fromrulewidth}}%
              \ifdim\@tempdima=\z@
                \setlength{\@tempdima}{\textwidth}%
              \fi
              \expandafter\rule\expandafter{\the\@tempdima}{%
                \useplength{fromrulethickness}}}%
          \fi
          \\
        }%
      }%
    }%
    \renewcommand*{\@alignfrom}{2}%
%    \end{macrocode}
% Damit ist die Umdefinierung beendet. In allen anderen Fällen wird
% der Kopf nicht verändert.
%    \begin{macrocode}
  \fi
}
%    \end{macrocode}
% \end{macro}
% \end{macro}
%
% Der Benutzer kann die Voreinstellung mit einer einfachen Anweisung ändern:
%    \begin{macrocode}
%<*class>
\newcommand{\firsthead}{%
  \ClassWarning{scrlttr2}{%
    Deprecated usage of \string\firsthead.\MessageBreak
    You should use \string\setkomavar{firsthead}\MessageBreak
    instead of \string\firsthead
  }
  \setkomavar{firsthead}%
}
%</class>
%    \end{macrocode}
% \end{macro}
%
% \subsubsection{Schrifteinstellungen für den Briefbogen}
%
% \begin{KOMAfont}{foldmark}
% \changes{v2.97c}{2007/05/08}{Neu}%^^A
% Die Schrift (hier ist die Farbe der entscheidende Faktor) der Faltmarken.
%    \begin{macrocode}
\newkomafont{foldmark}{}
%    \end{macrocode}
% \end{KOMAfont}
%
% \begin{KOMAfont}{fromname}
% \changes{v2.95}{2006/03/23}{jetzt mit \cs{newkomafont}}
% Der Absendername im Briefkopf.
%    \begin{macrocode}
\newkomafont{fromname}{}
%    \end{macrocode}
% \end{KOMAfont}
%
% \begin{KOMAfont}{fromaddress}
% \changes{v2.95}{2006/03/23}{jetzt mit \cs{newkomafont}}
% Die Absenderadresse im Briefkopf.
%    \begin{macrocode}
\newkomafont{fromaddress}{}
%    \end{macrocode}
% \end{KOMAfont}
%
% \begin{KOMAfont}{fromrule}
% \changes{v2.97c}{2007/05/08}{Neu}%^^A
% Die Schrift (hier ist die Farbe der entscheidende Faktor) der Linie im
% Absender.
%    \begin{macrocode}
\newkomafont{fromrule}{}
%    \end{macrocode}
% \end{KOMAfont}
%
% \begin{KOMAfont}{backaddress}
% \changes{v2.95}{2006/03/23}{jetzt mit \cs{newkomafont}}%^^A
% \changes{v3.21}{2016/05/19}{beachtet Option
%     \texttt{egregdoesnotlikesansseriftitles}}%^^A
% Die Rücksendeadresse.
%    \begin{macrocode}
\newkomafont{backaddress}{\@gr@gsffamily}
%    \end{macrocode}
% \end{KOMAfont}
%
% \begin{KOMAfont}{specialmail}
% \changes{v2.97c}{2007/07/06}{Neues Font-Element}%^^A
% Die Versandart.
%    \begin{macrocode}
\newkomafont{specialmail}{}
%    \end{macrocode}
% \end{KOMAfont}
%
% \begin{KOMAfont}{priority}
% \changes{v3.03}{2009/03/04}{Neues Font-Element}%^^A
% Grundschrift für die Priorität des Briefs, falls diese A oder B ist. Es wird
% ohne Durchschuss gearbeitet. An dieser Stelle ist Serifenlos quasi
% vorgeschrieben, so dass wir hier Option
% \texttt{egregdoesnotlikesansseriftitles} ignorieren.
%    \begin{macrocode}
\newkomafont{priority}{\fontsize{10pt}{10pt}\sffamily\bfseries}
%    \end{macrocode}
% \end{KOMAfont}
%
% \begin{KOMAfont}{prioritykey}
% \changes{v3.03}{2009/03/04}{Neues Font-Element}%^^A
% Schrift für den Buchstaben der Priorität des Briefs, falls diese A oder B
% ist. Wird nach \texttt{priority} angewandt. Eigentlich sollten hier 9\,mm
% verwendet werden. Dafür wird aber eine frei skalierbare Schrift benötig. Mit
% 24,88\,pt sind wir nur 0,25\,mm kleiner. Das sollte kein Problem
% darstellen. Es wird auch hier ohne Durchschuss gearbeitet.
%    \begin{macrocode}
\newkomafont{prioritykey}{\fontsize{24.88pt}{24.88pt}\selectfont}
%    \end{macrocode}
% \end{KOMAfont}
%
% \begin{KOMAfont}{PPlogo}
% \changes{v3.03}{2009/03/04}{Neues Font-Element}%^^A
% Schrift für das Logo \texttt{P.P.}, das im Anschriftmodus \texttt{PP}
% gesetzt wird. An dieser Stelle ist serifenlos quasi vorgeschrieben, so dass
% wir hier Option \texttt{egregdoesnotlikesansseriftitles} ignorieren.
%    \begin{macrocode}
\newkomafont{PPlogo}{\sffamily\bfseries}
%    \end{macrocode}
% \end{KOMAfont}
%
% \begin{KOMAfont}{PPdata}
% \changes{v3.03}{2009/03/04}{Neues Font-Element}%^^A
% Schrift, die innerhalb des PP-Kopfes für die Daten verwendet wird.
% gesetzt wird. And dieser Stelle ist serifenlos quasi vorgeschrieben, so dass
% wir hier Option \texttt{egregdoesnotlikesansseriftitles} ignorieren.
%    \begin{macrocode}
\newkomafont{PPdata}{\sffamily}
%    \end{macrocode}
% \end{KOMAfont}
%
% \begin{KOMAfont}{addressee}
% \changes{v2.97c}{2007/07/06}{Neues Font-Element}%^^A
%   Die komplette Anschrift.
%    \begin{macrocode}
\newkomafont{addressee}{}
%    \end{macrocode}
% \end{KOMAfont}
%
% \begin{KOMAfont}{toname}
% \changes{v2.97c}{2007/07/06}{Neues Font-Element}%^^A
%   Der Name des Empfängers in der Anschrift.
%    \begin{macrocode}
\newkomafont{toname}{}
%    \end{macrocode}
% \end{KOMAfont}
%
% \begin{KOMAfont}{toaddress}
% \changes{v2.97c}{2007/07/06}{Neues Font-Element}%^^A
%   Die Adresse des Empfängers in der Anschrift.
%    \begin{macrocode}
\newkomafont{toaddress}{}
%    \end{macrocode}
% \end{KOMAfont}
%
% \begin{KOMAfont}{refname}
% \changes{v2.97c}{2007/05/14}{Neues Font-Element}%^^A
% \changes{v3.21}{2016/05/19}{beachtet Option
%     \texttt{egregdoesnotlikesansseriftitles}}%^^A
% Die Feldnamen in der Geschäftszeile.
%    \begin{macrocode}
\newkomafont{refname}{\@gr@gsffamily\scriptsize}
%    \end{macrocode}
% \end{KOMAfont}
%
% \begin{KOMAfont}{refvalue}
% \changes{v2.97c}{2007/05/14}{Neues Font-Element}%^^A
% \changes{v2.97d}{2007/10/09}{Voreinstellung korrigiert}%^^A
% Die Feldwerte in der Geschäftszeile. Hier gibt es keinen Bedarf, Option
% \texttt{egregdoesnotlikesansseriftitles} zu beachten, da diese in
% Version~2.97 noch nicht existierte.
%    \begin{macrocode}
%<*class>
\ifnum\scr@compatibility=\@nameuse{scr@v@2.97}\relax
  \newkomafont{refvalue}{\sffamily\scriptsize}
\else
%</class>
  \newkomafont{refvalue}{}
%<class>\fi
%    \end{macrocode}
% \end{KOMAfont}
%
% \begin{KOMAfont}{placeanddate}
% \changes{v3.12}{2013/07/31}{Neues Font-Element}%^^A
% Wird an Stelle einer Geschäftszeile nur Ort und Datum gesetzt, so findet
% dieses Element Anwendung.
%    \begin{macrocode}
\newkomafont{placeanddate}{}
%    \end{macrocode}
% \end{KOMAfont}
%
% \begin{KOMAfont}{lettertitle}
% \changes{v3.17}{2015/02/21}{neu als Ersatz für \texttt{letter}}
% \changes{v3.21}{2016/05/19}{beachtet Option
%     \texttt{egregdoesnotlikesansseriftitles}}%^^A
% Der Brieftitel.
%    \begin{macrocode}
\newkomafont{lettertitle}{\normalcolor\@gr@gsffamily\bfseries}
%    \end{macrocode}
% \begin{KOMAfont}{title}
% \changes{v2.95}{2006/03/23}{jetzt mit \cs{newkomafont}}%^^A
% \changes{v3.17}{2015/02/21}{nur noch ein Alias}%^^A
% \changes{v3.17a}{2015/04/26}{Alias korrigiert}%^^A
%    \begin{macrocode}
%<package>\scr@ifundefinedorrelax{scr@fnt@title}{%
  \aliaskomafont{title}{lettertitle}%
%<package>}{}
%    \end{macrocode}
% \end{KOMAfont}
% \end{KOMAfont}
%
% \begin{KOMAfont}{lettersubject}
% \changes{v3.17}{2015/02/21}{neu als Ersatz für \texttt{subject}}
% \changes{v3.17a}{2015/04/26}{Alias korrigiert}%^^A
% Der Betreff im Brief.
%    \begin{macrocode}
\newkomafont{lettersubject}{\normalfont\bfseries}
%    \end{macrocode}
% \begin{KOMAfont}{subject}
% \changes{v2.95}{2006/03/23}{jetzt mit \cs{newkomafont}}%^^A
% \changes{v3.17}{2015/02/21}{nur noch ein Alias}%^^A
%    \begin{macrocode}
%<package>\scr@ifundefinedorrelax{scr@fnt@subject}{%
  \aliaskomafont{subject}{lettersubject}%
%<package>}{}
%    \end{macrocode}
% \end{KOMAfont}
% \end{KOMAfont}
%
% \subsubsection{Längen für den Briefbogen}
%
% \begin{macro}{\ltr@len@foldmarkhpos}
% \changes{v2.8q}{2001/10/06}{Neu}%^^A
% Abstand der horizontalen Faltmarken von der linken Papierkante.
%    \begin{macrocode}
\@newplength{foldmarkhpos}
%    \end{macrocode}
% \end{macro}
% \begin{macro}{\ltr@len@foldmarkvpos}
% \changes{v2.97e}{2007/11/10}{Neu}%^^A
% Abstand der vertikalen Faltmarken von der oberen Papierkante.
%    \begin{macrocode}
\@newplength{foldmarkvpos}
%    \end{macrocode}
% \end{macro}
% \begin{macro}{\ltr@len@tfoldmarkvpos}
% \changes{v2.8q}{2001/10/06}{Neu}%^^A
% \begin{macro}{\ltr@len@mfoldmarkvpos}
% \changes{v2.97e}{2007/11/08}{Neu}%^^A
% \begin{macro}{\ltr@len@bfoldmarkvpos}
% \changes{v2.8q}{2001/10/06}{Neu}%^^A
% Abstand der oberen, mittleren und unteren Faltmarken von der oberen
% Papierkante.
%    \begin{macrocode}
\@newplength{tfoldmarkvpos}
\@newplength{mfoldmarkvpos}
\@newplength{bfoldmarkvpos}
%    \end{macrocode}
% \end{macro}
% \end{macro}
% \end{macro}
% \begin{macro}{\ltr@len@lfoldmarkhpos}
% \changes{v2.97e}{2007/11/10}{Neu}%^^A
% Abstand linken Faltmarken von der linken Papierkante.
%    \begin{macrocode}
\@newplength{lfoldmarkhpos}
%    \end{macrocode}
% \end{macro}
% \begin{macro}{\ltr@len@pfoldmarklength}
% \changes{v2.97e}{2007/11/19}{Neu}%^^A
% \begin{macro}{\ltr@len@hfoldmarklength}
% \changes{v2.97e}{2007/11/10}{Neu}%^^A
% \begin{macro}{\ltr@len@mfoldmarklength}
% \changes{v2.97e}{2007/11/10}{Neu}%^^A
% \begin{macro}{\ltr@len@bfoldmarklength}
% \changes{v2.97e}{2001/11/10}{Neu}%^^A
% \begin{macro}{\ltr@len@lfoldmarklength}
% \changes{v2.97e}{2001/11/10}{Neu}%^^A
% Länge der Lochmarke und der oberen, mittleren und unteren horizontalen und
% der (linken) vertikalen Faltmarke.
%    \begin{macrocode}
\@newplength{pfoldmarklength}
\@newplength{tfoldmarklength}
\@newplength{mfoldmarklength}
\@newplength{bfoldmarklength}
\@newplength{lfoldmarklength}
%    \end{macrocode}
% \end{macro}
% \end{macro}
% \end{macro}
% \end{macro}
% \end{macro}
% \begin{macro}{\ltr@len@foldmarkthickness}
% \changes{v2.97c}{2007/05/08}{Neu}%^^A
% Dicke der Faltmarken.
%    \begin{macrocode}
\@newplength{foldmarkthickness}
\@setplength{foldmarkthickness}{.2pt}
%    \end{macrocode}
% \end{macro}
% \begin{macro}{\ltr@len@toaddrvpos}
% \changes{v2.8q}{2001/10/06}{Neu}%^^A
% Abstand des Adressfensters von der oberen Papierkante.
%    \begin{macrocode}
\@newplength{toaddrvpos}
%    \end{macrocode}
% \end{macro}
% \begin{macro}{\ltr@len@backaddrheight}
% \changes{v2.8q}{2001/10/06}{Neu}%^^A
% Höhe der Rücksendeadresse.
%    \begin{macrocode}
\@newplength{backaddrheight}
%    \end{macrocode}
% \end{macro}
% \begin{macro}{\ltr@len@toaddrhpos}
% \changes{v2.8q}{2001/10/06}{Neu}%^^A
% Abstand des Adressfeldes von der linken Papierkante bzw. wenn negativ von
% der rechten Papierkante.
%    \begin{macrocode}
\@newplength{toaddrhpos}
%    \end{macrocode}
% \end{macro}
% \begin{macro}{\ltr@len@specialmailindent}
% \changes{v2.8q}{2002/03/25}{Neu}%^^A
% Abstand der Versandtart vom linken Rand des Adressfensters.
%    \begin{macrocode}
\@newplength{specialmailindent}
%    \end{macrocode}
% \end{macro}
% \begin{macro}{\ltr@len@specialmailrightindent}
% \changes{v2.8q}{2002/03/25}{Neu}%^^A
% Abstand der Versandtart vom rechten Rand des Adressfensters.
%    \begin{macrocode}
\@newplength{specialmailrightindent}
%    \end{macrocode}
% \end{macro}
% \begin{macro}{\ltr@len@toaddrindent}
% \changes{v2.8q}{2001/10/06}{Neu}%^^A
% Abstand der Adresse vom linken Rand des Adressfensters. 0 bedeutet im
% Anschriftmodus \texttt{PP} 8\,mm, ansonsten wirklich 0\,mm. Außerdem wird im
% Anschriftmodus \texttt{PP} die Pseudolänge auch für den Abstand vom rechten
% Rand des Adressfensters verwendet.
%    \begin{macrocode}
\@newplength{toaddrindent}
%    \end{macrocode}
% \end{macro}
% \begin{macro}{\ltr@len@PPheadheight}
% \changes{v3.03}{2009/03/04}{Neu}%
% Höhe des PP-Kopfes in den Anschriftmodi \texttt{PP},
% \texttt{backgroundimage} und \texttt{image}. 0 bedeutet hier 20,74\,pt.
%    \begin{macrocode}
\@newplength{PPheadheight}
%    \end{macrocode}
% \end{macro}
% \begin{macro}{\ltr@len@PPheadwidth}
% \changes{v3.03}{2009/03/04}{Neu}%
% Breite des PP-Kopfes im Anschriftmodus \texttt{PP}. 0 bedeutet hier 42\,mm.
%    \begin{macrocode}
\@newplength{PPheadwidth}
%    \end{macrocode}
% \end{macro}
% \begin{macro}{\ltr@len@PPdatamatrixvskip}
% \changes{v3.03}{2009/03/04}{Neu}%
% Abstand zwischen PP-Kopf und DataMatrix im Anschriftmodus \texttt{PP}. 0
% bedeutet hier 9\,mm.
%    \begin{macrocode}
\@newplength{PPdatamatrixvskip}
%    \end{macrocode}
% \end{macro}
% \begin{macro}{\ltr@len@toaddrwidth}
% \changes{v2.8q}{2001/10/06}{Neu}%^^A
% Breite des Adressfensters
%    \begin{macrocode}
\@newplength{toaddrwidth}
%    \end{macrocode}
% \end{macro}
% \begin{macro}{\ltr@len@toaddrheight}
% \changes{v2.8q}{2001/10/06}{Neu}%^^A
% \changes{v3.03b}{2009/06/25}{Implementierung an Doku angepasst}%^^A
% Höhe des Adressfensters mit Rücksendeadresse
%    \begin{macrocode}
\@newplength{toaddrheight}
%    \end{macrocode}
% \end{macro}
% \begin{macro}{\ltr@len@locvpos}
% \changes{v2.97e}{2007/11/07}{Neu}%^^A
% Abstand der Absenderergänzung von der oberen Papierkante.
%    \begin{macrocode}
\@newplength{locvpos}
%    \end{macrocode}
% \end{macro}
% \begin{macro}{\ltr@len@locwidth}
% \changes{v2.8q}{2001/10/06}{Neu}%^^A
% Breite des Feldes neben dem Adressfenster
%    \begin{macrocode}
\@newplength{locwidth}
%    \end{macrocode}
% \end{macro}
% \begin{macro}{\ltr@len@lochpos}
% \changes{v2.97e}{2007/11/07}{Neu}%^^A
% Abstand der Absenderergänzung von der linken Papierkante bzw. wenn negativ
% von der rechten Papierkante.
%    \begin{macrocode}
\@newplength{lochpos}
%    \end{macrocode}
% \end{macro}
% \begin{macro}{\ltr@len@locheight}
% \changes{v2.97e}{2007/11/07}{Neu}%^^A
% Höhe der Absenderergänzung.
%    \begin{macrocode}
\@newplength{locheight}
%    \end{macrocode}
% \end{macro}
% \begin{macro}{\ltr@len@refvpos}
% \changes{v2.8q}{2001/10/06}{Neu}%^^A
% Abstand der Geschäftszeile von der Papieroberkante.
%    \begin{macrocode}
\@newplength{refvpos}
%    \end{macrocode}
% \end{macro}
% \begin{macro}{\ltr@len@refheight}
% \selectlanguage{english}%^^A
% \changes{v3.27}{2019/05/03}{new}%^^A
% Height of the reference line (if used).
%    \begin{macrocode}
%<*package&willbe>
\@newplength{refheight}
%</package&willbe>
%    \end{macrocode}
% \selectlanguage{ngerman}%^^A
% \end{macro}
% \begin{macro}{\ltr@len@refwidth}
% \changes{v2.8q}{2001/10/06}{Neu}%^^A
% Breite der Geschäftszeile.
%    \begin{macrocode}
\@newplength{refwidth}
%    \end{macrocode}
% \end{macro}
% \begin{macro}{\ltr@len@refhpos}
% \changes{v2.9l}{2003/01/28}{Neu}%^^A
% Horizontale Position der Geschäftszeile.
%    \begin{macrocode}
\@newplength{refhpos}
%    \end{macrocode}
% \end{macro}
% \begin{macro}{\ltr@len@refaftervskip}
% \changes{v2.8q}{2001/10/09}{Neu}%^^A
% Abstand nach der Geschäftszeile.
%    \begin{macrocode}
\@newplength{refaftervskip}
%    \end{macrocode}
% \end{macro}
% \begin{macro}{\ltr@len@sigbeforevskip}
% \changes{v2.8q}{2001/10/09}{Neu}%^^A
% Raum für die Unterschrift.
%    \begin{macrocode}
\@newplength{sigbeforevskip}
%    \end{macrocode}
% \end{macro}
% \begin{macro}{\ltr@len@sigindent}
% \changes{v2.8q}{2001/10/06}{Neu}%^^A
% Horizontaler Einzug des Schlussgrußes und der Signatur.
%    \begin{macrocode}
\@newplength{sigindent}
%    \end{macrocode}
% \end{macro}
% \begin{macro}{\ltr@len@@firstfootvpos}
% \changes{v2.8q}{2001/10/06}{Neu}%^^A
% Abstand des Fußes der ersten Seite von der Papieroberkante.
%    \begin{macrocode}
\@newplength{firstfootvpos}
%    \end{macrocode}
% \end{macro}
% \begin{macro}{\ltr@len@firstfootwidth}
% \changes{v2.8q}{2001/10/06}{Neu}%^^A
% Breite des Fußes der ersten Seite.
%    \begin{macrocode}
\@newplength{firstfootwidth}
%    \end{macrocode}
% \end{macro}
% \begin{macro}{\ltr@len@firstfoothpos}
% \changes{v3.05}{2009/11/09}{Neu}%^^A
% Abstand des Kopfes vom linken Rand der Seite.
%    \begin{macrocode}
\@newplength{firstfoothpos}
\@setplength{firstfoothpos}{\maxdimen}
%    \end{macrocode}
% \end{macro}
% \begin{macro}{\ltr@len@firstheadvspos}
% \changes{v2.8q}{2001/10/06}{Neu}%^^A
% Abstand des Kopfes der ersten Seite von der Papieroberkante.
%    \begin{macrocode}
\@newplength{firstheadvpos}
%    \end{macrocode}
% \end{macro}
% \begin{macro}{\ltr@len@firstheadwidth}
% \changes{v2.8q}{2001/10/06}{Neu}%^^A
% Breite des Kopfes der ersten Seite.
%    \begin{macrocode}
\@newplength{firstheadwidth}
%    \end{macrocode}
% \end{macro}
% \begin{macro}{\ltr@len@firstheadhpos}
% \changes{v3.05}{2009/11/09}{Neu}%^^A
% Abstand des Kopfes vom linken Rand der Seite.
%    \begin{macrocode}
\@newplength{firstheadhpos}
\@setplength{firstheadhpos}{\maxdimen}
%    \end{macrocode}
% \end{macro}
% \begin{macro}{\ltr@len@fromrulewidth}
% \changes{v2.8q}{2002/01/28}{Neu}%^^A
% Die Länge einer eventuell gesetzten Linie im Absender. Negative
% Werte bedeuten, dass die Länge automtaisch bestimmt wird.
%    \begin{macrocode}
\@newplength{fromrulewidth}
\@setplength{fromrulewidth}{0pt}
%    \end{macrocode}
% \end{macro}
% \begin{macro}{\ltr@len@fromrulethickness}
% \changes{v2.97c}{2007/05/08}{Neue Pseudolänge \texttt{fromrulewidth}}
% Die Dicke einer eventuell gesetzte Linie im Absender.
%    \begin{macrocode}
\@newplength{fromrulethickness}
\@setplength{fromrulethickness}{.4pt}
%    \end{macrocode}
% \end{macro}
%
%
% \subsubsection{Fuß des Briefbogens}
%
% \begin{macro}{\firstfoot}
% \changes{v3.08}{2010/10/28}{der Fuß des Briefbogens wird über eine
%     Variable realisiert}%^^A
% \begin{macro}{\@firstfoot}
% \changes{v3.08}{2010/10/28}{der Fuß des Briefbogens wird über eine
%     Variable realisiert}%^^A
% \begin{KOMAvar}{firstfoot}
% \changes{v3.08}{2010/10/28}{Neu}%^^A
%    \begin{macrocode}
% Der Fuß des Briefbogens ist in der Voreinstellung leer.
\newkomavar{firstfoot}
%<*class>
\newcommand*{\@firstfoot}{%
  \ClassWarning{scrlttr2}{%
    Deprecated usage of \string\@firstfoot.\MessageBreak
    \string\usekomavar{firstfoot} should be used\MessageBreak
    instead of \string\@firstfoot
  }%
  \usekomavar{firstfoot}%
}
%    \end{macrocode}
% Dies kann jedoch über eine einfache Benutzeranweisung geändert werden:
%    \begin{macrocode}
\newcommand{\firstfoot}{%
  \ClassWarning{scrlttr2}{%
    Deprecated usage of \string\firstfoot.\MessageBreak
    You should use \string\setkomavar{firstfoot}\MessageBreak
    instead of \string\firstfoot
  }
  \setkomavar{firstfoot}%
}
%</class>
%    \end{macrocode}
% \end{KOMAvar}
% \end{macro}
% \end{macro}
%
%
% \subsubsection{Ausgabe des Briefbogens}
%
% Die Ausgabe selbst erfolgt bei \textsf{scrlttr2} über die
% |\opening|-Anweisung, die in einem späteren Abschnitt implementiert ist. Bei
% \textsf{scrletter} erfolgt sie dagegen ab Version 3.27 über den Seitenstil
% \texttt{notepaper} (siehe \texttt{scrkernel-pagestyle.dtx}).
%
% \selectlanguage{english}%
% \begin{macro}{\scr@notepaper@printallelements}
% \changes{v3.27}{2019/04/16}{new (internal)}%^^A
% This is used by page style \texttt{notepaper} to print all elements.
% \begin{macro}{\scr@notepaper@printelementsoflist}
% \changes{v3.27}{2019/04/16}{new (internal)}%^^A
% And this is used to print a given list of elements.
% \begin{macro}{\scr@notepaper@printelement}
% \changes{v3.27}{2019/04/16}{new (internal)}%^^A
% Last but not least this prints one given element.
%    \begin{macrocode}
%<*package&body&willbe>
\newcommand*{\scr@notepaper@printallelements}{%
  \scr@notepaper@printelementsoflist\scr@notepaper@elements
}
\newcommand*{\scr@notepaper@printelementsoflist}[1]{%
  \expandafter\scr@ForEachTrimmedListElement\expandafter{#1}{%
    \scr@notepaper@printelement
  }%
}
\newcommand*{\scr@notepaper@printelement}[1]{%
  \@ifundefined{scr@notepaper@element@#1}{%
    \PackageWarning{scrletter}{%
      Unknown note paper element `#1'\MessageBreak
      will be ignored%
    }%
  }{%
    \@nameuse{scr@notepaper@element@#1}%
  }%
}
%    \end{macrocode}
% \end{macro}%^^A \scr@notepaper@printelement
% \end{macro}%^^A \scr@notepaper@printelementsoflist
% \end{macro}%^^A \scr@notepaper@printallelements
%
% \begin{macro}{\scr@foldmarks@printallelements}
% \changes{v3.27}{2019/04/16}{new (internal)}%^^A
% This is used by the page style \texttt{notepaper} to print all foldmarks.
%    \begin{macrocode}
\newcommand*{\scr@foldmarks@printallelements}{%
  \if@fold
    \ifdim \useplength{pfoldmarklength}=\z@
      \@setplength{pfoldmarklength}{4mm}%
    \fi
    \ifdim \useplength{tfoldmarklength}=\z@
      \@setplength{tfoldmarklength}{2mm}%
    \fi
    \ifdim \useplength{mfoldmarklength}=\z@
      \@setplength{mfoldmarklength}{2mm}%
    \fi
    \ifdim \useplength{bfoldmarklength}=\z@
      \@setplength{bfoldmarklength}{2mm}%
    \fi
    \ifdim \useplength{lfoldmarklength}=\z@
      \@setplength{lfoldmarklength}{4mm}%
    \fi
    \@hfoldmark{P}%  
    \@hfoldmark{T}%
    \@hfoldmark{M}%
    \@hfoldmark{B}%
    \@vfoldmark{L}%
  \fi
}
%    \end{macrocode}
% \end{macro}
%
% \begin{macro}{\scr@notepaper@elements}
% \changes{v3.27}{2019/03/22}{new (internal)}%^^A
% This is a comma-separated list of elements (aka blocks) to be printed on the
% notepaper. It is used by the page style layer \texttt{notepaper} to set all
% elements (see \texttt{scrkernel-pagestyle.dtx}). There are exactly four
% defaults (but you can add more):
%    \begin{macrocode}
\newcommand*{\scr@notepaper@elements}{%
  firsthead,addressee,location,firstfoot,refline%
}
%    \end{macrocode}
% \begin{macro}{\scr@notepaper@element@firsthead}
% \changes{v3.27}{2019/03/22}{new (internal)}%^^A
% How to print the notepaper element \texttt{firsthead}:
%    \begin{macrocode}
\newcommand*{\scr@notepaper@element@firsthead}{%
  \if@firsthead
%    \end{macrocode}
% The calculation of the real horizontal position could be done as part of the
% layer contents or while selection of the layer or while initialization of
% the layer. The in-contents code is used, so the layer can be used with
% different page styles. The position depends on \texttt{firstheadhpos} this
% way:
% \begin{itemize}
% \item If \texttt{firstheadhpos} is positive but less than the paper width,
%   \texttt{firstheadhpos} is the distance if the head from the left edge.
% \item If \texttt{firstheadhpos} is negative but the absolute value is less
%   then the paper width, the absolute value of \texttt{firstheadhpos} is the
%   distance from the right edge.
% \item If \texttt{firstheadhpos} is greater than the paper width, the head
%   will be centered.
% \item If \texttt{firstheadhpos} is less than the negative paper width, the
%   head will be left aligned to the text area.
% \end{itemize}
%    \begin{macrocode}
    \ifdim\useplength{firstheadhpos}<\paperwidth
      \ifdim \useplength{firstheadhpos}>-\paperwidth
        \ifdim \useplength{firstheadhpos}<\z@
          \setlength{\@tempdima}{\dimexpr
            \paperwidth + \useplength{firstheadhpos}
            - \useplength{firstheadwidth}
            \relax}%
        \else
          \setlength{\@tempdima}{\useplength{firstheadhpos}}%
        \fi
      \else
        \setlength{\@tempdima}{\dimexpr\oddsidemargin + 1in\relax}%
      \fi
    \else
      \setlength{\@tempdima}{\dimexpr
        (\paperwidth-\useplength{firstheadwidth})/2
        \relax}%
    \fi
    \put@notepaper@element
      (\@tempdima,\useplength{firstheadvpos})%
      (\useplength{firstheadwidth},%
        \useplength{toaddrvpos}-\useplength{firstheadvpos})%
      {Element `firsthead'}{\usekomavar{firsthead}}\relax
  \fi
}
%    \end{macrocode}
% \end{macro}%^^A \scr@notepaper@element@firsthead
% \begin{macro}{\scr@notepaper@element@firstfoot}
% \changes{v3.27}{2019/03/22}{new (internal)}%^^A
% Element \texttt{firstfoot} is very similar to \texttt{firsthead}. The
% differences are using of pseudo lengths for the foot instead the head and
% the calculation of the available height.
%    \begin{macrocode}
\newcommand*{\scr@notepaper@element@firstfoot}{%
  \if@firstfoot
    \ifdim\useplength{firstfoothpos}<\paperwidth
      \ifdim \useplength{firstfoothpos}>-\paperwidth
        \ifdim \useplength{firstfoothpos}<\z@
          \setlength{\@tempdima}{\dimexpr
            \paperwidth + \useplength{firstfoothpos}
            - \useplength{firstfootwidth}%
            \relax}%
        \else
          \setlength{\@tempdima}{\useplength{firstfoothpos}}%
        \fi
      \else
        \setlength{\@tempdima}{\dimexpr\oddsidemargin + 1in\relax}%
      \fi
    \else
      \setlength{\@tempdima}{\dimexpr
        (\paperwidth-\useplength{firstfootwidth})/2
        \relax}%
    \fi
    \put@notepaper@element
      (\@tempdima,\useplength{firstfootvpos})%
      (\useplength{firstfootwidth},
        \paperheight-\useplength{firstfootvpos})%
      {Element `firstfoot'}{\usekomavar{firstfoot}}\relax
  \fi
}
% \end{macro}%^^A \scr@notepaper@element@firstfoot
% \begin{macro}{\scr@notepaper@element@addressee}
% Element \texttt{addressee} is very complex and therefore split into several
% different sub-elements.
%    \begin{macrocode}
\newcommand*{\scr@notepaper@element@addressee}{%
  \setlengthtoplength{\@tempdima}{toaddrhpos}%
  \ifdim \@tempdima<\z@
    \addtolength{\@tempdima}{%
      \dimexpr \paperwidth-\useplength{toaddrwidth}\relax
    }%
  \fi
  \put@notepaper@element(\@tempdima,\useplength{toaddrvpos})%
                        (\useplength{toaddrwidth},\useplength{toaddrheight})%
    {address field}{%
      \setparsizes{\z@}{\z@}{\z@ plus 1fil}\par@updaterelative
      \ifcase \@afieldmode\relax
      \or
        \scr@notepaper@subelement@addressee@default
      \or
        \scr@notepaper@subelement@addressee@PP
      \or
        \scr@notepaper@subelement@addressee@PPbackgroundimage
      \or
        \scr@notepaper@subelement@addressee@image
      \else
        \scr@notepaper@subelement@addressee@default
      \fi
    }\relax
}
%    \end{macrocode}
% \begin{macro}{\scr@notepaper@subelement@addressee@default}
% \changes{v3.27}{2019/03/22}{new (internal)}%^^A
% \changes{v3.28}{2019/11/15}{\cs{ifkomavarempty} durch \cs{Ifkomavarempty}
%   ersetzt}%^^A
% \changes{v3.30}{2020/03/27}{Umdefinierung des Zeilenumbruchs
%   verschoben}%^^A
% The field with the recipience address. This should be used only on the first
% page of a letter.
%    \begin{macrocode}
\newcommand*{\scr@notepaper@subelement@addressee@default}{%
  \setbox\@tempboxa\vbox {%
    \hsize=\useplength{toaddrwidth}\linewidth=\hsize
    \color@begingroup
    \if@baddr
      \Ifkomavarempty{backaddress}{}{%
        \backaddr@format{\scriptsize
          {%
            \def\\{\usekomavar{backaddressseparator}\@ogobble}%
            \usekomafont{backaddress}%
            {\usekomavar{backaddress}}%
          }%
        }%
      }%
    \fi
    \color@endgroup  
  }%
  \ifdim \dimexpr\ht\@tempboxa+\dp\@tempboxa>\useplength{backaddrheight}\relax
    \PackageWarning{scrletter}{Element `backaddress' to high.\MessageBreak
      The content of `backaddress' could exceed\MessageBreak
      the address field%
    }%
  \fi
  \vbox to \useplength{backaddrheight}{\vss\usebox\@tempboxa}%
  \ifcase \ltr@priority
  \or
    \setkomavar{specialmail}{%
      {\usekomafont{priority}{%
          {\usekomafont{prioritykey}{B}}%
          -ECONOMY}}%
    }%
  \or
    \setkomavar{specialmail}{%
      {\usekomafont{priority}{%
          {\usekomafont{prioritykey}{A}}%
          -PRIORITY}}%
    }%
  \fi  
  \Ifkomavarempty{specialmail}{%
    \@tempdima=\z@
  }{%
    \setbox\@tempboxa\vbox{\hsize\useplength{toaddrwidth}%
      \color@begingroup
      \noindent\hspace*{\useplength{specialmailindent}}%
      \specialmail@format{%
        \usekomafont{specialmail}\usekomavar{specialmail}}%
      \hspace*{\useplength{specialmailrightindent}}%
      \color@endgroup
    }%
    \copy\@tempboxa
    \@tempdima=\dimexpr\ht\@tempboxa+\dp\@tempboxa\relax
  }%
  \setbox\@tempboxa\vbox{%
    \hsize\dimexpr \useplength{toaddrwidth}-2\useplength{toaddrindent}\relax
    \color@begingroup
    \raggedright
    \usekomafont{addressee}{%
%<*package>
      \usekomavar{addressee}\par%
%</package>
%<class>      {\usekomafont{toname}{\usekomavar{toname}\\}}%
%<class>      {\usekomafont{toaddress}{\usekomavar{toaddress}\par}}%
    }%
    \color@endgroup
  }%
  \ifnum\@afieldmode=\@ne
%    \end{macrocode}
% In opposite to the class, the package rounds the distance to lines.
%    \begin{macrocode}
    \vskip \dimexpr ((\useplength{toaddrheight}-\useplength{backaddrheight}
             -\@tempdima -\ht\@tempboxa -\dp\@tempboxa
             -\ht\strutbox)/(2*\baselineskip))
             *\baselineskip\relax
  \fi
  \hskip\useplength{toaddrindent}%
  \usebox\@tempboxa
}
%    \end{macrocode}
% \end{macro}%^^A \scr@notepaper@subelement@addressee@default
% \begin{macro}{\scr@notepaper@subelement@addressee@PP}
% \changes{v3.27}{2019/03/22}{new (internal)}%^^A
%    \begin{macrocode}
\newcommand*{\scr@notepaper@subelement@addressee@PP}{%
  \ifdim \useplength{toaddrindent}=\z@
    \setplength{toaddrindent}{8mm}%
  \fi
  \ifdim \useplength{PPheadheight}=\z@
    \setplength{PPheadheight}{20.74pt}%
  \fi
  \ifdim \useplength{PPheadwidth}=\z@
    \setplength{PPheadwidth}{42mm}%
  \fi
  \ifdim \useplength{PPdatamatrixvskip}=\z@
    \setplength{PPdatamatrixvskip}{3mm}%
  \fi
  \vskip 2pt\relax
  \hskip \useplength{toaddrindent}\relax
  \fbox{%
    \vbox to \dimexpr \useplength{PPheadheight}-2\fboxsep\relax{%
      \setlength{\parindent}{0pt}%
      \hsize \dimexpr \useplength{PPheadwidth}-2\fboxsep\relax
%    \end{macrocode}
% \changes{v3.17}{2015/04/17}{\cs{setbox} durch \cs{sbox} ersetzt}%^^A
% \changes{v3.25}{2017/11/15}{Verwendung von \cs{scr@endstrut} statt
%   \cs{unskip}\cs{strut}}%^^A
%    \begin{macrocode}
      \sbox\@tempboxa{%
        {\fontsize{20.74pt}{20.74pt}\selectfont
          \usekomafont{PPlogo}{P.P.}}\hskip 2mm%
      }%
      \dp\@tempboxa\z@
      \setlength{\@tempdima}{\hsize}%
      \addtolength{\@tempdima}{-\wd\@tempboxa}%
      \hbox{\box\@tempboxa
        \vbox to \dimexpr
        \useplength{PPheadheight}-2\fboxsep\relax{%
          \vss\hsize\@tempdima
          \raggedright
          \fontsize{8pt}{9pt}\selectfont\usekomafont{PPdata}%
          \usekomavar*{fromzipcode}\usekomavar{zipcodeseparator}%
          \usekomavar{fromzipcode}\scr@endstrut\\
          \usekomavar{place}\scr@endstrut
          \hfill
        }%
      }%
      \vss
    }%
  }%
  \hskip 1mm
  \mbox{%
    \vbox to \useplength{PPheadheight}{%
      \addtolengthplength[-]{\hsize}{PPheadwidth}%
      \addtolength{\hsize}{-1mm}%
      \addtolengthplength[-2]{\hsize}{toaddrindent}%
      \setlength{\parindent}{0pt}%
      \fontsize{8pt}{9pt}\selectfont\usekomafont{PPdata}%
      \hfill\usekomavar{PPcode}
      \vss
      \ifcase \ltr@priority
      \or
        \usekomafont{priority}{%
          {\usekomafont{prioritykey}{B}}%
          -ECONOMY\hss\vskip-\dp\strutbox
        }%
      \or
        \usekomafont{priority}{%
          {\usekomafont{prioritykey}{A}}%
          -PRIORITY\hss\vskip-\dp\strutbox
        }%
      \fi
      \vss
    }%
  }%
  \vskip 1mm
  \vskip-\ht\strutbox
  \noindent\hskip\useplength{toaddrindent}%
  \hrulefill%
  \hskip\useplength{toaddrindent}%
  \vbox to\z@{\hsize\z@
    \vskip \useplength{PPdatamatrixvskip}\relax
    \llap{\usekomavar{PPdatamatrix}\hskip\useplength{toaddrindent}}%
    \vss
  }%
  \vskip 3mm
  \noindent\hskip\useplength{toaddrindent}%
  \setlength{\@tempdima}{\useplength{toaddrwidth}}%
  \addtolengthplength[-2]{\@tempdima}{toaddrindent}%
  \vbox{\hsize\@tempdima
    \Ifkomavarempty{PPdatamatrix}{}{%
      \addtolength{\hsize}{-2mm}%
      \settowidth\@tempdima{\usekomavar{PPdatamatrix}}%
      \addtolength{\hsize}{-\@tempdima}%
    }%
    \raggedright
    \usekomafont{addressee}{%
%<*package>
      \usekomavar{addressee}\par%
%</package>
%<class>      {\usekomafont{toname}{\usekomavar{toname}\\}}%
%<class>      {\usekomafont{toaddress}{\usekomavar{toaddress}\par}}%
    }%
  }%
}
%    \end{macrocode}
% \end{macro}%^^A \scr@notepaper@subelement@addressee@PP
% \begin{macro}{\scr@notepaper@subelement@addressee@PPimage}
% \changes{v3.27}{2019/03/22}{new (internal)}%^^A
% TODO: Need to be tested!
%    \begin{macrocode}
\newcommand*{\scr@notepaper@subelement@addressee@PPimage}{%
  \ifdim \useplength{toaddrindent}=\z@
    \setplength{toaddrindent}{8mm}%
  \fi
  \vskip 1mm
  \noindent\expandafter\hskip\useplength{toaddrindent}%
  \vbox to \useplength{PPheadheight}{%
    \addtolengthplength[-2]{\hsize}{toaddrindent}%
    \setlength{\parindent}{0pt}%
    \usekomavar{addresseeimage}%
    \vss
  }%
}
%    \end{macrocode}
% \end{macro}%^^A \scr@notepaper@subelement@addressee@PPimage
% \begin{macro}{\scr@notepaper@subelement@addressee@backgroundimage}
% \changes{v3.27}{2019/03/22}{new (internal)}%^^A
% \changes{v3.28}{2019/11/15}{\cs{ifkomavarempty} durch \cs{Ifkomavarempty}
%   ersetzt}%^^A
% TODO: Need to be tested!
%    \begin{macrocode}
\newcommand*{\scr@notepaper@subelement@addressee@backgroundimage}{%
  \vskip 1mm
  \noindent\expandafter\hskip\useplength{toaddrindent}%
  \vbox to \useplength{PPheadheight}{%
    \addtolengthplength[-2]{\hsize}{toaddrindent}%
    \setlength{\parindent}{0pt}%
    \usekomavar{addresseeimage}%
    \vss
  }%
  \vskip 1mm
  \vskip\dp\strutbox
  \vfil%
  \noindent\hskip\useplength{toaddrindent}%
  \setlength{\@tempdima}{\useplength{toaddrwidth}}%
  \addtolengthplength[-2]{\@tempdima}{toaddrindent}%
  \vbox{\hsize\@tempdima
    \Ifkomavarempty{PPdatamatrix}{}{%
      \addtolength{\hsize}{-2mm}%
      \settowidth\@tempdima{\usekomavar{PPdatamatrix}}%
      \addtolength{\hsize}{-\@tempdima}%
    }%
    \raggedright
    \usekomafont{addressee}{%
%<*package>
      \usekomavar{addressee}\par%
%</package>
%<class>      {\usekomafont{toname}{\usekomavar{toname}\\}}%
%<class>      {\usekomafont{toaddress}{\usekomavar{toaddress}\par}}%
    }%
  }%
}
%    \end{macrocode}
% \end{macro}%^^A \scr@notepaper@subelement@addressee@backgroundimage
% \end{macro}%^^A \scr@notepaper@element@addressee
% \begin{macro}{\scr@notepaper@element@location}
% \changes{v3.27}{2019/03/22}{new (internal)}%^^A
% \changes{v3.28}{2019/11/15}{\cs{ifkomavarempty} durch \cs{Ifkomavarempty}
%   ersetzt}%^^A
%    \begin{macrocode}
\newcommand*{\scr@notepaper@element@location}{%
%    \end{macrocode}
% \selectlanguage{ngerman}%^^A
% \changes{v2.97e}{2007/11/13}{Der Absender kann nun ebenfalls in der
%   Absenderergänzung stehen}%^^A
% \changes{v2.98b}{2008/01/30}{Absenderergänzung immer setzen, wenn
%   verwendet}%^^A
% Je nach Absenderanordnung muss hier nun ebenfalls etwas getan werden.
% Der Fall 0, also linksbündiger Absender im Kopf spielt hier keine Rolle.
%    \begin{macrocode}
  \ifcase\@alignfrom
%    \end{macrocode}
% Im Fall 1, also zentrierter Absender im Kopf wird jedoch das Logo hier
% platziert, also darf die Absenderergänzung nicht leer sein.
%    \begin{macrocode}
  \or
    \if@logo
      \Ifkomavarempty{location}{\setkomavar{location}{\null}}{}%
    \fi
%    \end{macrocode}
% Im Fall 2, also rechtsbündiger Absender im Kopf, ist auch nichts zu tun.
%    \begin{macrocode}
  \or
%    \end{macrocode}
% Im Fall 3, also linksbündiger Absender in der Absenderergänzung wird hier
% eine Menge ausgegeben, also darf die Absenderergänzung nicht leer sein.
%    \begin{macrocode}
  \or
    \Ifkomavarempty{location}{\setkomavar{location}{\null}}{}%
%    \end{macrocode}
% Im Fall 3, also linksbündiger Absender in der Absenderergänzung wird hier
% eine Menge ausgegeben, also darf die Absenderergänzung nicht leer sein.
%    \begin{macrocode}
  \or
    \Ifkomavarempty{location}{\setkomavar{location}{\null}}{}%
%    \end{macrocode}
% Alle anderen Fälle sind wie hier ebenfalls ohne Bedeutung.
%    \begin{macrocode}
  \fi
%    \end{macrocode}
% \selectlanguage{english}%^^A
% Everything else only, if the field is not empty.
%    \begin{macrocode}
  \Ifkomavarempty{location}{}{%
%    \end{macrocode}  
% Calculation of height and vertical possition.
%    \begin{macrocode}
    \ifdim \useplength{locheight}=\z@%
      \setplength{locheight}{\useplength{toaddrheight}}%
    \fi
    \ifdim \useplength{locvpos}=\z@%
      \setplength{locvpos}{\useplength{toaddrvpos}}%
    \fi
%    \end{macrocode}
% Calculation of width and horizontal possition.
%    \begin{macrocode}
    \ifdim \useplength{locwidth}=\z@%
      \setplength{locwidth}{\paperwidth}%
      \ifdim \useplength{toaddrhpos}>\z@
        \addtoplength[-2]{locwidth}{\useplength{toaddrhpos}}%
      \else
        \addtoplength[2]{locwidth}{\useplength{toaddrhpos}}%
      \fi
      \addtoplength[-1]{locwidth}{\useplength{toaddrwidth}}%
      \if@bigloc
        \setplength[.66667]{locwidth}{\useplength{locwidth}}%
      \else
        \setplength[.5]{locwidth}{\useplength{locwidth}}%
      \fi
    \fi
    \ifdim \useplength{lochpos}=\z@%
      \setplength{lochpos}{\useplength{toaddrhpos}}%
    \fi
    \ifdim \useplength{lochpos}<\z@
      \setlength{\@tempdima}{-\useplength{lochpos}}%
    \else
      \setlength{\@tempdima}{\dimexpr
        \paperwidth-\useplength{lochpos}-\useplength{locwidth}
        \relax}%
    \fi
    \put@notepaper@element(\@tempdima,\useplength{locvpos})%
                          (\useplength{locwidth},\useplength{locheight})%
    {Element `location'}%
    {%
%    \end{macrocode}
% \selectlanguage{ngerman}%^^A
% In einigen Fällen der Absenderanordnung muss hier nun eine ganze Menge getan
% werden. In anderen ist hingegen nur wenig zu tun.  Im Fall 0, also
% linksbündiger Absender im Kopf wird nur die Absenderergänzung ausgegeben
% (das erfolgt nach der Fallunterscheidung).
%    \begin{macrocode}
      \ifcase\@alignfrom
%    \end{macrocode}
% Im Fall 1, also zentrierter Absender im Kopf wird jedoch das Logo hier
% platziert, die Absenderergänzung wird ggf. danach ausgegeben (das erfolgt
% nach der Fallunterscheidung).%^^A
% \changes{v2.98a}{2008/01/10}{bei zentriertem Absender wird die
%   Absenderergänzung ggf. unter dem Logo ausgegeben}%^^A
%    \begin{macrocode}
      \or
        \if@logo
          \Ifkomavarempty{fromlogo}{\null}{%
            {\raggedleft\noindent\usekomavar{fromlogo}\\
              \vskip\parskip}\noindent}%
        \fi
%    \end{macrocode}
% Im Fall 2, also rechtsbündiger Absender im Kopf, wird auch nur die
% Absenderergänzung ausgegeben (das erfolgt nach der Fallunterscheidung).
%    \begin{macrocode}
      \or
%    \end{macrocode}
% Im Fall 3, also linksbündiger Absender in der Absenderergänzung wird hier
% sozusagen der linksbündige Kopf nachgebaut. Das Logo steht dabei
% ggf. rechtsbündig oben.
%    \begin{macrocode}
      \or
        \if@logo
          \Ifkomavarempty{fromlogo}{}{%
            \raggedleft\usekomavar{fromlogo}\par}%
        \fi
        \raggedright
%    \end{macrocode}
% Zuerst wird der Name gesetzt. Dabei soll eine eigene Schriftart für den
% Namen verwendet werden.
%    \begin{macrocode}
        {\usekomafont{fromname}\strut\ignorespaces
          \usekomavar{fromname}%
%    \end{macrocode}
% Hat \verb|\@rulefrom| den nummerischen Wert 1, so soll direkt auf den
% Absender eine Linie folgen. Die Länge dieser Linie kann vom Anwender
% bestimmt oder automatisch gewählt werden. Es werden dabei die Pseudo-Längen
% \texttt{fromrulewidth} und \texttt{fromrulethickness} verwendet. Ist
% \texttt{fromrulewidth} 0, so wird sie automatisch bestimmt. Bei der
% automatischen Längenbestimmung wird einfach eine Linie über die Gesamtbreite
% der Absenderergänzung gesetzt.
%    \begin{macrocode}
          {%
            \ifnum\@rulefrom=1\\[-.5\baselineskip]%
              \usekomafont{fromrule}{%
                \setlength{\@tempdima}{\useplength{fromrulewidth}}%
                \ifdim\@tempdima=\z@
                  \setlength{\@tempdima}{\useplength{locwidth}}%
                \fi
                \expandafter\rule\expandafter{\the\@tempdima}{%
                  \useplength{fromrulethickness}}%
              }%
            \fi
          }%
          \\
        }%
%    \end{macrocode}
% Als Nächstes wird die Absenderadresse einschließlich aller optionalen Felder
% gesetzt. Dabei werden optionale Felder nur gesetzt, wenn der entsprechende
% Schalter \emph{true} ist. Die Adresse wird ebenfalls innerhalb einer Gruppe
% in einer eigenen Schriftart gesetzt.%
% \changes{v3.12}{2013/05/29}{Variable \texttt{frommobilephone} wird
% berücksichtigt}%^^A
%    \begin{macrocode}
        {\usekomafont{fromaddress}\strut\ignorespaces
          \usekomavar{fromaddress}%
          \if@phone\\\usekomavar*{fromphone}\usekomavar{fromphone}\fi%
          \if@mobilephone
            \\\usekomavar*{frommobilephone}\usekomavar{frommobilephone}%
          \fi%
          \if@fax\\\usekomavar*{fromfax}\usekomavar{fromfax}\fi%
          \if@email\\\usekomavar*{fromemail}\usekomavar{fromemail}\fi%
          \if@www\\\usekomavar*{fromurl}\usekomavar{fromurl}\fi%
%    \end{macrocode}
% Wie weiter oben erwähnt, kann auch an dieser Stelle eine Linie gesetzt
% werden.
%    \begin{macrocode}
          \ifnum\@rulefrom=2\\[-.5\baselineskip]%
            \usekomafont{fromrule}{%
              \setlength{\@tempdima}{\useplength{fromrulewidth}}%
              \ifdim\@tempdima=\z@
                \setlength{\@tempdima}{\useplength{locwidth}}%
              \fi
              \expandafter\rule\expandafter{\the\@tempdima}{%
                \useplength{fromrulethickness}}}%
          \fi
          \\
        }%
%    \end{macrocode}
% Damit ist die Definition des linksbündigen Absenders in der
% Absenderergänzung abgeschlossen. Im Fall 4, also rechtsbündiger Absender in
% der Absenderergänzung wird hier exakt wie für Fall 3 verfahren. Nur ist
% jetzt alles rechtsbündig.^^A%
% \changes{v3.12}{2013/05/29}{Variable \texttt{frommobilephone} wird
% berücksichtigt}%^^A%
%    \begin{macrocode}
      \or
        \raggedleft
        \if@logo
          \Ifkomavarempty{fromlogo}{}{\usekomavar{fromlogo}\par}%
        \fi
        {\usekomafont{fromname}\strut\ignorespaces
          \usekomavar{fromname}%
          {%
            \ifnum\@rulefrom=1\\[-.5\baselineskip]%
              \usekomafont{fromrule}{%
                \setlength{\@tempdima}{\useplength{fromrulewidth}}%
                \ifdim\@tempdima=\z@
                  \setlength{\@tempdima}{\useplength{locwidth}}%
                \fi
                \expandafter\rule\expandafter{\the\@tempdima}{%
                  \useplength{fromrulethickness}}%
              }%
            \fi
          }%
          \\
        }%
        {\usekomafont{fromaddress}\strut\ignorespaces
          \usekomavar{fromaddress}%
          \if@phone\\\usekomavar*{fromphone}\usekomavar{fromphone}\fi%
          \if@mobilephone%
            \\\usekomavar*{frommobilephone}\usekomavar{frommobilephone}%
          \fi%
          \if@fax\\\usekomavar*{fromfax}\usekomavar{fromfax}\fi%
          \if@email\\\usekomavar*{fromemail}\usekomavar{fromemail}\fi%
          \if@www\\\usekomavar*{fromurl}\usekomavar{fromurl}\fi%
          \ifnum\@rulefrom=2\\[-.5\baselineskip]%
            \usekomafont{fromrule}{%
              \setlength{\@tempdima}{\useplength{fromrulewidth}}%
              \ifdim\@tempdima=\z@
                \setlength{\@tempdima}{\useplength{locwidth}}%
              \fi
              \expandafter\rule\expandafter{\the\@tempdima}{%
                \useplength{fromrulethickness}}}%
          \fi
          \\
        }%
%    \end{macrocode}
% In allen anderen Fällen wird ebenfalls die Absenderergänzung gesetzt.
%    \begin{macrocode}
      \fi
      \usekomavar{location}%
%    \end{macrocode}
% \selectlanguage{english}%^^A
% That's it.
%    \begin{macrocode}
    }\relax
  }%
}
%    \end{macrocode}
% \end{macro}%^^A \scr@notepaper@element@location
% \begin{macro}{\scr@notepaper@element@refline}
% \changes{v3.27}{2019/05/03}{new (internal)}%^^A
% This notepaper element replaces the \cs{@datefield} of
% \textsf{scrlttr2}. In opposite to \textsf{scrlttr2} this element has a
% dedicated height \texttt{reflineheight}.
% TODO: Still missing!
%    \begin{macrocode}
\newcommand*{\scr@notepaper@element@refline}{%
%    \end{macrocode}
% First of all test, whether or not their has to be set anything.
%    \begin{macrocode}
  \@tempswafalse
  \if@refline
    \let\do@set@reffield\@test@reffield
%<trace>    \let\@tempa\@empty
    \@set@reffields
%    \end{macrocode}
% Maybe we have to add the date:
%    \begin{macrocode}
    \ifcase \@datepos
    \or
      \if@tempswa\addtoreffields{date}\fi
      \expandafter\let\csname if@refline\expandafter\endcsname
      \csname if@tempswa\endcsname
      \@test@reffield{date}%
    \or
      \if@tempswa
        \edef\expandafter\@set@reffields{%
          \unexpanded{\do@set@reffield{date}}%
          \unexpanded\expandafter{\@set@reffields}%
        }%
      \fi
      \expandafter\let\csname if@refline\expandafter\endcsname
      \csname if@tempswa\endcsname
      \@test@reffield{date}%
    \fi
  \fi  
  \if@tempswa
%    \end{macrocode}
% We have to print something, so let's calculate the position and height.
%    \begin{macrocode}
    \ifdim\useplength{refwidth}=\z@
%<trace>      \typeout{TRACE: refwidth by option}%
      \if@refwide
%<trace>        \typeout{TRACE: refline=wide}%
        \ifdim \useplength{toaddrhpos}>\z@
          \setplength{refhpos}{\useplength{toaddrhpos}}%
        \else
          \setplength[-]{refhpos}{\useplength{toaddrhpos}}%
        \fi
        \setplength{refwidth}{\paperwidth -2\useplength{refhpos}}%
      \else
%<trace>        \typeout{TRACE: refline=narrow}%
        \setplength{refwidth}{\textwidth}%
        \setplength{refhpos}{\oddsidemargin+1in}%
      \fi
    \else
%<trace>      \typeout{TRACE: refwidth by pseudo-length}%
      \ifdim\useplength{refhpos}=\z@
        \setlength{\@tempdima}{\dimexpr
          \textwidth-\useplength{refwidth}
          \relax}%
        \setlength{\@tempdimb}{\dimexpr \oddsidemarin+1in\relax}%
        \setlength{\@tempdimc}{\dimexpr \paperwidth-\textwidth\relax}%
        \divide\@tempdimb by 32768\relax
        \divide\@tempdimc by 32768\relax
        \ifdim\@tempdimc=\z@\else
          \multiply\@tempdima by\@tempdimb
          \divide\@tempdima by\@tempdimc
        \fi
        \setplength{refhpos}{\@tempdima}%
      \fi
    \fi
%<*trace>
    \typeout{TRACE: Reference line with \@tempa^^J
      \space\space\space\space\space\space
      at refhpos=\the\useplength{refhpos},^^J
      \space\space\space\space\space\space\space\space\space
      refvpos=\the\useplength{refvpos},^^J
      \space\space\space\space\space\space\space\space\space
      refwidth=\the\useplength{refwidth},^^J
      \space\space\space\space\space\space\space\space\space
      refheight=\the\useplength{refheight}}%
%</trace>
    \put@notepaper@element
      (\useplength{refhpos},\useplength{refvpos})%
      (\useplength{refwidth},\useplength{refheight})%
      {Element `refline'}%
      {%
%    \end{macrocode}
% Last but not least print it.
%    \begin{macrocode}
        \if@refline
%    \end{macrocode}
% Print a reference line.
%    \begin{macrocode}
          \let\do@set@reffield\@set@reffield\@tempswafalse\@set@reffields
        \else
%    \end{macrocode}
% Print only place and date.
%    \begin{macrocode}
          \ifnum \@datepos=1 \hspace*{\fill}\fi
          {%
            \usekomafont{placeanddate}{%
              \Ifkomavarempty{place}{}{%
                \usekomavar{place}\usekomavar{placeseparator}%
              }%
              \usekomavar{date}%
            }%
          }%
        \fi
      }%
  \fi
}
%    \end{macrocode}
% \end{macro}%^^A \scr@notepaper@element@refline
% \end{macro}%^^A \scr@notepaper@elements
% \begin{macro}{\put@notepager@element}
% \changes{v3.27}{2019/03/22}{new (internal)}%^^A
% \begin{macro}{\put@@notepager@@element}
% \changes{v3.27}{2019/05/03}{new (internal)}%^^A
% A helper macro to output one element at a specific
% position. \texttt{(\#1,\#2)} is the position, \texttt{(\#3,\#4)} the allowed
% width and height, \texttt{\#5} is the name of the element (used for warning
% messages only), \texttt{\#6} is the element content. The content is first
% put into a temporary box of type \texttt{vtop}. Optional code can be
% executed immediately before testing the total height of this box
% \cs{@tempboxa} and before using it.
%    \begin{macrocode}
\newcommand{\put@notepaper@element}{}
\long\def\put@notepaper@element(#1,#2)(#3,#4)#5#6{%
  \kernel@ifnextchar [%]
  {\put@@notepaper@@element({#1},{#2})({#3},{#4}){#5}{#6}}%
  {\put@@notepaper@@element({#1},{#2})({#3},{#4}){#5}{#6}[]}%
}
\newcommand{\put@@notepaper@@element}{}
\long\def\put@@notepaper@@element(#1,#2)(#3,#4)#5#6[#7]{%
  \put(\LenToUnit{\dimexpr #1\relax},\LenToUnit{-\dimexpr #2\relax}){%
    \setbox\@tempboxa \vtop{%
      \color@begingroup
      \vskip \z@
      \hsize=\dimexpr #3\relax
      \linewidth=\hsize
      \columnwidth=\linewidth
      \textwidth=\columnwidth
      \parfillskip=\z@ \@plus 1fil\relax
      #6
      \color@endgroup
    }%
    #7%
    \ifdim \dimexpr\ht\@tempboxa+\dp\@tempboxa\relax > \dimexpr #4\relax
      \PackageWarning{scrletter}{#5
        \the\dimexpr \ht\@tempboxa+\dp\@tempboxa-(#4)\relax\space
        to high.\MessageBreak
        The content could collide with other page contents\MessageBreak
        or break the page edges. You should reduce the\MessageBreak
        size before usage%
      }%
    \fi
    \usebox\@tempboxa
  }%
}
%</package&body&willbe>
%    \end{macrocode}
% \end{macro}%^^A \put@@notepaper@@element
% \end{macro}%^^A \put@notepaper@element
% \selectlanguage{ngerman}%^^A 
%
% \begin{macro}{\move@topt}
% \changes{v2.8n}{2001/09/07}{Neu}%^^A
% Für die folgenden Aktionen wird ein Makro benötigt, mit dem man vertikal von
% der relativen Seitenposition zur relativen Papierposition gelangen kann.
%    \begin{macrocode}
\newcommand*{\move@topt}{%
  \null\vskip -1in\vskip -\topmargin%
  \vskip -\headheight\vskip -\headsep%
  \vskip -\topskip%
  \ignorespaces%
}
%    \end{macrocode}
% \begin{macro}{\move@frompt}
% \changes{v2.8n}{2001/09/07}{Neu}%^^A
% Natürlich wird auch ein Makro benötigt, um vertikal von einer relativen
% Papierposition wieder auf die relative Seitenposition zurück zu gelangen.
%    \begin{macrocode}
\newcommand*{\move@frompt}{%
  \vskip 1in\vskip \topmargin%
  \vskip \headheight\vskip \headsep%
  \vskip \topskip%
  \vskip -\baselineskip%
  \ignorespaces%
}
%    \end{macrocode}
% \end{macro}
% \end{macro}
%
% \begin{macro}{\move@topl}
% \changes{v2.8n}{2001/09/07}{Neu}%^^A
% Für die folgenden Aktionen wird außerdem ein Makro benötigt, mit dem man
% horizontal von der relativen Seitenposition zur relativen Papierposition
% gelangen kann.
%    \begin{macrocode}
\newcommand*{\move@topl}{%
  \null\hskip -1in%
  \ifodd\value{page}\hskip -\oddsidemargin\else\hskip -\evensidemargin\fi
  \ignorespaces%
}
%    \end{macrocode}
% \begin{macro}{\move@frompl}
% \changes{v2.8n}{2001/09/07}{Neu}%^^A
% Natürlich wird auch ein Makro benötigt, um horizontal von einer relativen
% Papierposition wieder auf die relative Seitenposition zurück zu gelangen.
%    \begin{macrocode}
\newcommand*{\move@frompl}{%
  \hskip 1in%
  \ifodd\value{page}\hskip \oddsidemargin\else\hskip \evensidemargin\fi%
  \ignorespaces%
}
%    \end{macrocode}
% \end{macro}
% \end{macro}
%
%
% \begin{macro}{\vb@t@z}
% \changes{v2.8n}{2001/09/07}{Neu}%^^A
% \changes{v2.95}{2006/03/23}{Absatzformatierung mit \cs{setparsizes}}
% \changes{v2.97e}{2007/11/20}{\cs{long}}
% Ausserdem wird ein vertikales Äquivalent zu |\rlap| benötigt.
%    \begin{macrocode}
\newcommand{\vb@t@z}[1]{{%
    \setbox\z@\vbox to\z@{\color@begingroup #1\color@endgroup\vss}%
    \setparsizes{\z@}{\z@}{\z@ plus 1fil}\par@updaterelative
    \setlength{\baselineskip}{\z@}%
    \setlength{\@tempdima}{\ht0}\addtolength{\@tempdima}{\dp0}%
    \box\z@\vskip-\@tempdima%
  }%
}
%    \end{macrocode}
% \end{macro}
%
% \begin{macro}{\@foldmarks}
% \changes{v2.97e}{2007/11/08}{\texttt{mfoldmarkvpos} wird verwendet}%^^A
% Für die Faltmarken müssen nicht nur Abstände definiert werden, es wird auch
% ein Befehl benötigt, um die Faltmarken zu setzen.
% \begin{macro}{\@hfoldmark}
% \changes{v2.97e}{2007/11/08}{Neu (intern)}%^^A
% Dabei werden nur die horizontalen Faltmarken gesetzt, die tatsächlich
% existieren.
% \begin{macro}{\@foldmark}
% \changes{v2.8n}{2001/09/07}{Neu}%^^A
% \changes{v2.95}{2006/06/06}{\cs{vskip} in \cs{vb@t@z} verschoben, damit es
%   vom Satzspiegel unabhängig wird}%^^A
% \changes{v2.97c}{2007/05/08}{Dicke konfigurierbar}%^^A
% \changes{v2.97c}{2007/05/08}{Schrift (Farbe) konfigurierbar}%^^A
% \changes{v3.27}{2019/03/29}{different versions for package and class}%^^A
% Das geht am einfachsten, wenn wir auch einen Befehl haben, mit dem man eine
% einzelne Faltmarke relativ zur aktuellen Position setzen kann. Dies
% geschieht mit Hilfe von |\rlap|, damit die horizontale Position unverändert
% bleibt. Als Parameter werden der vertikale Abstand zur aktuellen Position
% und die Länder der Marke übergeben.
%    \begin{macrocode}
%<*class|!willbe>
\newcommand*{\@foldmark}[2]{%
  \vb@t@z{\setlength{\@tempdima}{#1}%
    \addtolengthplength[-.5]{\@tempdima}{foldmarkthickness}%
    \vskip \@tempdima
    \rlap{\move@topl\hskip\useplength{foldmarkhpos}%
      \usekomafont{foldmark}{%
        \rule{#2}{\useplength{foldmarkthickness}}}}}%
  \ignorespaces
}
%</class|!willbe>
%    \end{macrocode}
% \selectlanguage{english}%^^A
% The package version uses \cs{put} to set the marks. It has a possition, a
% direction and a length, so it can handle the horizontal and the vertical
% foldmarks.
%    \begin{macrocode}
%<*package&willbe>
\newcommand*{\@foldmark}{}
\def\@foldmark(#1,#2)(#3,#4)#5{%
  \put(\LenToUnit{#1},-\LenToUnit{#2}){%
    \linethickness{\useplength{foldmarkthickness}}%
    \line(#3,-#4){\LenToUnit{#5}}%
  }%
}
%</package&willbe>
\newcommand*{\@hfoldmark}[1]{%
  \begingroup
    \@expandtwoargs\in@{#1}{\@usefolds}%
    \ifin@
      \lowercase{%
        \edef\@tempa{%
%<*class|!willbe>
          \noexpand\ifdim \noexpand\useplength{#1foldmarkvpos}=\noexpand\z@
          \noexpand\else
            \noexpand\@foldmark{%
              \noexpand\useplength{#1foldmarkvpos}%
            }{%
              \noexpand\useplength{#1foldmarklength}%
            }%
          \noexpand\fi
%</class|!willbe>
%    \end{macrocode}
% \changes{v3.27}{2019/03/29}{different handling of the package version}%^^A
% The package version of \cs{@foldmark} has a new direction argument, because
% it handles horizontal and vertical foldmarks. And if there is not a
% pseudo-length for the vertical possition of the horizontal fold-mark the
% possition is half of the paper height. This is for the ``P''-mark.
% \selectlanguage{ngerman}%^^A
%    \begin{macrocode}
%<*package&willbe>
          \Ifplength{#1foldmarkvpos}{%
            \ifdim \useplength{#1foldmarkvpos}=\z@\else
              \noexpand\@foldmark(\noexpand\useplength{foldmarkhpos},%
                                    \noexpand\useplength{#1foldmarkvpos})%
                                 (1,0)%
              {%
                \noexpand\useplength{#1foldmarklength}%
              }%
            \fi
          }{%
            \noexpand\@foldmark(\noexpand\useplength{foldmarkhpos},%
                                  .5\noexpand\paperheight)%
                               (1,0)%
            {%
              \noexpand\useplength{#1foldmarklength}%
            }%
          }%
%</package&willbe>
        }%
      }%
      \@tempa
    \fi
  \endgroup
}
%    \end{macrocode}
% \begin{macro}{\@vfoldmark}
% \changes{v2.97e}{2007/11/10}{Neu (intern)}%^^A
% Für die vertikale(n) Faltmarke(n) gilt prinzipiell das gleiche. Dabei ist
% lediglich zu beachten, dass die Linie hier komplett oberhalb der Grundlinie
% gesetzt wird, also die vertikale Position um die Länge nach unten zu
% verschieben ist.
%    \begin{macrocode}
\newcommand*{\@vfoldmark}[1]{%
  \begingroup
    \ifdim \useplength{lfoldmarkhpos}=\z@\else
      \@expandtwoargs\in@{#1}{\@usefolds}%
      \ifin@
%    \end{macrocode}
% \selectlanguage{english}%^^A
% \changes{v3.27}{2019/03/29}{package version uses \cs{@foldmark}}%^^A
% The class version does still work using \cs{rule}.
%    \begin{macrocode}
%<*class|willbe>
        \vb@t@z{%
          \ifdim \useplength{foldmarkvpos}<\z@
            \nobreak\vskip \paperheight\nobreak
          \fi
          \vskip \useplength{foldmarkvpos}%
          \vskip \useplength{foldmarkthickness}%
          \rlap{\move@topl
            \setlength{\@tempdima}{\useplength{lfoldmarkhpos}}%
            \addtolengthplength[-.5]{\@tempdima}{foldmarkthickness}%
            \hskip\@tempdima
            \usekomafont{foldmark}{%
              \rule{\useplength{foldmarkthickness}}%
                   {\useplength{lfoldmarklength}}%
            }%
          }%
        }%
        \ignorespaces
%</class|willbe>
%    \end{macrocode}
% But the package version uses \cs{@foldmark}.
%    \begin{macrocode}
%<*package&willbe>
        \@foldmark(\useplength{lfoldmarkhpos},\useplength{foldmarkvpos})%
                  (0,1)%
                  {\useplength{lfoldmarklength}}%
%</package&willbe> 
      \fi
    \fi
  \endgroup
}
%    \end{macrocode}
% The main foldmark command is only needed by the class.
% \selectlanguage{ngerman}%^^A
%    \begin{macrocode}
%<*class|!willbe>
\newcommand*{\@foldmarks}{%
  \if@fold%
    \typeout{Foldmarks: yes}%
    \move@topt
    \ifdim \useplength{pfoldmarklength}=\z@
      \@setplength{pfoldmarklength}{4mm}%
    \fi
    \ifdim \useplength{tfoldmarklength}=\z@
      \@setplength{tfoldmarklength}{2mm}%
    \fi
    \ifdim \useplength{mfoldmarklength}=\z@
      \@setplength{mfoldmarklength}{2mm}%
    \fi
    \ifdim \useplength{bfoldmarklength}=\z@
      \@setplength{bfoldmarklength}{2mm}%
    \fi
    \ifdim \useplength{lfoldmarklength}=\z@
      \@setplength{lfoldmarklength}{4mm}%
    \fi
    \begingroup
      \@expandtwoargs\in@{P}{\@usefolds}%
      \ifin@
        \@foldmark{.5\paperheight}{\useplength{pfoldmarklength}}%
      \fi
    \endgroup
    \@hfoldmark{T}%
    \@hfoldmark{M}%
    \@hfoldmark{B}%
    \@vfoldmark{L}%
    \move@frompt
  \else
    \typeout{Foldmarks: no}%
  \fi%
}
%</class|!willbe>
%    \end{macrocode}
% \end{macro}
% \end{macro}
% \end{macro}
% \end{macro}
%
% \begin{macro}{\@savevbox}
% \changes{v2.8n}{2001/09/07}{Neu}%^^A
% \changes{v2.8q}{2001/10/30}{zusätzlich ein optionales Argument}%^^A
% \begin{macro}{\if@savevbox@warning}
% \begin{macro}{\@savevbox@warningtrue}
% \begin{macro}{\@savevbox@warningfalse}
% \changes{v2.8q}{2001/10/30}{Neu}%^^A
% Wenn eine der vertikalen Boxen in \cs{opening} zu einer \texttt{overbox
% \string\vbox\space} Meldung führt, weiss letztlich niemand, welche Box dafür
% verantwortlich war. Deshalb definieren wir hier ein Makro, das ggf. eine
% Meldung ausgibt, die erklärt, welches Element die Ursache war und was man
% tun kann. Das erste Argument ist dabei die Sollhöhe der Box, das zweite
% Argument ist der Boxinhalt, das dritte Argument is der Name des Elements,
% das zu hoch ist, das vierte der Name der Elementhöhe und das fünfte
% schließlich der Befehl, mit dem das Element vom Anwender definiert werden
% kann. Das optionale Argument wird nur bei der Ausgabe mit ausgegeben, findet
% jedoch bei der Berechnung keine Berücksichtigung.
%    \begin{macrocode}
\newif\if@savevbox@warning\@savevbox@warningtrue
\newcommand{\@savevbox}[6][]{%
  \begingroup%
    \vbox to #2{#1#3}%
    \if@savevbox@warning\@savevbox@warningfalse%
      \setbox\z@\vbox{\color@begingroup #3\color@endgroup}%
      \setlength{\@tempdima}{#2}%
      \setlength{\@tempdimb}{\ht0}\addtolength{\@tempdimb}{\dp0}%
      \ifdim \@tempdimb>\@tempdima%
        \addtolength{\@tempdimb}{-\@tempdima}%
%<class>        \ClassWarning{scrlttr2}{%
%<package>        \PackageWarning{scrletter}{%
          #4 is \the\@tempdimb\space too high.\MessageBreak
          You have to change `\string#5'\MessageBreak
          or you have to define a smaller\MessageBreak
          #4 using \string#6.\MessageBreak
          Because of this too high #4\MessageBreak
          you've got an overfull \string\vbox\space message%
        }%
      \fi%
    \fi%
  \endgroup%
}
%    \end{macrocode}
% \end{macro}
% \end{macro}
% \end{macro}
% \end{macro}
%
% \begin{macro}{\@firstheadfootfield}
% \changes{v2.8n}{2001/09/08}{Neu}%^^A
% \changes{v2.8q}{2001/10/30}{\cs{parfillskip} lokal auf 0 gesetzt}
% \changes{v2.8q}{2001/10/30}{Fuß explizit in eine Box gesetzt}%^^A
% \changes{v2.9l}{2003/01/23}{Positionierung verbessert}%^^A
% \changes{v2.97e}{2007/11/22}{Optionen \texttt{firsthead} und
%   \texttt{firstfoot} beachten}%^^A
% \changes{v3.05}{2009/11/09}{neue Pseudo-Länge \texttt{firstheadhpos}
%   beachtet}%^^A
% \changes{v3.05}{2009/11/09}{neue Pseudo-Länge \texttt{firstfoothpos}
%   beachtet}%^^A
% Ab Version 2.8n werden Kopf und Fuß der ersten Seite nicht mehr per
% Seitenstil \texttt{firstpage} gesetzt, sondern mit einem speziellen
% Feld. Dabei wird auch beachtet, dass dieses Feld im Kopf eine bestimmte aber
% beschränkte Höhe hat. Diese Höhe ist unabhängig von der normalen Kopfhöhe
% und richtet sich stattdessen nach der Position des Adressfeldes.
% \selectlanguage{english}%^^A
% \changes{v3.27}{2019/03/22}{package only enlarges page}%^^A
% From \KOMAScript~3.27 the \textsf{scrlayer} package does not print the page
% head or foot here. It only enlarges or shrinks the first page.
% \selectlanguage{ngerman}%^^A
%    \begin{macrocode}
\newcommand*{\@firstheadfootfield}{%
%<*class|!willbe>
  \if@firsthead
    \typeout{Head of first page}%
    \move@topt\vskip\useplength{firstheadvpos}%
    \vb@t@z{%
      \rlap{\move@topl
%    \end{macrocode}
% Wenn \texttt{firstheadhpos} positiv und kleiner als die Seitenbreite ist,
% dann ist es der Abstand des Kopfes vom linken Rand.
% Wenn \texttt{firstheadhpos} negativ und betragsmäßig kleiner als die
% Seitenbreite ist, dann ist es der Abstand des Kopfes vom rechten Rand.
% Wenn \texttt{firstheadhpos} größer als die Seitenbreite ist, dann wird der
% Kopf zentriert.
% Wenn \texttt{firstheadhpos} kleiner als die negative Seitenbreite ist, dann
% wird der Kopf am linken Rand des Satzspiegels ausgerichtet.
%    \begin{macrocode}
        \ifdim\useplength{firstheadhpos}<\paperwidth
          \ifdim \useplength{firstheadhpos}>-\paperwidth
            \ifdim \useplength{firstheadhpos}<\z@
              \setlength\@tempskipa{\paperwidth}%
              \addtolengthplength{\@tempskipa}{firstheadhpos}%
              \addtolengthplength[-]{\@tempskipa}{firstheadwidth}%
            \else
              \setlength{\@tempskipa}{\useplength{firstheadhpos}}%
            \fi
          \else
            \setlength\@tempskipa{\oddsidemargin}%
            \addtolength\@tempskipa{1in}%
          \fi
        \else
          \setlength\@tempskipa{.5\paperwidth}%
          \addtolengthplength[-.5]{\@tempskipa}{firstheadwidth}%
        \fi
        \hskip\@tempskipa
        \vbox{\hsize\useplength{firstheadwidth}%
          \setlength{\@tempskipa}{\useplength{toaddrvpos}}%
          \addtolengthplength[-1]{\@tempskipa}{firstheadvpos}%
          \addtolength{\@tempskipa}{-\baselineskip}%
%    \end{macrocode}
% Da sich einige -- beispielsweise |briefkopf.sty| -- darauf verlassen, dass
% |\textwidth| die Breite des Briefkopfes ist, wird diese Länge lokal
% entsprechend gesetzt. Außerdem wird dafür gesorgt, dass |\parfillskip| hier
% nicht berücksichtigt wird.
%    \begin{macrocode}
          \setlength{\textwidth}{\useplength{firstheadwidth}}%
          \setlength{\parfillskip}{\z@ \@plus 1fil}%
          \@savevbox{\@tempskipa}{\noindent\usekomavar{firsthead}%
            \vskip \z@ \@plus .00001fil}%
          {head of first page}{firstheadvpos}{\setkomavar}%
        }%
      }%
    }%
    \vskip-\useplength{firstheadvpos}\move@frompt
  \else
    \typeout{No head at first page}%
  \fi
  \move@topt
  \if@firstfoot
    \typeout{Foot of first page}%
    \vb@t@z{%
      \rlap{\move@topl
%    \end{macrocode}
% Wenn \texttt{firstfoothpos} positiv und kleiner als die Seitenbreite ist,
% dann ist es der Abstand des Kopfes vom linken Rand.
% Wenn \texttt{firstfoothpos} negativ und betragsmäßig kleiner als die
% Seitenbreite ist, dann ist es der Abstand des Kopfes vom rechten Rand.
% Wenn \texttt{firstfoothpos} größer als die Seitenbreite ist, dann wird der
% Kopf zentriert.
% Wenn \texttt{firstfoothpos} kleiner als die negative Seitenbreite ist, dann
% wird der Kopf am linken Rand des Satzspiegels ausgerichtet.
%    \begin{macrocode}
         \ifdim\useplength{firstfoothpos}<\paperwidth
          \ifdim \useplength{firstfoothpos}>-\paperwidth
            \ifdim \useplength{firstfoothpos}<\z@
              \setlength\@tempskipa{\paperwidth}%
              \addtolengthplength{\@tempskipa}{firstfoothpos}%
              \addtolengthplength[-]{\@tempskipa}{firstfootwidth}%
            \else
              \setlength{\@tempskipa}{\useplength{firstfoothpos}}%
            \fi
          \else
            \setlength\@tempskipa{\oddsidemargin}%
            \addtolength\@tempskipa{1in}%
          \fi
        \else
          \setlength\@tempskipa{.5\paperwidth}%
          \addtolengthplength[-.5]{\@tempskipa}{firstfootwidth}%
        \fi
        \hskip\@tempskipa
        \vbox{\hsize\useplength{firstfootwidth}%
          \vskip\useplength{firstfootvpos}%
%    \end{macrocode}
% Da sich einige -- beispielsweise |briefkopf.sty| -- darauf verlassen, dass
% |\textwidth| die Breite des Brieffußes ist, wird diese Länge lokal
% entsprechend gesetzt. Außerdem wird dafür gesorgt, dass |\parfillskip| hier
% nicht berücksichtigt wird.
%    \begin{macrocode}
          \setlength{\textwidth}{\useplength{firstfootwidth}}%
          \setlength{\parfillskip}{\z@ \@plus 1fil}%
          \vbox{\hsize\useplength{firstfootwidth}\noindent
            \usekomavar{firstfoot}}%
        }%
      }%
    }%
  \else
    \typeout{No foot at first page}%
  \fi
  \move@frompt%
%</class|!willbe>
  {%
    \setlength{\@tempdima}{1in}\addtolength{\@tempdima}{\topmargin}%
    \addtolength{\@tempdima}{\headheight}%
    \addtolength{\@tempdima}{\headsep}%
    \addtolength{\@tempdima}{\textheight}%
    \addtolength{\@tempdima}{\footskip}%
    \if@firstfoot
      \addtolengthplength[-1]{\@tempdima}{firstfootvpos}%
    \else
      \addtolength{\@tempdima}{-\paperheight}%
    \fi
    \ifdim\@tempdima>\z@\enlargethispage{-\@tempdima}%
    \else\if@enlargefp\enlargethispage{-\@tempdima}\fi\fi
  }%
}
%    \end{macrocode}
% \end{macro}
%
% \begin{macro}{\@addrfield}
% \changes{v2.8q}{2002/04/01}{\cs{parfillskip} lokal geändert}%^^A
% \changes{v2.8q}{2002/06/03}{Sonderbehandlung für Adressfeld rechts}%^^A
% \changes{v2.9o}{2003/02/20}{\cs{noindent} eingefügt bei
%   \texttt{specialmail}}%^^A
% \changes{v2.96}{2006/07/31}{Benutzung von \cs{backaddr@format}}%^^A
% \changes{v3.03}{2009/03/04}{Priorität kann Versandart ersetzen}%^^A
% \changes{v3.03}{2009/03/04}{Neue Anschriftmodi \texttt{PP},
%   \texttt{backgroundimage}, \texttt{addresseeimage}}%^^A
% \changes{v3.03b}{2009/06/30}{Schreibfehler bei »addressee« korrigiert}%^^A
% \changes{v3.06a}{2010/07/29}{Absatzparameter werden passend
%   eingestellt}%^^A
% \changes{v3.22}{2016/12/21}{zweimal fehlende Klammern bei \texttt{toname}
%   ergänzt}%^^A
% \changes{v3.22}{2016/12/21}{zweimal fehlende Klammern und \cs{endgraf} bei
%   \texttt{toaddress} ergänzt}%^^A
% \changes{v3.27}{2019/03/27}{only at \textsf{scrlttr2}}%^^A
% \changes{v3.28}{2019/11/15}{\cs{ifkomavarempty} durch \cs{Ifkomavarempty}
%   ersetzt}%^^A
% \changes{v3.30}{2020/03/27}{Umdefinierung des Zeilenumbruchs
%   verschoben}%^^A
% Dasselbe gilt auch für das Adressfeld für Fensterumschläge. Die
% Rücksendeadresse und die Briefart werden dabei mit eigenen
% Formatieranweisungen gesetzt.
%    \begin{macrocode}
%<*class|!willbe>
\newcommand*{\@addrfield}{%
  \ifcase \ifnum\@afieldmode=5 \@ne\else\@afieldmode\fi\relax
    \typeout{Address: No}%
  \or
    \begingroup%
      \if@baddr
        \Ifkomavarempty{backaddress}{%
          \let\@tempa\@empty\let\@tempb\@empty
        }{%
          \def\@tempa{backaddress}\def\@tempb{,\space}%
        }%
      \else
        \let\@tempa\@empty\let\@tempb\@empty
      \fi
      \ifcase \ltr@priority
        \Ifkomavarempty{specialmail}{}{%
          \edef\@tempa{\@tempa\@tempb specialmail}\def\@tempb{,\space}%
        }%
      \or
        \edef\@tempa{\@tempa\@tempb B-Economy}\def\@tempb{,\space}%
      \or
        \edef\@tempa{\@tempa\@tempb A-Priority}\def\@tempb{,\space}%
      \fi
      \typeout{Address (\@tempa\@tempb addressee)}%
    \endgroup%
    \move@topt\vskip\useplength{toaddrvpos}%
    \vb@t@z{{\setparsizes{\z@}{\z@}{\z@ plus 1fil}\par@updaterelative
        \rlap{\move@topl
          \setlengthtoplength{\@tempskipa}{toaddrhpos}%
          \ifdim \@tempskipa<\z@
            \addtolength{\@tempskipa}{\paperwidth}%
            \addtolengthplength[-]{\@tempskipa}{toaddrwidth}%
          \fi
          \hskip\@tempskipa
          \@savevbox{\useplength{toaddrheight}}{\hsize\useplength{toaddrwidth}%
            \@savevbox[\vss]{\useplength{backaddrheight}}{%
              \hsize\useplength{toaddrwidth}%
              \if@baddr
                \Ifkomavarempty{backaddress}{}{{%
                    \noindent%
                    \backaddr@format{\scriptsize
                      {%
                        \def\\{\usekomavar{backaddressseparator}\@ogobble}%
                        \usekomafont{backaddress}%
                        {\usekomavar{backaddress}}%
                      }}}}%
              \fi
            }{backaddress}{backaddrheight}{\setkomavar}%
            \ifcase \ltr@priority
            \or
              \setkomavar{specialmail}{%
                {\usekomafont{priority}{%
                    {\usekomafont{prioritykey}{B}}%
                    -ECONOMY}}%
              }%
            \or
              \setkomavar{specialmail}{%
                {\usekomafont{priority}{%
                    {\usekomafont{prioritykey}{A}}%
                    -PRIORITY}}%
              }%
            \fi
            \Ifkomavarempty{specialmail}{}{%
              \vbox{\hsize\useplength{toaddrwidth}%
                \noindent\hspace*{\useplength{specialmailindent}}%
                \specialmail@format{%
                  \usekomafont{specialmail}\usekomavar{specialmail}}%
                \hspace*{\useplength{specialmailrightindent}}}%
            }%
            \ifnum\@afieldmode=\@ne\vfil\fi%
            \noindent\hskip\useplength{toaddrindent}%
            \setlength{\@tempdima}{\useplength{toaddrwidth}}%
            \addtolengthplength[-2]{\@tempdima}{toaddrindent}%
            \vbox{\hsize\@tempdima\raggedright
              \usekomafont{addressee}{%
                {\usekomafont{toname}{\usekomavar{toname}\\}}%
                {\usekomafont{toaddress}{\usekomavar{toaddress}\endgraf}}}}%
            \vfil\null\vskip-\baselineskip}%
          {address field}{toaddrheight}%
          {\begin{letter} or you\MessageBreak
            have to define a smaller special purpose using\MessageBreak
            \string\specialmail}%
        }}}%
    \vskip-\useplength{toaddrvpos}\move@frompt
  \else
    \begingroup%
%    \end{macrocode}
% In den Modi \texttt{PP}, \texttt{backgroundimage} oder \texttt{addresseeimage}
% gibt es weder Rücksendeadresse noch Versandart aber ebenfalls eine Priorität.
%    \begin{macrocode}
      \if@baddr
        \let\@tempa\@empty\let\@tempb\@empty
        \ifcase \@afieldmode\relax\or\or
          \def\@tempa{PP-mode}\def\@tempb{,\space}%
        \or
          \def\@tempa{backgroundimage-mode}\def\@tempb{,\space}%
        \or
          \def\@tempa{image-mode}\def\@tempb{,\space}%
        \fi
        \Ifkomavarempty{backaddress}{}{%
          \edef\@tempa{\@tempa\@tempb backaddress ignored}\def\@tempb{,\space}%
        }%
      \else
        \let\@tempa\@empty\let\@tempb\@empty
      \fi
      \ifcase \ltr@priority
        \Ifkomavarempty{specialmail}{}{%
          \edef\@tempa{\@tempa\@tempb specialmail ignored}\def\@tempb{,\space}%
        }%
      \or
        \edef\@tempa{\@tempa\@tempb B-Economy}\def\@tempb{,\space}%
      \or
        \edef\@tempa{\@tempa\@tempb A-Priority}\def\@tempb{,\space}%
      \fi
      \typeout{Address (\@tempa\@tempb addressee)}%
    \endgroup%
    \move@topt\vskip\useplength{toaddrvpos}%
    \vb@t@z{{%
        \setparsizes{\z@}{\z@}{\z@ plus 1fil}\par@updaterelative
        \ifdim \useplength{toaddrindent}=\z@
          \@setplength{toaddrindent}{8mm}%
        \fi
        \ifdim \useplength{PPheadheight}=\z@
          \@setplength{PPheadheight}{20.74pt}%
        \fi
        \ifdim \useplength{PPheadwidth}=\z@
          \@setplength{PPheadwidth}{42mm}%
        \fi
        \ifdim \useplength{PPdatamatrixvskip}=\z@
          \@setplength{PPdatamatrixvskip}{9mm}%
        \fi
        \rlap{\move@topl
          \setlengthtoplength{\@tempskipa}{toaddrhpos}%
          \ifdim \@tempskipa<\z@
            \addtolength{\@tempskipa}{\paperwidth}%
            \addtolengthplength[-]{\@tempskipa}{toaddrwidth}%
          \fi
          \hskip\@tempskipa
          \@savevbox{\useplength{toaddrheight}}{\hsize\useplength{toaddrwidth}%
            \ifnum\@afieldmode=\tw@
%    \end{macrocode}
% Modus \texttt{PP}.
%    \begin{macrocode}
              \vskip 2pt\relax
              \hskip \useplength{toaddrindent}\relax
              \fbox{%
                \vbox to \dimexpr \useplength{PPheadheight}-2\fboxsep\relax{%
                  \setlength{\parindent}{0pt}%
                  \hsize \dimexpr \useplength{PPheadwidth}-2\fboxsep\relax
%    \end{macrocode}
% \changes{v3.17}{2015/04/17}{\cs{setbox} durch \cs{sbox} ersetzt}%^^A
% \changes{v3.25}{2017/11/15}{Verwendung von \cs{scr@endstrut} statt
%   \cs{unskip}\cs{strut}}%^^A
%    \begin{macrocode}
                  \sbox\@tempboxa{%
                    {\fontsize{20.74pt}{20.74pt}\selectfont
                      \usekomafont{PPlogo}{P.P.}}\hskip 2mm%
                  }%
                  \dp\@tempboxa\z@
                  \setlength{\@tempdima}{\hsize}%
                  \addtolength{\@tempdima}{-\wd\@tempboxa}%
                  \hbox{\box\@tempboxa
                    \vbox to \dimexpr
                    \useplength{PPheadheight}-2\fboxsep\relax{%
                      \vss\hsize\@tempdima
                      \raggedright
                      \fontsize{8pt}{9pt}\selectfont\usekomafont{PPdata}%
                      \usekomavar*{fromzipcode}\usekomavar{zipcodeseparator}%
                      \usekomavar{fromzipcode}\scr@endstrut\\
                      \usekomavar{place}\scr@endstrut
                      \hfill
                    }%
                  }%
                  \vss
                }%
              }%
              \hskip 1mm
              \mbox{%
                \vbox to \useplength{PPheadheight}{%
                  \addtolengthplength[-]{\hsize}{PPheadwidth}%
                  \addtolength{\hsize}{-1mm}%
                  \addtolengthplength[-2]{\hsize}{toaddrindent}%
                  \setlength{\parindent}{0pt}%
                  \fontsize{8pt}{9pt}\selectfont\usekomafont{PPdata}%
                  \hfill\usekomavar{PPcode}
                  \vss
                  \ifcase \ltr@priority
                  \or
                    \usekomafont{priority}{%
                      {\usekomafont{prioritykey}{B}}%
                      -ECONOMY\hss\vskip-\dp\strutbox
                    }%
                  \or
                    \usekomafont{priority}{%
                      {\usekomafont{prioritykey}{A}}%
                      -PRIORITY\hss\vskip-\dp\strutbox
                    }%
                  \fi
                  \vss
                }%
              }%
              \vskip 1mm
              \vskip-\ht\strutbox
              \noindent\hskip\useplength{toaddrindent}%
              \hrulefill%
              \hskip\useplength{toaddrindent}%
              \vbox to\z@{\hsize\z@
                \vskip \useplength{PPdatamatrixvskip}\relax
                \llap{\usekomavar{PPdatamatrix}\hskip\useplength{toaddrindent}}%
                \vss
              }%
            \else
%    \end{macrocode}
% Modus \texttt{backgroundimage} oder \texttt{image}.
%    \begin{macrocode}
              \vskip 1mm
              \noindent\expandafter\hskip\useplength{toaddrindent}%
              \vbox to \useplength{PPheadheight}{%
                \addtolengthplength[-2]{\hsize}{toaddrindent}%
                \setlength{\parindent}{0pt}%
                \usekomavar{addresseeimage}%
                \vss
              }%
              \vskip 1mm
              \vskip\dp\strutbox
            \fi
            \ifnum \@afieldmode=4 \else
              \vfil%
              \noindent\hskip\useplength{toaddrindent}%
              \setlength{\@tempdima}{\useplength{toaddrwidth}}%
              \addtolengthplength[-2]{\@tempdima}{toaddrindent}%
              \vbox{\hsize\@tempdima
                \Ifkomavarempty{PPdatamatrix}{}{%
                  \addtolength{\hsize}{-2mm}%
                  \settowidth\@tempdima{\usekomavar{PPdatamatrix}}%
                  \addtolength{\hsize}{-\@tempdima}%
                }%
                \raggedright
                \usekomafont{addressee}{%
                  {\usekomafont{toname}{\usekomavar{toname}\\}}%
                  {\usekomafont{toaddress}{\usekomavar{toaddress}\endgraf}}}}%
            \fi
            \vss\null}%
          {address field}{toaddrheight}%
          {\string\begin{letter}}%
        }}}%
    \vskip-\useplength{toaddrvpos}\move@frompt
  \fi
}
%</class|!willbe>
%    \end{macrocode}
% \begin{macro}{\specialmail@format}
% \changes{v2.97c}{2007/07/06}{neu (intern)}%^^A
%   Die Anweisung, die für das Aussehen der Versandart zuständig ist.
%    \begin{macrocode}
\newcommand*{\specialmail@format}[1]{}
\let\specialmail@format\underline
%    \end{macrocode}
% \end{macro}
%
% \begin{macro}{\@ogobble}
% \changes{v2.8q}{2002/01/22}{Neu}%^^A
% In obiger Definition wird das Makro \cs{@ogobble} verwendet, um
% einen optionalen Parameter zu vernichten, falls dieser existiert.
%    \begin{macrocode}
\newcommand*{\@ogobble}[1][]{}
%    \end{macrocode}
% \end{macro}
% \end{macro}
%
% \begin{macro}{\@locfield}
% \changes{v2.8o}{2001/09/17}{\cs{noindent} eingefügt}
% \changes{v2.8q}{2002/02/07}{\cs{noindent} nach dem Logo}
% \changes{v2.8q}{2002/07/23}{Berechnung der Breite hierher verschoben}%^^A
% \changes{v2.8q}{2002/06/03}{Sonderbehandlung für Adressfeld rechts}%^^A
% \changes{v3.27}{2019/03/27}{only at \textsf{scrlttr2}}%^^A
% \changes{v3.28}{2019/11/15}{\cs{ifkomavarempty} durch \cs{Ifkomavarempty}
%   ersetzt}%^^A
% Das "`locfield"' wird unabhängig von der Adresse gesetzt.
%    \begin{macrocode}
%<*class|!willbe>
\newcommand*{\@locfield}{%
  \ifdim \useplength{locwidth}=\z@%
    \@setplength{locwidth}{\paperwidth}%
    \ifdim \useplength{toaddrhpos}>\z@
      \@addtoplength[-2]{locwidth}{\useplength{toaddrhpos}}%
    \else
      \@addtoplength[2]{locwidth}{\useplength{toaddrhpos}}%
    \fi
    \@addtoplength[-1]{locwidth}{\useplength{toaddrwidth}}%
    \if@bigloc%
      \@setplength[.66667]{locwidth}{\useplength{locwidth}}%
    \else%
      \@setplength[.5]{locwidth}{\useplength{locwidth}}%
    \fi%
  \fi%
%    \end{macrocode}
% \changes{v2.97e}{2007/11/13}{Der Absender kann nun ebenfalls in der
%     Absenderergänzung stehen}%^^A
% \changes{v2.98b}{2008/01/30}{Absenderergänzung immer setzen, wenn
%     verwendet}%^^A
% Je nach Absenderanordnung muss hier nun ebenfalls etwas getan werden.
% Der Fall 0, also linksbündiger Absender im Kopf spielt hier keine Rolle.
%    \begin{macrocode}
  \ifcase\@alignfrom
%    \end{macrocode}
% Im Fall 1, also zentrierter Absender im Kopf wird jedoch das Logo hier
% platziert, also darf die Absenderergänzung nicht leer sein.
%    \begin{macrocode}
  \or
    \if@logo
      \Ifkomavarempty{location}{\setkomavar{location}{\null}}{}%
    \fi
%    \end{macrocode}
% Im Fall 2, also rechtsbündiger Absender im Kopf, ist auch nichts zu tun.
%    \begin{macrocode}
  \or
%    \end{macrocode}
% Im Fall 3, also linksbündiger Absender in der Absenderergänzung wird hier
% eine Menge ausgegeben, also darf die Absenderergänzung nicht leer sein.
%    \begin{macrocode}
  \or
    \Ifkomavarempty{location}{\setkomavar{location}{\null}}{}%
%    \end{macrocode}
% Im Fall 3, also linksbündiger Absender in der Absenderergänzung wird hier
% eine Menge ausgegeben, also darf die Absenderergänzung nicht leer sein.
%    \begin{macrocode}
  \or
    \Ifkomavarempty{location}{\setkomavar{location}{\null}}{}%
%    \end{macrocode}
% Alle anderen Fälle sind wie hier ebenfalls ohne Bedeutung.
%    \begin{macrocode}
  \fi
  \Ifkomavarempty{location}{%
    \typeout{Location field: empty}%
  }{%
    \typeout{Location field: yes}%
    \ifdim \useplength{lochpos}=\z@%
      \@setplength{lochpos}{\useplength{toaddrhpos}}%
    \fi
    \ifdim \useplength{locvpos}=\z@%
      \@setplength{locvpos}{\useplength{toaddrvpos}}%
    \fi
    \ifdim \useplength{locheight}=\z@%
      \@setplength{locheight}{\useplength{toaddrheight}}%
    \fi
    \move@topt\vskip\useplength{locvpos}%
    \vb@t@z{\noindent\makebox[0pt][l]{%
        \move@topl
        \setlengthtoplength[-]{\@tempskipa}{lochpos}%
        \ifdim \@tempskipa<\z@
          \addtolength{\@tempskipa}{\paperwidth}%
        \else
          \addtolengthplength{\@tempskipa}{locwidth}%
        \fi
        \hskip\@tempskipa
        \makebox[0pt][r]{%
          \@savevbox{\useplength{locheight}}{%
            \noindent\hsize\useplength{locwidth}%
%    \end{macrocode}
% In einigen Fällen der Absenderanordnung muss hier nun eine ganze Menge getan
% werden. In anderen ist hingegen nur wenig zu tun.
% Im Fall 0, also linksbündiger Absender im Kopf wird nur die
% Absenderergänzung ausgegeben (das erfolgt nach der Fallunterscheidung).
%    \begin{macrocode}
            \ifcase\@alignfrom
%    \end{macrocode}
% Im Fall 1, also zentrierter Absender im Kopf wird jedoch das Logo hier
% platziert, die Absenderergänzung wird ggf. danach ausgegeben (das erfolgt
% nach der Fallunterscheidung).
% \changes{v2.98a}{2008/01/10}{bei zentriertem Absender wird die
%   Absenderergänzung ggf. unter dem Logo ausgegeben}%^^A
%    \begin{macrocode}
            \or
              \if@logo
                \Ifkomavarempty{fromlogo}{\null}{%
                  {\raggedleft\noindent\usekomavar{fromlogo}\\
                    \vskip\parskip}\noindent}%
              \fi
%    \end{macrocode}
% Im Fall 2, also rechtsbündiger Absender im Kopf, wird auch nur die
% Absenderergänzung ausgegeben (das erfolgt nach der Fallunterscheidung).
%    \begin{macrocode}
            \or
%    \end{macrocode}
% Im Fall 3, also linksbündiger Absender in der Absenderergänzung wird hier
% sozusagen der linksbündige Kopf nachgebaut. Das Logo steht dabei
% ggf. rechtsbündig oben.
%    \begin{macrocode}
            \or
              \if@logo
                \Ifkomavarempty{fromlogo}{}{%
                  \raggedleft\usekomavar{fromlogo}\par}%
              \fi
              \raggedright
%    \end{macrocode}
% Zuerst wird der Name gesetzt. Dabei soll eine eigene Schriftart für
% den Namen verwendet werden.
%    \begin{macrocode}
              {\usekomafont{fromname}\strut\ignorespaces
                \usekomavar{fromname}%
%    \end{macrocode}
% Hat \verb|\@rulefrom| den nummerischen Wert 1, so soll direkt auf den
% Absender eine Linie folgen. Die Länge dieser Linie kann vom Anwender
% bestimmt oder automatisch gewählt werden. Es werden dabei die
% Pseudo-Längen \texttt{fromrulewidth} und \texttt{fromrulethickness}
% verwendet. Ist \texttt{fromrulewidth} 0, so wird sie automatisch
% bestimmt. Bei der automatischen Längenbestimmung wird einfach eine Linie
% über die Gesamtbreite der Absenderergänzung gesetzt.
%    \begin{macrocode}
                {%
                  \ifnum\@rulefrom=1\\[-.5\baselineskip]%
                    \usekomafont{fromrule}{%
                      \setlength{\@tempdima}{\useplength{fromrulewidth}}%
                      \ifdim\@tempdima=\z@
                        \setlength{\@tempdima}{\useplength{locwidth}}%
                      \fi
                      \expandafter\rule\expandafter{\the\@tempdima}{%
                        \useplength{fromrulethickness}}%
                    }%
                  \fi
                }%
                \\
              }%
%    \end{macrocode}
% Als Nächstes wird die Absenderadresse einschließlich aller
% optionalen Felder gesetzt. Dabei werden optionale Felder nur
% gesetzt, wenn der entsprechende Schalter \emph{true} ist. Die
% Adresse wird ebenfalls innerhalb einer Gruppe in einer eigenen
% Schriftart gesetzt.%
% \changes{v3.12}{2013/05/29}{Variable \texttt{frommobilephone} wird
%     berücksichtigt}%
%    \begin{macrocode}
              {\usekomafont{fromaddress}\strut\ignorespaces
                \usekomavar{fromaddress}%
                \if@phone\\\usekomavar*{fromphone}\usekomavar{fromphone}\fi%
                \if@mobilephone
                  \\\usekomavar*{frommobilephone}\usekomavar{frommobilephone}%
                \fi%
                \if@fax\\\usekomavar*{fromfax}\usekomavar{fromfax}\fi%
                \if@email\\\usekomavar*{fromemail}\usekomavar{fromemail}\fi%
                \if@www\\\usekomavar*{fromurl}\usekomavar{fromurl}\fi%
%    \end{macrocode}
% Wie weiter oben erwähnt, kann auch an dieser Stelle eine Linie
% gesetzt  werden.
%    \begin{macrocode}
                \ifnum\@rulefrom=2\\[-.5\baselineskip]%
                  \usekomafont{fromrule}{%
                    \setlength{\@tempdima}{\useplength{fromrulewidth}}%
                    \ifdim\@tempdima=\z@
                      \setlength{\@tempdima}{\useplength{locwidth}}%
                    \fi
                    \expandafter\rule\expandafter{\the\@tempdima}{%
                      \useplength{fromrulethickness}}}%
                \fi
                \\
              }%
%    \end{macrocode}
% Damit ist die Definition des linksbündigen Absenders in der
% Absenderergänzung abgeschlossen.
% Im Fall 4, also rechtsbündiger Absender in der Absenderergänzung wird hier
% exakt wie für Fall 3 verfahren. Nur ist jetzt alles rechtsbündig.%
% \changes{v3.12}{2013/05/29}{Variable \texttt{frommobilephone} wird
%     berücksichtigt}%
%    \begin{macrocode}
            \or
              \raggedleft
              \if@logo
                \Ifkomavarempty{fromlogo}{}{\usekomavar{fromlogo}\par}%
              \fi
              {\usekomafont{fromname}\strut\ignorespaces
                \usekomavar{fromname}%
                {%
                  \ifnum\@rulefrom=1\\[-.5\baselineskip]%
                    \usekomafont{fromrule}{%
                      \setlength{\@tempdima}{\useplength{fromrulewidth}}%
                      \ifdim\@tempdima=\z@
                        \setlength{\@tempdima}{\useplength{locwidth}}%
                      \fi
                      \expandafter\rule\expandafter{\the\@tempdima}{%
                        \useplength{fromrulethickness}}%
                    }%
                  \fi
                }%
                \\
              }%
              {\usekomafont{fromaddress}\strut\ignorespaces
                \usekomavar{fromaddress}%
                \if@phone\\\usekomavar*{fromphone}\usekomavar{fromphone}\fi%
                \if@mobilephone%
                  \\\usekomavar*{frommobilephone}\usekomavar{frommobilephone}%
                \fi%
                \if@fax\\\usekomavar*{fromfax}\usekomavar{fromfax}\fi%
                \if@email\\\usekomavar*{fromemail}\usekomavar{fromemail}\fi%
                \if@www\\\usekomavar*{fromurl}\usekomavar{fromurl}\fi%
                \ifnum\@rulefrom=2\\[-.5\baselineskip]%
                  \usekomafont{fromrule}{%
                    \setlength{\@tempdima}{\useplength{fromrulewidth}}%
                    \ifdim\@tempdima=\z@
                      \setlength{\@tempdima}{\useplength{locwidth}}%
                    \fi
                    \expandafter\rule\expandafter{\the\@tempdima}{%
                      \useplength{fromrulethickness}}}%
                \fi
                \\
              }%
%    \end{macrocode}
% In allen anderen Fällen wird ebenfalls die Absenderergänzung gesetzt.
%    \begin{macrocode}
            \fi
            \usekomavar{location}%
%    \end{macrocode}
% Zum Schluss wird noch verhindert, dass wegen nicht vollen vertikalen Boxen
% gemotzt wird.
%    \begin{macrocode}
            \vskip \z@ \@plus .00001fil}%
          {location field}{locheight}{\setkomavar}%
        }}%
    }%
    \vskip-\useplength{locvpos}\move@frompt
  }%
}
%</class|!willbe>
%    \end{macrocode}
% \end{macro}
%
% \begin{macro}{\@datefield}
% \changes{v2.8q}{2002/01/14}{Schalter \cs{if@tempwa} wird bereits hier auf
%     \texttt{true} gesetzt}
% \changes{v2.97c}{2007/05/14}{Neue Fontelemente}%^^A
% Je nachdem, welche Felder definiert sind, muss eine geschäftsmäßiger
% Standardbrief oder ein eher persönliches Aussehen gewählt werden.
% \begin{macro}{\do@set@reffield}
% \changes{v3.09}{2011/03/06}{Neu (intern)}%^^A
% Das ist ein \cs{do} für die Felder und bekommt innerhalb von
% \cs{@datefield} unterschiedliche Bedeutungen. An dieser Stelle wird es daher
% nur reserviert.
%    \begin{macrocode}
\newcommand*{\do@set@reffield}{}
%    \end{macrocode}
% \begin{macro}{\@set@reffield}
% \changes{v3.17}{2015/04/17}{\cs{setbox} durch \cs{sbox} ersetzt}
% Einzelne Felder werden mit Hilfe dieses Makros gesetzt. Als Parameter wird
% der Name des Feldes übergeben. Es wird dann davon ausgegangen, dass der
% Name des Feldes in der jeweiligen Sprache im Makro
% \verb|\|\texttt{\emph{\#1}}\verb|name| und der Wert des Feldes in
% \verb|\var|\texttt{\emph{\#1}} zu finden ist.
%    \begin{macrocode}
\newcommand*{\@set@reffield}[1]{%
  \Ifkomavarempty{#1}{}{%
    \settowidth{\@tempdima}{\strut\usekomafont{refvalue}\usekomavar{#1}}%
    \sbox\z@{\strut\usekomafont{refname}\usekomavar*{#1}}%
    \ifdim \@tempdima<\wd0\setlength{\@tempdima}{\wd0}\fi
    \if@tempswa\enskip\hfill\fi
    \parbox[t]{\@tempdima}{\noindent\raggedright
        \strut\box\z@\\\usekomafont{refvalue}\strut\usekomavar{#1}}%
    \@tempswatrue
  }%
}
%    \end{macrocode}
% \end{macro}
% \begin{macro}{\@test@reffield}
% \changes{v3.09}{2011/03/06}{Neu (intern)}%^^A
% Dieses Makro testet, ob ein Feld tatsächlich zu setzen ist und setzt
% ggf. \cs{@tempswa}.
%    \begin{macrocode}
\newcommand*{\@test@reffield}[1]{%
  \Ifkomavarempty{#1}{}{%
%<trace>    \if@tempswa\l@addto@macro\@tempa{, }\fi
%<trace>    \l@addto@macro\@tempa{#1}%
    \@tempswatrue
  }%
}
%    \end{macrocode}
% \end{macro}
% \begin{macro}{\@set@reffields}
% \changes{v2.8q}{2002/01/14}{Neu}%^^A
% Dieses Makro enthält alle Aufrufe von \cs{@set@reffield}, die durchzuführen
% sind. Das Makro kann mit Hilfe von \cs{addtoreffields} erweitert werden.
%    \begin{macrocode}
\newcommand*{\@set@reffields}{}%
%    \end{macrocode}
% \end{macro}
% \begin{macro}{\addtoreffields}
% \changes{v3.09}{2011/03/06}{Verwendung von \cs{do@set@reffield} an Stelle
%     von \cs{@set@reffield}}
% Dieses Makro erwartet als obligatorisches Argument den Namen einer
% KOMA-Script Variable. Der Befehl arbeitet lokal zur aktuellen Umgebung.
%    \begin{macrocode}
\newcommand*{\addtoreffields}[1]{%
  \l@addto@macro{\@set@reffields}{\do@set@reffield{#1}}%
}
%    \end{macrocode}
% \end{macro}
%
% \begin{macro}{\removereffields}
% \changes{v2.8q}{2002/01/14}{Neu}%^^A
% Es gibt auch einen Befehl, um alle Felder aus der Geschäftszeile zu
% entfernen.
%    \begin{macrocode}
\newcommand*{\removereffields}{%
  \renewcommand*{\@set@reffields}{}%
}
%    \end{macrocode}
% \end{macro}
%
% \begin{macro}{\defaultreffields}
% \changes{v2.8q}{2002/01/14}{Neu}%^^A
% Ebenso ein Befehl, um die üblichen Felder zu setzen.
%    \begin{macrocode}
\newcommand*{\defaultreffields}{%
  \removereffields
  \addtoreffields{yourref}%
  \addtoreffields{yourmail}%
  \addtoreffields{myref}%
  \addtoreffields{customer}%
  \addtoreffields{invoice}%
}
%    \end{macrocode}
% Genau diesen Befehl rufen wir auch gleich auf:
%    \begin{macrocode}
\defaultreffields
%    \end{macrocode}
% \end{macro}
%
% \changes{v2.3e}{1996/05/31}{Abstand zwischen Ort und Datum eingefügt}%^^A
% \changes{v2.8o}{2001/09/18}{für den Brieftyp \texttt{KOMAold} anschließend
%   einen zusätzlichen vertikalen Abstand eingefügt}%^^A
% \changes{v2.8q}{2001/11/13}{\cs{rlap} eingefügt}%^^A
% \changes{v2.8q}{2001/11/13}{\cs{@parskipfalse}\cs{@parskip@indent}}%^^A
% \changes{v2.8q}{2002/01/14}{Leertests entfernt, da in \cs{@set@reffield}
%   bereits enthalten}%^^A
% \changes{v2.9l}{2003/01/28}{Berechnung von \texttt{refwidth} an die
%   Dokumentation angepasst}%^^A
% \changes{v2.9l}{2003/01/28}{Berechnung von \texttt{refhpos} hinzugefügt}%^^A
% \changes{v2.95}{2006/03/23}{Absatzeinzug mit \cs{setparsizes} abschalten}%^^A
% \changes{v2.97c}{2007/05/14}{Neue Font-Elemente}%^^A
% \changes{v3.09}{2011/03/06}{\cs{do@set@reffield} muss initialisiert
%   werden}%^^A
% \changes{v3.28}{2019/11/15}{\cs{ifkomavarempty} durch \cs{Ifkomavarempty}
%   ersetzt}%^^A
%    \begin{macrocode}
\newcommand*{\@datefield}{%
  {%
    \setparsizes{\z@}{\z@}{\z@ plus 1fil}\par@updaterelative
    \ifdim\useplength{refwidth}=\z@
      \if@refwide
        \@setplength{refwidth}{\paperwidth}%
        \ifdim \useplength{toaddrhpos}>\z@
          \@setplength{refhpos}{\useplength{toaddrhpos}}%
        \else
          \@setplength[-]{refhpos}{\useplength{toaddrhpos}}%
        \fi
        \@addtoplength[-2]{refwidth}{\useplength{refhpos}}%
      \else
        \@setplength{refwidth}{\textwidth}%
        \@setplength{refhpos}{\oddsidemargin}%
        \@addtoplength{refhpos}{1in}%
      \fi
    \else
      \ifdim\useplength{refhpos}=\z@
        \begingroup
%    \end{macrocode}
% Wir bestimmen zunächst, um wieviel die Geschäftszeile schmaler ist als der
% Textbereich.
%    \begin{macrocode}
          \setlength\@tempdima{\textwidth}%
          \addtolengthplength[-]{\@tempdima}{refwidth}%
%    \end{macrocode}
% Dann wird bestimmt, wie groß der linke Rand tatsächlich ist.
%    \begin{macrocode}
          \setlength\@tempdimb{\oddsidemargin}%
          \addtolength\@tempdimb{1in}%
%    \end{macrocode}
% Ebenso wird bestimmt, wie groß der linke und rechte Rand zusammen sind.
%    \begin{macrocode}
          \setlength{\@tempdimc}{\paperwidth}%
          \addtolength{\@tempdimc}{-\textwidth}%
%    \end{macrocode}
% Jetzt wird aus dem Gesamtrand der Geschäftszeile der linke Rand der
% Geschäftszeile so bestimmt, dass das Randverhältnis dabei unverändert
% bleibt. Dabei erlauben wir uns eine gewisse Rechenungenauigkeit beim
% Verhältnis.
%    \begin{macrocode}
          \divide\@tempdimb by 32768\relax
          \divide\@tempdimc by 32768\relax
          \ifdim\@tempdimc=\z@\else
            \multiply\@tempdima by\@tempdimb
            \divide\@tempdima by\@tempdimc
          \fi
          \edef\@tempa{\noexpand\endgroup
            \noexpand\@setplength{refhpos}{\the\@tempdima}}%
        \@tempa
    \fi\fi
    \@tempswafalse
    \move@topt\vskip\useplength{refvpos}%
%    \end{macrocode}
% Jetzt kontrollieren wir, ob überhaupt etwas auszugeben ist:
%    \begin{macrocode}
    \@tempswafalse
    \let\do@set@reffield\@test@reffield
%<trace>    \let\@tempa\@empty
    \@set@reffields
    \ifcase \@datepos\else\@test@reffield{date}\fi
    \if@tempswa
%<trace>      \typeout{Reference line(s) with: \@tempa}%
      \@tempswafalse
%    \end{macrocode}
% Jetzt erfolgt noch die Absatzeinzugkorrektur. Danach kommt die eigentliche
% Arbeit.
%    \begin{macrocode}
      \rlap{\noindent\move@topl\hskip\useplength{refhpos}%
        \vbox{\hsize\useplength{refwidth}%
          \noindent
          \ifnum \@datepos=2
            \@set@reffields
            \if@tempswa
              \@tempswafalse
              \@set@reffield{date}%
            \fi
          \fi
          \let\do@set@reffield\@set@reffield
          \@set@reffields%
          \if@tempswa
            \ifnum \@datepos=1 \@set@reffield{date}\fi%
%    \end{macrocode}
% Seit Version~2.97c wird hier auch noch getestet, ob die Geschäftszeile
% möglicherweise mit zu großer Schrift gesetzt wird.
% \changes{v3.17}{2015/04/17}{\cs{setbox} durch \cs{sbox} ersetzt}%^^A
%    \begin{macrocode}
            \sbox\z@{\usekomafont{refname}\strutbox}%
            \ifdim\ht\z@>\ht\strutbox
%<class>              \ClassWarningNoLine{scrlttr2}{%
%<package>              \PackageWarningNoLine{scrletter}{%
                Font element `refname' should not be larger than
                \string\normalsize!}%
            \fi
            \sbox\z@{\usekomafont{refvalue}\strutbox}%
            \ifdim\ht\z@>\ht\strutbox
%<class>              \ClassWarningNoLine{scrlttr2}{%
%<package>              \PackageWarningNoLine{scrletter}{%
                Font element `refvalue' should not be larger than
                \string\normalsize!}%
            \fi
          \else
%    \end{macrocode}
% Hier landen wir schließlich, wenn gar keine Felder ausgegeben wurden, also
% nur Ort und Datum zu setzen sind.
% \changes{v3.12}{2013/07/31}{neues Element \texttt{placeanddate} wird
%   verwendet}%^^A
%    \begin{macrocode}
            \ifcase \@datepos\else
              \ifnum \@datepos=1 \enskip\hfill\fi
              {\usekomafont{placeanddate}{\Ifkomavarempty{place}{}{%
                    \usekomavar{place}\usekomavar{placeseparator}}%
                  \usekomavar{date}}}%
            \fi
          \fi
        }%
      }%
%<trace>    \else\typeout{No reference line.}%
    \fi
    \vskip\useplength{refaftervskip}%
  }%
}
%    \end{macrocode}
% \end{macro}
% \end{macro}
%
% \begin{macro}{\@maketitle}
% \changes{v2.3b}{1996/01/14}{Verwendung von \cs{sectfont} für den
%     \cs{title}}%^^A
% \changes{v2.8p}{2001/09/22}{\cs{titlefont} an Stelle von \cs{sectfont}}%^^A
% \changes{v2.8q}{2001/11/13}{\cs{@parskipfalse}\cs{@parskip@indent}}%^^A
% \changes{v2.9t}{2004/09/16}{\cs{par} in Größe und Schrift des Titels}%^^A
% \changes{v2.95}{2006/03/22}{\cs{usekomafont} and Stelle von
%     \cs{titlefont}}%^^A
% \changes{v3.13}{2014/06/26}{umbenannt in \cs{@makelettertitle}}%^^A
% \end{macro}
% \begin{macro}{\@makelettertitle}
% \changes{v3.13}{2014/06/26}{neu (ehemals \cs{@maketitle}}%^^A
% \changes{v3.15}{2015/02/21}{\texttt{title} durch \texttt{lettertitle}
%     ersetzt}%^^A
% \changes{v3.28}{2019/11/15}{\cs{ifkomavarempty} durch \cs{Ifkomavarempty}
%   ersetzt}%^^A
% \begin{macro}{\@subjfield}
% \changes{v2.8q}{2001/10/19}{\cs{centerline} durch \cs{centering}
%     ersetzt}%^^A
% \changes{v2.8q}{2001/11/13}{\cs{@parskipfalse}\cs{@parskip@indent}}%^^A
% \changes{v2.8q}{2002/01/31}{wird mit hängendem Einzug gesetzt}%^^A
% \changes{v2.95}{2006/03/23}{Absatzeinzug mit cs{setparsizes}%^^A
%     einstellen}%^^A
% \changes{v3.01}{2008/11/03}{Verwendung der neuen Pseudolängen
%     \texttt{subjectbeforevskip} und \texttt{subjectaftervskip}}%^^A
% \changes{v3.15}{2015/02/21}{\texttt{subject} durch \texttt{lettersubject}
%     ersetzt}%^^A
% \changes{v3.28}{2019/11/15}{\cs{ifkomavarempty} durch \cs{Ifkomavarempty}
%   ersetzt}%^^A
% Der eigentliche Titel wir mit |\@maketitle| gesetzt. Dies sollte jedoch
% ebenso wie beim Betreff (|\@subjfield|) \emph{nicht} manuell geschehen.
%    \begin{macrocode}
\newcommand*{\@makelettertitle}{%
  \Ifkomavarempty{title}{%
    \typeout{Title: no}%
  }{%
    \typeout{Title: yes}%
    {\setparsizes{\z@}{\z@}{\z@ plus 1fil}\par@updaterelative
      \centering\LARGE\usekomafont{lettertitle}\usekomavar{title}\par}%
    \vskip\baselineskip%
  }%
}
\newcommand*{\@subjfield}{{%
    \setparsizes{\z@}{\z@}{\z@ plus 1fil}\par@updaterelative
    \Ifkomavarempty{subject}{%
      \typeout{Subject: no}%
    }{%
      \ifx\raggedsubject\@empty
        \ifdim\useplength{subjectvpos}=\z@
          \if@subjafter%
            \typeout{Subject: after opening}%
            \let\raggedsubject\centering%
          \else
            \typeout{Subject: before opening}%
            \def\raggedsubject{\noindent\@hangfrom}%
          \fi
        \else
          \typeout{Subject: at absolute vertical position}%
          \def\raggedsubject{\noindent\@hangfrom}%
        \fi
      \fi
      \ifdim\useplength{subjectvpos}=\z@
        \vskip\useplength{subjectbeforevskip}%
        \begingroup
      \else
        \move@topt\vskip\useplength{subjectvpos}%
        \expandafter\vb@t@z\expandafter{%
      \fi
      \usekomafont{lettersubject}%
      \raggedsubject{\if@subj \subject@format{\strut\ignorespaces
          \usekomavar*{subject}}\fi}%
      \subject@format{\strut\ignorespaces\usekomavar{subject}}\par
      \ifdim\useplength{subjectvpos}=\z@
        \endgroup
        \vskip\useplength{subjectaftervskip}%
      \else
        }%
        \vskip-\useplength{subjectvpos}\move@frompt
      \fi
    }%
  }%
}
%    \end{macrocode}
% \begin{macro}{\ltr@len@subjectvpos}
% \changes{v3.01}{2008/11/03}{Neue Pseudolänge \texttt{subjectvpos}}
%   Manchmal ist es günstiger, wenn man den Betreff absolut positionieren
%   kann. Das geschieht über diese Pseudolänge, wenn sie \emph{nicht} 0 ist.
%    \begin{macrocode}
\@newplength{subjectvpos}
\@setplength{subjectvpos}{0pt}
%    \end{macrocode}
% \end{macro}
% \begin{macro}{\ltr@len@subjectaftervskip}
% \changes{v3.01}{2008/11/03}{Neue Pseudolänge \texttt{subjectaftervskip}}
%   Abstand nach dem Betreff. Dieser Abstand wird aus Gründen der
%   Kompatibilität zu früheren Versionen mit 2\cs{baselineskip}
%   voreingestellt.
%    \begin{macrocode}
\@newplength{subjectaftervskip}
\@setplength{subjectaftervskip}{2\baselineskip}
%    \end{macrocode}
% \end{macro}
% \begin{macro}{\ltr@len@subjectbeforevskip}
% \changes{v3.01}{2008/11/03}{Neue Pseudolänge \texttt{subjectbeforevskip}}
%   Abstand zusätzlich vor dem Betreff. Dieser Abstand wird aus Gründen der
%   Kompatibilität zu früheren Versionen mit \cs{baselineskip}
%   voreingestellt. Es sei darauf hingewiesen, dass hier bereits ein Abstand
%   beispielsweise vom Titel existieren kann!
%    \begin{macrocode}
\@newplength{subjectbeforevskip}
\@setplength{subjectbeforevskip}{0pt}
%    \end{macrocode}
% \end{macro}
% \end{macro}
% \end{macro}
%
%
% \subsection{Anweisungen für das Briefpapier}
%
% Für das Briefpapier gibt es etwas weniger Anweisungen. Teilweise werden hier
% auch die Anweisungen für den Briefbogen mit verwendet.
%
% \subsubsection{Kopf des Briefpapiers}
%
% Der Kopf des Briefpapiers wird wieder ganz normal über den Seitenstil
% festgelegt. Allerdings kann er entsprechende dem Briefbogen manuell
% definiert werden.
%
% \begin{macro}{\nexthead}
% \changes{v3.08}{2010/10/28}{der Kopf der Folgeseiten wird über eine
%     Variable realisiert}%^^A
% \begin{macro}{\@nexthead}
% \changes{v3.08}{2010/10/28}{der Kopf der Folgeseiten wird über eine
%     Variable realisiert}%^^A
% \begin{KOMAvar}{nexthead}
% \changes{v3.08}{2010/10/28}{Neu}%^^A
% Die Voreinstellung ist ebenfalls von Optionen abhängig.
%    \begin{macrocode}
%<*class>
\newcommand*{\@nexthead}{%
  \ClassWarning{scrlttr2}{%
    Deprecated usage of \string\@nexthead.\MessageBreak
    \string\usekomavar{nexthead} should be used\MessageBreak
    instead of \string\@nexthead
  }%
  \usekomavar{nexthead}%
}
%</class>
\newkomavar{nexthead}
\setkomavar{nexthead}{%
  \rlap{\parbox[c]{\textwidth}{\raggedright\strut
      \ifnum\@pageat=0\pagemark\else\leftmark\fi\strut}}%
  \parbox[c]{\textwidth}{%
    \centering\strut
    \ifcase\@pageat
      \rightmark
    \or
      \pagemark
    \or
      \rightmark
    \fi\strut}%
  \llap{\parbox[c]{\textwidth}{\raggedleft\strut
      \ifcase\@pageat
        \leftmark
      \or
        \rightmark
      \or
        \pagemark
      \else
        \rightmark
      \fi\strut}}%
}%
%    \end{macrocode}
% \end{KOMAvar}
% \end{macro}
% Die Voreinstellung ist wiederum mit einer einfachen Benutzeranweisung
% änderbar.
%    \begin{macrocode}
%<*class>
\newcommand{\nexthead}{%
  \ClassWarning{scrlttr2}{%
    Deprecated usage of \string\nexthead.\MessageBreak
    You should use \string\setkomavar{nexthead}\MessageBreak
    instead of \string\nexthead
  }
  \setkomavar{nexthead}%
}
%</class>
%    \end{macrocode}
% \end{macro}
%
%
% \subsubsection{Fuß des Briefpapiers}
%
% \begin{macro}{\nextfoot}
% \changes{v3.08}{2010/10/28}{der Fuß der Folgeseiten wird über eine
%     Variable realisiert}%^^A
% \begin{macro}{\@nextfoot}
% \changes{v3.08}{2010/10/28}{der Fuß der Folgeseiten wird über eine
%     Variable realisiert}%^^A
% \begin{KOMAvar}{nextfoot}
% \changes{v3.08}{2010/10/28}{Neu}%^^A
% Je nach Optioneneinstellung kann sich die Seitenmarke in der Voreinstellung
% auch im Fuß befinden.
%    \begin{macrocode}
%<*class>
\newcommand*{\@nextfoot}{%
  \ClassWarning{scrlttr2}{%
    Deprecated usage of \string\@nextfoot.\MessageBreak
    \string\usekomavar{nextfoot} should be used\MessageBreak
    instead of \string\@nextfoot
  }%
  \usekomavar{nextfoot}%
}
%</class>
\newkomavar{nextfoot}
\setkomavar{nextfoot}{%
  \ifnum\@pageat>2
    \parbox[c]{\textwidth}{%
      \ifcase\@pageat\or\or\or\raggedright\or\centering\or\raggedleft\fi
      \strut\pagemark%
      }%
  \fi
}
%    \end{macrocode}
% \end{KOMAvar}
% \end{macro}
% Auch diese Einstellung ist per Benutzeranweisung leicht zu ändern.
%    \begin{macrocode}
%<*class>
\newcommand{\nextfoot}{%
  \ClassWarning{scrlttr2}{%
    Deprecated usage of \string\nextfoot.\MessageBreak
    You should use \string\setkomavar{nextfoot}\MessageBreak
    instead of \string\nextfoot
  }
  \setkomavar{nextfoot}%
}
%</class>
%    \end{macrocode}
% \end{macro}
%
%
% \subsection{Die Brief-Umgebung}
%
% \begin{environment}{letter}
% \changes{v2.8q}{2002/01/16}{optionales Argument hinzugefügt}%^^A
% \changes{v2.8q}{2002/01/16}{Fußnotenzähler wird zurückgesetzt}%^^A
% \changes{v2.9d}{2002/08/12}{Fußnotenzähler wird korrekt zurückgesetzt}%^^A
% \changes{v3.17}{2015/03/20}{Ersatz von \cs{pagemark}}%^^A
% \changes{v3.19}{2015/07/15}{\cs{letterpagestyle} wird verwendet}%^^A
% \changes{v3.19}{2015/07/15}{Briefe werden referenzierbar nummeriert}%^^A
% Diese Umgebung ist etwas anders definiert, als normalerweise von
% \LaTeX gewohnt. Mit ihr können innerhalb eines Dokuments beliebig
% viele Briefe erzeugt werden. Eine der Abweichungen von der
% Standardklasse besteht darin, dass als optionales Argument Optionen
% gesetzt werden können.
%    \begin{macrocode}
\newenvironment{letter}[2][]{%
%    \end{macrocode}
% \changes{v3.27}{2019/03/05}{zweipaltige Briefe werden nicht unterstützt}%^^A
% Falls wir uns im zweispaltigen Modus befinden, ist es jetzt höchste Zeit,
% eine Warnung auszugeben und auf einspaltig umzuschalten. Zwar wäre es
% theoretisch noch möglich, alles innerhalb von \cs{opening} in das optionale
% Argument von \cs{twocolumn} zu packen, aber spätestens beim Schlussgruß
% scheitert das dann. Also besser gleich gar nicht unterstützen.
%    \begin{macrocode}
  \if@twocolumn
%<class>    \ClassWarning{\KOMAClassName}{%
%<package>    \PackageWarning{scrletter}{%
      twocolumn mode not supported.\MessageBreak
      If you really want a twocolumned letter, you should\MessageBreak
      use package `multicol' and add a `multicols'\MessageBreak
      environment inside your letter.\MessageBreak
      Switching back to oneside%
    }%
    \letter@restonecoltrue
    \expandafter\onecolumn
  \else
    \letter@restonecolfalse
  \fi
%    \end{macrocode}
% Als erstes wird eine neue Seite begonnen. Dies kann ggf. auch eine neue
% rechte Seite sein. Dabei wird auch auf den korrekten Seitenstil
% geachtet. Danach werden alle relevanten Zähler und Strafpunkte
% zurückgesetzt und der obligatorische Parameter ausgewertet.
% \changes{v3.19}{2015/07/15}{wegen des Pakets
%   \cs{cleardoubleoddemptypage}}%^^A
% \changes{v3.27}{2019/03/05}{Briefe verwenden \cs{raggedbottom}}%^^A
%    \begin{macrocode}
  \cleardoubleoddemptypage
  \raggedbottom
  \scr@ifundefinedorrelax{letterpagestyle}{}{%
    \ifx\letterpagestyle\@empty\else
      \pagestyle{\letterpagestyle}%
    \fi
  }%
%<package>  \let\pagemark\letterpagemark
  \refstepcounter{letter}%
  \let\thisletter\@currentlabel
  \setcounter{page}{\@ne}%
  \setcounter{footnote}{0}%
  \interlinepenalty=200\@processto{#2}%
%    \end{macrocode}
% Dann werden ggf. die übergebenen Optionen ausgeführt.
%    \begin{macrocode}
  \KOMAoptions{#1}%
%    \end{macrocode}
% \selectlanguage{english}%^^A
% \changes{v3.27}{2019/03/31}{first-page head generation moved}%^^A
% From \KOMAScript~3.27 \textsf{scrletter} generates the first-page head in
% the \texttt{letter}-environment instead of the command \cs{opening}.
%    \begin{macrocode}
%<package&willbe>  \@gen@firsthead
%    \end{macrocode}
% \changes{v3.27}{2019/03/22}{\textsf{scrletter} sets the page style}%^^A
% From \KOMAScript~3.27 \textsf{scrletter} sets page style
% \texttt{notepaper} for the first page before executing the begin hook.
% \selectlanguage{ngerman}%^^A
%    \begin{macrocode}
%<package&willbe>  \thispagestyle{notepaper}%
%    \end{macrocode}
% Zum Schluss werden die per \cs{AtBeginLetter} angegeben Anweisungen
% ausgeführt.
%    \begin{macrocode}
%<*package&willbe>
  \ifdim\useplength{refheight}>\z@
  \else
    \begingroup
      \newkomavar[\strut]{@internal@testvar@}%
      \setkomavar{@internal@testvar@}{\strut}%
      \setplengthtototalheight{refheight}{\@set@reffield{@internal@testvar@}}%
      \edef\reserved@a{\noexpand\endgroup
        \noexpand\setplength{refheight}{\useplength{refheight}}%
      }%
    \reserved@a
  \fi
%</package&willbe>
  \@beginletterhook
%<package&willbe>\initial@letter@vskip
}{%
  \stopletter
%    \end{macrocode}
% Danach wird noch ein Label gesetzt.
% \changes{v3.19}{2015/07/15}{ein Label wird gesetzt}%^^A
%    \begin{macrocode}
  \let\@currentlabel\thisletter
  \label{\thisletter.lastpage}%
  \@@par\pagebreak\@@par%
  \ifletter@restonecol\twocolumn\fi
%    \end{macrocode}
% \changes{v3.27}{2019/03/31}{cleanup running head}%^^A
% The package also cleans up the running head after the end of the letter.
%    \begin{macrocode}
%<package&willbe>  \@mkdouble\@empty
}
%    \end{macrocode}
% \begin{macro}{\initial@letter@vskip}
% \changes{v3.27}{2019/05/03}{new (internal)}%^^A
% Go to the start posistion of the letter. Usually to jump below the reference
% line inside the \texttt{letter} environment.
%    \begin{macrocode}
%<*package&willbe>
\newcommand*{\initial@letter@vskip}{%
  \move@topt
  \vskip \dimexpr \useplength{refvpos}
                  \if@refline +\useplength{refheight}\fi
                  +\useplength{refaftervskip}\relax
}
%</package&willbe>
%    \end{macrocode}
% \end{macro}
% \begin{macro}{\ifletter@restonecol}
% \changes{v3.27}{2019/03/05}{neu intern}
% Dieser Schalter bestimmt, ob am Ende der \texttt{letter}-Umgebung auf
% zweispaltig zurückgeschaltet werden muss.
%    \begin{macrocode}
\newif\ifletter@restonecol
%    \end{macrocode}
% \end{macro}
% \begin{macro}{\thisletter}
% \changes{v3.19}{2015/07/15}{neue Anweisung}%^^A
% Gibt den Wert des aktuellen Labels wieder. Man benötigt dies beispielsweise,
% um per |\ref{\thisletter.lastpage}| auf die letzte Seite zugreifen zu
% können.
%    \begin{macrocode}
\newcommand*{\thisletter}{outside.letter.undefined}
%    \end{macrocode}
% \end{macro}%^^A \thisletter
% \begin{macro}{\letterlastpage}
% \changes{v3.19}{2015/07/15}{neue Anweisung}%^^A
%    \begin{macrocode}
\newcommand*{\letterlastpage}{\pageref{\thisletter.lastpage}}
%    \end{macrocode}%^^A \letterlastpage
% \end{macro}
% \begin{macro}{\stopletter}
% Damit man Material am Ende des Briefes einschleusen kann.
%    \begin{macrocode}
\newcommand*{\stopletter}{}
%    \end{macrocode}
% \end{macro}%^^A \stopletter
% \begin{macro}{\@processto}
% \changes{v2.9r}{2004/06/16}{robust gegen leeren Empfänger}%^^A
% \changes{v3.27}{2019/03/24}{extended setting of variables}%^^A
% \changes{v3.28}{2019/11/15}{\cs{ifkomavarempty} durch \cs{Ifkomavarempty}
%   ersetzt}%^^A
% \begin{macro}{\@xproc}
% Die Adresse, die als Argument der Briefumgebung übergeben wird, muss in Name
% und Adresse zerlegt werden.
%    \begin{macrocode}
%    \begin{macrocode}
\newcommand{\@processto}[1]{%
%    \end{macrocode}
% If \cs{@processto} argument (\texttt{letter} mandatory argument) is empty:
%    \begin{macrocode}
  \IfArgIsEmpty{#1}{%
%    \end{macrocode}
% try to build it from \texttt{toname} and \texttt{toaddress}.
%    \begin{macrocode}
%<*package&willbe>
    \Ifkomavarempty{toname}{%
      \Ifkomavarempty{tofamilyname}{}{%
        \Ifkomavarempty{togivenname}{%
          \usekomavar[\setkomavar{toname}]{tofamilyname}%
        }{%
          % Fixme: The \cs{protected@edef}s are critical!
          \usekomavar[\protected@edef\reserved@a]{togivenname}%
          \usekomavar[\protected@edef\reserved@b]{tofamilyname}%
          \edef\reserved@a{%
            \noexpand\setkomavar{toname}{%
              \unexpanded\expandafter{\reserved@a}\ %
              \unexpanded\expandafter{\reserved@b}}%
          }\reserved@a
        }%
      }%
    }{}%
    \Ifkomavarempty{toaddress}{%
      % Fixme: The \cs{protected@edef}s are critical!
      \usekomavar[\protected@edef\reserved@a]{toname}%
      \protected@edef\reserved@a{%
        \noexpand\usekomafont{toname}{\reserved@a}%
      }\reserved@a
    }{%
      \Ifkomavarempty{toname}{%
        % Fixme: The \cs{protected@edef}s are critical!
        \usekomavar[\protected@edef\reserved@a]{toaddress}%
        \protected@edef\reserved@a{%
          \noexpand\usekomafont{toaddress}{\reserved@a}%
        }\reserved@a
      }{%
        % Fixme: The \cs{protected@edef}s are critical!
        \usekomavar[\protected@edef\scr@local@toname]{toname}%
        \usekomavar[\protected@edef\reserved@b]{toaddress}%
        \protected@edef\reserved@a{%
          \noexpand\setkomavar{addressee}{%
            {\noexpand\usekomafont{toname}{\scr@local@toname}}\\
            \noexpand\usekomafont{toaddress}{\reserved@b}%
          }%
        }\reserved@a
      }%
    }%
%</package&willbe>
  }{%
%    \end{macrocode}
% Split into part before first \cmd\\ and it:
%    \begin{macrocode}
    \expandafter\@xproc#1\\\@nil
    \ifx\reserved@b\@empty\else\expandafter\@xproc#1\@nil\fi
%    \end{macrocode}
% For the package set variable \texttt{addressee}:
%    \begin{macrocode}
%<*package&willbe>
    % Fixme: The \cs{protected@edef}s are critical!
    \protected@edef\reserved@c{%
      \noexpand\setkomavar{addressee}{%
        {\noexpand\usekomafont{toname}{\reserved@a}}%
        \ifx\reserved@b\@empty\else
          \\\noexpand\usekomafont{toaddress}{\reserved@b}%
        \fi
      }%
    }%
%    \end{macrocode}
% If \texttt{toname} is empty, set it:
%    \begin{macrocode}
    \Ifkomavarempty{toname}{%
      % Fixme: The \cs{protected@edef}s are critical!
      \protected@edef\reserved@a{%
        \noexpand\reserved@c{\noexpand\setkomavar{toname}{\reserved@a}}}%
    }{\let\reserved@a\reserved@c}%
%    \end{macrocode}
% If \texttt{toaddress} is empty, set it:
%    \begin{macrocode}
    \Ifkomavarempty{toaddress}{%
      % Fixme: The \cs{protected@edef}s are critical!
      \protected@edef\reserved@b{\noexpand\reserved@a
        \noexpand\setkomavar{toaddress}{\reserved@b}}%
      \reserved@b
    }{\reserved@a}%
%</package&willbe>
%    \end{macrocode}
% For the class simply set \texttt{toname} und \texttt{toaddress}:
% \changes{v3.27}{2019/10/28}{don't expand the contents of \cs{reserved@a} and
%   \cs{reserved@b}}%^^A
%    \begin{macrocode}
%<*class|!willbe>
    \edef\reserved@a{%
      \noexpand\setkomavar{toname}{\unexpanded\expandafter{\reserved@a}}%
      \noexpand\setkomavar{toaddress}{\unexpanded\expandafter{\reserved@b}}%
    }\reserved@a
%</class|!willbe>
  }%
}
\newcommand*{\@xproc}{}
\long\def\@xproc#1\\#2\@nil{%
%    \end{macrocode}
% \changes{v3.27}{2019/03/24}{do not directly set variables but
%   \cs{reserved@a} and \cs{reserved@b}}%^^A
%    \begin{macrocode}
  \def\reserved@a{#1}\def\reserved@b{#2}%
}
%    \end{macrocode}
% \end{macro}%^^A \@xproc
% \end{macro}%^^A \@processto
% \begin{macro}{\flushbottom}
% \changes{v2.97c}{2007/05/17}{neue Erweiterung}%^^A
% Das Verhalten muss auf der letzten Briefseite wieder \cs{raggedbottom}
% ähneln. Das wird über \cs{stopletter} erreicht.
%    \begin{macrocode}
\g@addto@macro\flushbottom{%
  \g@addto@macro\stopletter{\vskip \z@\@plus.0001fil\relax}%
}
%    \end{macrocode}
% \end{macro}%^^A \flushbottom
% \begin{Counter}{letter}
% \changes{v3.19}{2015/07/15}{neuer Zähler}%^^A
% Die Briefe werden durchnummeriert.
%    \begin{macrocode}
\newcounter{letter}
%    \end{macrocode}
% \end{Counter}
% \end{environment}%^^A letter
%
%
% \begin{macro}{\letterpagestyle}
% \changes{v3.19}{2015/07/15}{neue Anweisung}%^^A
% Analog zu \cs{titlepagestyle}, \cs{indexpagestyle} etc. gibt es jetzt auch
% ein \cs{letterpagestyle}. In der Voreinstellung ist dieser bei der Klasse
% leer. Dagegen wird beim Paket \texttt{plain.letter} verwendet. Durch den
% Unterschied wird faktisch die Kompatibilität des Pakets mit der Klasse
% jedoch verbessert!
%    \begin{macrocode}
\newcommand*{\letterpagestyle}{%
%<package>  plain.letter%
}
%    \end{macrocode}
% \end{macro}
%
%
% \begin{macro}{\AtBeginLetter}
% \changes{v2.8n}{2001/09/08}{Neu}%^^A
% \begin{macro}{\@beginletterhook}
% \changes{v2.8n}{2001/09/08}{Neu}%^^A
% Wie oben zu sehen, wurde in den Anfang der Briefumgebung ein Haken
% eingefügt. Dieser wird auch tatsächlich intern genutzt. Er funktioniert
% prinzipiell wie der |\AtBeginDocument|-Haken. Zunächst macht der Haken rein
% gar nichts.
%    \begin{macrocode}
\newcommand{\AtBeginLetter}{\g@addto@macro\@beginletterhook}
\newcommand{\@beginletterhook}{}
%    \end{macrocode}
% \end{macro}
% \end{macro}
%
% \begin{macro}{\AtEndLetter}
% \changes{v2.95}{2006/03/22}{Neu}%^^A
% Analog zu |\AtBeginLetter| gibt es auch ein |\AtEndLetter|. Als Haken wird
% hierbei |\stopletter| verwendet.
%    \begin{macrocode}
\newcommand{\AtEndLetter}{\g@addto@macro\stopletter}
%    \end{macrocode}
% \end{macro}
% 
% \begin{macro}{\stopbreaks}
% \begin{macro}{\startbreaks}
% \changes{v3.33}{2021/03/05}{\cs{relax} added}
% \begin{macro}{\@nobreakvspace}
% \changes{v3.20}{2016/04/12}{\cs{@ifstar} durch \cs{kernel@ifstar}
%   ersetzt}
% \changes{v3.33}{2021/03/05}{robust definition}
% \begin{macro}{\@nobreakvspacex}
% \begin{macro}{\@nobreakcr}
% \changes{v3.20}{2016/04/12}{\cs{@ifstar} durch \cs{kernel@ifstar}
%   ersetzt}
% \changes{v3.33}{2021/03/05}{simplified}
% Innerhalb von Briefen wird der Umbruch etwas anders gehandhabt.
% Dadurch soll der Brieftext besser positioniert werden.
%    \begin{macrocode}
\newcommand*{\stopbreaks}{\interlinepenalty \@M
 \def\par{\@@par\nobreak}\let\\=\@nobreakcr
 \let\vspace\@nobreakvspace}
\DeclareRobustCommand*{\@nobreakvspace}{%
  \kernel@ifstar{\@nobreakvspacex}{\@nobreakvspacex}}
\newcommand*{\@nobreakvspacex}[1]{\ifvmode\nobreak\vskip #1\relax\else
 \@bsphack\vadjust{\nobreak\vskip #1}\@esphack\fi}
\newcommand*{\@nobreakcr}{\kernel@ifstar{\@normalcr*}{\@normalcr*}}
\newcommand*{\startbreaks}{\let\\=\@normalcr
 \interlinepenalty 200\def\par{\@@par\penalty 200\relax}}
%    \end{macrocode}
% \end{macro}
% \end{macro}
% \end{macro}
% \end{macro}
% \end{macro}
%
% \begin{macro}{\opening}
% \changes{v2.8n}{2001/09/08}{Komplett überarbeitet}%^^A
% \changes{v2.8q}{2001/11/13}{\cs{@parskipfalse}\cs{@parskip@indent}}%^^A
% \changes{v2.8q}{2002/03/25}{\cs{@PapersizeWarning}}%^^A
% \changes{v2.95}{2003/03/23}{Absatzformatierung mit \cs{setparsizes}
%   einstellen}%^^A
% \changes{v3.27}{2019/03/27}{Only \textsf{scrlttr2} uses \cs{@addrfield},
%   \cs{@locfield} and \cs{@foldmarks}}%^^A
% \changes{v3.27}{2019/03/27}{Only \textsf{scrlttr2} sets pagestyle}%^^A
% \changes{v3.27}{2019/03/31}{Only \textsf{scrlttr2} generates the first-page
%   head here}%^^A
% Denn beim Eröffnungsgruß mit |\opening| wird neben sämtlichen Feldern auch
% der Titel und der Betreff gesetzt.
%    \begin{macrocode}
\newcommand{\opening}[1]{%
  \@PapersizeWarning
%<class|!willbe>  \@gen@firsthead
  \@mkboth{\scr@fromname@var}{\scr@subject@var}%
%<class|!willbe>  \thispagestyle{empty}%
  \noindent
%<class|!willbe>  \@foldmarks\@firstheadfootfield
%<class|!willbe>  \@addrfield\@locfield
  \ifdim\useplength{subjectvpos}=\z@\else\@subjfield\fi
%<class|!willbe>  \@datefield
  \@makelettertitle
  \ifdim\useplength{subjectvpos}=\z@\if@subjafter\else\@subjfield\fi\fi
  {\setparsizes{\z@}{\z@}{\z@ plus 1fil}\par@updaterelative#1\par}%
  {\setlength{\@tempdima}{\baselineskip}%
    \addtolength{\@tempdima}{-\parskip}%
    \ifdim \@tempdima>\z@\vskip\@tempdima\fi}%
  \ifdim\useplength{subjectvpos}=\z@\if@subjafter\@subjfield\fi\fi
  \@afterindentfalse\@afterheading
}
%    \end{macrocode}
% \end{macro}
%
%
% \begin{macro}{\closing}
% \changes{v2.2b}{1995/02/16}{{\cmd\fromsig} wird nicht mehr auf \cs{empty}
%     getestet sondern direkt eingesetzt}
% \changes{v2.3b}{1996/01/14}{Abstand zwischen Schlussgruß und Signatur
%     erhöht}%^^A
% \changes{v2.5d}{2000/06/10}{Abstand zwischen Schlussgruß und Signatur
%     variabel}%^^A
% \changes{v2.8q}{2001/11/13}{\cs{@parskipfalse}\cs{@parskip@indent}}
% \changes{v2.8q}{2002/02/05}{nicht mehr \cs{long}}
% \changes{v2.8q}{2002/02/05}{darf \texttt{\cmd{\\}[\dots]} enthalten}
% \changes{v2.95}{2003/03/23}{Absatzformatierung mit \cs{setparsizes}%^^A
%     einstellen}
% Außer dem Eröffnungsgruß mit |\opening| gibt es natürlich auch einen
% Schlussgruß mit |\closing|, bei dem außerdem die Unterschrift u.\,ä. gesetzt
% wird.
% \begin{macro}{\presig@skip}
% \changes{v2.5d}{2000/06/10}{Neues Makro, das den Abstand zwischen
%     Schlussgruß und Signatur beinhaltet}%^^A
% \changes{v2.8q}{2001/10/09}{Ersetzt durch Pseudo-Länge sigbeforevskip}%^^A
% \end{macro}
% \begin{macro}{\setpresigskip}
% \changes{v2.5d}{2000/06/10}{Neues Makro, das den Abstand zwischen
%     Schlussgruß und Signatur setzt}%^^A
% \changes{v2.8q}{2001/10/09}{Mit Hilfe der neuen Pseudo-Längen definiert}%^^A
% Das Makro |\presig@skip| beinhaltet dabei den Abstand zwischen Schlussgruß
% und Signatur. Da der Wert über ein Benutzerinterface per |\setpresigskip|
% gesetzt wird, muss hier kein kostbares Längenregister (skip oder dim)
% verschwendet werden, sondern ein Makro reicht aus. In |\setpresigskip| wird
% allerdings noch sichergestellt, dass die Fehlermeldung bei falschem
% Parameter möglichst nicht erst bei Verwendung auftritt.
%    \begin{macrocode}
%</letter>
%<lco&KOMAold>\def\setpresigskip{\@setplength{sigbeforevskip}}
%<*letter>
%    \end{macrocode}
% \end{macro}
% \begin{macro}{\@xtrimtempa}
% \changes{v2.8q}{2002/02/05}{Neu (intern)}%^^A
% \changes{v3.17}{2014/02/19}{entfernt}%^^A
% \begin{macro}{\@@xtrimtempa}
% \changes{v2.8q}{2002/02/05}{Neu (intern)}%^^A
% \changes{v3.17}{2014/02/19}{entfernt}%^^A
% \end{macro}
% \end{macro}
% \begin{macro}{\@tempdimaismaxwidthof}
% \changes{v2.8q}{2002/02/05}{Neu (intern)}%^^A
% \changes{v3.17}{2014/02/19}{entfernt}%^^A
% \end{macro}
% Jetzt endlich zum eigentlichen Befehl.
% \changes{v2.9r}{2004/01/28}{Längenbegrenzung auf \cs{hsize}}%^^A
% \changes{v2.9r}{2004/01/28}{das Argument darf leer sein}%^^A
% \changes{v3.17}{2015/02/19}{aus Rücksicht auf Leute, die ohne Rücksicht
%   auf die Anleitung einfach machen, was ihnen gerade einfällt, neu
%   implementiert}%^^A
% \changes{v3.17}{2015/02/20}{Weil manche Leute es unbedingt auf die Spitze
%   treiben wollen, \cs{}\cs{} durch \cs{tabularnewline} ersetzt.}%^^A
% \changes{v3.17}{2015/02/20}{Weil manche Leute es unbedingt auf die Spitze
%   treiben wollen, mehr \cs{strut} eingefügt.}%^^A
% \changes{v3.20}{2015/10/08}{Umbruch nach dem Gruß nur, wenn wir uns im
%   horizontalen Modus befinden}%^^A
% \changes{v3.25}{2017/11/15}{Verwendung von \cs{scr@endstrut} statt
%   \cs{unskip}\cs{strut}}%^^A
%    \begin{macrocode}
\newcommand*{\closing}[1]{{%
    \setparsizes{\z@}{\z@}{\z@ plus 1fil}\par@updaterelative\nobreak\par
    \nobreak\vskip\baselineskip
    \stopbreaks\noindent
    \settowidth{\@tempdima}{%
      \begin{tabular}{@{}l@{}}
        #1\tabularnewline
        \usekomavar{signature}\tabularnewline
      \end{tabular}
    }%
    \setlength{\@tempdimb}{\hsize}%
    \addtolength{\@tempdimb}{-\useplength{sigindent}}%
    \ifdim\@tempdima>\@tempdimb\setlength{\@tempdima}{\@tempdimb}\fi
    \hspace{\useplength{sigindent}}%
    \parbox{\@tempdima}{\raggedsignature\strut\ignorespaces
      #1\ifhmode\scr@endstrut\\[\useplength{sigbeforevskip}]
        \else\vskip \useplength{sigbeforevskip}\fi
      \strut\ignorespaces
      \usekomavar{signature}\ifhmode\scr@endstrut\fi}}\par
  \nobreak\vskip\baselineskip
}
%    \end{macrocode}
% \end{macro}
%
% \begin{macro}{\ps}
% \changes{v2.8n}{2001/09/08}{Komplett überarbeitet}%^^A
% Ebenfalls möglich ist ein Postscriptum. Dieser Befehl erwartet das
% Postscriptum nicht als Argument, sondern schaltet zum Nachtext um.
%    \begin{macrocode}
\newcommand\ps{\par%
  \ifdim\parskip=\z@%
    \vskip\baselineskip%
  \fi\startbreaks\noindent\ignorespaces%
}
%    \end{macrocode}
% \end{macro}
%
% \begin{macro}{\cc}
% Darüber hinaus gibt es noch einen Verteiler. Dieser wird sauber formatiert.
% \changes{v2.6}{2001/01/03}{\cs{def} durch \cs{newcommand*} ersetzt}
% \changes{v2.8n}{2001/09/08}{Komplett "uberarbeitet}%^^A
% \changes{v2.9o}{2003/04/22}{\cs{parfillskip} angepasst}%^^A
% \changes{v2.9q}{2004/04/26}{\cs{parfillskip} auf \texttt{1em plus 1fil}
%     geändert}%^^A
% \changes{v2.9r}{2004/05/25}{\cs{parfillskip} auf \texttt{\cs{z@} plus
%       1fil} geändert}%^^A
% \changes{v3.22}{2016/12/15}{fehlender Stern bei \cs{ifkomavaremptry}
%     ergänzt}%^^A
%    \begin{macrocode}
\newcommand*{\cc}[1]{\par%
  \ifdim\parskip=\z@%
    \vskip\baselineskip%
  \fi\noindent%
  \begingroup
    \parbox[t]{\textwidth}{%
      \Ifkomavarempty*{ccseparator}{}{%
        \@hangfrom{%
          \strut\usekomavar*{ccseparator}\usekomavar{ccseparator}}%
        }%
      \ignorespaces #1\strut}%
    \setlength{\parfillskip}{\z@ \@plus 1fil}\par
  \endgroup
}
%    \end{macrocode}
% \end{macro}
%
% \begin{macro}{\encl}
% Auch Anlagen werden ordentlich gesetzt.
% \changes{v2.4a}{1997/06/06}{der Doppelpunkt bei den Anlagen wird nur noch
%     gesetzt, wenn \cs{enclname} nicht leer ist}
% \changes{v2.6}{2001/01/03}{\cs{def} durch \cs{newcommand*} ersetzt}
% \changes{v2.8n}{2001/09/08}{Komplett überarbeiten}%^^A
% \changes{v2.8q}{2002/01/31}{Fehlendes \% ergänzt}%^^A
% \changes{v2.9o}{2003/04/22}{\cs{parfillskip} angepasst}
% \changes{v2.9q}{2004/04/26}{\cs{parfillskip} auf \texttt{1em plus 1fil}
%     geändert}
% \changes{v2.9r}{2004/05/25}{\cs{parfillskip} auf \texttt{\cs{z@} plus
%       1fil} geändert}
%    \begin{macrocode}
\newcommand*{\encl}[1]{\par%
  \ifdim\parskip=\z@%
    \vskip\baselineskip%
  \fi\noindent%
  \begingroup
    \parbox[t]{\textwidth}{%
      \Ifkomavarempty*{enclseparator}{}{%
        \@hangfrom{%
          \strut\usekomavar*{enclseparator}\usekomavar{enclseparator}}%
        }%
      \ignorespaces #1\strut}%
    \setlength{\parfillskip}{\z@ \@plus 1fil}\par
  \endgroup
}
%    \end{macrocode}
% \end{macro}
%
%
% \subsection{Briefspezifische Feld-Variablen}
%
% \begin{KOMAvar}{signature}
% \changes{v2.8q}{2002/01/16}{Neu}%^^A
% \begin{KOMAvar}{place}
% \changes{v2.8q}{2002/01/16}{Neu}%^^A
% \begin{KOMAvar}{location}
% \changes{v2.8q}{2002/01/16}{Neu}%^^A
% \begin{KOMAvar}{backaddress}
% \changes{v2.8q}{2002/01/16}{Neu}%^^A
% \changes{v3.28}{2019/11/15}{\cs{ifkomavarempty} durch \cs{Ifkomavarempty}
%   ersetzt}%^^A
% \begin{KOMAvar}{specialmail}
% \changes{v2.8q}{2002/01/16}{Neu}%^^A
% \begin{KOMAvar}{title}
% \changes{v2.8q}{2002/01/16}{Neu}%^^A
% \begin{KOMAvar}{fromlogo}
% \changes{v2.8q}{2002/01/28}{Neu}%^^A
% \begin{KOMAvar}{fromzipcode}
% \changes{v3.03}{2009/03/04}{Neu}%^^A
% \begin{KOMAvar}{PPcode}
% \changes{v3.03}{2009/03/04}{Neu}%^^A
% \begin{KOMAvar}{PPdatamatrix}
% \changes{v3.03}{2009/03/04}{Neu}%^^A
% \begin{KOMAvar}{addresseeimage}
% \changes{v3.03}{2009/03/04}{Neu}%^^A
% \changes{v3.03b}{2009/06/30}{Schreibfehler korrigiert}%^^A
% Zunächst besitzt die Briefklasse einige Variablen, die von Haus aus nicht
% über einen Ausgabenamen verfügen, weil sie üblicherweise ohne Ausgabe einer
% Bezeichung verwendet werden. Diese Variablen haben teilweise auch keinen
% vordefinierten Inhalt. Der Inhalt muss erst vom Anwender gesetzt
% werden. Teilweise gibt es aber auch einen vordefinierten Inhalt, der von
% anderen Variablen abhängt.
%    \begin{macrocode}
\newkomavar{signature}
\setkomavar{signature}{\usekomavar{fromname}}
\newkomavar{place}
\newkomavar{location}
\newkomavar{backaddress}
\setkomavar{backaddress}{%
  \Ifkomavarempty{fromname}{}{%
    \strut\ignorespaces\usekomavar{fromname}%
    \Ifkomavarempty{fromaddress}{}{\\}}%
  \Ifkomavarempty{fromaddress}{}{%
    \strut\ignorespaces\usekomavar{fromaddress}}%
  }
\newkomavar{specialmail}
\newkomavar{title}
\newkomavar{fromlogo}
\newkomavar{fromzipcode}
\newkomavar{PPcode}
\newkomavar{PPdatamatrix}
\newkomavar{addresseeimage}
%</letter>
%</body>
%    \end{macrocode}
% \end{KOMAvar}
% \end{KOMAvar}
% \end{KOMAvar}
% \end{KOMAvar}
% \end{KOMAvar}
% \end{KOMAvar}
% \end{KOMAvar}
% \end{KOMAvar}
% \end{KOMAvar}
% \end{KOMAvar}
% \end{KOMAvar}
%
% \begin{option}{symbolicnames}
% \changes{v3.12}{2013/05/13}{neue Option}%^^A
% \changes{v3.17}{2015/03/16}{interne Speicherung der Werte}%^^A
% \changes{v3.28}{2019/11/18}{\cs{ifstr} umbenannt in \cs{Ifstr}}%^^A
% \begin{macro}{\if@scrsymbols}
% \changes{v3.12}{2013/05/13}{neu (intern)}%^^A
% \begin{macro}{\@scrsymbolstrue}
% \changes{v3.12}{2013/05/13}{neu (intern)}%^^A
% \begin{macro}{\@scrsymbolsfalse}
% \changes{v3.12}{2013/05/13}{neu (intern)}%^^A
% \changes{v3.27}{2019/03/21}{neue Werte}%^^A
% Über diese Option kann eingestellt werden, dass an Stelle von
% sprachabhängigen Bezeichnern für einige Variablen Symbole verwendet
% werden. Statt nur boolsche Werte, sind ab \KOMAScript~3.27 auch die Werte
% \texttt{marvosym} oder \texttt{fontawesome} möglich. \texttt{marvosym}
% entspricht dabei dem bisherigen \texttt{true}.
%    \begin{macrocode}
%<*option&letter>
\newif\if@scrsymbols
\KOMA@key{symbolicnames}[true]{%
  \KOMA@set@ifkey{symbolicnames}{@scrsymbols}{#1}%
  \ifx\FamilyKeyState\FamilyKeyStateProcessed
    \def\scr@symbols@font{marvosym}%
  \else
    \KOMA@set@ncmdkey{symbolicnames}{reserved@a}{%
      {marvosym}{0},%
      {fontawesome}{1},{awesome}{1},%
    }{#1}%
    \ifx \FamilyKeyState\FamilyKeyStateProcessed
      \@scrsymbolstrue
      \ifcase \number\reserved@a
        \def\scr@symbols@font{marvosym}%
      \else
        \def\scr@symbols@font{fontawesome}%
      \fi
    \fi
  \fi
  \ifx \FamilyKeyState\FamilyKeyStateProcessed
    \KOMA@kav@xreplacevalue{.%
%<class>      \KOMAClassFileName    
%<package>      scrletter.\scr@pkgextension
    }{symbolicnames}{#1}%
    \if@scrsymbols
      \Ifstr{\scr@symbols@font}{marvosym}{%
        \def\scr@symbol@phone{\Telefon}%
        \def\scr@symbol@fax{\FAX}%
        \def\scr@symbol@email{\Email}%
        \def\scr@symbol@mobilephone{\Mobilefone}%
        \def\scr@symbol@www{\@empty}%
      }{%
        \def\scr@symbol@phone{\faPhone}%
        \def\scr@symbol@fax{\faFax}%
        \def\scr@symbol@email{\faEnvelopeO}%
        \def\scr@symbol@mobilephone{\faMobilePhone}%
        \def\scr@symbol@www{\faGlobe}%
      }%
      \if@atdocument
        \begingroup
          \@tempswafalse
          \expandafter\ifdefined\scr@symbol@phone\else\@tempswatrue\fi
          \expandafter\ifdefined\scr@symbol@fax\else\@tempswatrue\fi
          \expandafter\ifdefined\scr@symbol@email\else\@tempswatrue\fi
          \expandafter\ifdefined\scr@symbol@mobilephone\else\@tempswatrue\fi
          \expandafter\ifdefined\scr@symbol@www\else\@tempswatrue\fi
          \if@tempswa
            \aftergroup\@scrsymbolsfalse
%<class>            \ClassWarning{\KOMAClassName}{%
%<package>            \PackageWarning{scrletter}{%
              Option `symbolicnames=#1'
              ignorred.\MessageBreak
              You've tried to set option `symbolicnames'\MessageBreak
              after \string\begin{document}, but without\MessageBreak
              loading package `\scr@symbols@font' before.\MessageBreak
              You should either switch on option\MessageBreak
              `symbolicnames' before \string\begin{document}\MessageBreak
              or load package `\scr@letter@symbols@font' in the
              document\MessageBreak
              preamble to avoid this warning%
            }%
          \fi
        \endgroup
      \else
%    \end{macrocode}
% \changes{v3.20}{2016/04/01}{Name der Klasse oder des Pakets ist für die
%   Ladeverzögerung von \textsf{marvosym} egal}%^^A
% \changes{v3.20}{2016/04/01}{Ladeverzögerung für das Font-Paket
%   nur, wenn erforderlich und möglich}%^^A
% In der Dokumentpräambel kann das Paket \textsf{marvosym} selbst geladen
% werden. Allerdings funktioniert das nicht, solange wir im Optionen-Abschnitt
% sind. Also muss es ggf. verzögert werden. Wie lange es zu verzögern ist,
% hängt davon ab, ob wir uns in einer Klasse, einem Paket oder sonst einer
% Datei mit Optionen-Abschnitt befinden.
%    \begin{macrocode}
        \ifx\@fileswith@pti@ns\@badrequireerror
          \Ifstr{\@currext}{\@clsextension}{%
            \expandafter\AtEndOfClass
          }{%
            \Ifstr{\@currext}{\@pkgextension}{%
              \expandafter\AtEndOfPackage
            }{%
              \expandafter\AtBeginDocument
            }%
          }%
        \else
          \expandafter\@firstofone
        \fi
        {\RequirePackage{\scr@symbols@font}}%
      \fi
    \fi
  \fi
}
\KOMA@kav@add{.%
%<class>  \KOMAClassFileName
%<package>  scrletter.\scr@pkgextension
}{symbolicnames}{false}%
%    \end{macrocode}
% \selectlanguage{english}
% \begin{macro}{\scr@symbols@font}
% \changes{v3.27}{2019/03/21}{new (internal)}%^^A
% The name of the font packages used for the symbols if symbols instead of
% names are activated.
% \begin{macro}{\scr@symbol@phone}
% \changes{v3.27}{2019/03/21}{new (internal)}%^^A
% The phone symbole.
% \begin{macro}{\scr@symbol@fax}
% \changes{v3.27}{2019/03/21}{new (internal)}%^^A
% The fax symbole.
% \begin{macro}{\scr@symbol@email}
% \changes{v3.27}{2019/03/21}{new (internal)}%^^A
% The e-mail symbol.
% \begin{macro}{\scr@symbol@mobilephone}
% \changes{v3.27}{2019/03/21}{new (internal)}%^^A
% The mobile-phone symbol.
% \begin{macro}{\scr@symbol@www}
% \changes{v3.27}{2019/03/21}{new (internal)}%^^A
% The world-wide-web symbol.
%    \begin{macrocode}
\newcommand*{\scr@symbols@font}{}
\newcommand*{\scr@symbol@phone}{\undefined}
\newcommand*{\scr@symbol@fax}{\undefined}
\newcommand*{\scr@symbol@email}{\undefined}
\newcommand*{\scr@symbol@mobilephone}{\undefined}
\newcommand*{\scr@symbol@www}{\undefined}
%</option&letter>
%    \end{macrocode}
% \end{macro}
% \end{macro}
% \end{macro}
% \end{macro}
% \end{macro}
% \end{macro}
% \selectlanguage{ngerman}
% \end{macro}
% \end{macro}
% \end{macro}
% \end{option}
%
% \begin{KOMAvar}{placeseparator}
% \changes{v2.8q}{2002/01/16}{Neu}%^^A
% \begin{KOMAvar}{subjectseparator}
% \changes{v2.8q}{2002/01/16}{Neu}%^^A
% \begin{KOMAvar}{ccseparator}
% \changes{v2.8q}{2002/01/16}{Neu}%^^A
% \begin{KOMAvar}{enclseparator}
% \changes{v2.8q}{2002/01/16}{Neu}%^^A
% \begin{KOMAvar}{backaddressseparator}
% \changes{v2.8q}{2002/01/22}{Neu}%^^A
% \begin{KOMAvar}{phoneseparator}
% \changes{v2.8q}{2002/03/25}{Neu}%^^A
% \changes{v3.12}{2013/05/29}{Option \texttt{symbolicnames} wird beachtet}%^^A
% \changes{v3.27}{2019/03/21}{Symbole sind indirekt gespeichert}%^^A
% \begin{KOMAvar}{faxseparator}
% \changes{v2.8q}{2002/03/25}{Neu}%^^A
% \changes{v3.12}{2013/05/29}{Option \texttt{symbolicnames} wird beachtet}%^^A
% \changes{v3.27}{2019/03/21}{Symbole sind indirekt gespeichert}%^^A
% \begin{KOMAvar}{emailseparator}
% \changes{v2.8q}{2002/03/25}{Neu}%^^A
% \changes{v3.12}{2013/05/29}{Option \texttt{symbolicnames} wird beachtet}%^^A
% \changes{v3.27}{2019/03/21}{Symbole sind indirekt gespeichert}%^^A
% \begin{KOMAvar}{urlseparator}
% \changes{v2.8q}{2002/03/25}{Neu}%^^A
% \changes{v3.12}{2013/05/29}{Option \texttt{symbolicnames} wird beachtet}%^^A
% \changes{v3.27}{2019/03/21}{Symbole sind indirekt gespeichert}%^^A
% \begin{KOMAvar}{mobilephoneseparator}
% \changes{v3.12}{2013/05/29}{Neu}%^^A
% \changes{v3.27}{2019/03/21}{Symbole sind indirekt gespeichert}%^^A
% \begin{KOMAvar}{zipcodeseparator}
% \changes{v3.03}{2009/03/04}{Neu}%^^A
% Daneben gibt es einige Trennzeichen. Diese haben ggf. als Ausgabenamen das
% Feld, für das sie als Trennzeichen dienen. Die Trennzeichen sind
% vordefiniert.
%    \begin{macrocode}
%<*body>
%<*letter>
\newkomavar[\usekomavar{place}]{placeseparator}
\setkomavar{placeseparator}{,\nobreakspace}
\newkomavar[\subjectname]{subjectseparator}
\setkomavar{subjectseparator}{: }
\newkomavar[\ccname]{ccseparator}
\setkomavar{ccseparator}{: }
\newkomavar[\enclname]{enclseparator}
\setkomavar{enclseparator}{: }
\newkomavar{backaddressseparator}
\setkomavar{backaddressseparator}{,\nobreakspace}
\newkomavar[\if@scrsymbols\scr@symbol@phone\else\phonename\fi]{phoneseparator}
\setkomavar{phoneseparator}{\if@scrsymbols\else:\fi\nobreakspace}
\newkomavar[%
  \if@scrsymbols\scr@symbol@mobilephone\else\mobilephonename\fi
]{mobilephoneseparator}
\setkomavar{mobilephoneseparator}{\usekomavar{phoneseparator}}
\newkomavar[\if@scrsymbols\scr@symbol@fax\else\faxname\fi]{faxseparator}
\setkomavar{faxseparator}{\if@scrsymbols\else:\fi\nobreakspace}
\newkomavar[\if@scrsymbols\scr@symbol@email\else\emailname\fi]{emailseparator}
\setkomavar{emailseparator}{\if@scrsymbols\else:\fi\nobreakspace}
\newkomavar[\if@scrsymbols\scr@symbol@www\else\wwwname\fi]{urlseparator}
\setkomavar{urlseparator}{\if@scrsymbols\else:\nobreakspace\fi}
\newkomavar{zipcodeseparator}
\setkomavar{zipcodeseparator}{\,--\,}
%    \end{macrocode}
% \end{KOMAvar}
% \end{KOMAvar}
% \end{KOMAvar}
% \end{KOMAvar}
% \end{KOMAvar}
% \end{KOMAvar}
% \end{KOMAvar}
% \end{KOMAvar} 
% \end{KOMAvar}
% \end{KOMAvar}
% \end{KOMAvar}
%
% \selectlanguage{english}%^^A
% \begin{KOMAvar}{addressee}
% \changes{v3.27}{2019/03/24}{new variable}%^^A
% The package needs a variable for the addressee (used for the address field).
% \begin{KOMAvar}{togivenname}
% \changes{v3.27}{2019/03/24}{new variable}%^^A
% \begin{KOMAvar}{tofamilyname}
% \changes{v3.27}{2019/03/24}{new variable}%^^A
% The package allows to build \texttt{toname} by \texttt{togivenname} and
% \texttt{tofamilyname}.
%    \begin{macrocode}
%<*package&willbe>
\newkomavar{addressee}
\newkomavar{togivenname}
\newkomavar{tofamilyname}
%</package&willbe>
%    \end{macrocode}
% \end{KOMAvar}%^^A tofamilyname
% \end{KOMAvar}%^^A togivenname
% \end{KOMAvar}%^^A addressee
% \selectlanguage{ngerman}%^^A
%
% \begin{KOMAvar}{fromname}
% \changes{v2.8q}{2002/01/16}{Neu}%^^A
% \begin{KOMAvar}{fromaddress}
% \changes{v2.8q}{2002/01/16}{Neu}%^^A
% \begin{KOMAvar}{fromemail}
% \changes{v2.8q}{2002/01/28}{Neu}%^^A
% \begin{KOMAvar}{fromurl}
% \changes{v2.8q}{2002/01/28}{Neu}%^^A
% \begin{KOMAvar}{frombank}
% \changes{v2.8q}{2002/01/28}{Neu}%^^A
% \begin{KOMAvar}{toname}
% \changes{v2.8q}{2002/01/16}{Neu}%^^A
% \begin{KOMAvar}{toaddress}
% \changes{v2.8q}{2002/01/16}{Neu}%^^A
% \begin{KOMAvar}{fromphone}
% \changes{v2.8q}{2002/01/16}{Neu}%^^A
% \begin{KOMAvar}{frommobilephone}
% \changes{v3.12}{2013/05/29}{Neu}%^^A
% \begin{KOMAvar}{fromfax}
% \changes{v2.8q}{2002/01/28}{Neu}%^^A
% \begin{KOMAvar}{yourmail}
% \changes{v2.8q}{2002/01/16}{Neu}%^^A
% \begin{KOMAvar}{yourref}
% \changes{v2.8q}{2002/01/16}{Neu}%^^A
% \begin{KOMAvar}{myref}
% \changes{v2.8q}{2002/01/16}{Neu}%^^A
% \begin{KOMAvar}{customer}
% \changes{v2.8q}{2002/01/16}{Neu}%^^A
% \begin{KOMAvar}{invoice}
% \changes{v2.8q}{2002/01/16}{Neu}%^^A
% \begin{KOMAvar}{subject}
% \changes{v2.8q}{2002/01/16}{Neu}%^^A
% \begin{KOMAvar}{date}
% \changes{v2.8q}{2002/01/16}{Neu}%^^A
% \begin{macro}{\date}
% \changes{v3.17}{2015/03/05}{das Makro gibt eine Warnung aus}%^^A
% \begin{macro}{\scr@saved@date}
% \changes{v3.17}{2015/03/05}{Neu (intern)}%^^A
% Desweiteren gibt es einige Variable, die auch einen Ausgabenamen haben, weil
% dieser üblicherweise vorangestellt wird. Eine Besonderheit stellt dabei
% \texttt{subject} dar. Der Ausgabename beinhaltet dabei nämlich auch die
% Trennzeichen, die hinter dem Namen üblicherweise stehen. Diese sind wiederum
% ein Variable, deren Name der Ausgabename des Feldes ist. Will man also nur
% den Namen aber nicht die Trennzeichen ändern, so ändert man statt dem Namen
% von \texttt{subject} den Namen von \texttt{subjectseparator}.
%
% Ähnlich ist es mit \texttt{toaddress} und \texttt{fromaddress}. Dort ist der
% Wert von \texttt{toname} bzw. \texttt{fromname} der Ausgabename. Bei
% \texttt{toaddress} und \texttt{fromaddress} wird davon jedoch intern kein
% Gebrauch gemacht. Die Variable \texttt{date} wird bereits so vordefiniert,
% dass auch weiterhin \cs{date} funktioniert.
%    \begin{macrocode}
\newkomavar[\headfromname]{fromname}
\newkomavar[\usekomavar{fromname}]{fromaddress}
\newkomavar[\headtoname]{toname}
\newkomavar[\usekomavar{toname}]{toaddress}
\newkomavar[\usekomavar*{phoneseparator}%
  \usekomavar{phoneseparator}]{fromphone}
\newkomavar[\usekomavar*{mobilephoneseparator}%
  \usekomavar{mobilephoneseparator}]{frommobilephone}
\newkomavar[\usekomavar*{faxseparator}%
  \usekomavar{faxseparator}]{fromfax}
\newkomavar[\usekomavar*{emailseparator}%
  \usekomavar{emailseparator}]{fromemail}
\newkomavar[\usekomavar*{urlseparator}%
  \usekomavar{urlseparator}]{fromurl}
\newkomavar[\bankname]{frombank}
\newkomavar[\yourrefname]{yourref}
\newkomavar[\yourmailname]{yourmail}
\newkomavar[\myrefname]{myref}
\newkomavar[\customername]{customer}
\newkomavar[\invoicename]{invoice}
\newkomavar[\usekomavar*{subjectseparator}%
  \usekomavar{subjectseparator}]{subject}
\newkomavar[\datename]{date}
\setkomavar{date}{\@date}
\newcommand*{\scr@saved@date}{}
\let\scr@saved@date\date
\renewcommand*{\date}[1]{%
%<class>  \ClassWarning{\KOMAClassName}{%
%<package>  \PackageWarning{scrletter}{%
    \string\date\space changes the internal \string\LaTeX\space date
    only.\MessageBreak
    You should note that this changes the default\MessageBreak
    value of variable `date' only as long as nobody\MessageBreak
    uses `\string\setkomavar` to change the content of\MessageBreak
    variable `date'.\MessageBreak
    Therefore, usage of `\string\setkomavar' is recommended\MessageBreak
    to change the letter's date%
  }%
  \scr@saved@date{#1}%
}
%    \end{macrocode}
% \end{macro}%^^A \scr@saved@date
% \end{macro}%^^A \date
% \end{KOMAvar}
% \end{KOMAvar}
% \end{KOMAvar}
% \end{KOMAvar}
% \end{KOMAvar}
% \end{KOMAvar}
% \end{KOMAvar}
% \end{KOMAvar}
% \end{KOMAvar}
% \end{KOMAvar}
% \end{KOMAvar}
% \end{KOMAvar}
% \end{KOMAvar}
% \end{KOMAvar}
% \end{KOMAvar}
% \end{KOMAvar}
% \end{KOMAvar}
%
% \iffalse
%</letter>
% \fi
%
% \iffalse
%</body>
% \fi
%
% \Finale
%
\endinput
%
% end of file `scrkernel-notepaper.dtx'
%%% Local Variables:
%%% mode: doctex
%%% TeX-master: t
%%% End:
