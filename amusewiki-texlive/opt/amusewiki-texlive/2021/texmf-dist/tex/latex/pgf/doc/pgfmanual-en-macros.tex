% Copyright 2019 by Till Tantau
%
% This file may be distributed and/or modified
%
% 1. under the LaTeX Project Public License and/or
% 2. under the GNU Free Documentation License.
%
% See the file doc/generic/pgf/licenses/LICENSE for more details.

% $Header$


\newcount\pgfmanualtargetcount

\colorlet{examplefill}{yellow!80!black}
\definecolor{graphicbackground}{rgb}{0.96,0.96,0.8}
\definecolor{codebackground}{rgb}{0.9,0.9,1}
\definecolor{animationgraphicbackground}{rgb}{0.96,0.96,0.8}

\newenvironment{pgfmanualentry}{\list{}{\leftmargin=2em\itemindent-\leftmargin\def\makelabel##1{\hss##1}}}{\endlist}
\newcounter{pgfmanualentry}
\newcommand\pgfmanualentryheadline[1]{%
  \itemsep=0pt\parskip=0pt{\raggedright\item\refstepcounter{pgfmanualentry}\strut{#1}\par}\topsep=0pt}
\newcommand\pgfmanualbody{\parskip3pt}

\let\origtexttt=\texttt
\def\texttt#1{{\def\textunderscore{\char`\_}\def\textbraceleft{\char`\{}\def\textbraceright{\char`\}}\origtexttt{#1}}}
\def\exclamationmarktext{!}
\def\atmarktext{@}

{
  \catcode`\|=12
  \gdef\pgfmanualnormalbar{|}
  \catcode`\|=13
  \AtBeginDocument{\gdef|{\ifmmode\pgfmanualnormalbar\else\expandafter\verb\expandafter|\fi}}
}



\newenvironment{pgflayout}[1]{
  \begin{pgfmanualentry}
    \pgfmanualentryheadline{%
      \pgfmanualpdflabel{#1}{}%
      \texttt{\string\pgfpagesuselayout\char`\{\declare{#1}\char`\}}\oarg{options}%
    }
    \index{#1@\protect\texttt{#1} layout}%
    \index{Page layouts!#1@\protect\texttt{#1}}%
    \pgfmanualbody
}
{
  \end{pgfmanualentry}
}


\newenvironment{sysanimateattribute}[1]{
  \begin{pgfmanualentry}
    \pgfmanualentryheadline{%
      \pgfmanualpdflabel{#1}{}%
      \texttt{\string\pgfsysanimate\char`\{\declare{#1}\char`\}}%
    }
    \index{#1@\protect\texttt{#1} system layer animation attribute}%
    \index{Animation attributes (system layer)!#1@\protect\texttt{#1}}%
    \pgfmanualbody
}
{
  \end{pgfmanualentry}
}


\newenvironment{animateattribute}[1]{
  \begin{pgfmanualentry}
    \pgfmanualentryheadline{%
      \pgfmanualpdflabel{#1}{}%
      \texttt{\string\pgfanimateattribute\char`\{\declare{#1}\char`\}\marg{options}}%
    }
    \index{#1@\protect\texttt{#1} basic layer animation attribute}%
    \index{Animation attributes (basic layer)!#1@\protect\texttt{#1}}%
    \pgfmanualbody
}
{
  \end{pgfmanualentry}
}


\newenvironment{tikzanimateattribute}[1]{
  \begin{pgfmanualentry}
    \pgfmanualentryheadline{%
      \foreach \attr in{#1} {\expandafter\pgfmanualpdflabel\expandafter{\attr}{}}%
      \textbf{Animation attribute} \foreach \attr[count=\i]
      in{#1}{{\ifnum\i>1 \textbf,\fi} \texttt{:\declare{\attr}}}%
    }
    \foreach\attr in{#1}{%
      \edef\indexcall{%
        \noexpand\index{\attr@\noexpand\protect\noexpand\texttt{\attr} animation attribute}%
        \noexpand\index{Animation attributes!\attr@\noexpand\protect\noexpand\texttt{\attr}}%
      }%
      \indexcall%
    }%
    \pgfmanualbody
}
{
  \end{pgfmanualentry}
}


\newenvironment{command}[1]{
  \begin{pgfmanualentry}
    \extractcommand#1\@@
    \pgfmanualbody
}
{
  \end{pgfmanualentry}
}

\makeatletter

\def\includeluadocumentationof#1{
  \directlua{require 'pgf.manual.DocumentParser'}
  \directlua{pgf.manual.DocumentParser.include '#1'}
}

\newenvironment{luageneric}[4]{
  \pgfmanualentry
    \pgfmanualentryheadline{#4 \texttt{#1\declare{#2}}#3}
    \index{#2@\protect\texttt{#2} (Lua)}%
    \def\temp{#1}
    \ifx\temp\pgfutil@empty\else
      \index{#1@\protect\texttt{#1}!#2@\protect\texttt{#2} (Lua)}%
    \fi
  \pgfmanualbody
}{\endpgfmanualentry}

\newenvironment{luatable}[3]{
  \medskip
  \luageneric{#1}{#2}{ (declared in \texttt{#3})}{\textbf{Lua table}}
}{\endluageneric}

\newenvironment{luafield}[1]{
  \pgfmanualentry
    \pgfmanualentryheadline{Field \texttt{\declare{#1}}}
  \pgfmanualbody
}{\endpgfmanualentry}


\newenvironment{lualibrary}[1]{
  \pgfmanualentry
  \pgfmanualentryheadline{%
    \pgfmanualpdflabel{#1}{}%
    \textbf{Graph Drawing Library} \texttt{\declare{#1}}%
  }
    \index{#1@\protect\texttt{#1} graph drawing library}%
    \index{Libraries!#1@\protect\texttt{#1}}%
    \index{Graph drawing libraries!#1@\protect\texttt{#1}}%
    \vskip.25em
    {\ttfamily\char`\\usegdlibrary\char`\{\declare{#1}\char`\}\space\space \char`\%\space\space  \LaTeX\space and plain \TeX}\\
    {\ttfamily\char`\\usegdlibrary[\declare{#1}]\space \char`\%\space\space Con\TeX t}\smallskip\par
    \pgfmanualbody
}{\endpgfmanualentry}

\newenvironment{luadeclare}[4]{
  \pgfmanualentry
  \def\manual@temp@default{#3}%
  \def\manual@temp@initial{#4}%
  \def\manual@temp@{#3#4}%
  \pgfmanualentryheadline{%
    \pgfmanualpdflabel{#1}{}%
    {\ttfamily/graph
      drawing/\declare{#1}\opt{=}}\opt{#2}\hfill%
    \ifx\manual@temp@\pgfutil@empty\else%
    (\ifx\manual@temp@default\pgfutil@empty\else%
    default {\ttfamily #3}\ifx\manual@temp@initial\pgfutil@empty\else, \fi%
    \fi%
    \ifx\manual@temp@initial\pgfutil@empty\else%
    initially {\ttfamily #4}%
    \fi%
    )\fi%
  }%
  \index{#1@\protect\texttt{#1} key}%
  \pgfmanualbody
  \gdef\myname{#1}%
%  \keyalias{tikz}
%  \keyalias{tikz/graphs}
}{\endpgfmanualentry}

\newenvironment{luadeclarestyle}[4]{
  \pgfmanualentry
  \def\manual@temp@para{#2}%
  \def\manual@temp@default{#3}%
  \def\manual@temp@initial{#4}%
  \def\manual@temp@{#3#4}%
  \pgfmanualentryheadline{%
    \pgfmanualpdflabel{#1}{}%
    {\ttfamily/graph drawing/\declare{#1}}\ifx\manual@temp@para\pgfutil@empty\else\opt{\texttt=}\opt{#2}\fi\hfill%
    (style\ifx\manual@temp@\pgfutil@empty\else, %
    \ifx\manual@temp@default\pgfutil@empty\else%
    default {\ttfamily #3}\ifx\manual@temp@initial\pgfutil@empty\else, \fi%
    \fi%
    \ifx\manual@temp@initial\pgfutil@empty\else%
    initially {\ttfamily #4}%
    \fi%
    \fi)%
  }%
  \index{#1@\protect\texttt{#1} key}%
  \pgfmanualbody%
  \gdef\myname{#1}%
%  \keyalias{tikz}
%  \keyalias{tikz/graphs}
}{\endpgfmanualentry}

\newenvironment{luanamespace}[2]{
  \luageneric{#1}{#2}{}{\textbf{Lua namespace}}
}{\endluageneric}

\newenvironment{luafiledescription}[1]{}{}

\newenvironment{luacommand}[4]{
  \hypertarget{pgf/lua/#1}{\luageneric{#2}{#3}{\texttt{(#4)}}{\texttt{function}}}
}{\endluageneric}

\newenvironment{luaparameters}{\par\emph{Parameters:}%
  \parametercount=0\relax%
  \let\item=\parameteritem%
  \let\list=\restorelist%
}
{\par
}

\newenvironment{luareturns}{\par\emph{Returns:}%
  \parametercount=0\relax%
  \let\item=\parameteritem%
  \let\list=\restorelist%
}
{\par
}

\newcount\parametercount

\newenvironment{parameterdescription}{\unskip%
  \parametercount=0\relax%
  \let\item=\parameteritem%
  \let\list=\restorelist%
}
{\par
}
\let\saveditemcommand=\item
\let\savedlistcommand=\list
\def\denselist#1#2{\savedlistcommand{#1}{#2}\parskip0pt\itemsep0pt}
\def\restorelist{\let\item=\saveditemcommand\denselist}
\def\parameteritem{\pgfutil@ifnextchar[\parameteritem@{}}%}
\def\parameteritem@[#1]{\advance\parametercount by1\relax\hskip0.15em plus 1em\emph{\the\parametercount.}\kern1ex\def\test{#1}\ifx\test\pgfutil@empty\else#1\kern.5em\fi}

\newenvironment{commandlist}[1]{%
  \begin{pgfmanualentry}
  \foreach \xx in {#1} {%
    \expandafter\extractcommand\xx\@@
  }%
  \pgfmanualbody
}{%
  \end{pgfmanualentry}
}%

% \begin{internallist}[register]{\pgf@xa}
% \end{internallist}
%
% \begin{internallist}[register]{\pgf@xa,\pgf@xb}
% \end{internallist}
\newenvironment{internallist}[2][register]{%
  \begin{pgfmanualentry}
  \foreach \xx in {#2} {%
    \expandafter\extractinternalcommand\expandafter{\xx}{#1}%
  }%
  \pgfmanualbody
}{%
  \end{pgfmanualentry}
}%
\def\extractinternalcommand#1#2{%
  \removeats{#1}%
  \pgfmanualentryheadline{%
    \pgfmanualpdflabel{\textbackslash\strippedat}{}%
    Internal #2 \declare{\texttt{\string#1}}}%
  \index{Internals!\strippedat @\protect\myprintocmmand{\strippedat}}%
  \index{\strippedat @\protect\myprintocmmand{\strippedat}}%
}

%% MW: START MATH MACROS
\def\mvar#1{{\ifmmode\textrm{\textit{#1}}\else\rmfamily\textit{#1}\fi}}

\makeatletter

\def\extractmathfunctionname#1{\extractmathfunctionname@#1(,)\tmpa\tmpb}
\def\extractmathfunctionname@#1(#2)#3\tmpb{\def\mathname{#1}}

\makeatother
  
\newenvironment{math-function}[1]{
  \def\mathdefaultname{#1}
  \extractmathfunctionname{#1}
  \edef\mathurl{{math:\mathname}}\expandafter\hypertarget\expandafter{\mathurl}{}%
  \begin{pgfmanualentry}
    \pgfmanualentryheadline{\texttt{#1}}%
    \index{\mathname @\protect\texttt{\mathname} math function}%
    \index{Math functions!\mathname @\protect\texttt{\mathname}}%
    \pgfmanualbody
}
{
  \end{pgfmanualentry}
}

\def\pgfmanualemptytext{}
\def\pgfmanualvbarvbar{\char`\|\char`\|}

\newenvironment{math-operator}[4][]{%
  \begin{pgfmanualentry}
  \csname math#3operator\endcsname{#2}{#4}
  \def\mathtest{#4}%
  \ifx\mathtest\pgfmanualemptytext%
    \def\mathtype{(#3 operator)}
  \else%
    \def\mathtype{(#3 operator; uses the \texttt{#4} function)}
  \fi%
  \pgfmanualentryheadline{\mathexample\hfill\mathtype}%
  \def\mathtest{#1}%
  \ifx\mathtest\pgfmanualemptytext%
    \index{#2@\protect\texttt{#2} #3 math operator}%  
    \index{Math operators!#2@\protect\texttt{#2}}%
  \fi%
  \pgfmanualbody
}
{\end{pgfmanualentry}}

\newenvironment{math-operators}[5][]{%
  \begin{pgfmanualentry}
  \csname math#4operator\endcsname{#2}{#3}
  \def\mathtest{#5}%
  \ifx\mathtest\pgfmanualemptytext%
    \def\mathtype{(#4 operators)}
  \else%
    \def\mathtype{(#4 operators; use the \texttt{#5} function)}
  \fi%
  \pgfmanualentryheadline{\mathexample\hfill\mathtype}%
  \def\mathtest{#1}%
  \ifx\mathtest\pgfmanualemptytext%
    \index{#2#3@\protect\texttt{#2\protect\ #3} #4 math operators}% 
    \index{Math operators!#2#3@\protect\texttt{#2\protect\ #3}}%
  \fi%
  \pgfmanualbody
}
{\end{pgfmanualentry}}

\def\mathinfixoperator#1#2{%
  \def\mathoperator{\texttt{#1}}%
  \def\mathexample{\mvar{x}\space\texttt{#1}\space\mvar{y}}%
}

\def\mathprefixoperator#1#2{%
  \def\mathoperator{\texttt{#1}}%
  \def\mathexample{\texttt{#1}\mvar{x}}%
}

\def\mathpostfixoperator#1#2{%
  \def\mathoperator{\texttt{#1}}
  \def\mathexample{\mvar{x}\texttt{#1}}%
}

\def\mathgroupoperator#1#2{%
  \def\mathoperator{\texttt{#1\ #2}}%
  \def\mathexample{\texttt{#1}\mvar{x}\texttt{#2}}%
}

\expandafter\let\csname matharray accessoperator\endcsname=\mathgroupoperator
\expandafter\let\csname matharrayoperator\endcsname=\mathgroupoperator

\def\mathconditionaloperator#1#2{%
  \def\mathoperator{#1\space#2}
  \def\mathexample{\mvar{x}\ \texttt{#1}\ \mvar{y}\ {\texttt{#2}}\ \mvar{z}}
}

\newcommand\mathcommand[1][\mathdefaultname]{%
  \expandafter\makemathcommand#1(\empty)\stop%
  \expandafter\extractcommand\mathcommandname\@@%
  \medskip
}
\makeatletter

\def\makemathcommand#1(#2)#3\stop{%
  \expandafter\def\expandafter\mathcommandname\expandafter{\csname pgfmath#1\endcsname}%
  \ifx#2\empty%
  \else%
    \@makemathcommand#2,\stop,
  \fi}
\def\@makemathcommand#1,{%
  \ifx#1\stop%
  \else%
    \expandafter\def\expandafter\mathcommandname\expandafter{\mathcommandname{\ttfamily\char`\{#1\char`\}}}%
    \expandafter\@makemathcommand%
  \fi}
\makeatother

\def\calcname{\textsc{calc}}

\newenvironment{math-keyword}[1]{
  \extracttikzmathkeyword#1@
  \begin{pgfmanualentry}
    \pgfmanualentryheadline{\texttt{\color{red}\mathname}\mathrest}%
    \index{\mathname @\protect\texttt{\mathname} tikz math function}%
    \index{TikZ math functions!\mathname @\protect\texttt{\mathname}}%
    \pgfmanualbody
}
{
  \end{pgfmanualentry}
}

\def\extracttikzmathkeyword#1#2@{%
  \def\mathname{#1}%
  \def\mathrest{#2}%
}

%% MW: END MATH MACROS


\def\extractcommand#1#2\@@{%
  \removeats{#1}%
  \pgfmanualentryheadline{%
    \pgfmanualpdflabel{\textbackslash\strippedat}{}%
    \declare{\expandafter\texttt\expandafter{\string#1}}#2%
  }%
  \index{\strippedat @\protect\myprintocmmand{\strippedat}}
}

\def\luaextractcommand#1#2\relax{%
  \declare{\texttt{\string#1}}#2\par%
%  \removeats{#1}%
 % \index{\strippedat @\protect\myprintocmmand{\strippedat}}
 % \pgfmanualpdflabel{\textbackslash\strippedat}{}%
}


% \begin{environment}{{name}\marg{arguments}}
\renewenvironment{environment}[1]{
  \begin{pgfmanualentry}
    \extractenvironement#1\@@
    \pgfmanualbody
}
{
  \end{pgfmanualentry}
}

\def\extractenvironement#1#2\@@{%
  \pgfmanualentryheadline{%
    \pgfmanualpdflabel{#1}{}%
    {\ttfamily\char`\\begin\char`\{\declare{#1}\char`\}}#2%
  }%
  \pgfmanualentryheadline{{\ttfamily\ \ }\meta{environment contents}}%
  \pgfmanualentryheadline{{\ttfamily\char`\\end\char`\{\declare{#1}\char`\}}}%
  \index{#1@\protect\texttt{#1} environment}%
  \index{Environments!#1@\protect\texttt{#1}}
}


\newenvironment{plainenvironment}[1]{
  \begin{pgfmanualentry}
    \extractplainenvironement#1\@@
    \pgfmanualbody
}
{
  \end{pgfmanualentry}
}

\def\extractplainenvironement#1#2\@@{%
  \pgfmanualentryheadline{{\ttfamily\declare{\char`\\#1}}#2}%
  \pgfmanualentryheadline{{\ttfamily\ \ }\meta{environment contents}}%
  \pgfmanualentryheadline{{\ttfamily\declare{\char`\\end#1}}}%
  \index{#1@\protect\texttt{#1} environment}%
  \index{Environments!#1@\protect\texttt{#1}}%
}


\newenvironment{contextenvironment}[1]{
  \begin{pgfmanualentry}
    \extractcontextenvironement#1\@@
    \pgfmanualbody
}
{
  \end{pgfmanualentry}
}

\def\extractcontextenvironement#1#2\@@{%
  \pgfmanualentryheadline{{\ttfamily\declare{\char`\\start#1}}#2}%
  \pgfmanualentryheadline{{\ttfamily\ \ }\meta{environment contents}}%
  \pgfmanualentryheadline{{\ttfamily\declare{\char`\\stop#1}}}%
  \index{#1@\protect\texttt{#1} environment}%
  \index{Environments!#1@\protect\texttt{#1}}}


\newenvironment{shape}[1]{
  \begin{pgfmanualentry}
    \pgfmanualentryheadline{%
      \pgfmanualpdflabel{#1}{}%
      \textbf{Shape} {\ttfamily\declare{#1}}%
    }%
    \index{#1@\protect\texttt{#1} shape}%
    \index{Shapes!#1@\protect\texttt{#1}}
    \pgfmanualbody
}
{
  \end{pgfmanualentry}
}

\newenvironment{pictype}[2]{
  \begin{pgfmanualentry}
    \pgfmanualentryheadline{%
      \pgfmanualpdflabel{#1}{}%
      \textbf{Pic type} {\ttfamily\declare{#1}#2}%
    }%
    \index{#1@\protect\texttt{#1} pic type}%
    \index{Pic Types!#1@\protect\texttt{#1}}
    \pgfmanualbody
}
{
  \end{pgfmanualentry}
}

\newenvironment{shading}[1]{
  \begin{pgfmanualentry}
    \pgfmanualentryheadline{%
      \pgfmanualpdflabel{#1}{}%
      \textbf{Shading} {\ttfamily\declare{#1}}}%
    \index{#1@\protect\texttt{#1} shading}%
    \index{Shadings!#1@\protect\texttt{#1}}
    \pgfmanualbody
}
{
  \end{pgfmanualentry}
}


\newenvironment{graph}[1]{
  \begin{pgfmanualentry}
    \pgfmanualentryheadline{%
      \pgfmanualpdflabel{#1}{}%
      \textbf{Graph} {\ttfamily\declare{#1}}}%
    \index{#1@\protect\texttt{#1} graph}%
    \index{Graphs!#1@\protect\texttt{#1}}
    \pgfmanualbody
}
{
  \end{pgfmanualentry}
}

\newenvironment{gdalgorithm}[2]{
  \begin{pgfmanualentry}
    \pgfmanualentryheadline{%
      \pgfmanualpdflabel{#1}{}%
      \textbf{Layout} {\ttfamily/graph drawing/\declare{#1}\opt{=}}\opt{\meta{options}}}%
    \index{#1@\protect\texttt{#1} layout}%
    \index{Layouts!#1@\protect\texttt{#1}}%
    \foreach \algo in {#2}
    {\edef\marshal{\noexpand\index{#2@\noexpand\protect\noexpand\texttt{#2} algorithm}}\marshal}%
    \index{Graph drawing layouts!#1@\protect\texttt{#1}}
    \item{\small alias {\ttfamily/tikz/#1}}\par
    \item{\small alias {\ttfamily/tikz/graphs/#1}}\par
    \item{\small Employs {\ttfamily algorithm=#2}}\par
    \pgfmanualbody
}
{
  \end{pgfmanualentry}
}

\newenvironment{dataformat}[1]{
  \begin{pgfmanualentry}
    \pgfmanualentryheadline{%
      \pgfmanualpdflabel{#1}{}%
      \textbf{Format} {\ttfamily\declare{#1}}}%
    \index{#1@\protect\texttt{#1} format}%
    \index{Formats!#1@\protect\texttt{#1}}
    \pgfmanualbody
}
{
  \end{pgfmanualentry}
}

\newenvironment{stylesheet}[1]{
  \begin{pgfmanualentry}
    \pgfmanualentryheadline{%
      \pgfmanualpdflabel{#1}{}%
      \textbf{Style sheet} {\ttfamily\declare{#1}}}%
    \index{#1@\protect\texttt{#1} style sheet}%
    \index{Style sheets!#1@\protect\texttt{#1}}
    \pgfmanualbody
}
{
  \end{pgfmanualentry}
}

\newenvironment{handler}[1]{
  \begin{pgfmanualentry}
    \extracthandler#1\@nil%
    \pgfmanualbody
}
{
  \end{pgfmanualentry}
}

\def\gobble#1{}
\def\extracthandler#1#2\@nil{%
  \pgfmanualentryheadline{%
    \pgfmanualpdflabel{/handlers/#1}{}%
    \textbf{Key handler} \meta{key}{\ttfamily/\declare{#1}}#2}%
  \index{\gobble#1@\protect\texttt{#1} handler}%
  \index{Key handlers!#1@\protect\texttt{#1}}
}


\makeatletter


\newenvironment{stylekey}[1]{
  \begin{pgfmanualentry}
    \def\extrakeytext{style, }
    \extractkey#1\@nil%
    \pgfmanualbody
}
{
  \end{pgfmanualentry}
}

\def\choicesep{$\vert$}%
\def\choicearg#1{\texttt{#1}}

\newif\iffirstchoice

% \mchoice{choice1,choice2,choice3}
\newcommand\mchoice[1]{%
  \begingroup
  \firstchoicetrue
  \foreach \mchoice@ in {#1} {%
    \iffirstchoice
      \global\firstchoicefalse
    \else
      \choicesep
    \fi
    \choicearg{\mchoice@}%
  }%
  \endgroup
}%

% \begin{key}{/path/x=value}
% \begin{key}{/path/x=value (initially XXX)}
% \begin{key}{/path/x=value (default XXX)}
\newenvironment{key}[1]{
  \begin{pgfmanualentry}
    \def\extrakeytext{}
    %\def\altpath{\emph{\color{gray}or}}%
    \extractkey#1\@nil%
    \pgfmanualbody
}
{
  \end{pgfmanualentry}
}

% \insertpathifneeded{a key}{/pgf} -> assign mykey={/pgf/a key}
% \insertpathifneeded{/tikz/a key}{/pgf} -> assign mykey={/tikz/a key}
%
% #1: the key
% #2: a default path (or empty)
\def\insertpathifneeded#1#2{%
  \def\insertpathifneeded@@{#2}%
  \ifx\insertpathifneeded@@\empty
    \def\mykey{#1}%
  \else
    \insertpathifneeded@#2\@nil
    \ifpgfutil@in@
      \def\mykey{#2/#1}%
    \else
      \def\mykey{#1}%
    \fi
  \fi
}%
\def\insertpathifneeded@#1#2\@nil{%
  \def\insertpathifneeded@@{#1}%
  \def\insertpathifneeded@@@{/}%
  \ifx\insertpathifneeded@@\insertpathifneeded@@@
    \pgfutil@in@true
  \else
    \pgfutil@in@false
  \fi
}%

% \begin{keylist}[default path]
%   {/path/option 1=value,/path/option 2=value2}
% \end{keylist}
\newenvironment{keylist}[2][]{%
  \begin{pgfmanualentry}
    \def\extrakeytext{}%
  \foreach \xx in {#2} {%
    \expandafter\insertpathifneeded\expandafter{\xx}{#1}%
    \expandafter\extractkey\mykey\@nil%
  }%
  \pgfmanualbody
}{%
  \end{pgfmanualentry}
}%

\def\extractkey#1\@nil{%
  \pgfutil@in@={#1}%
  \ifpgfutil@in@%
    \extractkeyequal#1\@nil
  \else%
    \pgfutil@in@{(initial}{#1}%
    \ifpgfutil@in@%
      \extractequalinitial#1\@nil%
    \else
      \pgfmanualentryheadline{%
      \def\mykey{#1}%
      \def\mypath{}%
      \gdef\myname{}%
      \firsttimetrue%
      \pgfmanualdecomposecount=0\relax%
      \decompose#1/\nil%
        {\ttfamily\declare{#1}}\hfill(\extrakeytext no value)}%
    \fi
  \fi%
}

\def\extractkeyequal#1=#2\@nil{%
  \pgfutil@in@{(default}{#2}%
  \ifpgfutil@in@%
    \extractdefault{#1}#2\@nil%
  \else%
    \pgfutil@in@{(initial}{#2}%
    \ifpgfutil@in@%
      \extractinitial{#1}#2\@nil%
    \else
      \pgfmanualentryheadline{%
        \def\mykey{#1}%
        \def\mypath{}%
        \gdef\myname{}%
        \firsttimetrue%
        \pgfmanualdecomposecount=0\relax%
        \decompose#1/\nil%
        {\ttfamily\declare{#1}=}#2\hfill(\extrakeytext no default)}%
    \fi%
  \fi%
}

\def\extractdefault#1#2(default #3)\@nil{%
  \pgfmanualentryheadline{%
    \def\mykey{#1}%
    \def\mypath{}%
    \gdef\myname{}%
    \firsttimetrue%
    \pgfmanualdecomposecount=0\relax%
    \decompose#1/\nil%
    {\ttfamily\declare{#1}\opt{=}}\opt{#2}\hfill (\extrakeytext default {\ttfamily#3})}%
}

\def\extractinitial#1#2(initially #3)\@nil{%
  \pgfmanualentryheadline{%
    \def\mykey{#1}%
    \def\mypath{}%
    \gdef\myname{}%
    \firsttimetrue%
    \pgfmanualdecomposecount=0\relax%
    \decompose#1/\nil%
    {\ttfamily\declare{#1}=}#2\hfill (\extrakeytext no default, initially {\ttfamily#3})}%
}

\def\extractequalinitial#1 (initially #2)\@nil{%
  \pgfmanualentryheadline{%
    \def\mykey{#1}%
    \def\mypath{}%
    \gdef\myname{}%
    \firsttimetrue%
    \pgfmanualdecomposecount=0\relax%
    \decompose#1/\nil%
    {\ttfamily\declare{#1}}\hfill (\extrakeytext initially {\ttfamily#2})}%
}

% Introduces a key alias '/#1/<name of current key>'
% to be used inside of \begin{key} ... \end{key}
\def\keyalias#1{\vspace{-3pt}\item{\small alias {\ttfamily/#1/\myname}}\vspace{-2pt}\par
  \pgfmanualpdflabel{/#1/\myname}{}%
}

\newif\iffirsttime
\newcount\pgfmanualdecomposecount

\makeatother

\def\decompose/#1/#2\nil{%
  \def\test{#2}%
  \ifx\test\empty%
    % aha.
    \index{#1@\protect\texttt{#1} key}%
    \index{\mypath#1@\protect\texttt{#1}}%
    \gdef\myname{#1}%
    \pgfmanualpdflabel{#1}{}
  \else%
    \advance\pgfmanualdecomposecount by1\relax%
    \ifnum\pgfmanualdecomposecount>2\relax%
      \decomposetoodeep#1/#2\nil%
    \else%
      \iffirsttime%
        \begingroup%  
          % also make a pdf link anchor with full key path.
          \def\hyperlabelwithoutslash##1/\nil{%
            \pgfmanualpdflabel{##1}{}%
          }%
          \hyperlabelwithoutslash/#1/#2\nil%
        \endgroup%
        \def\mypath{#1@\protect\texttt{/#1/}!}%
        \firsttimefalse%
      \else%
        \expandafter\def\expandafter\mypath\expandafter{\mypath#1@\protect\texttt{#1/}!}%
      \fi%
      \def\firsttime{}%
      \decompose/#2\nil%
    \fi%
  \fi%
}

\def\decomposetoodeep#1/#2/\nil{%
  % avoid too-deep nesting in index
  \index{#1/#2@\protect\texttt{#1/#2} key}%
  \index{\mypath#1/#2@\protect\texttt{#1/#2}}%
  \decomposefindlast/#1/#2/\nil%
}
\makeatletter
\def\decomposefindlast/#1/#2\nil{%
  \def\test{#2}%
  \ifx\test\pgfutil@empty%
    \gdef\myname{#1}%
  \else%
    \decomposefindlast/#2\nil%
  \fi%
}
\makeatother
\def\indexkey#1{%
  \def\mypath{}%
  \decompose#1/\nil%
}

\newenvironment{predefinedmethod}[1]{
  \begin{pgfmanualentry}
    \extractpredefinedmethod#1\@nil
    \pgfmanualbody
}
{
  \end{pgfmanualentry}
}
\def\extractpredefinedmethod#1(#2)\@nil{%
  \pgfmanualentryheadline{%
    \pgfmanualpdflabel{#1}{}%
    Method \declare{\ttfamily #1}\texttt(#2\texttt) \hfill(predefined for all classes)}
  \index{#1@\protect\texttt{#1} method}%
  \index{Methods!#1@\protect\texttt{#1}}
}


\newenvironment{ooclass}[1]{
  \begin{pgfmanualentry}
    \def\currentclass{#1}
    \pgfmanualentryheadline{%
      \pgfmanualpdflabel{#1}{}%
      \textbf{Class} \declare{\texttt{#1}}}
    \index{#1@\protect\texttt{#1} class}%
    \index{Class #1@Class \protect\texttt{#1}}%
    \index{Classes!#1@\protect\texttt{#1}}
    \pgfmanualbody
}
{
  \end{pgfmanualentry}
}

\newenvironment{method}[1]{
  \begin{pgfmanualentry}
    \extractmethod#1\@nil
    \pgfmanualbody
}
{
  \end{pgfmanualentry}
}
\def\extractmethod#1(#2)\@nil{%
  \def\test{#1}
  \ifx\test\currentclass
    \pgfmanualentryheadline{%
      \pgfmanualpdflabel{#1}{}%
      Constructor \declare{\ttfamily #1}\texttt(#2\texttt)}
  \else
    \pgfmanualentryheadline{%
      \pgfmanualpdflabel{#1}{}%
      Method \declare{\ttfamily #1}\texttt(#2\texttt)}
  \fi
  \index{#1@\protect\texttt{#1} method}%
  \index{Methods!#1@\protect\texttt{#1}}
  \index{Class \currentclass!#1@\protect\texttt{#1}}%
}

\newenvironment{classattribute}[1]{
  \begin{pgfmanualentry}
    \extractattribute#1\@nil
    \pgfmanualbody
}
{
  \end{pgfmanualentry}
}
\def\extractattribute#1=#2;\@nil{%
  \def\test{#2}%
  \ifx\test\@empty
    \pgfmanualentryheadline{%
      \pgfmanualpdflabel{#1}{}%
      Private attribute \declare{\ttfamily #1} \hfill (initially empty)}
  \else
    \pgfmanualentryheadline{%
      \pgfmanualpdflabel{#1}{}%
      Private attribute \declare{\ttfamily #1} \hfill (initially {\ttfamily #2})}
  \fi
  \index{#1@\protect\texttt{#1} attribute}%
  \index{Attributes!#1@\protect\texttt{#1}}
  \index{Class \currentclass!#1@\protect\texttt{#1}}%
}



\newenvironment{predefinednode}[1]{
  \begin{pgfmanualentry}
    \pgfmanualentryheadline{%
      \pgfmanualpdflabel{#1}{}%
      \textbf{Predefined node} {\ttfamily\declare{#1}}}%
    \index{#1@\protect\texttt{#1} node}%
    \index{Predefined node!#1@\protect\texttt{#1}}
    \pgfmanualbody
}
{
  \end{pgfmanualentry}
}

\newenvironment{coordinatesystem}[1]{
  \begin{pgfmanualentry}
    \pgfmanualentryheadline{%
      \pgfmanualpdflabel{#1}{}%
      \textbf{Coordinate system} {\ttfamily\declare{#1}}}%
    \index{#1@\protect\texttt{#1} coordinate system}%
    \index{Coordinate systems!#1@\protect\texttt{#1}}
    \pgfmanualbody
}
{
  \end{pgfmanualentry}
}

\newenvironment{snake}[1]{
  \begin{pgfmanualentry}
    \pgfmanualentryheadline{\textbf{Snake} {\ttfamily\declare{#1}}}%
    \index{#1@\protect\texttt{#1} snake}%
    \index{Snakes!#1@\protect\texttt{#1}}
    \pgfmanualbody
}
{
  \end{pgfmanualentry}
}

\newenvironment{decoration}[1]{
  \begin{pgfmanualentry}
    \pgfmanualentryheadline{\textbf{Decoration} {\ttfamily\declare{#1}}}%
    \index{#1@\protect\texttt{#1} decoration}%
    \index{Decorations!#1@\protect\texttt{#1}}
    \pgfmanualbody
}
{
  \end{pgfmanualentry}
}


\def\pgfmanualbar{\char`\|}
\makeatletter
\newenvironment{pathoperation}[3][]{
  \begin{pgfmanualentry}
    \def\pgfmanualtest{#1}%
    \pgfmanualentryheadline{%
      \ifx\pgfmanualtest\@empty%
        \pgfmanualpdflabel{#2}{}%
      \fi%
      \textcolor{gray}{{\ttfamily\char`\\path}\
        \ \dots}
      \declare{\texttt{\noligs{#2}}}#3\ \textcolor{gray}{\dots\texttt{;}}}%
    \ifx\pgfmanualtest\@empty%
      \index{#2@\protect\texttt{#2} path operation}%
      \index{Path operations!#2@\protect\texttt{#2}}%
    \fi%
    \pgfmanualbody
}
{
  \end{pgfmanualentry}
}
\newenvironment{datavisualizationoperation}[3][]{
  \begin{pgfmanualentry}
    \def\pgfmanualtest{#1}%
    \pgfmanualentryheadline{%
      \ifx\pgfmanualtest\@empty%
        \pgfmanualpdflabel{#2}{}%
      \fi%
      \textcolor{gray}{{\ttfamily\char`\\datavisualization}\
        \ \dots}
      \declare{\texttt{\noligs{#2}}}#3\ \textcolor{gray}{\dots\texttt{;}}}%
    \ifx\pgfmanualtest\@empty%
      \index{#2@\protect\texttt{#2} (data visualization)}%
      \index{Data visualization!#2@\protect\texttt{#2}}%
    \fi%
    \pgfmanualbody
}
{
  \end{pgfmanualentry}
}
\makeatother

\def\doublebs{\texttt{\char`\\\char`\\}}


\newenvironment{package}[1]{
  \begin{pgfmanualentry}
    \pgfmanualentryheadline{%
      \pgfmanualpdflabel{#1}{}%
      {\ttfamily\char`\\usepackage\char`\{\declare{#1}\char`\}\space\space \char`\%\space\space  \LaTeX}}
    \index{#1@\protect\texttt{#1} package}%
    \index{Packages and files!#1@\protect\texttt{#1}}%
    \pgfmanualentryheadline{{\ttfamily\char`\\input \declare{#1}.tex\space\space\space \char`\%\space\space  plain \TeX}}
    \pgfmanualentryheadline{{\ttfamily\char`\\usemodule[\declare{#1}]\space\space \char`\%\space\space  Con\TeX t}}
    \pgfmanualbody
}
{
  \end{pgfmanualentry}
}


\newenvironment{pgfmodule}[1]{
  \begin{pgfmanualentry}
    \pgfmanualentryheadline{%
      \pgfmanualpdflabel{#1}{}%
      {\ttfamily\char`\\usepgfmodule\char`\{\declare{#1}\char`\}\space\space\space
        \char`\%\space\space  \LaTeX\space and plain \TeX\space and pure pgf}}
    \index{#1@\protect\texttt{#1} module}%
    \index{Modules!#1@\protect\texttt{#1}}%
    \pgfmanualentryheadline{{\ttfamily\char`\\usepgfmodule[\declare{#1}]\space\space \char`\%\space\space  Con\TeX t\space and pure pgf}}
    \pgfmanualbody
}
{
  \end{pgfmanualentry}
}

\newenvironment{pgflibrary}[1]{
  \begin{pgfmanualentry}
    \pgfmanualentryheadline{%
      \pgfmanualpdflabel{#1}{}%
      \textbf{\tikzname\ Library} \texttt{\declare{#1}}}
    \index{#1@\protect\texttt{#1} library}%
    \index{Libraries!#1@\protect\texttt{#1}}%
    \vskip.25em%
    {{\ttfamily\char`\\usepgflibrary\char`\{\declare{#1}\char`\}\space\space\space
        \char`\%\space\space  \LaTeX\space and plain \TeX\space and pure pgf}}\\
    {{\ttfamily\char`\\usepgflibrary[\declare{#1}]\space\space \char`\%\space\space  Con\TeX t\space and pure pgf}}\\
    {{\ttfamily\char`\\usetikzlibrary\char`\{\declare{#1}\char`\}\space\space
        \char`\%\space\space  \LaTeX\space and plain \TeX\space when using \tikzname}}\\
    {{\ttfamily\char`\\usetikzlibrary[\declare{#1}]\space
        \char`\%\space\space  Con\TeX t\space when using \tikzname}}\\[.5em]
    \pgfmanualbody
}
{
  \end{pgfmanualentry}
}

\newenvironment{purepgflibrary}[1]{
  \begin{pgfmanualentry}
    \pgfmanualentryheadline{%
      \pgfmanualpdflabel{#1}{}%
      \textbf{{\small PGF} Library} \texttt{\declare{#1}}}
    \index{#1@\protect\texttt{#1} library}%
    \index{Libraries!#1@\protect\texttt{#1}}%
    \vskip.25em%
    {{\ttfamily\char`\\usepgflibrary\char`\{\declare{#1}\char`\}\space\space\space
        \char`\%\space\space  \LaTeX\space and plain \TeX}}\\
    {{\ttfamily\char`\\usepgflibrary[\declare{#1}]\space\space \char`\%\space\space  Con\TeX t}}\\[.5em]
    \pgfmanualbody
}
{
  \end{pgfmanualentry}
}

\newenvironment{tikzlibrary}[1]{
  \begin{pgfmanualentry}
    \pgfmanualentryheadline{%
      \pgfmanualpdflabel{#1}{}%
      \textbf{\tikzname\ Library} \texttt{\declare{#1}}}
    \index{#1@\protect\texttt{#1} library}%
    \index{Libraries!#1@\protect\texttt{#1}}%
    \vskip.25em%
    {{\ttfamily\char`\\usetikzlibrary\char`\{\declare{#1}\char`\}\space\space \char`\%\space\space  \LaTeX\space and plain \TeX}}\\
    {{\ttfamily\char`\\usetikzlibrary[\declare{#1}]\space \char`\%\space\space Con\TeX t}}\\[.5em]
    \pgfmanualbody
}
{
  \end{pgfmanualentry}
}



\newenvironment{filedescription}[1]{
  \begin{pgfmanualentry}
    \pgfmanualentryheadline{File {\ttfamily\declare{#1}}}%
    \index{#1@\protect\texttt{#1} file}%
    \index{Packages and files!#1@\protect\texttt{#1}}%
    \pgfmanualbody
}
{
  \end{pgfmanualentry}
}


\newenvironment{packageoption}[1]{
  \begin{pgfmanualentry}
    \pgfmanualentryheadline{{\ttfamily\char`\\usepackage[\declare{#1}]\char`\{pgf\char`\}}}
    \index{#1@\protect\texttt{#1} package option}%
    \index{Package options for \textsc{pgf}!#1@\protect\texttt{#1}}%
    \pgfmanualbody
}
{
  \end{pgfmanualentry}
}



\newcommand\opt[1]{{\color{black!50!green}#1}}
\newcommand\ooarg[1]{{\ttfamily[}\meta{#1}{\ttfamily]}}

\def\opt{\afterassignment\pgfmanualopt\let\next=}
\def\pgfmanualopt{\ifx\next\bgroup\bgroup\color{black!50!green}\else{\color{black!50!green}\next}\fi}



\def\beamer{\textsc{beamer}}
\def\pdf{\textsc{pdf}}
\def\eps{\texttt{eps}}
\def\pgfname{\textsc{pgf}}
\def\tikzname{Ti\emph{k}Z}
\def\pstricks{\textsc{pstricks}}
\def\prosper{\textsc{prosper}}
\def\seminar{\textsc{seminar}}
\def\texpower{\textsc{texpower}}
\def\foils{\textsc{foils}}

{
  \makeatletter
  \global\let\myempty=\@empty
  \global\let\mygobble=\@gobble
  \catcode`\@=12
  \gdef\getridofats#1@#2\relax{%
    \def\getridtest{#2}%
    \ifx\getridtest\myempty%
      \expandafter\def\expandafter\strippedat\expandafter{\strippedat#1}
    \else%
      \expandafter\def\expandafter\strippedat\expandafter{\strippedat#1\protect\printanat}
      \getridofats#2\relax%
    \fi%
  }

  \gdef\removeats#1{%
    \let\strippedat\myempty%
    \edef\strippedtext{\stripcommand#1}%
    \expandafter\getridofats\strippedtext @\relax%
  }
  
  \gdef\stripcommand#1{\expandafter\mygobble\string#1}
}

\def\printanat{\char`\@}

\def\declare{\afterassignment\pgfmanualdeclare\let\next=}
\def\pgfmanualdeclare{\ifx\next\bgroup\bgroup\color{red!75!black}\else{\color{red!75!black}\next}\fi}


\let\textoken=\command
\let\endtextoken=\endcommand

\def\myprintocmmand#1{\texttt{\char`\\#1}}

\def\example{\par\smallskip\noindent\textit{Example: }}
\def\themeauthor{\par\smallskip\noindent\textit{Theme author: }}


\def\indexoption#1{%
  \index{#1@\protect\texttt{#1} option}%
  \index{Graphic options and styles!#1@\protect\texttt{#1}}%
}

\def\itemcalendaroption#1{\item \declare{\texttt{#1}}%
  \index{#1@\protect\texttt{#1} date test}%
  \index{Date tests!#1@\protect\texttt{#1}}%
}



\def\class#1{\list{}{\leftmargin=2em\itemindent-\leftmargin\def\makelabel##1{\hss##1}}%
\extractclass#1@\par\topsep=0pt}
\def\endclass{\endlist}
\def\extractclass#1#2@{%
\item{{{\ttfamily\char`\\documentclass}#2{\ttfamily\char`\{\declare{#1}\char`\}}}}%
  \index{#1@\protect\texttt{#1} class}%
  \index{Classes!#1@\protect\texttt{#1}}}

\def\partname{Part}

\makeatletter
\def\index@prologue{\section*{Index}\addcontentsline{toc}{section}{Index}
  This index only contains automatically generated entries. A good
  index should also contain carefully selected keywords. This index is
  not a good index.
  \bigskip
}
\c@IndexColumns=2
  \def\theindex{\@restonecoltrue
    \columnseprule \z@  \columnsep 29\p@
    \twocolumn[\index@prologue]%
       \parindent -30pt
       \columnsep 15pt
       \parskip 0pt plus 1pt
       \leftskip 30pt
       \rightskip 0pt plus 2cm
       \small
       \def\@idxitem{\par}%
    \let\item\@idxitem \ignorespaces}
  \def\endtheindex{\onecolumn}
\def\noindexing{\let\index=\@gobble}


\newenvironment{arrowtipsimple}[1]{
  \begin{pgfmanualentry}
    \pgfmanualentryheadline{\textbf{Arrow Tip Kind} {\ttfamily#1}}
    \index{#1@\protect\texttt{#1} arrow tip}%
    \index{Arrow tips!#1@\protect\texttt{#1}}%
    \def\currentarrowtype{#1}
    \pgfmanualbody}
{
  \end{pgfmanualentry}
}

\newenvironment{arrowtip}[4]{
  \begin{pgfmanualentry}
    \pgfmanualentryheadline{\textbf{Arrow Tip Kind} {\ttfamily#1}}
    \index{#1@\protect\texttt{#1} arrow tip}%
    \index{Arrow tips!#1@\protect\texttt{#1}}%
    \pgfmanualbody
    \def\currentarrowtype{#1}
    \begin{minipage}[t]{10.25cm}
      #2
    \end{minipage}\hskip5mm\begin{minipage}[t]{4.75cm}
      \leavevmode\vskip-2em
    \tikz{
      \draw [black!50,line width=5mm,-{#1[#3,color=black]}] (-4,0) -- (0,0);
      \foreach \action in {#4}
      { \expandafter\processaction\action\relax }
    }
    \end{minipage}\par\smallskip
  }
{
  \end{pgfmanualentry}
}

\newenvironment{arrowcap}[5]{
  \begin{pgfmanualentry}
    \pgfmanualentryheadline{\textbf{Arrow Tip Kind} {\ttfamily#1}}
    \index{#1@\protect\texttt{#1} arrow tip}%
    \index{Arrow tips!#1@\protect\texttt{#1}}%
    \pgfmanualbody
    \def\currentarrowtype{#1}
    \begin{minipage}[t]{10.25cm}
      #2
    \end{minipage}\hskip5mm\begin{minipage}[t]{4.75cm}
      \leavevmode\vskip-2em
    \tikz{
      \path [tips, line width=10mm,-{#1[#3,color=black]}] (-4,0) -- (0,0);
      \draw [line width=10mm,black!50] (-3,0) -- (#5,0);
      \foreach \action in {#4}
      { \expandafter\processaction\action\relax }
    }
    \end{minipage}\par\smallskip
  }
{
  \end{pgfmanualentry}
}

\newenvironment{pattern}[1]{
  \begin{pgfmanualentry}
    \pgfmanualentryheadline{\textbf{Pattern} {\ttfamily#1}}
    \index{#1@\protect\texttt{#1} pattern}%
    \index{Patterns!#1@\protect\texttt{#1}}%
    \pgfmanualbody
}
{
  \end{pgfmanualentry}
}

\def\processaction#1=#2\relax{
  \expandafter\let\expandafter\pgf@temp\csname manual@action@#1\endcsname
  \ifx\pgf@temp\relax\else
    \pgf@temp#2/0/\relax
  \fi
}
\def\manual@action@length#1/#2/#3\relax{%
  \draw [red,|<->|,semithick,xshift=#2] ([yshift=4pt]current bounding
  box.north -| -#1,0) coordinate (last length) -- node
  [above=-2pt] {|length|} ++(#1,0);
}
\def\manual@action@width#1/#2/#3\relax{%
  \draw [overlay, red,|<->|,semithick] (.5,-#1/2) -- node [below,sloped] {|width|} (.5,#1/2);
}
\def\manual@action@inset#1/#2/#3\relax{%
  \draw [red,|<->|,semithick,xshift=#2] ([yshift=-4pt]current bounding
  box.south -| last length) -- node [below] {|inset|} ++(#1,0);
}

\newenvironment{arrowexamples}
{\begin{tabbing}
    \hbox to \dimexpr\linewidth-5.5cm\relax{\emph{Appearance of the below at line width} \hfil} \= 
     \hbox to 1.9cm{\emph{0.4pt}\hfil} \= \hbox to 2cm{\emph{0.8pt}\hfil} \= \emph{1.6pt} \\
  }
{\end{tabbing}\vskip-1em}

\newenvironment{arrowcapexamples}
{\begin{tabbing}
    \hbox to \dimexpr\linewidth-5.5cm\relax{\emph{Appearance of the below at line width} \hfil} \= 
     \hbox to 1.9cm{\emph{1ex}\hfil} \= \hbox to 2cm{\emph{1em}\hfil} \\
  }
{\end{tabbing}\vskip-1em}

\def\arrowcapexample#1[#2]{\def\temp{#1}\ifx\temp\pgfutil@empty\arrowcapexample@\currentarrowtype[{#2}]\else\arrowcapexample@#1[{#2}]\fi}
\def\arrowcapexample@#1[#2]{%
  {\sfcode`\.1000\small\texttt{#1[#2]}} \>
  \kern-.5ex\tikz [baseline,>={#1[#2]}] \draw [line
  width=1ex,->] (0,.5ex) -- (2em,.5ex);  \>
  \kern-.5em\tikz [baseline,>={#1[#2]}] \draw [line
  width=1em,->] (0,.5ex) -- (2em,.5ex);  \\
}

\def\arrowexample#1[#2]{\def\temp{#1}\ifx\temp\pgfutil@empty\arrowexample@\currentarrowtype[{#2}]\else\arrowexample@#1[{#2}]\fi}
\def\arrowexample@#1[#2]{%
  {\sfcode`\.1000\small\texttt{#1[#2]}} \>
  \tikz [baseline,>={#1[#2]}] \draw [line
  width=0.4pt,->] (0,.5ex) -- (2em,.5ex); thin \>
  \tikz [baseline,>={#1[#2]}] \draw [line
  width=0.8pt,->] (0,.5ex) -- (2em,.5ex); \textbf{thick} \>
  \tikz [baseline,>={#1[#2]}] \draw [line
  width=1.6pt,->] (0,.5ex) -- (3em,.5ex); \\
}
\def\arrowexampledup[#1]{\arrowexample[{#1] \currentarrowtype[}]}
\def\arrowexampledupdot[#1]{\arrowexample[{#1] . \currentarrowtype[}]}

\def\arrowexampledouble#1[#2]{\def\temp{#1}\ifx\temp\pgfutil@empty\arrowexampledouble@\currentarrowtype[{#2}]\else\arrowexampledouble@#1[{#2}]\fi}
\def\arrowexampledouble@#1[#2]{%
  {\sfcode`\.1000\small\texttt{#1[#2]} on double line} \>
  \tikz [baseline,>={#1[#2]}]
    \draw [double equal sign distance,line width=0.4pt,->] (0,.5ex) -- (2em,.5ex); thin \>
  \tikz [baseline,>={#1[#2]}]
    \draw [double equal sign distance,line width=0.8pt,->] (0,.5ex) -- (2em,.5ex); \textbf{thick} \>
  \tikz [baseline,>={#1[#2]}]
    \draw [double equal sign distance, line width=1.6pt,->] (0,.5ex) -- (3em,.5ex); \\
}



\newcommand\symarrow[1]{%
  \index{#1@\protect\texttt{#1} arrow tip}%
  \index{Arrow tips!#1@\protect\texttt{#1}}%
  \texttt{#1}& yields thick  
  \begin{tikzpicture}[arrows={#1-#1},thick,baseline]
    \useasboundingbox (-1mm,-0.5ex) rectangle (1.1cm,2ex);
    \fill [black!15] (1cm,-.5ex) rectangle (1.1cm,1.5ex) (-1mm,-.5ex) rectangle (0mm,1.5ex) ;
    \draw (0pt,.5ex) -- (1cm,.5ex);
  \end{tikzpicture} and thin
  \begin{tikzpicture}[arrows={#1-#1},thin,baseline]
    \useasboundingbox (-1mm,-0.5ex) rectangle (1.1cm,2ex);
    \fill [black!15] (1cm,-.5ex) rectangle (1.1cm,1.5ex) (-1mm,-.5ex) rectangle (0mm,1.5ex) ;
    \draw (0pt,.5ex) -- (1cm,.5ex);
  \end{tikzpicture}
}
\newcommand\symarrowdouble[1]{%
  \index{#1@\protect\texttt{#1} arrow tip}%
  \index{Arrow tips!#1@\protect\texttt{#1}}%
  \texttt{#1}& yields thick  
  \begin{tikzpicture}[arrows={#1-#1},thick,baseline]
    \useasboundingbox (-1mm,-0.5ex) rectangle (1.1cm,2ex);
    \fill [black!15] (1cm,-.5ex) rectangle (1.1cm,1.5ex) (-1mm,-.5ex) rectangle (0mm,1.5ex) ;
    \draw (0pt,.5ex) -- (1cm,.5ex);
  \end{tikzpicture}
  and thin
  \begin{tikzpicture}[arrows={#1-#1},thin,baseline]
    \useasboundingbox (-1mm,-0.5ex) rectangle (1.1cm,2ex);
    \fill [black!15] (1cm,-.5ex) rectangle (1.1cm,1.5ex) (-1mm,-.5ex) rectangle (0mm,1.5ex) ;
    \draw (0pt,.5ex) -- (1cm,.5ex);
  \end{tikzpicture}, double 
  \begin{tikzpicture}[arrows={#1-#1},thick,baseline]
    \useasboundingbox (-1mm,-0.5ex) rectangle (1.1cm,2ex);
    \fill [black!15] (1cm,-.5ex) rectangle (1.1cm,1.5ex) (-1mm,-.5ex) rectangle (0mm,1.5ex) ;
    \draw[double,double equal sign distance] (0pt,.5ex) -- (1cm,.5ex);
  \end{tikzpicture} and 
  \begin{tikzpicture}[arrows={#1-#1},thin,baseline]
    \useasboundingbox (-1mm,-0.5ex) rectangle (1.1cm,2ex);
    \fill [black!15] (1cm,-.5ex) rectangle (1.1cm,1.5ex) (-1mm,-.5ex) rectangle (0mm,1.5ex) ;
    \draw[double,double equal sign distance] (0pt,.5ex) -- (1cm,.5ex);
  \end{tikzpicture}
}

\newcommand\sarrow[2]{%
  \index{#1@\protect\texttt{#1} arrow tip}%
  \index{Arrow tips!#1@\protect\texttt{#1}}%
  \index{#2@\protect\texttt{#2} arrow tip}%
  \index{Arrow tips!#2@\protect\texttt{#2}}%
  \texttt{#1-#2}& yields thick  
  \begin{tikzpicture}[arrows={#1-#2},thick,baseline]
    \useasboundingbox (-1mm,-0.5ex) rectangle (1.1cm,2ex);
    \fill [black!15] (1cm,-.5ex) rectangle (1.1cm,1.5ex) (-1mm,-.5ex) rectangle (0mm,1.5ex) ;
    \draw (0pt,.5ex) -- (1cm,.5ex);
  \end{tikzpicture} and thin
  \begin{tikzpicture}[arrows={#1-#2},thin,baseline]
    \useasboundingbox (-1mm,-0.5ex) rectangle (1.1cm,2ex);
    \fill [black!15] (1cm,-.5ex) rectangle (1.1cm,1.5ex) (-1mm,-.5ex) rectangle (0mm,1.5ex) ;
    \draw (0pt,.5ex) -- (1cm,.5ex);
  \end{tikzpicture}
}

\newcommand\sarrowdouble[2]{%
  \index{#1@\protect\texttt{#1} arrow tip}%
  \index{Arrow tips!#1@\protect\texttt{#1}}%
  \index{#2@\protect\texttt{#2} arrow tip}%
  \index{Arrow tips!#2@\protect\texttt{#2}}%
  \texttt{#1-#2}& yields thick  
  \begin{tikzpicture}[arrows={#1-#2},thick,baseline]
    \useasboundingbox (-1mm,-0.5ex) rectangle (1.1cm,2ex);
    \fill [black!15] (1cm,-.5ex) rectangle (1.1cm,1.5ex) (-1mm,-.5ex) rectangle (0mm,1.5ex) ;
    \draw (0pt,.5ex) -- (1cm,.5ex);
  \end{tikzpicture} and thin
  \begin{tikzpicture}[arrows={#1-#2},thin,baseline]
    \useasboundingbox (-1mm,-0.5ex) rectangle (1.1cm,2ex);
    \fill [black!15] (1cm,-.5ex) rectangle (1.1cm,1.5ex) (-1mm,-.5ex) rectangle (0mm,1.5ex) ;
    \draw (0pt,.5ex) -- (1cm,.5ex);
  \end{tikzpicture}, double 
  \begin{tikzpicture}[arrows={#1-#2},thick,baseline]
    \useasboundingbox (-1mm,-0.5ex) rectangle (1.1cm,2ex);
    \fill [black!15] (1cm,-.5ex) rectangle (1.1cm,1.5ex) (-1mm,-.5ex) rectangle (0mm,1.5ex) ;
    \draw[double,double equal sign distance] (0pt,.5ex) -- (1cm,.5ex);
  \end{tikzpicture} and 
  \begin{tikzpicture}[arrows={#1-#2},thin,baseline]
    \useasboundingbox (-1mm,-0.5ex) rectangle (1.1cm,2ex);
    \fill [black!15] (1cm,-.5ex) rectangle (1.1cm,1.5ex) (-1mm,-.5ex) rectangle (0mm,1.5ex) ;
    \draw[double,double equal sign distance] (0pt,.5ex) -- (1cm,.5ex);
  \end{tikzpicture}
}

\newcommand\carrow[1]{%
  \index{#1@\protect\texttt{#1} arrow tip}%
  \index{Arrow tips!#1@\protect\texttt{#1}}%
  \texttt{#1}& yields for line width 1ex
  \begin{tikzpicture}[arrows={#1-#1},line width=1ex,baseline]
    \useasboundingbox (-1mm,-0.5ex) rectangle (1.6cm,2ex);
    \fill [black!15] (1.5cm,-.5ex) rectangle (1.6cm,1.5ex) (-1mm,-.5ex) rectangle (0mm,1.5ex) ;
    \draw (0pt,.5ex) -- (1.5cm,.5ex);
  \end{tikzpicture}
}
\def\myvbar{\char`\|}
\newcommand\plotmarkentry[1]{%
  \index{#1@\protect\texttt{#1} plot mark}%
  \index{Plot marks!#1@\protect\texttt{#1}}
  \texttt{\char`\\pgfuseplotmark\char`\{\declare{\noligs{#1}}\char`\}} &
  \tikz\draw[color=black!25] plot[mark=#1,mark options={fill=examplefill,draw=black}] coordinates{(0,0) (.5,0.2) (1,0) (1.5,0.2)};\\
}
\newcommand\plotmarkentrytikz[1]{%
  \index{#1@\protect\texttt{#1} plot mark}%
  \index{Plot marks!#1@\protect\texttt{#1}}
  \texttt{mark=\declare{\noligs{#1}}} & \tikz\draw[color=black!25]
  plot[mark=#1,mark options={fill=examplefill,draw=black}] 
    coordinates {(0,0) (.5,0.2) (1,0) (1.5,0.2)};\\
}



\ifx\scantokens\@undefined
  \PackageError{pgfmanual-macros}{You need to use extended latex
    (elatex) or (pdfelatex) to process this document}{}
\fi

\begingroup
\catcode`|=0
\catcode`[= 1
\catcode`]=2
\catcode`\{=12
\catcode `\}=12
\catcode`\\=12 |gdef|find@example#1\end{codeexample}[|endofcodeexample[#1]]
|endgroup

% define \returntospace.
%
% It should define NEWLINE as {}, spaces and tabs as \space.
\begingroup
\catcode`\^=7
\catcode`\^^M=13
\catcode`\^^I=13
\catcode`\ =13%
\gdef\returntospace{\catcode`\ =13\def {\space}\catcode`\^^I=13\def^^I{\space}}
\gdef\showreturn{\show^^M}
\endgroup

\begingroup
\catcode`\%=13
\catcode`\^^M=13
\gdef\commenthandler{\catcode`\%=13\def%{\@gobble@till@return}}
\gdef\@gobble@till@return#1^^M{}
\gdef\@gobble@till@return@ignore#1^^M{\ignorespaces}
\gdef\typesetcomment{\catcode`\%=13\def%{\@typeset@till@return}}
\gdef\@typeset@till@return#1^^M{{\def%{\char`\%}\textsl{\char`\%#1}}\par}
\endgroup

% Define tab-implementation functions
%   \codeexample@tabinit@replacementchars@
% and
%   \codeexample@tabinit@catcode@
%
% They should ONLY be used in case that tab replacement is active.
%
% This here is merely a preparation step.
%
% Idea:
% \codeexample@tabinit@catcode@ will make TAB active
% and
% \codeexample@tabinit@replacementchars@ will insert as many spaces as
% /codeexample/tabsize contains.
{
\catcode`\^^I=13
% ATTENTION: do NOT use tabs in these definitions!!
\gdef\codeexample@tabinit@replacementchars@{%
 \begingroup
 \count0=\pgfkeysvalueof{/codeexample/tabsize}\relax
 \toks0={}%
 \loop
 \ifnum\count0>0
  \advance\count0 by-1
  \toks0=\expandafter{\the\toks0\ }%
 \repeat
 \xdef\codeexample@tabinit@replacementchars@@{\the\toks0}%
 \endgroup
 \let^^I=\codeexample@tabinit@replacementchars@@
}%
\gdef\codeexample@tabinit@catcode@{\catcode`\^^I=13}%
}%

% Called after any options have been set. It assigns
%   \codeexample@tabinit@catcode
% and
%   \codeexample@tabinit@replacementchars
% which are used inside of 
%\begin{codeexample}
% ...
%\end{codeexample}
%
% \codeexample@tabinit@catcode  is either \relax or it makes tab
% active.
%
% \codeexample@tabinit@replacementchars is either \relax or it inserts
% a proper replacement sequence for tabs (as many spaces as
% configured)
\def\codeexample@tabinit{%
  \ifnum\pgfkeysvalueof{/codeexample/tabsize}=0\relax
    \let\codeexample@tabinit@replacementchars=\relax
    \let\codeexample@tabinit@catcode=\relax
  \else
    \let\codeexample@tabinit@catcode=\codeexample@tabinit@catcode@
    \let\codeexample@tabinit@replacementchars=\codeexample@tabinit@replacementchars@
  \fi
}

\newif\ifpgfmanualtikzsyntaxhilighting

\pgfqkeys{/codeexample}{%
  width/.code=  {\setlength\codeexamplewidth{#1}},
  graphic/.code=  {\colorlet{graphicbackground}{#1}},
  code/.code=  {\colorlet{codebackground}{#1}},
  execute code/.is if=code@execute,
  hidden/.is if=code@hidden,
  code only/.code=  {\code@executefalse},
  setup code/.code=  {\pgfmanual@setup@codetrue\code@executefalse},
  multipage/.code=  {\code@executefalse\pgfmanual@multipage@codetrue},
  pre/.store in=\code@pre,
  post/.store in=\code@post,
  % #1 is the *complete* environment contents as it shall be
  % typeset. In particular, the catcodes are NOT the normal ones.
  typeset listing/.code=  {#1},
  render instead/.store in=\code@render,
  vbox/.code=  {\def\code@pre{\vbox\bgroup\setlength{\hsize}{\linewidth-6pt}}\def\code@post{\egroup}},
  ignorespaces/.code=  {\let\@gobble@till@return=\@gobble@till@return@ignore},
  leave comments/.code=  {\def\code@catcode@hook{\catcode`\%=12}\let\commenthandler=\relax\let\typesetcomment=\relax},
  tabsize/.initial=0,% FIXME : this here is merely used for indentation. It is just a TAB REPLACEMENT.
  every codeexample/.style={width=4cm+7pt, tikz syntax=true},
  from file/.code={\codeexamplefromfiletrue\def\codeexamplesource{#1}},
  tikz syntax/.is if=pgfmanualtikzsyntaxhilighting,
  animation list/.store in=\code@animation@list,
  animation pre/.store in=\code@animation@pre,
  animation post/.store in=\code@animation@post,
  animation scale/.store in=\pgfmanualanimscale,
  animation bb/.style={
    animation pre={
      \tikzset{
        every picture/.style={
          execute at begin picture={
            \useasboundingbox[clip] #1;}
        }
      }
    }
  },
  preamble/.store in=\code@preamble,
}

\def\pgfmanualanimscale{.5}

\newread\examplesource


% Opening, reading and closing the results file

\def\opensource#1{
  \immediate\openin\examplesource=#1
}
\def\do@codeexamplefromfile{%
  \immediate\openin\examplesource\expandafter{\codeexamplesource}%
  \def\examplelines{}%
  \readexamplelines
  \closein\examplesource
  \expandafter\endofcodeexample\expandafter{\examplelines}%
}

\def\readexamplelines{
  \ifeof\examplesource%
  \else
    \immediate\read\examplesource to \exampleline
    \expandafter\expandafter\expandafter\def\expandafter\expandafter\expandafter\examplelines\expandafter\expandafter\expandafter{\expandafter\examplelines\exampleline}
    \expandafter\readexamplelines%
  \fi
}

\let\code@animation@pre\pgfutil@empty
\let\code@animation@post\pgfutil@empty
\let\code@animation@list\pgfutil@empty

\let\code@pre\pgfutil@empty
\let\code@post\pgfutil@empty
\let\code@render\pgfutil@empty
\let\code@preamble\pgfutil@empty
\def\code@catcode@hook{}

\newif\ifpgfmanual@multipage@code
\newif\ifpgfmanual@setup@code
\newif\ifcodeexamplefromfile
\newdimen\codeexamplewidth
\newif\ifcode@execute
\newif\ifcode@hidden
\newbox\codeexamplebox
\def\codeexample[#1]{%
  \global\let\pgfmanual@do@this\relax%
  \aftergroup\pgfmanual@do@this%
  \begingroup%
  \code@executetrue
  \pgfqkeys{/codeexample}{every codeexample,#1}%
  \pgfmanualswitchoncolors%
  \ifcodeexamplefromfile\begingroup\fi
  \codeexample@tabinit% assigns \codeexample@tabinit@[catcode,replacementchars]
  \parindent0pt
  \begingroup%
  \par% this \par is not inside \ifcode@hidden because we want to switch to vmode
  \ifcode@hidden\else
    \medskip%
  \fi
  \let\do\@makeother%
  \dospecials%
  \obeylines%
  \@vobeyspaces%
  \catcode`\%=13%
  \catcode`\^^M=13%
  \code@catcode@hook%
  \codeexample@tabinit@catcode
  \relax%
  \ifcodeexamplefromfile%
    \expandafter\do@codeexamplefromfile%
  \else%
    \expandafter\find@example%
  \fi}
\def\endofcodeexample#1{%
  \endgroup%
  \ifpgfmanual@setup@code%
    \gdef\pgfmanual@do@this{%
      {%
        \returntospace%
        \commenthandler%
        \xdef\code@temp{#1}% removes returns and comments
      }%
      \edef\pgfmanualmcatcode{\the\catcode`\^^M}%
      \catcode`\^^M=9\relax%
      \expandafter\scantokens\expandafter{\code@temp}%
      \catcode`\^^M=\pgfmanualmcatcode%
    }%
  \fi%
  \ifcode@hidden\else
    \ifcode@execute%
      \setbox\codeexamplebox=\hbox{%
        \ifx\code@render\pgfutil@empty%
        {%
          {%
            \returntospace%
            \commenthandler%
            \xdef\code@temp{#1}% removes returns and comments
          }%
          \catcode`\^^M=9%
          \colorbox{graphicbackground}{\color{black}\ignorespaces%
            \code@pre\expandafter\scantokens\expandafter{\code@temp\ignorespaces}\code@post\ignorespaces}%
        }%
        \else%
          \global\let\code@temp\code@render%
          \colorbox{graphicbackground}{\color{black}\ignorespaces%
            \code@render}%
        \fi%
      }%
      \ifx\code@animation@list\pgfutil@empty%
      \else%
      \setbox\codeexampleboxanim=\vbox{%
        \rightskip0pt\leftskip0pt plus1filll%
        \ifdim\wd\codeexamplebox>\codeexamplewidth%
        \else%
          \hsize\codeexamplewidth%
          \advance\hsize by2cm%
        \fi%
        \leavevmode\catcode`\^^M=9%
        \foreach \pgfmanualtime/\pgfmanualtimehow in\code@animation@list{%
          \setbox\codeexampleboxanim=\hbox{\colorbox{animationgraphicbackground}{%
              \tikzset{make snapshot of=\pgfmanualtime}%
              \scalebox{\pgfmanualanimscale}{\color{black}\ignorespaces%
                \code@animation@pre\expandafter\scantokens\expandafter{\code@temp\ignorespaces}\code@animation@post\ignorespaces}%
            }}%
          \space\raise4pt\hbox to0pt{\vrule width0pt height1em\hbox
            to\wd\codeexampleboxanim{\hfil\scriptsize$t{=}\pgfmanualtimehow \mathrm s$\hfil}\hss}%
          \lower\ht\codeexampleboxanim\box\codeexampleboxanim\hfil\penalty0\hskip0ptplus-1fil%
        }%
      }%
      \setbox\codeexampleboxanim=\hbox{\hbox{}\hskip-2cm\box\codeexampleboxanim}%
      \fi%
      \ifdim\wd\codeexamplebox>\codeexamplewidth%
        \def\code@start{\par}%
        \def\code@flushstart{}\def\code@flushend{}%
        \def\code@mid{\parskip2pt\par\noindent}%
        \def\code@width{\linewidth-6pt}%
        \def\code@end{}%
      \else%
        \def\code@start{%
          \linewidth=\textwidth%
          \parshape \@ne 0pt \linewidth
          \leavevmode%
          \hbox\bgroup}%
        \def\code@flushstart{\hfill}%
        \def\code@flushend{\hbox{}}%
        \def\code@mid{\hskip6pt}%
        \def\code@width{\linewidth-12pt-\codeexamplewidth}%
        \def\code@end{\egroup}%
      \fi%
      \code@start%
      \noindent%
      \begin{minipage}[t]{\codeexamplewidth}\raggedright
        \hrule width0pt%
        \footnotesize\vskip-1em%
        \code@flushstart\box\codeexamplebox\code@flushend%
        \vskip0pt%
        \leavevmode%
        \box\codeexampleboxanim%
        \vskip-1ex
        \leavevmode%
      \end{minipage}%
    \else%
      \def\code@mid{\par}
      \def\code@width{\linewidth-6pt}
      \def\code@end{}
    \fi%
    \code@mid%
      \ifpgfmanual@multipage@code%
        {%
          \pgfkeysvalueof{/codeexample/prettyprint/base color}%
          \pgfmanualdolisting{#1}%
        }%
      \else%
        \colorbox{codebackground}{%
          \pgfkeysvalueof{/codeexample/prettyprint/base color}%
          \begin{minipage}[t]{\code@width}%
            \pgfmanualdolisting{#1}%
          \end{minipage}}%
      \fi%
    \code@end%
    \par%
    \medskip
  \fi
  \endcodeexample\endgroup%
}

\def\endcodeexample{\endgroup}
\newbox\codeexampleboxanim

\def\pgfmanualdolisting#1{%
      {%
        \let\do\@makeother
        \dospecials
        \frenchspacing\@vobeyspaces
        \normalfont\ttfamily\footnotesize
        \typesetcomment%
        \codeexample@tabinit@replacementchars
        \@tempswafalse
        \def\par{%
          \if@tempswa
          \leavevmode \null \@@par\penalty\interlinepenalty
          \else
          \@tempswatrue
          \ifhmode\@@par\penalty\interlinepenalty\fi
          \fi}%
        \obeylines
        \everypar \expandafter{\the\everypar \unpenalty}%
        \ifx\code@preamble\pgfutil@empty\else
          \pgfutil@tempdima=\hsize
          \vbox{\hsize=\pgfutil@tempdima
              \raggedright\scriptsize\detokenize\expandafter{\code@preamble}}%
        \fi
        \pgfkeysvalueof{/codeexample/typeset listing/.@cmd}{#1}\pgfeov
      }%
}

\makeatother

\usepackage{pgfmanual}


% autoxref is now always on

% \makeatletter
% % \pgfautoxrefs will be defined by 'make dist'
% \pgfutil@ifundefined{pgfautoxrefs}{%
%   \renewcommand\pgfmanualpdflabel[3][]{#3}% NO-OP
%   \def\pgfmanualpdfref#1#2{#2}%
%   \pgfkeys{
%     /pdflinks/codeexample links=false,% DISABLED.
%   }%
% }{}
% \makeatother

\newdimen\pgfmanualcslinkpreskip

% Styling of the pretty printer
\pgfkeys{
  /codeexample/syntax hilighting/.style={
    /codeexample/prettyprint/key name/.code={\textcolor{keycolor}{\pgfmanualpdfref{##1}{\noligs{##1}}}},
    /codeexample/prettyprint/key name with handler/.code 2 args={\textcolor{keycolor}{\pgfmanualpdfref{##1}{\noligs{##1}}}/\textcolor{blue!70!black}{\pgfmanualpdfref{/handlers/##2}{\noligs{##2}}}},
    /codeexample/prettyprint/key value display only/.code={\textcolor{keycolor}{{\itshape{\let\pgfmanualwordstartup\relax\pgfmanualprettyprintcode{##1}}}}},
    /codeexample/prettyprint/cs/.code={\textcolor{cscolor}{\pgfmanualcslinkpreskip4.25pt\pgfmanualpdfref{##1}{\noligs{##1}}}},
    /codeexample/prettyprint/cs with args/.code 2 args={\textcolor{black}{\pgfmanualcslinkpreskip4.25pt\pgfmanualpdfref{##1}{\noligs{##1}}}\{\textcolor{black}{\pgfmanualprettyprintcode{##2}}\pgfmanualclosebrace},
    /codeexample/prettyprint/cs arguments/pgfkeys/.initial=1,
    /codeexample/prettyprint/cs/pgfkeys/.code 2 args={\textcolor{black}{\pgfmanualcslinkpreskip4.25pt\pgfmanualpdfref{##1}{\noligs{##1}}}\{\textcolor{black}{\pgfmanualprettyprintpgfkeys{##2}}\pgfmanualclosebrace},
    /codeexample/prettyprint/cs arguments/begin/.initial=1,
    /codeexample/prettyprint/cs/begin/.code 2 args={\textcolor{black}{##1}\{\textcolor{cscolor}{\pgfmanualpdfref{##2}{\noligs{##2}}}\pgfmanualclosebrace},
    /codeexample/prettyprint/cs arguments/end/.initial=1,
    /codeexample/prettyprint/cs/end/.code 2 args={\textcolor{black}{##1}\{\textcolor{cscolor}{\pgfmanualpdfref{##2}{\noligs{##2}}}\pgfmanualclosebrace},
    /codeexample/prettyprint/word/.code={\pgfmanualwordstartup{\begingroup\pgfkeyssetvalue{/pdflinks/search key prefixes in}{}\pgfmanualpdfref{##1}{\noligs{##1}}\endgroup}},
    /codeexample/prettyprint/point/.code={\textcolor{pointcolor}{\noligs{##1}}},%
    /codeexample/prettyprint/point with cs/.code 2 args={\textcolor{pointcolor}{(\pgfmanualpdfref{##1}{\noligs{##1}}:\noligs{##2}}},%
    /codeexample/prettyprint/comment font=\itshape,
    /codeexample/prettyprint/base color/.initial=\color{basecolor},
    /pdflinks/render hyperlink/.code={%
      {\setbox0=\hbox{##1}%
        \rlap{{\color{linkcolor}\dimen0\wd0\advance\dimen0by-\pgfmanualcslinkpreskip\hskip\pgfmanualcslinkpreskip\vrule width\dimen0 height-1pt depth1.6pt}}%
        \box0%
      }%
    }
  },/codeexample/syntax hilighting
}

\colorlet{keycolor}{black}
\colorlet{cscolor}{black}
\colorlet{pointcolor}{black}
\colorlet{basecolor}{black}
\colorlet{linkcolor}{black!8}

\def\pgfmanualswitchoncolors{%
  \colorlet{keycolor}{green!50!black}%
  \colorlet{cscolor}{blue!70!black}
  \colorlet{pointcolor}{violet}
  \colorlet{basecolor}{black!55}
  \colorlet{linkcolor}{white}
}

\makeatletter

\def\pgfmanualwordstartup{\textcolor{black}}

\def\noligs#1{\pgfmanualnoligs#1\kern0pt--\pgf@stop}%
\def\pgfmanualnoligs#1--{%
  \pgfutil@ifnextchar\pgf@stop{#1\pgfutil@gobble}{#1-\kern0pt-\kern0pt\pgfmanualnoligs}%
}
\makeatother


%%% Local Variables: 
%%% mode: latex
%%% TeX-master: "beameruserguide"
%%% End: 
