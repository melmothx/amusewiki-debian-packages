% Copyright 1997 Apostolos Syropoulos
%
% This file is part of the babel-greek package.
% ---------------------------------------------
%
% It may be distributed and/or modified under the
% conditions of the LaTeX Project Public License, either version 1.3
% of this license or (at your option) any later version.
% The latest version of this license is in
%   http://www.latex-project.org/lppl.txt
% and version 1.3 or later is part of all distributions of LaTeX
% version 2003/12/01 or later.
%
% This work has the LPPL maintenance status "maintained".
%
% The Current Maintainer of this work is Günter Milde.

\documentclass[11pt]{article}
\usepackage[american,greek]{babel}
\languageattribute{greek}{polutoniko}
% \languageattribute{greek}{ancient}
\newcommand{\langGreek}{\foreignlanguage{greek}}

\usepackage{athnum,grmath}
\newcommand{\sg}{\selectlanguage{greek}}
\newcommand{\sa}{\selectlanguage{american}}
\begin{document}
% \show\extrasgreek
%\show\extraspolutonikogreek
\selectlanguage{american}

\title{Writing Greek with the \ttfamily greek\rmfamily\  option of the
\ttfamily babel\rmfamily\ package}
\author{Apostolos Syropoulos\\
        366, 28th October Str.\\
        GR-671 00 Xanthi, GREECE\\
        e-mail: \texttt{apostolo@platon.ee.duth.gr}}
\date{October 15, 1997}
\maketitle

\abstract{\noindent
This document describes the use of the Latin transliteration for Greek that
is defined by the LGR font encoding. Today, all modern LaTeX distributions
support literal input of Greek characters, which is the preferred method for
new documents. [G. Milde 2013/12/02]}


%%%%%%%%%%%%%%%%%%%%%%%%%%%%%%%%%%%%%%%%%%%%%%%%%%%%%%%%%%%%%%%%%%%%%%%%
\section{Overview}

The \texttt{greek} option of the \texttt{babel} package is an attempt to
make it possible for someone to write Greek text with \LaTeX. The current
version of the \texttt{greek} option supports the
\langGreek{monotonik'o} and \langGreek{polutonik'o}
accentual systems of the Greek language.
Moreover, there is now support for Greek numerals. One can produce easily
valid Greek numerals both in uppercase and lowercase forms, e.g,
\langGreek{\greeknumeral{1997}}\ and \langGreek{\Greeknumeral{1997}}. The
labels in second and fourth level enumerations are lowercase
and uppercase Greek numerals correspondingly.
%%%%%%%%%%%%%%%%%%%%%%%%%%%%%%%%%%%%%%%%%%%%%%%%%%%%%%%%%%%%%%%%%%%%%%%%%
\section{Typing Greek Text}
By default, \TeX\ understands only 7-bit ASCII characters, so it is not
possible to enter directly Greek letters.%
\footnote{Literal input of Greek characters
is possible with XeTeX, LuaTeX, or the greek-inputenc LaTeX package.
G. Milde, 2013/07/19}
Instead, someone enters Latin letters
which are mapped to their Greek ``counterparts'' by \TeX. The following
table shows the transliteration employed:
\begin{center}
\begin{tabular}{|lllllllllllll|}\hline
\langGreek{a}&
\langGreek{b}&
\langGreek{g}&
\langGreek{d}&
\langGreek{e}&
\langGreek{z}&
\langGreek{h}&
\langGreek{j}&
\langGreek{i}&
\langGreek{k}&
\langGreek{l}&
\langGreek{m}&
\langGreek{n}\\
a& b& g& d&  e&  z&  h&  j&  i&  k&  l&  m&  n\\
\hline
\langGreek{x}&
\langGreek{o}&
\langGreek{p}&
\langGreek{r}&
\langGreek{sv}&
\langGreek{t}&
\langGreek{u}&
\langGreek{f}&
\langGreek{q}&
\langGreek{y}&
\langGreek{w}&
\langGreek{c}& \hbox{ } \\
x&  o&  p&  r&  s&
t&  u&  f&  q&  y&  w& c& \hbox{ }\\ \hline
\end{tabular}
\end{center}
Please, note that in order to produce the letter \langGreek{sv} in isolation
one has to type \texttt{sv}. This feature is due to the strong ligature
that \TeX\ employs.
In the ``modern'' \langGreek{monotonik'o} accentual system only one accent is
used---\langGreek{oxe'ia} (acute). In the traditional \langGreek{polutonik'o}
accentual system we
need more accents and breathing signs. We can produce an accented letter by
prefixing the letter with he symbol that denotes the accent, e.g.,
\texttt{>a'erac} produces the word \sg >a'erac.\sa\footnote{For the
technically inclined reader, we must say that \TeX\ uses the ligature table of
the font in order to determine the character that corresponds to the
input character sequence.} Here are the symbols that are recognized:

\begin{center}
\begin{tabular}{cccc}\hline
Accent & Symbol & Example & Output\\ \hline
acute  & \texttt{'} & \texttt{g'ata} & \langGreek{g'ata}\\
grave  & \texttt{`} & \texttt{dad`i} & \langGreek{dad`i}\\
circumflex & \verb+~+ & \verb+ful~hc+ & \sg\langGreek{ful~hc}\sa\\
rough breathing & \verb+<+ & \verb+<'otan+ & \sg\langGreek{<'otan}\sa\\
smooth breathing & \verb+>+ & \verb+>'aneu+ & \sg\langGreek{>'aneu}\sa\\
subscript & \texttt{|} & \verb+>anate'ilh|+ & \sg\langGreek{>anate'ilh|}\\
dieresis & \texttt{"}& \texttt{qa"ide'uh|c} & \sg\langGreek{qa"ide'uh|c}\\
\hline
\end{tabular}
\end{center}
Note that the subscript symbol is placed \textbf{after} the letter.
The last thing someone must know in order to be able to write normal Greek
text is the punctuation marks used in the language:
\begin{center}
\begin{tabular}{ccc}\hline
Punctuation Sign & Symbol & Output\\ \hline
period   & \texttt{.} & \sg\langGreek{.}\sa\\
semicolon & \texttt{;} & \sg\langGreek{;}\sa\\
exclamation mark & \texttt{!} & \sg\langGreek{!}\sa\\
comma & \texttt{,} & \sg\langGreek{,}\sa\\
colon & \texttt{:} & \sg\langGreek{:}\sa\\
question mark & \texttt{?} & \sg\langGreek{?}\sa\\
left apostrophe & \texttt{``} & \sg\langGreek{``}\sa\\
right apostrophe & \texttt{''} & \sg\langGreek{''}\sa\\
right apostrophe (alias) & \texttt{"} & \sg\langGreek{"}\sa\\
left quotation mark & \texttt{((} & \sg\langGreek{((}\sa\\
right quotation mark & \texttt{))} & \sg\langGreek{))}\sa\\ \hline
\end{tabular}
\end{center}
Using these conventions it is a straightforward exercise to write Greek
\langGreek{polutoniko} text. For example the following excerpt from
\langGreek{D'uskoloc} of \langGreek{M'enandroc}
\sg
\begin{quote}
T'i f'hic? <Id`wn >enj'ede pa~id'' >eleuj'eran\\
t`ac plhs'ion N'umfac stefano~usan, S'wstrate,\\
>er~wn 'ap~hljec e>uj'uc?
\end{quote}
\sa can be produced by the following \LaTeX\ code:
\begin{center}
\begin{tabular}{l}
\verb+T'i f'hic? <Id`wn >enj'ede pa~id'' >eleuj'eran+\\
\verb+t`ac plhs'ion N'umfac stefano~usan, S'wstrate,+\\
\verb+>er~wn 'ap~hljec e>uj'uc?+
\end{tabular}
\end{center}
%%%%%%%%%%%%%%%%%%%%%%%%%%%%%%%%%%%%%%%%%%%%%%%%%%%%%%%%%%%%%%%%%%%%%
\section{Producing Greek Text}
Once the Greek language is selected with the command
\begin{center}
\verb+\selectlanguage{greek}+
\end{center}
whatever we type will be typeset with the Greek fonts. The command
\verb+\ensureascii+ can be used for short passages in some language that
uses the Latin alphabet.
However, all words will be hyphenated by following the Greek hyphenation
rules! Therefore it is better to switch to the other language
with \verb+\foreignlanguage+. The commands \verb+\lgrfont+ and
\verb+\greektext+ switch to the Greek LGR font encoding.
For example, the word \lgrfont{M'imhc} has been produced with the command
\verb+\lgrfont{M'imhc}+. Please note that Greek hyphenation and upcasing rules
are not applied and certain symbols cannot have
their expected result for Greek text, unless someone has selected the Greek
language, e.g., \verb+~+ is such a symbol.

As we have mentioned above this version of the \texttt{greek} option of the
\texttt{babel} package supports the use of Greek numerals. The commands
\verb+\greeknumeral+ and \verb+\Greeknumeral+ produce the lowercase and
the uppercase Greek numeral, e.g.,
\begin{center}
\begin{tabular}{cc}\hline
Command & Output\\ \hline
\verb+\Greeknumeral{9999}+ & \sg\langGreek{\Greeknumeral{9999}}\\
\verb+\greeknumeral{9999}+ & \sg\langGreek{\greeknumeral{9999}}\\
\hline
\end{tabular}
\end{center}
In order to correctly typeset the greek numerals the greek option file
uses the following commands:
\begin{center}
\begin{tabular}{cc}\hline
Command & Output\\ \hline
% symbol names updated to current defaults [GM]
% name used by package "greek-fontenc"                         % obsolete name
\verb+\textdexiakeraia+    & \langGreek{\textdexiakeraia}\\    % \anwtonos
\verb+\textaristerikeraia+ & \langGreek{\textaristerikeraia}\\ % \katwtonos
\verb+\textkoppa+          & \langGreek{\textkoppa}\\          % \qoppa
\verb+\textsampi+          & \langGreek{\textsampi}\\          % \sampi
\verb+\textstigma+         & \langGreek{\textstigma}\\         % \stigma
\verb+\textKoppa+          & \langGreek{\textKoppa}\\          % n.a. (\qoppa)
\verb+\textSampi+          & \langGreek{\textSampi}\\
\verb+\textStigma+         & \langGreek{\textStigma}\\
\hline
\end{tabular}
\end{center}

Additional symbols are available:
\begin{center}
\begin{tabular}{cc}\hline
Command & Output\\ \hline
% symbol names updated to current defaults [GM]
% name used by package "greek-fontenc"                   % obsolete name
\verb+\textDigamma+     & \langGreek{\textDigamma}\\     % \Digamma
\verb+\textdigamma+ 	& \langGreek{\textdigamma}\\     % \ddigamma
\verb+\textQoppa+ 	& \langGreek{\textQoppa}\\       % \varqoppa
\verb+\textqoppa+ 	& \langGreek{\textqoppa}\\       % \VarQoppa
\verb+\textvarstigma+   & \langGreek{\textvarstigma}\\   % \varstigma
\verb+\texteuro+        & \langGreek{\texteuro}\\        % \texteuro
\verb+\textperthousand+ & \langGreek{\textperthousand}\\ % \permill
\hline
\end{tabular}
\end{center}

In traditional Greek typography the first paragraph after a header is
always indented, contrary to the habit of, say, American typography. This
effect can be achieved by using the package \verb+indentfirst+.

The package \verb|athnum| provides the command \verb|\athnum|, with which
one can produce the so called \textit{Athenian numerals}:
\begin{center}
\begin{tabular}{cc}\hline
Command & Output\\ \hline
\verb|\athnum{1997}| & \langGreek{\athnum{1997}}\\
\hline
\end{tabular}
\end{center}

The package \verb|grmath| renames the basic log-like functions with their
greek counterparts:
\begin{center}
\begin{tabular}{cc}\hline
Command & Output\\ \hline
\verb|$\sin^{2}x+\cos^{2}x=1$| & $\sin^{2}x+\cos^{2}x=1$\\
\hline
\end{tabular}
\end{center}
\end{document}
