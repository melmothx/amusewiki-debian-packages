%% $Id: pst-news21.tex 382 2021-12-29 19:19:18Z herbert $
%\RequirePackage{pdfmanagement-testphase}
\DocumentMetadata{}
\documentclass[11pt,english,BCOR=10mm,DIV=12,bibliography=totoc,parskip=false,headings=small,
    headinclude=false,footinclude=false,twoside,usegeometry,dvipsnames]{pst-doc}
    
\usepackage{libertinus}
\usepackage{hvlogos}
\listfiles
%\usepackage[svgnames,dvipsnames,x11names,pspdf=-dALLOWPSTRANSPARENCY]{pstricks-pdf}
\usepackage{showexpl,pst-arrow,pst-plot,pst-geometrictools}
\lstset{explpreset={pos=l,width=-99pt,overhang=0pt,hsep=\columnsep,vsep=\bigskipamount,rframe={},extendedchars},
    escapechar=?}

\usepackage{biblatex}
\addbibresource{PSTricks.bib}
\def\Lcs#1{\texttt{\textbackslash#1}}
\begin{document}

\title{\texttt{News -- \the\year}\\ \Large new macros and bugfixes for the basic package.}
\author{Herbert Voß}
\date{\today}

\settitle

\tableofcontents

\part{\texttt{pstricks} -- package}

%--------------------------------------------------------------------------------------
\section{\texttt{pstricks.sty}}
%--------------------------------------------------------------------------------------
With \LuaLaTeX\ the package now tries to detect if the command \Lcs{DocumentMetadata} is
needed. If yes, than it is defined as \Verb|\DocumentMetadata{}|. This code does not really
worked, the reason why the user has to set \Lcs{DocumentMetadata} by it's own.


%--------------------------------------------------------------------------------------
\section{\texttt{pstricks.tex} (v. 3.19 -- 2023/04/30)}
%--------------------------------------------------------------------------------------

Added a test for \verb|lualatex| before the default linewidth and color are set.
See last line in \verb|pstricks.tex|

\section{\LuaLaTeX}\label{lua}
This version has a stable basic support for the lua package
\LPack{luapstricks.lua}, available from \url{https://github.com/zauguin/luapstricks}.
This is also part of every \MiKTeX\ or \TeXLive\ installation.  This documentation was
run with \verb|lualatex|, which creates directly the pdf. No GhostScript needed.

Example:


  \def\myline#1{\psline[linecolor=red,linewidth=0.5pt,arrowscale=1.5]{#1}(0,1ex)(1.3,1ex)}%
  \def\mylineA#1{\psline[linecolor=red,linewidth=0.5pt,arrowscale=4.5]{#1}(0,1ex)(2,1ex)}%
  \psset{arrowscale=1.5}
  \begin{longtable}{@{} c @{\qquad} p{3cm} l @{}}%
    Value & Example & Name \\[2pt]\hline
    \Lnotation{-}      & \myline{-}      & None\\
    \Lnotation{<->}    & \myline{<->}    & Arrowheads.\\
    \Lnotation{>-<}    & \myline{>-<}    & Reverse arrowheads.\\
    \Lnotation{<{<}-{>}>}  & \myline{<<->>}  & Double arrowheads.\\
    \Lnotation{{>}>-{<}<}  & \myline{>>-<<}  & Double reverse arrowheads.\\
    \Lnotation{{|}-{|}}    & \myline{|-|}    & T-bars, flush to endpoints.\\
    \Lnotation{{|}*-{|}*}  & \myline{|*-|*}  & T-bars, centered on endpoints.\\
    \Lnotation{[-]}    & \myline{[-]}    & Square brackets.\\
    \Lnotation{]-[}    & \myline{]-[}    & Reversed square brackets.\\
    \Lnotation{(-)}    & \myline{(-)}    & Rounded brackets.\\
    \Lnotation{)-(}    & \myline{)-(}    & Reversed rounded brackets.\\
    \Lnotation{o-o}    & \myline{o-o}    & Circles, centered on endpoints.\\
    \Lnotation{*-*}    & \myline{*-*}    & Disks, centered on endpoints.\\
    \Lnotation{oo-oo}  & \myline{oo-oo}  & Circles, flush to endpoints.\\
    \Lnotation{**-**}  & \myline{**-**}  & Disks, flush to endpoints.\\
    \Lnotation{{|}<->{|}}  & \myline{|<->|}  & T-bars and arrows.\\
    \Lnotation{{|}>-<{|}}  & \myline{|>-<|}  & T-bars and reverse arrows.\\
    \Lnotation{h-h{}}   & \myline{h-h}    & left/right hook arrows.\\
    \Lnotation{H-H{}}   & \myline{H-H}    & left/right hook arrows.\\
    \Lnotation{v-v}   & \myline{v-v}    & left/right inside vee arrows.\\
    \Lnotation{V-V}   & \myline{V-V}    & left/right outside vee arrows.\\
    \Lnotation{f-f}   & \myline{f-f}    & left/right inside filled arrows.\\
    \Lnotation{F-F}   & \myline{F-F}    & left/right outside filled arrows.\\
    \Lnotation{t-t}   & \myline{t-t}    & left/right inside slash arrows.\\[5pt]
    \Lnotation{T-T}   & \myline{T-T}    & left/right outside slash arrows.\\
%
    \Lnotation{<D-D>}   & \mylineA{<D-D>}    & curved  arrows.\\
    \Lnotation{<D<D-D>D>}   & \mylineA{<D<D-D>D>}    & curved doubled arrows.\\
    \Lnotation{D>-<D}   & \mylineA{D>-<D}    & curved  arrows, tip inside.\\
    \Lnotation{<T-T>}   & \myline{<T-T>}    & curved lines.\\
%    \Lnotation{>T-T<}   & \mylineA{>T-T<}    & \TikZ\ like arrows.\\
    \hline
  \end{longtable}




\nocite{*}
\printbibliography


\end{document}

