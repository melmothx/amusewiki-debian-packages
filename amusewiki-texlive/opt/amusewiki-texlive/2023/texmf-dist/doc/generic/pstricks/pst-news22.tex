%% $Id: pst-news21.tex 382 2021-12-29 19:19:18Z herbert $
\RequirePackage{pdfmanagement-testphase}
\DeclareDocumentMetadata{}
\documentclass[11pt,english,BCOR=10mm,DIV=12,bibliography=totoc,parskip=false,headings=small,
    headinclude=false,footinclude=false,twoside,usegeometry,dvipsnames]{pst-doc}
    
\usepackage{libertinus}
\usepackage{hvlogos}
\listfiles
%\usepackage[svgnames,dvipsnames,x11names,pspdf=-dALLOWPSTRANSPARENCY]{pstricks-pdf}
\usepackage{showexpl,pst-arrow,pst-plot,pst-geometrictools}
\lstset{explpreset={pos=l,width=-99pt,overhang=0pt,hsep=\columnsep,vsep=\bigskipamount,rframe={},extendedchars},
    escapechar=?}

\usepackage{biblatex}
\addbibresource{PSTricks.bib}
\def\Lcs#1{\texttt{\textbackslash#1}}
\begin{document}

\title{\texttt{News -- \the\year}\\ \Large new macros and bugfixes for the basic package.}
\author{Herbert Voß}
\date{\today}

\settitle

\tableofcontents

\part{\texttt{pstricks} -- package}

This version of the News was run with \verb|lualatex| \emph{without} using Ghostscript.
The PDF file was created in a direkt way by Lua. If you run \texttt{lualatex} then the
packahe \texttt{luapstricks} is automatically loaded.


%--------------------------------------------------------------------------------------
\section{\texttt{pstricks.sty}}
%--------------------------------------------------------------------------------------
The optional argument \texttt{gsfonts} can be used to load only the symbol font from GhostScript.
Otherwise the one from URW or the system is used, which is the default.



%--------------------------------------------------------------------------------------
\section{\texttt{pstricks.tex} (v. 3.17 -- 2022/10/22)}
%--------------------------------------------------------------------------------------


\subsection{pgf library}

This version fixes a problem with the upcoming pgf from which PSTricks uses the \Lcs{foreach}
command as \Lcs{pgfforeach}.

\subsection{Arrows}


There are new arrow types and a new optional argument \Lkeyword{tipcolor}.
Checking \Lkeyword{tipcolor} can be suppressed by setting

\begin{verbatim}
\makeatletter
\ps@check@tipcplor{}
\makeatother
\end{verbatim}

This is only needed in some rare cases, e.g. \Lcs{pscustom} with \Lcs{code} and
color setting on PS level.


\nocite{*}
\printbibliography


\end{document}

