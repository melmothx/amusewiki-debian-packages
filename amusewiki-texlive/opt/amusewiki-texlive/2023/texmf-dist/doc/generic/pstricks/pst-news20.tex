%% $Id: pst-news20.tex 25 2020-09-18 06:59:21Z herbert $
\documentclass[11pt,english,BCOR=10mm,DIV=12,bibliography=totoc,parskip=false,headings=small,
    headinclude=false,footinclude=false,twoside]{scrartcl}

\listfiles
\usepackage[T1]{fontenc}
\usepackage{listings}
\lstset{basicstyle=\ttfamily\small}
\usepackage{libertinus}
\usepackage[scaled=0.88]{beramono}
\usepackage{babel}
\usepackage[svgnames,dvipsnames,x11names,pspdf=-dALLOWPSTRANSPARENCY]{pstricks-pdf}
\usepackage{pstricks-add}

\usepackage{biblatex}
\addbibresource{PSTricks.bib}
\def\Lcs#1{\texttt{\textbackslash#1}}
\begin{document}

\title{\texttt{News -- \the\year}\\ \Large new macros and bugfixes for the basic package.}
\author{Herbert Voß}
\date{\today}

\maketitle

\tableofcontents

\part{\texttt{pstricks} -- package}

%--------------------------------------------------------------------------------------
\section{\texttt{pstricks.sty} -- \texttt{pstricks-pdf.sty}}
%--------------------------------------------------------------------------------------

With the package \texttt{pstricks-pdf} you can now use 

\begin{verbatim}
pdflatex --shell-escape <file>
\end{verbatim}

This document was created this way. Remember that you have to use the environment \texttt{postscript}
if you do not use the environment \texttt{pspicture} or a lot of PS-code outside this environment:

\begin{lstlisting}
\begin{postscript}
\pstVerb{/LL 1 def /RR 140 def /CCmy 6 def /RsqC RR dup mul 1000 div CCmy mul 1000 div def 
/omegam LL RR div 1000 mul RR div CCmy div 1000 mul 1 sub sqrt RR mul 3 sqrt div LL div def 
/phiomegafunc {/omega exch def LL RsqC sub omega CCmy mul 1000 div omega mul 1000 div 
    LL mul LL mul sub omega mul dup 0 ge {RR atan}{RR atan 360 sub} ifelse} def 
/phimax omegam phiomegafunc def}
\begin{center}
\begin{psgraph}[axesstyle=frame,yAxisLabel=$\varphi$,xAxisLabel=$\omega$/Hz,
	yticksize=0 16cm,xticksize=-90 
	90,subticksize=1,Dy=20,Dx=100,xsubticks=2](0,0)(0,-90)(800,90){16cm}{8cm}
\psplot[linecolor=Blue1,plotpoints=200,linewidth=2pt]{0}{800}{x phiomegafunc}
\uput{0pt}[0](10,75){$R=$\psPrintValue{RR}\hspace{2em}$\Omega$}
\uput{0pt}[0](10,65){$C=$\psPrintValue{CCmy}\hspace{0.75em}$\mu$F}
\uput{0pt}[0](10,55){$L=\psPrintValue{LL}\hspace{0.75em}\text{H}$}
\pscircle*[linecolor=Red1](!omegam phimax){2pt}
\end{psgraph}
\end{center}
\end{postscript}
\end{lstlisting}

\resizebox{\linewidth}{!}{%
\begin{postscript}
\pstVerb{/LL 1 def /RR 140 def /CCmy 6 def /RsqC RR dup mul 1000 div CCmy mul 1000 div def /omegam 
LL RR div 1000 mul RR div CCmy div 1000 mul 1 sub sqrt RR mul 3 sqrt div LL div def /phiomegafunc 
{/omega exch def LL RsqC sub omega CCmy mul 1000 div omega mul 1000 div LL mul LL mul sub omega mul 
dup 0 ge {RR atan}{RR atan 360 sub} ifelse} def /phimax omegam phiomegafunc def}
\begin{psgraph}[axesstyle=frame,yAxisLabel=$\varphi$,xAxisLabel=$\omega$/Hz,
	yticksize=0 16cm,xticksize=-90 
	90,subticksize=1,Dy=20,Dx=100,xsubticks=2](0,0)(0,-90)(800,90){16cm}{8cm}
\psplot[linecolor=Blue1,plotpoints=200,linewidth=2pt]{0}{800}{x phiomegafunc}
\uput*{0pt}[0](10,75){$R=$\psPrintValue{RR}\hspace{2em}$\Omega$}
\uput*{0pt}[0](10,65){$C=$\psPrintValue{CCmy}\hspace{0.75em}$\mu$F}
\uput*{0pt}[0](10,55){$L=\psPrintValue{LL}\hspace{0.75em}\text{H}$}
\psdot[linecolor=Red1](!omegam phimax)
\end{psgraph}
\end{postscript}%
}

(Example by Poul Riis)
%--------------------------------------------------------------------------------------
\section{\texttt{pstricks.tex} (v. 2.99 -- 2020/06/09)}
%--------------------------------------------------------------------------------------

New optional arguments \texttt{griddx} and \texttt{griddy}, which are only valid
for \Lcs{psgrid}:

\begin{lstlisting}[basicstyle=\small\ttfamily]
\psset{unit=5mm}
\newpsstyle{gridstyle}{gridlabels=8pt, gridfont=Helvetica, gridcolor=red, 
  subgridcolor=gray, subgriddiv=5, gridwidth=.8pt, subgridwidth=.4pt, 
  griddots=10, subgriddots=5,
  griddx=5, griddy=2 }
\begin{pspicture}[showgrid](25,25)
\end{pspicture}
\end{lstlisting}


\bigskip
\begin{postscript}
\psset{unit=5mm}
\newpsstyle{gridstyle}{gridlabels=8pt, gridfont=Helvetica, gridcolor=red, 
  subgridcolor=gray, subgriddiv=5, gridwidth=.8pt, subgridwidth=.4pt, 
  griddots=10, subgriddots=5,
  griddx=5, griddy=2 }
\begin{pspicture}[showgrid](25,25)
\end{pspicture}
\end{postscript}

%--------------------------------------------------------------------------------------
\section{\texttt{pstricks.pro}}
%--------------------------------------------------------------------------------------

The function \texttt{Grid} supports GridDX and GridDY.

The current version 1.32 should handle transparency for all Ghostscript versions.
For versions > 9.52 you need for \verb|ps2pdf| the optional argument \verb|-dALLOWPSTRANSPARENCY|
instead of \verb|-dNOSAFER|, which is still needed if you want to write Postscript files from within
the \TeX-run.


\nocite{*}
\printbibliography


\end{document}

