%# -*- coding: utf-8 -*-
\ifx\epTeXinputencoding\undefined\else % defined in e-pTeX (> TL2016)
  \epTeXinputencoding utf8    % ensure utf-8 encoding for platex
\fi

\documentclass[a4paper]{jsarticle}% supports \verb in \footnote
\usepackage[nohyperref]{doc}% doc v3
\usepackage{pldocverb}
\GetFileInfo{pldocverb.sty}
\title{Package \textsf{pldocverb} \fileversion}
\author{Hironobu Yamashita}
\date{\filedate}
\begin{document}

\maketitle

This package \textsf{pldocverb} provides small patches to
\textsf{doc} package of \textsf{latex(-base)} for use with
Japanese p\LaTeX/up\LaTeX.
Current package supports re-definition of \verb+\verb+ command.

This package is part of \textsf{platex-tools} bundle:
\begin{verbatim}
  https://github.com/aminophen/platex-tools
\end{verbatim}

\bigskip

この\textsf{pldocverb}パッケージは、
\textsf{latex(-base)}バンドルの\textsf{doc}パッケージによって
上書きされて無効化されてしまうp\LaTeX/up\LaTeX カーネルの修正点を
再有効化します。具体的には、\verb+\verb+コマンドの直前に
\verb+\xkanjiskip+挿入を許可するための修正に対応します。

\bigskip
\begin{minipage}{0.6\linewidth}
\noautoxspacing %% on purpose
\begin{verbatim}
  \documentclass{jarticle} % if you are using pLaTeX,
  \usepackage{pldocverb}   % load this!
  \usepackage{minijs}
  \begin{document}
  \setlength{\xkanjiskip}{10pt}
  これが\verb+test+で\verb+テスト+です。
  \end{document}
\end{verbatim}
\end{minipage}
\begin{minipage}{0.35\linewidth}
  \setlength{\xkanjiskip}{10pt}
  これが\verb+test+で\verb+テスト+です。
\end{minipage}
\bigskip

\end{document}
