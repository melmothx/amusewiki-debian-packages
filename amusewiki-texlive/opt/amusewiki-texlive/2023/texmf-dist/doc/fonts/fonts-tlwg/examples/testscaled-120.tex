%%
%% Test for font scaling at 120%
%% 2017/09/11 Abhabongse Janthong <abhabongse@gmail.edu>
%%     - write a new test introducing the usage of package option
%%
\documentclass[a4paper]{article}
\usepackage[english,thai]{babel}
\usepackage[utf8x]{inputenc}
\usepackage[scale=1.2]{fonts-tlwg}

\newcommand{\testthaipoem}[3]{%
  \usefont{LTH}{#1}{#2}{#3}
  \noindent
  \begin{tabbing}
  {\usefont{LTH}{norasi}{b}{n}
  XXXXXXXXXXXXXXXXXXXXXXXXX} \=
  {\usefont{LTH}{norasi}{b}{n}
  XXXXXXXXXXXXXXXXXXXXXXXXX}\kill
  \hspace{1em}๏ เป็นมนุษย์สุดประเสริฐเลิศคุณค่า \> กว่าบรรดาฝูงสัตว์เดรัจฉาน \\
  จงฝ่าฟันพัฒนาวิชาการ          \> อย่าล้างผลาญฤๅเข่นฆ่าบีฑาใคร\\
  ไม่ถือโทษโกรธแช่งซัดฮึดฮัดด่า    \> หัดอภัยเหมือนกีฬาอัชฌาสัย \\
  ปฏิบัติประพฤติกฎกำหนดใจ       \> พูดจาให้จ๊ะๆ จ๋าๆ น่าฟังเอย ฯ\\
  \end{tabbing}}

\newcommand{\testenglish}[3]{%
  \usefont{LTH}{#1}{#2}{#3}
  \noindent
  A quick brown fox jumps over the lazy dog.}

\newcommand{\testEnglish}[3]{%
  \usefont{LTH}{#1}{#2}{#3}
  \noindent
  \MakeUppercase{A quick brown fox jumps over the lazy dog.}}

\newcommand{\testligkern}[3]{%
  \usefont{LTH}{#1}{#2}{#3}
  \noindent
  ที่ ท่า ทิ้ง ท้า กิ๊ง ก๊ง ตี๋ ต๋า บ่น ป่น, บ้น ป้น, บ๊น ป๊น, บ๋น ป๋น บิน ปิน บีน ปีน บิ่น ปิ่น บัน ปั่น บั่น
  ก็ ป็ ปู่ ญ ญุ ญู ญฺ ฐ ฐุ ฐู ฐฺ กุ ฎุ ฎู ฎฺ ฏุ ฏู ฏฺ บำ บ่ำ ปำ ป่ำ -\textyamakkan{}
  \textfongmun{} \textangkhankhu{} \textkhomut{}
  ปะเฺติ็ลฺ โฺญฺ็จฺ ปั็วฮฺ ทฺ็อง เปฺิ็ว มูํย
  แต็่ง เจฺํอ เปรฺิ่ห์ โจ๊่ เปฺี่ย โฺทร ม็่อง เติ็ง อาื ยาึ ปิํปี็ป็่ป๊่ปฺ่
  จือรฺุ การฺู
  - -- --- `` '' \dag{} \ddag{} \S{} \P{} \${} \ae{} \AE{} \oe{} \OE{} \aa{}
  \AA{} \ss{} \copyright{} \textregistered{} \texttrademark{} \textbackslash{}
  \textasciicircum{} \textasciitilde{} \textbar{} \textbraceleft{}
  \textbraceright{} ?` !` ff fi fl ffi ffl tt ti AV\\}

\newcommand{\testpali}[3]{%
  \usefont{LTH}{#1}{#2}{#3}
  \noindent
  \textpali{หตฺเถสุ ภิกฺขเว สติ, อาทานนิกฺเขปนํ ปญฺญายติ}\\
  \textpali{เอวเมว โข ภิกฺขเว}\\
  \textpali{จกฺขุสมิํปิ สติ}\\
  \textpali{จกฺขุสมฺผสฺสปจฺจยา อุปฺปชฺชติ อชฺฌตฺตํ สุขทุกฺขํ}\\
  \textpali{ทิฏฺฐา มยา ภิกฺขเว ฉ ผสฺสายตนิกา นาม นิรยา}\\}

\begin{document}
\pagestyle{empty}
\vfil
\begin{figure*}
\Huge
\hspace*{.2\textwidth}\usefont{LTH}{kinnari}{m}{n}แบบอักษรไทยใน \LaTeX\\
\hspace*{.2\textwidth}\usefont{LTH}{garuda}{m}{n}แบบอักษรไทยใน \LaTeX\\
\hspace*{.2\textwidth}\usefont{LTH}{norasi}{m}{n}แบบอักษรไทยใน \LaTeX\\
\hspace*{.2\textwidth}\usefont{LTH}{laksaman}{m}{n}แบบอักษรไทยใน \LaTeX\\
\hspace*{.2\textwidth}\usefont{LTH}{ttype}{m}{n}แบบอักษรไทยใน \LaTeX\\
\hspace*{.2\textwidth}\usefont{LTH}{ttypist}{m}{n}แบบอักษรไทยใน \LaTeX\\
\hspace*{.2\textwidth}\usefont{LTH}{purisa}{m}{n}แบบอักษรไทยใน \LaTeX\\
\hspace*{.2\textwidth}\usefont{LTH}{loma}{m}{n}แบบอักษรไทยใน \LaTeX\\
\hspace*{.2\textwidth}\usefont{LTH}{waree}{m}{n}แบบอักษรไทยใน \LaTeX\\
\hspace*{.2\textwidth}\usefont{LTH}{umpush}{m}{n}แบบอักษรไทยใน \LaTeX\\
\hspace*{.2\textwidth}\usefont{LTH}{sawasdee}{m}{n}แบบอักษรไทยใน \LaTeX\\
\end{figure*}
\vfil
\clearpage

\pagestyle{plain}
\section{\usefont{LTH}{kinnari}{b}{n}Kinnari -- กินรี\protect\footnote{จากโครงการฟอนต์แห่งชาติ (National Font Project)}}

\subsection{ตัวอย่างประโยคภาษาไทย\protect\footnote{โดยสมาคมคอมพิวเตอร์แห่งประเทศไทยในพระบรมราชูปถัมภ์}}

\testthaipoem{kinnari}{m}{n}

\testthaipoem{kinnari}{b}{n}

\testthaipoem{kinnari}{m}{it}

\testthaipoem{kinnari}{b}{it}

\testthaipoem{kinnari}{m}{sl}

\testthaipoem{kinnari}{b}{sl}

\subsection{ตัวอย่างภาษาอังกฤษ}

\testenglish{kinnari}{m}{n}

\testenglish{kinnari}{b}{n}

\testenglish{kinnari}{m}{it}

\testenglish{kinnari}{b}{it}

\testenglish{kinnari}{m}{sl}

\testenglish{kinnari}{b}{sl}

\testEnglish{kinnari}{m}{n}

\testEnglish{kinnari}{b}{n}

\testEnglish{kinnari}{m}{it}

\testEnglish{kinnari}{b}{it}

\testEnglish{kinnari}{m}{sl}

\testEnglish{kinnari}{b}{sl}


\subsection{การจัดระดับตัวอักษรและตัวอักษรพิเศษ}
\noindent
\testligkern{kinnari}{m}{n}

\testligkern{kinnari}{b}{n}

\testligkern{kinnari}{m}{it}

\testligkern{kinnari}{b}{it}

\testligkern{kinnari}{m}{sl}

\testligkern{kinnari}{b}{sl}

\subsection{ภาษาบาลี-สันสกฤต}
\testpali{kinnari}{m}{n}

\testpali{kinnari}{b}{n}

\testpali{kinnari}{m}{it}

\testpali{kinnari}{b}{it}

\testpali{kinnari}{m}{sl}

\testpali{kinnari}{b}{sl}

\vfil\pagebreak


\section{\usefont{LTH}{garuda}{b}{n}Garuda -- ครุฑ\protect\footnote{จากโครงการฟอนต์แห่งชาติ (National Font Project)}}

\subsection{ตัวอย่างประโยคภาษาไทย}

\testthaipoem{garuda}{m}{n}

\testthaipoem{garuda}{b}{n}

\testthaipoem{garuda}{m}{it}

\testthaipoem{garuda}{b}{it}

\subsection{ตัวอย่างภาษาอังกฤษ}

\testenglish{garuda}{m}{n}

\testenglish{garuda}{b}{n}

\testenglish{garuda}{m}{it}

\testenglish{garuda}{b}{it}

\testEnglish{garuda}{m}{n}

\testEnglish{garuda}{b}{n}

\testEnglish{garuda}{m}{it}

\testEnglish{garuda}{b}{it}


\subsection{การจัดระดับตัวอักษรและตัวอักษรพิเศษ}

\testligkern{garuda}{m}{n}

\testligkern{garuda}{b}{n}

\testligkern{garuda}{m}{it}

\testligkern{garuda}{b}{it}


\subsection{ภาษาบาลี-สันสกฤต}

\testpali{garuda}{m}{n}

\testpali{garuda}{b}{n}

\testpali{garuda}{m}{it}

\testpali{garuda}{b}{it}

\vfil\pagebreak

\section{\usefont{LTH}{norasi}{b}{n}Norasi -- นรสีห์\protect\footnote{จากโครงการฟอนต์แห่งชาติ (National Font Project)}}

\subsection{ตัวอย่างประโยคภาษาไทย}

\testthaipoem{norasi}{m}{n}

\testthaipoem{norasi}{b}{n}

\testthaipoem{norasi}{m}{it}

\testthaipoem{norasi}{b}{it}

\testthaipoem{norasi}{m}{sl}

\testthaipoem{norasi}{b}{sl}

\subsection{ตัวอย่างภาษาอังกฤษ}
\testenglish{norasi}{m}{n}

\testenglish{norasi}{b}{n}

\testenglish{norasi}{m}{it}

\testenglish{norasi}{b}{it}

\testenglish{norasi}{m}{sl}

\testenglish{norasi}{b}{sl}

\testEnglish{norasi}{m}{n}

\testEnglish{norasi}{b}{n}

\testEnglish{norasi}{m}{it}

\testEnglish{norasi}{b}{it}

\testEnglish{norasi}{m}{sl}

\testEnglish{norasi}{b}{sl}


\subsection{การจัดระดับตัวอักษรและตัวอักษรพิเศษ}

\testligkern{norasi}{m}{n}

\testligkern{norasi}{b}{n}

\testligkern{norasi}{m}{it}

\testligkern{norasi}{b}{it}

\testligkern{norasi}{m}{sl}

\testligkern{norasi}{b}{sl}

\subsection{ภาษาบาลี-สันสกฤต}

\testpali{norasi}{m}{n}

\testpali{norasi}{b}{n}

\testpali{norasi}{m}{it}

\testpali{norasi}{b}{it}

\vfil\pagebreak


\section{\usefont{LTH}{laksaman}{b}{n}Laksaman -- ลักษมัณ\protect\footnote{ดัดแปลงจาก TH Sarabun New ของคุณศุภกิจ เฉลิมลาภ}}

\subsection{ตัวอย่างประโยคภาษาไทย}

\testthaipoem{laksaman}{m}{n}

\testthaipoem{laksaman}{b}{n}

\testthaipoem{laksaman}{m}{it}

\testthaipoem{laksaman}{b}{it}

\subsection{ตัวอย่างภาษาอังกฤษ}

\testenglish{laksaman}{m}{n}

\testenglish{laksaman}{b}{n}

\testenglish{laksaman}{m}{it}

\testenglish{laksaman}{b}{it}

\testEnglish{laksaman}{m}{n}

\testEnglish{laksaman}{b}{n}

\testEnglish{laksaman}{m}{it}

\testEnglish{laksaman}{b}{it}


\subsection{การจัดระดับตัวอักษรและตัวอักษรพิเศษ}

\testligkern{laksaman}{m}{n}

\testligkern{laksaman}{b}{n}

\testligkern{laksaman}{m}{it}

\testligkern{laksaman}{b}{it}


\subsection{ภาษาบาลี-สันสกฤต}

\testpali{laksaman}{m}{n}

\testpali{laksaman}{b}{n}

\testpali{laksaman}{m}{it}

\testpali{laksaman}{b}{it}

\vfil\pagebreak


\section{\usefont{LTH}{ttype}{b}{n}Tlwg Typewriter\protect\footnote{โดย Thai Linux Working Group (TLWG)}}

\subsection{ตัวอย่างประโยคภาษาไทย}

\testthaipoem{ttype}{m}{n}

\testthaipoem{ttype}{b}{n}

\testthaipoem{ttype}{m}{it}

\testthaipoem{ttype}{b}{it}

\subsection{ตัวอย่างภาษาอังกฤษ}

\testenglish{ttype}{m}{n}

\testenglish{ttype}{b}{n}

\testenglish{ttype}{m}{it}

\testenglish{ttype}{b}{it}

\testEnglish{ttype}{m}{n}

\testEnglish{ttype}{b}{n}

\testEnglish{ttype}{m}{it}

\testEnglish{ttype}{b}{it}


\subsection{การจัดระดับตัวอักษรและตัวอักษรพิเศษ}

\testligkern{ttype}{m}{n}

\testligkern{ttype}{b}{n}

\testligkern{ttype}{m}{it}

\testligkern{ttype}{b}{it}


\subsection{ภาษาบาลี-สันสกฤต}

\testpali{ttype}{m}{n}

\testpali{ttype}{b}{n}

\testpali{ttype}{m}{it}

\testpali{ttype}{b}{it}

\vfil\pagebreak


\section{\usefont{LTH}{ttypist}{b}{n}Tlwg Typist\protect\footnote{โดย Thai Linux Working Group (TLWG)}}

\subsection{ตัวอย่างประโยคภาษาไทย}

\testthaipoem{ttypist}{m}{n}

\testthaipoem{ttypist}{b}{n}

\testthaipoem{ttypist}{m}{it}

\testthaipoem{ttypist}{b}{it}

\subsection{ตัวอย่างภาษาอังกฤษ}

\testenglish{ttypist}{m}{n}

\testenglish{ttypist}{b}{n}

\testenglish{ttypist}{m}{it}

\testenglish{ttypist}{b}{it}

\testEnglish{ttypist}{m}{n}

\testEnglish{ttypist}{b}{n}

\testEnglish{ttypist}{m}{it}

\testEnglish{ttypist}{b}{it}


\subsection{การจัดระดับตัวอักษรและตัวอักษรพิเศษ}

\testligkern{ttypist}{m}{n}

\testligkern{ttypist}{b}{n}

\testligkern{ttypist}{m}{it}

\testligkern{ttypist}{b}{it}


\subsection{ภาษาบาลี-สันสกฤต}

\testpali{ttypist}{m}{n}

\testpali{ttypist}{b}{n}

\testpali{ttypist}{m}{it}

\testpali{ttypist}{b}{it}

\vfil\pagebreak


\section{\usefont{LTH}{purisa}{m}{n}Purisa -- ภูริสา\protect\footnote{โดย Thai Linux Working Group (TLWG)}}

\subsection{ตัวอย่างประโยคภาษาไทย}

\testthaipoem{purisa}{m}{n}

\testthaipoem{purisa}{b}{n}

\testthaipoem{purisa}{m}{it}

\testthaipoem{purisa}{b}{it}

\subsection{ตัวอย่างภาษาอังกฤษ}

\testenglish{purisa}{m}{n}

\testenglish{purisa}{b}{n}

\testenglish{purisa}{m}{it}

\testenglish{purisa}{b}{it}

\testEnglish{purisa}{m}{n}

\testEnglish{purisa}{b}{n}

\testEnglish{purisa}{m}{it}

\testEnglish{purisa}{b}{it}


\subsection{การจัดระดับตัวอักษรและตัวอักษรพิเศษ}

\testligkern{purisa}{m}{n}

\testligkern{purisa}{b}{n}

\testligkern{purisa}{m}{it}

\testligkern{purisa}{b}{it}


\subsection{ภาษาบาลี-สันสกฤต}

\testpali{purisa}{m}{n}

\testpali{purisa}{b}{n}

\testpali{purisa}{m}{it}

\testpali{purisa}{b}{it}

\vfil\pagebreak


\section{\usefont{LTH}{loma}{b}{n}Loma -- โลมา\protect\footnote{จากศูนย์เทคโนโลยีอิเล็กทรอนิกส์และคอมพิวเตอร์แห่งชาติ (NECTEC)}}

\subsection{ตัวอย่างประโยคภาษาไทย}

\testthaipoem{loma}{m}{n}

\testthaipoem{loma}{b}{n}

\testthaipoem{loma}{m}{it}

\testthaipoem{loma}{b}{it}

\subsection{ตัวอย่างภาษาอังกฤษ}

\testenglish{loma}{m}{n}

\testenglish{loma}{b}{n}

\testenglish{loma}{m}{it}

\testenglish{loma}{b}{it}

\testEnglish{loma}{m}{n}

\testEnglish{loma}{b}{n}

\testEnglish{loma}{m}{it}

\testEnglish{loma}{b}{it}


\subsection{การจัดระดับตัวอักษรและตัวอักษรพิเศษ}

\testligkern{loma}{m}{n}

\testligkern{loma}{b}{n}

\testligkern{loma}{m}{it}

\testligkern{loma}{b}{it}


\subsection{ภาษาบาลี-สันสกฤต}

\testpali{loma}{m}{n}

\testpali{loma}{b}{n}

\testpali{loma}{m}{it}

\testpali{loma}{b}{it}

\vfil\pagebreak


\section{\usefont{LTH}{waree}{b}{n}Waree -- วารี\protect\footnote{โดย คุณวิทยา ไตรสารวัฒนะ}}

\subsection{ตัวอย่างประโยคภาษาไทย}

\testthaipoem{waree}{m}{n}

\testthaipoem{waree}{b}{n}

\testthaipoem{waree}{m}{it}

\testthaipoem{waree}{b}{it}

\subsection{ตัวอย่างภาษาอังกฤษ}

\testenglish{waree}{m}{n}

\testenglish{waree}{b}{n}

\testenglish{waree}{m}{it}

\testenglish{waree}{b}{it}

\testEnglish{waree}{m}{n}

\testEnglish{waree}{b}{n}

\testEnglish{waree}{m}{it}

\testEnglish{waree}{b}{it}


\subsection{การจัดระดับตัวอักษรและตัวอักษรพิเศษ}

\testligkern{waree}{m}{n}

\testligkern{waree}{b}{n}

\testligkern{waree}{m}{it}

\testligkern{waree}{b}{it}

\subsection{ภาษาบาลี-สันสกฤต}

\testpali{waree}{m}{n}

\testpali{waree}{b}{n}

\testpali{waree}{m}{it}

\testpali{waree}{b}{it}

\vfil\pagebreak


\section{\usefont{LTH}{umpush}{b}{n}Umpush -- อัมพุช\protect\footnote{โดย คุณวิทยา ไตรสารวัฒนะ}}

\subsection{ตัวอย่างประโยคภาษาไทย}

\testthaipoem{umpush}{l}{n}

\testthaipoem{umpush}{m}{n}

\testthaipoem{umpush}{b}{n}

\testthaipoem{umpush}{l}{it}

\testthaipoem{umpush}{m}{it}

\testthaipoem{umpush}{b}{it}

\subsection{ตัวอย่างภาษาอังกฤษ}

\testenglish{umpush}{l}{n}

\testenglish{umpush}{m}{n}

\testenglish{umpush}{b}{n}

\testenglish{umpush}{l}{it}

\testenglish{umpush}{m}{it}

\testenglish{umpush}{b}{it}

\testEnglish{umpush}{l}{n}

\testEnglish{umpush}{m}{n}

\testEnglish{umpush}{b}{n}

\testEnglish{umpush}{l}{it}

\testEnglish{umpush}{m}{it}

\testEnglish{umpush}{b}{it}


\subsection{การจัดระดับตัวอักษรและตัวอักษรพิเศษ}

\testligkern{umpush}{l}{n}

\testligkern{umpush}{m}{n}

\testligkern{umpush}{b}{n}

\testligkern{umpush}{l}{it}

\testligkern{umpush}{m}{it}

\testligkern{umpush}{b}{it}


\subsection{ภาษาบาลี-สันสกฤต}

\testpali{umpush}{l}{n}

\testpali{umpush}{m}{n}

\testpali{umpush}{b}{n}

\testpali{umpush}{l}{it}

\testpali{umpush}{m}{it}

\testpali{umpush}{b}{it}

\vfil\pagebreak


\section{\usefont{LTH}{sawasdee}{b}{n}Sawasdee -- สวัสดี\protect\footnote{โดย คุณพล อุดมวิทยานุกูล}}

\subsection{ตัวอย่างประโยคภาษาไทย}

\testthaipoem{sawasdee}{m}{n}

\testthaipoem{sawasdee}{b}{n}

\testthaipoem{sawasdee}{m}{it}

\testthaipoem{sawasdee}{b}{it}

\subsection{ตัวอย่างภาษาอังกฤษ}

\testenglish{sawasdee}{m}{n}

\testenglish{sawasdee}{b}{n}

\testenglish{sawasdee}{m}{it}

\testenglish{sawasdee}{b}{it}

\testEnglish{sawasdee}{m}{n}

\testEnglish{sawasdee}{b}{n}

\testEnglish{sawasdee}{m}{it}

\testEnglish{sawasdee}{b}{it}


\subsection{การจัดระดับตัวอักษรและตัวอักษรพิเศษ}

\testligkern{sawasdee}{m}{n}

\testligkern{sawasdee}{b}{n}

\testligkern{sawasdee}{m}{it}

\testligkern{sawasdee}{b}{it}


\subsection{ภาษาบาลี-สันสกฤต}

\testpali{sawasdee}{m}{n}

\testpali{sawasdee}{b}{n}

\testpali{sawasdee}{m}{it}

\testpali{sawasdee}{b}{it}

\vfil\pagebreak


\end{document}
