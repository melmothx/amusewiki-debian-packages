\documentclass[a4paper]{article}
\usepackage[thai]{babel}
\usepackage[utf8x]{inputenc}
\usepackage[norasi-osf]{fonts-tlwg}

\usepackage[bf,sc]{titlesec}

% For "มงคล 38 ประการ"
\usepackage{enumitem}

\newcommand{\texturl}[1]{\textit{#1}}

\title{Old Style Figures Demonstration}
\author{}
\date{}

\newcommand{\pali}[1]{\textit{\textpali{#1}}}
\newcommand{\source}[2]{%
  \par\noindent\rule{\textwidth}{0.4pt}
  \emph{#1:} #2}

\begin{document}
\maketitle

\section{Ten Commandments}

\begin{enumerate}
  \item I am the Lord thy God. Thou shalt have no other gods before me.
        Thou shalt not make unto thee any graven image.
  \item Thou shalt not take the name of the Lord thy God in vain.
  \item Remember the sabbath day, to keep it holy.
  \item Honour thy father and thy mother.
  \item Thou shalt not murder.
  \item Thou shalt not commit adultery.
  \item Thou shalt not steal.
  \item Thou shalt not bear false witness against thy neighbour.
  \item Thou shalt not covet thy neighbour's wife.
  \item Thou shalt not covet thy neighbour's house,
        or his slaves, or his animals, or anything of thy neighbour.
\end{enumerate}

You shall set up these stones, which I command you today, on Mount Gerizim.

\source{Source}{%
  Wikipedia
  \texturl{https://en.wikipedia.org/wiki/Ten\_Commandments}}

\section{มงคล 38 ประการ}

สมัยหนึ่ง พระผู้มีพระภาคประทับอยู่ ณ พระวิหารเชตวัน อารามของท่านอนาถบิณฑิกเศรษฐี
ใกล้พระนครสาวัตถี ครั้งนั้นแล ครั้นปฐมยามล่วงไป เทวดาตนหนึ่งมีรัศมีงามยิ่งนัก
ยังพระวิหารเชตวันทั้งสิ้นให้สว่างไสว เข้าไปเฝ้าพระผู้มีพระภาคถึงที่ประทับ ถวายบังคมแล้วยืนอยู่ ณ
ที่ควรส่วนข้างหนึ่ง ครั้นแล้ว ได้กราบทูลพระผู้มีพระภาคด้วยคาถาว่า

\begin{quote}
เทวดาและมนุษย์เป็นอันมาก ผู้หวังความสวัสดี ได้พากันคิดมงคลทั้งหลาย ขอพระองค์จงตรัสอุดมมงคล
\end{quote}

พระผู้มีพระภาคตรัสพระคาถาตอบว่า (จัดหมวดหมู่ตามบทคาถา)

\subsection{ฝึกให้เป็นคนดี}
\begin{enumerate}
  \item \pali{อเสวนา จ พาลานํ} (การไม่คบคนพาล ๑)
  \item \pali{ปณฺฑิตานญฺจ เสวนา} (การคบบัณฑิต ๑)
  \item \pali{ปูชา จ ปูชนียานํ} (การบูชาบุคคลที่ควรบูชา ๑)
\end{enumerate}
\pali{เอตมฺมงฺคลมุตฺตมํ ฯ} (นี้เป็นอุดมมงคล)

\subsection{สร้างความพร้อมในการฝึกตนเอง}
\begin{enumerate}[resume]
  \item \pali{ปฏิรูปเทสวาโส จ} (การอยู่ในประเทศอันสมควร ๑)
  \item \pali{ปุพฺเพ จ กตปุญฺญตา} (ความเป็นผู้มีบุญอันกระทำแล้วในกาลก่อน ๑)
  \item \pali{อตฺตสมฺมาปณิธิ จ} (การตั้งตนไว้ชอบ ๑)
\end{enumerate}
\pali{เอตมฺมงฺคลมุตฺตมํ ฯ} (นี้เป็นอุดมมงคล)

\subsection{ฝึกตนให้เป็นคนมีประโยชน์}
\begin{enumerate}[resume]
  \item \pali{พาหุสจฺจญฺจ} (พาหุสัจจะ ๑ ---มีความรอบรู้, ความเป็นพหูสูต)
  \item \pali{สิปฺปญฺจ} (ศิลป ๑)
  \item \pali{วินโย จ สุสิกฺขิโต} (วินัยที่ศึกษาดีแล้ว ๑)
  \item \pali{สุภาสิตา จ ยา วาจา} (วาจาสุภาษิต ๑)
\end{enumerate}
\pali{เอตมฺมงฺคลมุตฺตมํ ฯ} (นี้เป็นอุดมมงคล)

\subsection{บำเพ็ญประโยชน์ต่อครอบครัว}
\begin{enumerate}[resume]
  \item \pali{มาตาปิตุอุปฏฺฐานํ} (การบำรุงบิดามารดา ๑)
  \item \pali{ปุตฺตสงฺคโห} (การเลี้ยงดูบุตร ๑ ---แยกมาจาก \pali{ปุตฺตทารสฺส สงฺคโห})
  \item \pali{ทารสฺส สงฺคโห} (การสงเคราะห์ภรรยา-สามี ๑
                            ---แยกมาจาก \pali{ปุตฺตทารสฺส สงฺคโห})
  \item \pali{อนากุลา จ กมฺมนฺตา} (การงานอันไม่อากูล ๑ ---การงานไม่คั่งค้าง)
\end{enumerate}
\pali{เอตมฺมงฺคลมุตฺตมํ ฯ} (นี้เป็นอุดมมงคล)

\subsection{บำเพ็ญประโยชน์ต่อสังคม}
\begin{enumerate}[resume]
  \item \pali{ทานญฺจ} (ทาน ๑)
  \item \pali{ธมฺมจริยา จ} (การประพฤติธรรม ๑)
  \item \pali{ญาตกานญฺจ สงฺคโห} (การสงเคราะห์ญาติ ๑)
  \item \pali{อนวชฺชานิ กมฺมานิ} (กรรมอันไม่มีโทษ ๑)
\end{enumerate}
\pali{เอตมฺมงฺคลมุตฺตมํ ฯ} (นี้เป็นอุดมมงคล)

\subsection{ปรับเตรียมสภาพใจให้พร้อม}
\begin{enumerate}[resume]
  \item \pali{อารตี วิรตี ปาปา} (การงดเว้นจากบาป ๑)
  \item \pali{มชฺชปานา จ สญฺญโม} (ความสำรวมจากการดื่มน้ำเมา ๑)
  \item \pali{อปฺปมาโท จ ธมฺเมสุ} (ความไม่ประมาทในธรรมทั้งหลาย ๑)
\end{enumerate}
\pali{เอตมฺมงฺคลมุตฺตมํ ฯ} (นี้เป็นอุดมมงคล)

\subsection{การแสวงหาธรรมะเบื้องต้นใส่ตัว}
\begin{enumerate}[resume]
  \item \pali{คารโว จ} (ความเคารพ ๑)
  \item \pali{นิวาโต จ} (ความประพฤติถ่อมตน ๑)
  \item \pali{สนฺตุฏฺฐี จ} (ความสันโดษ ๑)
  \item \pali{กตญฺญุตา} (ความกตัญญู ๑)
  \item \pali{กาเลน ธมฺมสฺสวนํ} (การฟังธรรมโดยกาล ๑)
\end{enumerate}
\pali{เอตมฺมงฺคลมุตฺตมํ ฯ} (นี้เป็นอุดมมงคล)

\subsection{การแสวงหาธรรมะเบื้องสูงใส่ตัวให้เต็มที่}
\begin{enumerate}[resume]
  \item \pali{ขนฺตี จ} (ความอดทน ๑)
  \item \pali{โสวจสฺสตา} (ความเป็นผู้ว่าง่าย ๑)
  \item \pali{สมณานญฺจ ทสฺสนํ} (การได้เห็นสมณะทั้งหลาย ๑)
  \item \pali{กาเลน ธมฺมสากจฺฉา} (การสนทนาธรรมโดยกาล ๑)
\end{enumerate}
\pali{เอตมฺมงฺคลมุตฺตมํ ฯ} (นี้เป็นอุดมมงคล)

\subsection{การฝึกภาคปฏิบัติเพื่อกำจัดกิเลสให้สิ้นไป}
\begin{enumerate}[resume]
  \item \pali{ตโป จ} (ความเพียร ๑)
  \item \pali{พฺรหฺมจริยญฺจ} (พรหมจรรย์ ๑)
  \item \pali{อริยสจฺจานทสฺสนํ} (การเห็นอริยสัจ ๑)
  \item \pali{นิพฺพานสจฺฉิกิริยา จ} (การกระทำนิพพานให้แจ้ง ๑)
\end{enumerate}
\pali{เอตมฺมงฺคลมุตฺตมํ ฯ} (นี้เป็นอุดมมงคล)

\subsection{ผลจากการปฏิบัติตนจนหมดกิเลส}
\begin{enumerate}[resume]
  \item \pali{ผุฏฺฐสฺส โลกธมฺเมหิ จิตฺตํ ยสฺส น กมฺปติ} (จิตของผู้ใดอันโลกธรรมทั้งหลายถูกต้องแล้ว
        ย่อมไม่หวั่นไหว ๑)
  \item \pali{อโสกํ} (ไม่เศร้าโศก ๑)
  \item \pali{วิรชํ} (ปราศจากธุลี ๑)
  \item \pali{เขมํ} (เป็นจิตเกษม ๑)
\end{enumerate}
\pali{เอตมฺมงฺคลมุตฺตมํ ฯ} (นี้เป็นอุดมมงคล)

เทวดาและมนุษย์ทั้งหลาย ทำมงคลเช่นนี้แล้ว เป็นผู้ไม่ปราชัยในข้าศึกทุกหมู่เหล่า
ย่อมถึงความสวัสดีในที่ทุกสถาน นี้เป็นอุดมมงคลของเทวดาและมนุษย์เหล่านั้น ฯ

\source{ที่มา}{%
  \begin{itemize}
    \item มงคลสูตรในขุททกปาฐะ พระสุตตันตปิฎก ขุททกนิกาย \\
          \texturl{https://84000.org/tipitaka/book/v.php?B=25\&A=41\&Z=72} และ \\
          \texturl{https://84000.org/tipitaka/pali/pali\_item\_s.php?book=25\&item=5\&items=2}
    \item พ่อน้องกันต์. มงคลชีวิต 38 ประการ.
          \texturl{https://www.gotoknow.org/posts/382786}
  \end{itemize}}

\section{Reply from the Observatory of Cambridge}

Barbicane, however, lost not one moment amid all the enthusiasm of which he
had become the object. His first care was to reassemble his colleagues in the
board-room of the Gun Club. There, after some discussion, it was agreed to
consult the astronomers regarding the astronomical part of the enterprise.
Their reply once ascertained, they could then discuss the mechanical means, and
nothing should be wanting to ensure the success of this great experiment.

A note couched in precise terms, containing special interrogatories, was then
drawn up and addressed to the Observatory of Cambridge in Massachusetts. This
city, where the first university of the United States was founded, is justly
celebrated for its astronomical staff. There are to be found assembled all the
most eminent men of science. Here is to be seen at work that powerful telescope
which enabled Bond to resolve the nebula of Andromeda, and Clarke to discover
the satellite of Sirius. This celebrated institution fully justified on all
points the confidence reposed in it by the Gun Club. So, after two days, the
reply so impatiently awaited was placed in the hands of President Barbicane.

It was couched in the following terms:

\emph{The Director of the Cambridge Observatory to the President of the Gun Club at Baltimore.}

CAMBRIDGE, October 7. On the receipt of your favor of the 6th instant,
addressed to the Observatory of Cambridge in the name of the members of the
Baltimore Gun Club, our staff was immediately called together, and it was
judged expedient to reply as follows:

The questions which have been proposed to it are these---

``1.~Is it possible to transmit a projectile up to the moon?

``2.~What is the exact distance which separates the earth from its satellite?

``3.~What will be the period of transit of the projectile when endowed with
sufficient initial velocity? and, consequently, at what moment ought it to be
discharged in order that it may touch the moon at a particular point?

``4.~At what precise moment will the moon present herself in the most favorable
position to be reached by the projectile?

``5.~What point in the heavens ought the cannon to be aimed at which is
intended to discharge the projectile?

``6.~What place will the moon occupy in the heavens at the moment of the
projectile’s departure?''

Regarding the \emph{first} question, ``Is it possible to transmit a projectile
up to the moon?''

\emph{Answer.}---Yes; provided it possess an initial velocity of 1,200 yards per
second; calculations prove that to be sufficient. In proportion as we recede
from the earth the action of gravitation diminishes in the inverse ratio of the
square of the distance; that is to say, \emph{at three times a given distance
the action is nine times less.} Consequently, the weight of a shot will
decrease, and will become reduced to zero at the instant that the attraction of
the moon exactly counterpoises that of the earth; that is to say at 47/52 of
its passage. At that instant the projectile will have no weight whatever; and,
if it passes that point, it will fall into the moon by the sole effect of the
lunar attraction. The \emph{theoretical possibility} of the experiment is
therefore absolutely demonstrated; its \emph{success} must depend upon the
power of the engine employed.

As to the \emph{second} question, ``What is the exact distance which separates
the earth from its satellite?''

\emph{Answer.}---The moon does not describe a \emph{circle} round the earth,
but rather an \emph{ellipse,} of which our earth occupies one of the
\emph{foci;} the consequence, therefore, is, that at certain times it
approaches nearer to, and at others it recedes farther from, the earth; in
astronomical language, it is at one time in \emph{apogee,} at another in
\emph{perigee.} Now the difference between its greatest and its least distance
is too considerable to be left out of consideration. In point of fact, in its
apogee the moon is 247,552 miles, and in its perigee, 218,657 miles only
distant; a fact which makes a difference of 28,895 miles, or more than
one-ninth of the entire distance. The perigee distance, therefore, is that
which ought to serve as the basis of all calculations.

To the \emph{third} question:---

\emph{Answer.}---If the shot should preserve continuously its initial velocity
of 12,000 yards per second, it would require little more than nine hours to
reach its destination; but, inasmuch as that initial velocity will be
continually decreasing, it will occupy 300,000 seconds, that is 83hrs.\ 20m.\ in
reaching the point where the attraction of the earth and moon will be in
\emph{equilibrio.} From this point it will fall into the moon in 50,000
seconds, or 13hrs.\ 53m.\ 20sec. It will be desirable, therefore, to discharge
it 97hrs.\ 13m.\ 20sec.\ before the arrival of the moon at the point aimed at.

Regarding question \emph{four,} ``At what precise moment will the moon present
herself in the most favorable position, etc.?''

\emph{Answer.}---After what has been said above, it will be necessary, first of
all, to choose the period when the moon will be in perigee, and \emph{also} the
moment when she will be crossing the zenith, which latter event will further
diminish the entire distance by a length equal to the radius of the earth,
\emph{i.~e.}\ 3,919 miles; the result of which will be that the final passage
remaining to be accomplished will be 214,976 miles. But although the moon
passes her perigee every month, she does not reach the zenith always \emph{at
exactly the same moment.} She does not appear under these two conditions
simultaneously, except at long intervals of time. It will be necessary,
therefore, to wait for the moment when her passage in perigee shall coincide
with that in the zenith. Now, by a fortunate circumstance, on the 4th of
December in the ensuing year the moon \emph{will} present these two conditions.
At midnight she will be in perigee, that is, at her shortest distance from the
earth, and at the same moment she will be crossing the zenith.

On the \emph{fifth} question, ``At what point in the heavens ought the cannon
to be aimed?''

\emph{Answer.}---The preceding remarks being admitted, the cannon ought to be
pointed to the zenith of the place. Its fire, therefore, will be perpendicular
to the plane of the horizon; and the projectile will soonest pass beyond the
range of the terrestrial attraction. But, in order that the moon should reach
the zenith of a given place, it is necessary that the place should not exceed
in latitude the declination of the luminary; in other words, it must be
comprised within the degrees 0$^\circ$ and 28$^\circ$ of lat.\ N.\ or S.
In every other spot the fire must necessarily be oblique, which would seriously
militate against the success of the experiment.

As to the \emph{sixth} question, ``What place will the moon occupy in the
heavens at the moment of the projectile’s departure?''

\emph{Answer.}---At the moment when the projectile shall be discharged into
space, the moon, which travels daily forward 13$^\circ$ 10$'$ 35$''$, will be
distant from the zenith point by four times that quantity, \emph{i.~e.}\ by
52$^\circ$ 41$'$ 20$''$, a space which corresponds to the path which she will
describe during the entire journey of the projectile. But, inasmuch as it is
equally necessary to take into account the deviation which the rotary motion of
the earth will impart to the shot, and as the shot cannot reach the moon until
after a deviation equal to 16 radii of the earth, which, calculated upon the
moon’s orbit, are equal to about eleven degrees, it becomes necessary to add
these eleven degrees to those which express the retardation of the moon just
mentioned: that is to say, in round numbers, about sixty-four degrees.
Consequently, at the moment of firing the visual radius applied to the moon
will describe, with the vertical line of the place, an angle of sixty-four
degrees.

These are our answers to the questions proposed to the Observatory of Cambridge
by the members of the Gun Club:---

To sum up---

1st. The cannon ought to be planted in a country situated between 0$^\circ$ and
28$^\circ$ of N.\ or S.\ lat.

2nd. It ought to be pointed directly toward the zenith of the place.

3rd. The projectile ought to be propelled with an initial velocity of 12,000
yards per second.

4th. It ought to be discharged at 10hrs.\ 46m.\ 40sec.\ of the 1st of December
of the ensuing year.

5th. It will meet the moon four days after its discharge, precisely at midnight
on the 4th of December, at the moment of its transit across the zenith.

The members of the Gun Club ought, therefore, without delay, to commence the
works necessary for such an experiment, and to be prepared to set to work at
the moment determined upon; for, if they should suffer this 4th of December to
go by, they will not find the moon again under the same conditions of perigee
and of zenith until eighteen years and eleven days afterward.

The staff of the Cambridge Observatory place themselves entirely at their
disposal in respect of all questions of theoretical astronomy; and herewith add
their congratulations to those of all the rest of America. For the Astronomical
Staff, J. M. BELFAST, \emph{Director of the Observatory of Cambridge.}

\source{Source}{%
  The Project Gutenberg eBook of From the Earth to the Moon, by
  Jules Verne. \texturl{https://www.gutenberg.org/files/83/83-h/83-h.htm}}

\end{document}
